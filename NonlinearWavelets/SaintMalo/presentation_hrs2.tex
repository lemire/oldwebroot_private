%% LyX 1.2 created this file.  For more info, see http://www.lyx.org/.
%% Do not edit unless you really know what you are doing.
\documentclass[25pt,english,landscape]{foils}
\usepackage[T1]{fontenc}
\usepackage[latin1]{inputenc}
\usepackage{a4wide}
\usepackage{geometry}
\geometry{verbose,landscape,letterpaper,lmargin=1cm,rmargin=1cm}
\usepackage{fancyhdr}
\pagestyle{fancy}
\setcounter{secnumdepth}{5}
\setcounter{tocdepth}{5}
\setlength\parskip{\smallskipamount}
\setlength\parindent{0pt}
\usepackage{graphicx}
\usepackage{setspace}
\onehalfspacing
\usepackage{amssymb}

\makeatletter

%%%%%%%%%%%%%%%%%%%%%%%%%%%%%% LyX specific LaTeX commands.
\providecommand{\LyX}{L\kern-.1667em\lower.25em\hbox{Y}\kern-.125emX\@}

%%%%%%%%%%%%%%%%%%%%%%%%%%%%%% User specified LaTeX commands.
 \usepackage{color}
 \usepackage{pslatex}
 \usepackage{background}
 \usepackage[colorlinks=true]{hyperref}
 \newcommand{\pagetransition}[1]{%
     \hypersetup{pdfpagetransition={#1}}
 }                                                                               
 \usepackage{pause}
 \usepackage{backref}
 \usepackage{pp4slide}
%% \pagestyle{fancy}
 \usepackage{lastpage}     

%% Background of blue palette
 \definecolor{webblackblue}{rgb}{0.0,0.0,0.2}
 \definecolor{webblue}{rgb}{0.0, 0.0, 0.6}

%% Background of red palette
 \definecolor{webblackred}{rgb}{0.2,0.0,0.0}
 \definecolor{webred}{rgb}{0.6,0.0,0.0}

%% Background of green palette
 \definecolor{webblackgreen}{rgb}{0.0,0.2,0.0}                                   
 \definecolor{webgreen}{rgb}{0.0,0.6,0.0} 

%% Background of magenta palette
 \definecolor{webblackmagenta}{rgb}{0.14,0.0,0.14}                                   
 \definecolor{webmagenta}{rgb}{0.42,0.0,0.42} 

%% Background of cyan palette
 \definecolor{webblackcyan}{rgb}{0.0,0.14,0.14}                                   
 \definecolor{webcyan}{rgb}{0.0,0.42,0.42} 

%% Background of yellow pallete
 \definecolor{webblackyellow}{rgb}{0.14,0.14,0.0}                                   
 \definecolor{webyellow}{rgb}{0.85,0.85,0.0} 

 \definecolor{webgray}{rgb}{0.75,0.75,0.75}
 \definecolor{weborange}{rgb}{1.0, 0.6, 0.0}
 \usepackage{ae}
 \usepackage{colortbl}
 \usepackage{amsmath}
 \usepackage{amssymb}
 \usepackage{amsfonts}
 \usepackage{mathrsfs}
 \usepackage{amsthm}
 \usepackage{amscd}
 \usepackage{bbm}
 \usepackage{stmaryrd}
 \theoremstyle{plain}    
 \newtheorem{thm}{Theorem}
\usepackage{multicol}
\usepackage{pp4link}
\renewcommand{\headrulewidth}{0pt}
\renewcommand{\footrulewidth}{0pt}
%\hypersetup{pdfpagemode={FullScreen}} 
%\usepackage[transition={Replace}]{foliage}
\usepackage{foliage}
\makeatother
\AtBeginDocument{
  \renewcommand{\labelitemi}{\textcolor{yellow}{\P}}
  \renewcommand{\labelitemii}{\textcolor{yellow}{\S}}
  \renewcommand{\labelitemiii}{\textcolor{yellow}{\(\sharp\)}}
}

\usepackage{babel}
\begin{document}
\vpagecolor{webblue}

\color{yellow}


\foilhead{A Family of 4-point Dyadic High Resolution Subdivision Schemes}

\begin{center}\textbf{\large \href{mailto:lemire@ondelette.com}{Daniel Lemire}}\end{center}{\large \par}

\begin{center}Research Officer, National Research Council of Canada
(NRC) \end{center}

\begin{center}email: lemire@ondelette.com\end{center}

\begin{flushright}Curves and Surfaces\end{flushright}

\begin{flushright}Saint-Malo, France\end{flushright}

\begin{flushright}June-July 2002\end{flushright}


\foilhead{Subdivision: why care?}

\hypertarget{the_start}{}

\begin{multicols}{2}

\begin{itemize}
\item \textcolor{yellow}{\toplink{ITEMIZE}{Itemize}}
\item \textcolor{yellow}{\toplink{MATHED}{Mathed}}
\item \toplink{REFERENCES}{REFERENCES}
\end{itemize}
\end{multicols}

\cfoot{\hyperlink{the_start}{Index}}


\foilhead{ITEMIZE}

\toptarget{ITEMIZE}

This an example of itemize environment with pause:

\begin{itemize}
\item Primer itemize \pause
\item Segundo itemize \pause
\end{itemize}
270


\foilhead{MATHED}

\toptarget{MATHED}

Another example showing the capabilities of pause in one equation\begin{equation}
\int _{a}^{b}\, f(x)dx=\pause F(b)-F(b)\label{ec:first}\end{equation}


\hpagecolor[webblue]{white} 


\foilhead{CHANGING COLOR OF THE BACKGROUND}

In this slide one variation in the color degradation is done, the
gradient of color is horizontal


\foilhead{{\color{weborange}BLUE COLOUR}}

\color{white}

The basic references are \cite{1}, \cite{2}, 

\begin{enumerate}
\item This an example of enumerated environment with pause\pause
\item One equation is showed \pause\[
x=y+1\pause \]

\item The font has been changed to white
\end{enumerate}
\begin{itemize}
\item \textbf{An example of equation (\ref{ec:first}) with hyperref \hyperref[ec:first]{equation~\ref*{ec:first}}}
\item An another example of reference, \textbf{\autoref{ec:first}}
\newpage
\item We can use figures with links, click the figure below to return to
the index \pause
\end{itemize}
\vspace{0.375cm}
\begin{center}\hyperlink{the_start}{fds}\includegraphics[  width=0.25\paperwidth,bb = 0 0 200 100, draft, type=eps]{icon not found!}\end{center}
\vspace{0.375cm}


\foilhead{\color{weborange}RED COLOUR}

\vpagecolor[webred]{red}

\color{webyellow}

The basic references are \cite{1}, \cite{2}, this is an example
of bibliographic references.

90

\hypertarget{the_end}{}

\toptarget{REFERENCES}

Of course you can use bibtex possibilities. 

\begin{thebibliography}{1}
\bibitem{1}First reference
\bibitem{2}Second refernce\end{thebibliography}

\end{document}

