%----------------------- Beginning of TeX file --------------------------
%
%                  Abstract for CURVES and SURFACES
%                           SAINT-MALO 2002
%
%------------------------------------------------------------------------

%------------------------------------------------------------------------
% The modifiable part starts with the comment : "BEGIN HERE"
%------------------------------------------------------------------------

%------------------------------------------------------------------------
% Specific macros for Curves and Surfaces -- St-Malo 2002 abstracts
%------------------------------------------------------------------------
\font\fa=cmr17
\font\fb=cmr12
\newcount\refnum
\refnum = 0

\def\tit#1{\bigskip\bigskip\medskip\centerline{\fa#1}}
\def\titwo#1{\medskip \centerline{\fa#1}}
\def\tithree#1{\medskip \centerline{\fa#1}}
\def\aut{\bigskip\bigskip\centerline{\fb\author} }
\def\keys{\bigskip\noindent{\bf Keywords:\ }\keywords}
\long\def\abstext{\bigskip\noindent {\bf Abstract:\ }\par\abstract\bigskip\medskip}
\def\ref{\global\advance\refnum by 1 \item{\the\refnum .}}
\def\addr{\medskip\name\par\affiliate\par\address\par\email\par}

%------------------------------------------------------------------------
% Just "convenient" macros
%------------------------------------------------------------------------

\def\NN{\hbox{I\kern-.2em\hbox{N}}} % positive integers
\def\ZZ{{{\rm Z}\kern-.28em{\rm Z}}} % set of integers
\def\RR{\mathop{{\rm I}\kern-.2em{\rm R}}\nolimits} % reals

%------------------------------------------------------------------------
%
% BEGIN HERE
%
%------------------------------------------------------------------------
% Please, complete the following items :
%  1. title
%  2. author(s)
%  3. keyword(s)
%  4. abstract text
%  5. reference(s)
%  6. address of each author
%------------------------------------------------------------------------
%
%%%%%%%%%%%%%%%%%%%%%%%%%%%%%%%%%%%%%%%%%%%%%%%%%%%%%%%%%%%%%%%%%%%%%%%%%
% 1. TITLE
%    Please DO NOT insert any carriage return within your title.
%	 If the title is too long to fit into one line,
%	 split it and use titwo (and tithree if necessary).
%	 Capitalize the first letter of each nontrivial word.
%
% Eg : \tit{Part one of your title (mandatory)}
%      \titwo{Part two of your title if needed}
%      \tithree{Part three of your title if needed}
%
\tit{} % <--- Insert your title inside the braces.
%
%
%%%%%%%%%%%%%%%%%%%%%%%%%%%%%%%%%%%%%%%%%%%%%%%%%%%%%%%%%%%%%%%%%%%%%%%%%
% 2. AUTHORS
%    Please DO NOT insert any carriage return within the authors's list.
%    If there are several, please supply a comma separated list and put 
%    $^*$ after the speaker's name.
%
% Eg : \def\author{A. Cohen} defines one author
%      \def\author{x, y$^*$, z} defines 3 authors (y being the speaker)
%
\def\author{} % <--- Insert the list of author(s) inside the braces.
%
%
%%%%%%%%%%%%%%%%%%%%%%%%%%%%%%%%%%%%%%%%%%%%%%%%%%%%%%%%%%%%%%%%%%%%%%%%%
% 3. KEYWORDS
%    Please supply the keyword(s) of your communication as a comma 
%    separated list.
%
% Eg : \keywords{keyword1,keyword2,more,keywords} 
%
\def\keywords{} % <--- Insert the list of keyword(s) inside the braces.
%
%
%%%%%%%%%%%%%%%%%%%%%%%%%%%%%%%%%%%%%%%%%%%%%%%%%%%%%%%%%%%%%%%%%%%%%%%%%
% 4. ABSTRACT
%    The whole document MUST fit inside one page, preferably half a page. 
%    You may include here some of your favorite macros but, please, 
%    avoid to do so if possible.
%
% Insert the text of your abstract inside the \def\abstract braces.
%
\def\abstract{

} % End of \def\abstract (Do not edit this line)
%
%
%%%%%%%%%%%%%%%%%%%%%%%%%%%%%%%%%%%%%%%%%%%%%%%%%%%%%%%%%%%%%%%%%%%%%%%%%
% 5. REFERENCES
%    1.  Arrange your references in alphabetical order.
%    2.  Use the \ref macro. 
%        This macro will automatically number your references.
%    3.  Examples :
%
% Eg: \ref Cohen A.,
%	  {\sl Wavelet Methods in Numerical Analysis},
%	  Elsevier, Amsterdam, 2000.
%
%	  \ref Dubuc S. and J.-L. Merrien,
%	  A 4-point Hermite subdivision scheme,
%	  in {\sl Mathematical Methods in CAGD: Oslo 2000},
%	  T. Lyche and L.L. Schumaker (eds),
%	  Vanderbilt University Press, Nashville, 2001, xxx--yyy.
%
%	  \ref Laurent P.J., P. Sablonni\`ere, and L.L. Schumaker,
%	  {\sl Curve and Surface Design: Saint-Malo 1999},
%	  Vanderbilt University Press, Nashville, 2000.
%
%	  \ref Mazure M.-L., Blossoming: a geometrical approach,
%	  {\sl Constructive Approximation} {\bf 15} (1999), 33--68.
%
%	  \ref Ramshaw L., Blossoming: a connect-the-dots approach to splines,
%	  tech. report, Digital Systems Research Center, Palo Alto, (1987).
%
% Insert your references between the \def\refs braces
%
\def\refs{

} % End of \def\refs (Do not edit this line)
%
%
%%%%%%%%%%%%%%%%%%%%%%%%%%%%%%%%%%%%%%%%%%%%%%%%%%%%%%%%%%%%%%%%%%%%%%%%%
% 6. ADDRESS
%    Give the address of each of the authors :
%
% Eg : \def\name(Marie-Laurence Mazure}
%      \def\affiliate{Laboratoire LMC/IMAG}  
%      \def\address{51, rue des Mathematiques - 38041 Grenoble Cedex 9}    
%      \def\email{Marie-Laurence.Mazure@imag.fr}      
%      \addr
\def\adresses{
%------------------------------------------------------------------------
% First author
% Delete the first percent mark on the following 5 lines and fill in the braces 
%\def\name{}       % Insert the name of one author
%\def\affiliate{}  % Insert the name of his/her institution
%\def\address{}    % Insert the address of his/her institution
%\def\email{}      % Insert the author's E-mail address
%\addr             % DON'T FORGET to delete the first percent mark 
%
%------------------------------------------------------------------------
% Second author
% if any, delete the percent mark on the following 5 lines and fill in the braces 
%\def\name{}
%\def\affiliate{}
%\def\address{}
%\def\email{}
%\addr
%
%------------------------------------------------------------------------
% Third author
% if any, delete the percent mark on the following 5 lines and fill in the braces 
%\def\name{}
%\def\affiliate{}
%\def\address{}
%\def\email{}
%\addr
%
% And so on...
} % End of \def\adresses (Do not edit this line)
%
%
%%%%%%%%%%%%%%%%%%%%%%%%%%%%%%%%%%%%%%%%%%%%%%%%%%%%%%%%%%%%%%%%%%%%%%%%%
% DO NOT MODIFY ANYTHING BELOW THIS LINE.
%%%%%%%%%%%%%%%%%%%%%%%%%%%%%%%%%%%%%%%%%%%%%%%%%%%%%%%%%%%%%%%%%%%%%%%%%
% Thanks for having completed all the needed information.
% The remaining lines just print the whole document.
%%%%%%%%%%%%%%%%%%%%%%%%%%%%%%%%%%%%%%%%%%%%%%%%%%%%%%%%%%%%%%%%%%%%%%%%%
\aut
\keys
\abstext
\refs
\adresses
\bye 
%--------------------------- End of TeX file -----------------------------
%-------------------------------------------------------------------------
