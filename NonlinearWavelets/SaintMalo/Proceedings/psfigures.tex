%***************** STMALO2 MACROS ******************
% FILE NAME: psfigures.tex
% COPYRIGHT: 10/15/97, updated 6/1/2002
% BY: Larry Schumaker
% EMAIL: s@mars.cas.vanderbilt.edu
% WEB: http://www.math.vanderbilt.edu/~schumake/

\count100= 1
\count101= 7
\input malo2
\input lls00
%\draft

\title{Inserting Postscript Figures Using}
\titwo{the \conf Macros}
\author{Larry L. Schumaker}
\def\shorttitle{Inserting Postscript Figures}
\def\shortauthor{L. L. Schumaker}

\def\back{{\tt\char'134}}
\def\lc{$\{$}
\def\rc{$\}$}
\def\tb#1{{\tt\back #1}}
\def\Tex{{\TeX~}}

\abstract{The purpose of this note (which corresponds to the \Tex
file {\tt psfigures.tex}) is to explain how to use the macro file
{\tt epsf.tex} along with the macro files {\tt\macname.tex} and
 {\tt lls00.tex} to insert postscript figures into a \Tex document.
Macintosh users may want to use {\tt epsfmac.tex} instead.}

\sect{1. Introduction}
There are a number of public domain \Tex macros which can be used
to insert postscript files into a \Tex document. I have found
the macro file {\tt epsf.tex} to be particularly easy to use.
To use this macro file, you need three things:
\ms
\item{1)} a copy of {\tt epsf.tex}
\sm
\item{2)} a dvi driver which is capable of working with {\tt epsf.tex}.
One which works on unix machines is {\tt dvips}.
\sm
\item{3)} postscript files of the figures you wish to insert.  These
can be created using any convenient software.
\ms

To use the macro file {\tt epsf.tex}, you have to input it with
the command \tb{input epsf.tex} at the beginning of your \Tex
file.  This command is already included
in the {\tt lls00.tex} macro file.
You should put {\tt epsf.tex} in the same directory
as the \TeX{} file you are processing.

The macros in {\tt epsf.tex} allow for considerable control over the size
and position of the figures to be inserted.  It is not my aim here to
explain {\tt epsf} in detail. In order to simplify matters,
I assume that you are also using the macro files
{\tt \macname.tex} and  {\tt lls00.tex} which were
created for the \conf proceedings.
The latter  file includes three special macros which can be used along
with {\tt epsf} to insert figures.

\sect{2. Inserting a Single Figure}
The basic  macro for inserting a figure is called
\tb{psone}. It has two parameters.  The
first parameter gives the name of the file containing the figure, and
the second parameter sets the size of the figure in inches
(1 inch = 2.5 cm).

To illustrate the use of \tb{psone}, I will now insert a figure
called {\tt schumaker1.fig}.  This is a postscript file, but instead of
calling it {\tt schumaker1.ps} I have named it {\tt schumaker1.fig}.  This
helps me remember that it is a figure, and prevents me from deleting it
accidently when removing all postscript files in a directory.
Here is the \Tex code for inserting my figure:
\medskip
{\obeylines \tt
\back topinsert
\back psone\lc schumaker1.fig\rc \lc 1.5\rc
\back figcap\lc 1\rc \lc A circle\rc\lc 0\rc
\back endinsert
}
\ms

In inserting this figure, I have also made use of a macro in
{\tt \macname.tex}
called \tb{figcap} which is used to create a caption.
It has three parameters.
The first gives the figure number, the second gives the actual caption,
and the third gives the number of inches of space which you wish to insert
between the figure and the caption. When inserting postscript figures
with proper bounding boxes, this would
generally be 0. If you use \tb{figcap} to insert space for a figure
to be pasted in, this parameter would be the amount of space to
leave for the figure.

\topinsert
\psone{schumaker1.fig}{1.5}
\figcap {1} {A circle}{0}
\endinsert

To get a smaller version of the same figure, I simply change the
second parameter in \tb{psone} from 1.5 to 1:
\ms
{\obeylines \tt
\back topinsert
\back vskip .2 in
\back psone\lc schumaker1.fig\rc \lc 1\rc
\back figcap\lc 2\rc \lc A smaller circle\rc\lc{.2}\rc
\back endinsert
}
\medskip

\midinsert
\vskip .2 in
\psone{schumaker1.fig}{1}
\figcap {2} {A smaller circle}{.2}
\endinsert

The size of the first figure is set to 1.5 inch while the
size of the second one is set to 1 inch.  Unfortunately, these
numbers may not always give the actual size of the figure --- it depends on
how the postscript file was created. Thus,
it may take some experimentation to find the setting for this parameter
which will give a figure of the desired size.

In Fig.~2, I have  used \tb{vskip .2 in} to
insert an extra .2 in of space above the figure.
In addition, by choosing .2 as the third parameter in\break \tb{figcap},
I have inserted .2 inches of space between the figure and the caption.
It is also possible to skip by a negative number of inches to take
space out.

\sect{3. Inserting Figures Side by Side}
To insert two figures side by side, I have written a macro called
\tb{pstwo}. Its use is
completely analogous to the use of \tb{psone}:
\medskip
{\obeylines \tt
\back midinsert
\back pstwo\lc schumaker1.fig\rc \lc 1.4\rc \lc schumaker3.fig\rc \lc 2.8\rc
\back figcap\lc 3\rc \lc Two figures side by side\rc\lc 0\rc
\back endinsert
}
\medskip

\midinsert
\pstwo{schumaker1.fig}{1.4}{schumaker3.fig}{2.8}%
\figcap {3} {Two figures side by side}{0}
\endinsert

Because the sizes of these two figures in the postscript files
were not the same,
to get them to match in vertical size, I had to set the
parameter controlling the size of  {\tt schumaker1.fig}
to 1.4, and the parameter controlling the size of {\tt schumaker3.fig}
to 2.8. In practice some experimentation is usually required to size
the figures correctly.

The right-hand figure in Fig.~3 includes some writing which is really
too small to read.  If you want to include words on your figure,
be sure they are written large enough so that they can be read
at the final size.

The same idea can be used to put three or more figures side by side.
Here is an example with three figures:
\medskip
{\obeylines \tt
\back midinsert
\back psthree\lc schumaker1.fig\rc\lc 1\rc
  \lc schumaker1.fig\rc\lc 1\rc
  \lc schumaker1.fig\rc\lc 1\rc
  \back figcap\lc 4\rc\lc Three figures side by side\rc\lc 0\rc
  \back  endinsert
}

\midinsert
\psthree{schumaker1.fig}{1}
     {schumaker1.fig}{1}
     {schumaker1.fig}{1}
\figcap{4}{Three figures side by side}{0}
\endinsert

%\eject
\sect{4. Creating the Postscript Files}
A postscript file is identified by its first line which should
begin with the two symbols {\tt \%!}.  These symbols may be followed by a
comment identifying the software which produced the file.  If these two symbols
are not present and you send the file to a laser printer, you will
probably get a line-by-line print-out of the entire file (which can
be huge). Here is the first line of the file {\tt schumaker1.fig}:
\ms
$$\hbox{\tt \%!PS-Adobe-2.0 EPSF-2.0} $$
\ms

\noindent Postscript files describing figures can be created by a variety of
software. These include
\sm
\item{1)} La\TeX
\item{2)} Pic\TeX
\item{3)} Mathematica
\item{4)} SGI ShowCase
\item{5)} MATLAB.
\sm

\noin
It is also possible to create your own postscript files directly,
see the manuals [1,2]. In any case, if you do use postscript,
make sure that you give us a complete file which prints correctly
on a standard laser printer, and if possible verify that it can
be inserted into your \TeX{} file correctly.

\sect{5. Bounding Boxes}
The macros in {\tt epsf.tex}  will work with most postscript files,
provided that the file includes
a line which describes the size of a \dword{bounding box} enclosing
the figure. Here is the line for the file {\tt schumaker1.fig}:
\medskip
$$\hbox{\tt \%\%BoundingBox: 18 679 104 765} $$
\medskip
\noindent The four numbers here are in \dword{postscript units}.
	There are
exactly 72 postscript units per inch.  The first pair of numbers describe
the location of the lower-left corner of the bounding box relative
to the \eword{origin}, which is taken to be the
lower-left corner of the page.	In this case the lower-left corner of the
bounding box is 18 units
to the right and 679 units up from the origin.
The second pair of numbers describes
where the upper-right-hand corner of the bounding box lies.  In this case
it is 104 units to the right and 765 units up from the origin.

Some drawing packages will furnish a precise bounding box, but I have seen
others which don't give any bounding box at all, and others which
give an incorrect one (often the whole page).
Our next figure illustrates what happens with a file where the bounding
box is not properly set. It
 was inserted using the code
\medskip

{\obeylines \tt
\back midinsert
\back psone \lc schumaker2.fig\rc \lc 4\rc
\back figcap \lc 5\rc \lc A circle surrounded by too much space\rc\lc 0\rc
\back endinsert
}
\medskip

\midinsert
\psone{schumaker2.fig}{4}
\figcap {5} {A circle surrounded by too much space}{0}
\endinsert

%\eject
If you compare the files {\tt schumaker.fig1} and {\tt schumaker2.fig}, you
will see that the only difference is in their second lines where the size
of\break the \dword{bounding box} is described. In particular,
the line setting the bounding box\break in  {\tt schumaker2.fig} is
$$\hbox{\tt \%\%BoundingBox: 0 0 612 792} $$
\noindent This describes a full ($8.5 \times 11$ inch) page. The
circle only occupies a small part of the page.	But now because of
the bounding box, the entire page is inserted, explaining all the white
space in Fig.~5.

If you have a file with no bounding box or an incorrectly sized
one, you can fix it with your editor.  First print the file on
a laser printer, then take a ruler and physically measure the
location of the lower-left and upper-right corners of an appropriate
bounding box.  If there is already a statement defining the
bounding box, just edit
the file to set the correct dimensions. Remember that 1 inch is
equal to 72 units.  If there is no
such statement, insert one in the file.
(I would put it as the second line in the file, after the first
line which identifies the file as a postscript file).

\eject
\sect{6. Remarks}
In case you are wondering, if you
use {\tt epsf} to insert figures and later decide to
change the magnification of your document, everything (including
the figures) is automatically rescaled.

Another set of macros which I have seen is called {\tt psfig.tex}.
They work in much the same way as {\tt epsf.tex}. With some dvi drivers,
it is possible to insert postscript figures using a {\tt \back special}
command. It would be nice if there were a standard way to create and handle
postscript files -- but unfortunately (analogous to the La\TeX, AMS\Tex
mess), this is not the case.

The postscript files for the figures included in this
document are relatively small (on the order of 5K each).  If you create
postscript files by some bitmapping process (for example using
``snapshot" to capture a screen on an SGI), you can get postscript files which
are on the order of several megabytes. It is better to avoid such
huge figure files if possible.
If you are working with such
huge files, don't be surprised if your laser printer takes 10 to 20
minutes or more to print a page.
I suggest that you take a photo of your screen in such cases.

Finally, I would like to mention that the macro file {\tt gridbox.tex}
contains macros created by Carl de\thinspace{}Boor
which are very handy for putting text into figures (and can also be
used to arrange figures side-by-side, etc.)  Instructions on how to
use these macro are contained in the file.
To see how they work, try the file {\tt gridtest.tex}.

\References

\ref {\sl Postscript Language, Reference Manual}, Addison-Wesley,
Reading, MA, 1985.

\ref {\sl Postscript Language, Tutorial and Cookbook}, Addison-Wesley,
Reading, MA, 1985.

\ref Schumaker, L. L., Instructions for using the \conf \TeX{} Macros,
see the \Tex file {\tt instruct.tex}
included in the \conf  Macro Package
available on the conference home page.

\Address{
Larry L. Schumaker
Dept. of Mathematics
Vanderbilt University
Nashville, TN 37240
{\tt s@mars.cas.vanderbilt.edu
http://www.math.vanderbilt.edu/$\sim$schumake/}
}

\bye

%%  END of psfigures.tex
