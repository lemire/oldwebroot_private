%***************** MALO2 MACROS  ******************
% FILE NAME: gridtest.tex
% COPYRIGHT: 5/1/97 -- updated 6/1/2003
% BY: Larry Schumaker +
% EMAIL: s@mars.cas.vanderbilt.edu
% WEB: http://www.math.vanderbilt.edu/~schumake/
% ******************************************************************
\magnification\magstep1
\hsize 4.66in
\vsize 7.66in
\def\tb#1{{\tt \char'134#1}}
\input gridbox
\input epsf
%\showgridfalse   %Uncomment this to remove the grid lines

\centerline{\bf A Test of Gridbox}
\bigskip
This file shows how to use the macros \tb{gridbox}, \tb{point},
\tb{cpoint}, and \tb{ppoint}, which are contained in the file {\tt
gridbox.tex}, to precisely place figures and label them.
For example, you can use \tb{centerline} and \tb{vbox}
to create a centered box:
\bigskip

\centerline{\vbox{
\gridbox6/7/{
\cpoint(3,3){\epsfxsize4truecm\epsfbox{schumaker1.fig}}
\ppoint11(3,3){$o$}
\point(1,4.1){\hbox{This is a circle}}
}}}

\bigskip
You can use \tb{midinsert} to create a left-justified box (and turn off the
grid
by using \tb{showgridfalse}):

\midinsert
\showgridfalse
\gridbox6/7/{
\ppoint11(3,3){\epsfxsize4truecm\epsfbox{schumaker1.fig}}
\cpoint(3,3){$o$}
\ppoint00(1,4.1){\hbox{This is a circle}}
}
\endinsert
In figuring out where to place things, it is very helpful to
use a viewer which shows cursor position.
{\tt ghostview} does this, but it shows position in
{\tt truept} rather than {\tt truecm}.  In this case,
you can use \tb{ptgridbox} instead of \tb{gridbox}. In using
\tb{ptgridbox} you specify locations in points -- {\tt ghostview}
translates this to true points. Thus,
in the following example we have specified
gridlines to be 20 points apart, but since
this document is at magnification  \tb{magstep1},
{\tt ghostview} will show them to be 24 points apart.
\smallskip

\centerline{\vbox{\ptgridbox40/80/{
\ppoint10(20,10){\epsfxsize40pt\epsfbox{schumaker1.fig}}
\ppoint12(20,70){\epsfxsize40pt\epsfbox{schumaker1.fig}}
\cpoint(20,40){o}
}}}
\bye

%END of gridtest.tex
