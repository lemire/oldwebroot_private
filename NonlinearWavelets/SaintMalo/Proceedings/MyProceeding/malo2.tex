%******************** ST MALO2 MACROS  ******************
% FILE NAME: malo2.tex
% COPYRIGHT: 10/21/97, updated 06/1/02
% BY: Larry Schumaker
% EMAIL: s@mars.cas.vanderbilt.edu
% WEB: http://www.math.vanderbilt.edu/~schumake/

\overfullrule 0pt
\def\conf{{StMalo2\ }}
\def\macname{{malo2}}

%******************** PAGE SIZE ************************************
\magnification\magstep1
%\hsize 4.66in
%\vsize 7.66in
\hsize 4.43in
\vsize 7.28in

%********************* OFFSET FOR PRINTER ***************************
% You may have to adjust the offset for your printer. A typical
% command looks like
% \voffset .5 in
% \hoffset 1 in

%  ************************ FONTS ***********************************
% If you don't have some of these fonts, replace them by size 10
% For example, if you don't have cmmi6, replace it by cmmi10

\font\cmsc=cmcsc10 %%%% small caps font
\font\bfab = cmbx9
\font\fa=cmr17
\font\fb=cmr12
\font\fc=cmr10
\font\fab=cmr9
\font\sfab=cmr7
\font\fs=cmr6
\font\fd=cmr5
\font\slab=cmsl9
\font\tab= cmmi9
\font\sab= cmmi7
\font\ssab= cmmi6
\font\abst= cmsy9
\font\sabs= cmsy7
\font\ssabs= cmsy6
\font\itab= cmti9
\font\dfont=cmss10
\font\efont=cmti10
\font\Bbb=msbm10


\def\figfont{
    \textfont0 = \fab
    \scriptfont0 = \sfab
    \scriptscriptfont0 = \fs
    \textfont1 = \tab
    \scriptfont1 = \sab
    \scriptscriptfont1 = \ssab
    \textfont2 = \abst
    \scriptfont2 = \sabs
    \scriptscriptfont2 = \ssabs
    \let \sl = \slab
    \let \bf = \bfab
    \let \it = \itab
    \baselineskip 9pt
    \fab}

%********************* OPEN FACE LETTERS **********************

\def\CC{{\rm C\kern-.18cm\vrule width.6pt height 6pt depth-.2pt
\kern.18cm}}

\def\NN{{\mathop{{\rm I}\kern-.2em{\rm N}}\nolimits}}

\def\PP{{\mathop{{\rm I}\kern-.2em{\rm P}}\nolimits}}

\def\RR{{\mathop{{\rm I}\kern-.2em{\rm R}}\nolimits}}

\def\RRt{{\fa I}\kern-.2em{\fa R}}
  %% This version of \RR can be used in the title of the paper

%\def\TT{{\mathop{{\rm I}\kern-.42em{\rm T}}\nolimits}}
\def\TT{\hbox{{\Bbb{T}}}}

\def\ZZ{{\mathop{{\rm Z}\kern-.28em{\rm Z}}\nolimits}}

\def\Cbb{\hbox{{\Bbb{C}}}}
\def\Dbb{\hbox{{\Bbb{D}}}}
\def\Kbb{\hbox{{\Bbb{K}}}}
\def\Nbb{\hbox{{\Bbb{N}}}}
\def\Pbb{\hbox{{\Bbb{P}}}}
\def\Qbb{\hbox{{\Bbb{Q}}}}
\def\Rbb{\hbox{{\Bbb{R}}}}
\def\Tbb{\hbox{{\Bbb{T}}}}
\def\Zbb{\hbox{{\Bbb{Z}}}}


%******************* BOLDFACE ************************************

% Boldface math
\def\bfm#1{{\dimen0=.01em\dimen1=.009em\makebold{$#1$}}}

% Boldface subscript
\def\bfs#1{{\dimen0=.008em\dimen1=.0073em
  \makebold{$\scriptstyle#1$}}}

% Boldface subsubscript
\def\bfss#1{{\dimen0=.006em\dimen1=.0057em
   \makebold{$\scriptscriptstyle#1$}}}

\def\makebold#1{\mathord{\setbox0=\hbox{#1}%
       \copy0\kern-\wd0%
       \raise\dimen1\copy0\kern-\wd0%
       {\advance\dimen1 by \dimen1\raise\dimen1\copy0}\kern-\wd0%
       \kern\dimen0\raise\dimen1\copy0\kern-\wd0%
       {\advance\dimen1 by \dimen1\raise\dimen1\copy0}\kern-\wd0%
       \kern\dimen0\raise\dimen1\copy0\kern-\wd0%
       {\advance\dimen1 by \dimen1\raise\dimen1\copy0}\kern-\wd0%
       \kern\dimen0\raise\dimen1\copy0\kern-\wd0%
       \kern\dimen0\box0}}

%********************** BOXIT ******************************
%  This macro was written by Yvon Lafranche

\newbox\maboite

\def\boxit#1#2{\setbox\maboite=\hbox{\kern#1{#2}\kern#1}%
    \dimen1=\ht\maboite \advance\dimen1 by #1 \dimen2=\dp\maboite
\advance\dimen2 by #1%
    \setbox\maboite=\hbox{\vrule height\dimen1
depth\dimen2\box\maboite\vrule}%
    \setbox\maboite=\vbox{\hrule\box\maboite\hrule}%
    \advance\dimen1 by .4truept \ht\maboite=\dimen1%
    \advance\dimen2 by .4truept \dp\maboite=\dimen2 \box\maboite\relax}

%*********************** USEFUL MACROS ************************

\def\bs{\bigskip}

\def\bull{\item{$\bullet$}}

\def\cro{\cr \noalign{\vskip 4pt}}
  %%  Use in place of \cr for extra space between cases

\def\croo{\cr \noalign{\vskip 6pt}}
  %%  Use in place of \cr for extra space between cases

\def \dword#1{{\dfont #1}}

\def \eword#1{{\efont #1}}

\def\eg{{\it e.g.}}%

\def\frac#1#2{{#1 \over #2}}
   % #1 = numerator
   % #2 = denominator

\def\ie{{\it i.e.},}    % includes the comma

\def \lrm {{\rm (}}
\def \rrm {{\rm )}}

\def\ms{\medskip}

\def\noin{\noindent}

\def\ritem#1{\item{\rm #1}}

\def\ritemitem#1{\itemitem{\rm #1}}

\def\sm{\smallskip}

\def\st{\/:\ }
 %%  Use \st for "such that" in definitions of sets

\def\thc{{$\thinspace$:}}

%********************** FIGURE MACROS ************************

\def\Fig#1{Fig.\thinspace#1}
\def\Figs#1{Figs.\thinspace#1}

\def\figcap#1#2#3{\null\vskip 1pt \vskip  #3in {\vfill}
  {\figfont \centerline{{\bf Fig.~#1.} #2.}}}
   %% #1 = figure number
   %% #2 = figure caption
   %% #3 = number of inches of space (1 inch = 2.5 cm)

\def\figcaplong#1#2#3{\null\vskip 1pt \vskip #3in \figfont    
 \centerline{\hbox{\noindent{\bf Fig.~#1.}
 \vtop{\hsize 4in \noin #2.\smallskip}}}
}
   %% #1 = figure number
   %% #2 = figure caption
   %% #3 = number of inches of space (1 inch = 2.5 cm)

%********************** TABLE MACROS ************************

\def\tabcap#1#2#3{\null\vskip 1pt \vbox to #3in {\vfill}
  {\figfont \centerline{{\bf Tab.~#1.} #2.}}}
   %% #1 = figure number
   %% #2 = figure caption
   %% #3 = number of inches of space (1 inch = 2.5 cm)

\def\tabcaplong#1#2#3{\null\vskip 1pt \vbox to #3in {\vfill}
  {\figfont {\narrower \noindent{\bf Tab.~#1.} #2.\smallskip}}}
   %% #1 = figure number
   %% #2 = figure caption
   %% #3 = number of inches of space (1 inch = 2.5 cm)

%********************* PROOF and end OF PROOF ******************

\def\nopf{\medskip}
  %% If there is no proof, use this to insert space

\def\pf{\noindent{\bf Proof: }}

\def\eop{\makeblanksquare6{.4}\ms}
  %% Mark the end of a proof

\def\makeblanksquare#1#2{
\dimen0=#1pt\advance\dimen0 by -#2pt
      \vrule height#1pt width#2pt depth0pt\kern-#2pt
      \vrule height#1pt width#1pt depth-\dimen0 \kern-#1pt
      \vrule height#2pt width#1pt depth0pt \kern-#2pt
      \vrule height#1pt width#2pt depth0pt
}

\def\meop{~~\makeblanksquare6{.4}}
 %%  Use this inside of displayed equations instead of \eop

\def\dis{\noindent{\bf Discussion: }}

%********************* FIRST PAGE  ******************************

\def\abstract#1{\bigskip\bigskip\medskip%
    {\narrower \baselineskip 9pt \fab \noindent {\bf Abstract.~~}%
    \textfont0 = \fab
    \scriptfont0 = \sfab
    \scriptscriptfont0 = \fs
    \textfont1 = \tab
    \scriptfont1 = \sab
    \scriptscriptfont1 = \ssab
    \textfont2 = \abst
    \scriptfont2 = \sabs
    \scriptscriptfont2 = \ssabs
    \let \it = \itab
    \let \sl = \slab
    #1\bigskip}\medskip}
\def\author#1{\bigskip\bigskip\centerline{\fb #1}}
\def\autwo#1{\smallskip \centerline {\fb #1}}
\def\copyright{\hbox{{\fb o}\kern-.61em \raise .46ex \hbox{\fd c}}}
\def\footnoterule{\kern -3pt \hrule width 0truein \kern 2.6pt}
\def\leftheadline{\ifnum\pageno=\count100 \hfill%
  \else\rm\folio\hfil\it\shortauthor\fi}
\def\rightheadline{\ifnum\pageno=\count100 \hfill%
  \else\it\shorttitle\hfil\rm\folio\fi}
\def\title#1{\centerline {\fa #1}}
\def\titwo#1{\medskip \centerline {\fa #1}}
\def\tithree#1{\medskip \centerline {\fa #1}}
\def\titexp#1#2{\hbox{{\fa #1} \kern-.25em \raise .90ex \hbox{\fb #2}}\/}
  %% Use to create superscripts in the title
\def\titsub#1#2{\hbox{{\fa #1} \kern-.25em \lower .60ex \hbox{\fb #2}}\/}
  %% Use to create subscripts in the title

\nopagenumbers
\headline{\ifodd\pageno\rightheadline \else\leftheadline\fi}
\footline{\hfil}
\null\vskip 18pt
\centerline{}
\pageno=\count100
\count102=\count100
\advance\count102 by -1
\advance\count102 by \count101

%********************* FOOTNOTE on FIRST PAGE  ********************

\insert\footins{\fs
\medskip
\baselineskip 8pt
\leftline{Curve and Surface Fitting: Saint-Malo 2002
  \hfill {\fab \the\pageno}}
\leftline{XXX (eds.),
    pp. \the\pageno--\the\count102.}
\leftline{Copyright \copyright\ 2002 by Nashboro Press,
Nashville, TN.}
\leftline{ISBN XXX.}
\leftline{All rights of reproduction in any form reserved.}
\smallskip
\par\allowbreak}

%********************* SECTIONS  ******************************

\def\sect#1{\goodbreak\bigskip\smallskip\centerline{\bf\S #1}\medskip
    \noindent\ignorespaces}
  %% DEFINE SECTION HEADING

\def\sectlong#1#2{\goodbreak\bigskip\smallskip\leftline{\bf\S #1}
   {\bf #2}\medskip \noindent\ignorespaces}
   % LONG SECTION
   % #1 = First part of title
   % #2 = Second part of title

\def\subsect#1{\goodbreak\bigskip\leftline{\bf#1}\medskip
  \noindent\ignorespaces}
  %% DEFINE SUBSECTION HEADING

\def\subsectlong#1#2{\goodbreak\bigskip\leftline{\bf#1}
  {\bf #2} \medskip \noindent\ignorespaces}
  %% DEFINE SUBSECTION HEADING

\def\fsubsect#1{\goodbreak\leftline{\bf#1}\medskip
   \noindent\ignorespaces}
  %% USE in place of \subsect if it immediately follows \sect

%********************** end MATTER ****************************

\def\Acknowledgments{\goodbreak\bigskip\noindent{\bf
   Acknowledgments.\ }}

\def\Address{\nonfrenchspacing\goodbreak\bigskip\obeylines}

\def\Remark#1.{\goodbreak\medskip\noin {\bf Remark#1.}}
\def\Example#1.{\goodbreak\medskip\noin {\bf Example#1.}}

%************************ REFERENCES *********************

\def\ref{\smallskip\global\advance\refnum by 1 \item{\the\refnum.}}
\newcount\refnum \refnum = 0

\def\References{\goodbreak\bigskip\centerline{\bf References}\bigskip
   \frenchspacing}

%***************************** JOURNALS **************************

\def\AAM{\AiAM}
\def\ApA{Appl.\ Anal.}
\def\ACHA{Appl.\ Comput.\ Harmonic Anal.}
\def\ACJ{Australian Comput.\ J.}
\def\ACMTMS{ACM Trans.\ Math.\ Software}
\def\ACMTG{ACM Trans.\ on Graphics}
\def\AC{Adv.\ Comput.}
\def\AJM{Amer.\ J. Math.}
\def\AMASH{Acta Math.\ Acad.\ Sci.\ Hungar.}
\def\AMC{Appl.\ Math.\ Comp.}
\def\AMM{Appl.\ Math.\ Mech.}
\def\AMMo{Amer.\ Math.\ Monthly}
\def\AMS{Acta Math.\ Sinica}
\def\AM{Aequationes Math.}
\def\AN{Acta Numerica}
\def\AoNM{Annals of Numerical Mathematics}
\def\ANM{Appl.\ Numer.\ Math.}
\def\ANTA{Anal.\ Num\'er.\ Th\'eor.\ Approx.}
\def\ATA{Approx.\ Theory Appl.}
\def\AiM{Advances in Math.}
\def\AiAM{Advances in Appl.\ Math.}
\def\AiCM{Advances in Comp.\ Math.}
\def\AfM{Arkiv for Matematik}
\def\BAMS{Bull.\ Amer.\ Math.\ Soc.}
\def\BIT{BIT}
\def\C{Computing}
\def\CACM{Commun.\ ACM}
\def\CAD{Computer-Aided Design}
\def\CAGD{Comput.\ Aided Geom.\ Design}
\def\CA{Constr.\ Approx.}
\def\CGA{Comp.\ Graphics and Applics.}
\def\CGIP{Comp.\ Graphics and Image Proc.}
\def\CJM{Canad.\ J. Math.}
\def\CJ{Computer J.}
\def\CM{Contemp.\ Math.}
\def\CMA{Comput.\ Math.\ Appl.}
\def\CMB{Canad.\ Math.\ Bull.}
\def\CMP{Comm.\ Math.\ Phys.}
\def\CMSCP{Canad.\ Math.\ Soc.\ Conf.\ Proceedings}
\def\CRABS{C. R. Acad.\ Bulgare Sci.}
\def\CRASP{C. R. Acad.\ Sci.\ Paris}
\def\CVGIP{Computer Vision, Graphics, and Image Processing}
\def\DMJ{Duke Math.\ J.}
\def\EJA{East J. Approx.}
\def\IBMJRD{IBM J. Res.\ Dev.}
\def\IBMJR{IBM J. Res.}
\def\ICGA{IEEE Comp.\ Graph.\ Appl.}
\def\IJM{Illinois J. Math.}
\def\IJNME{Intern.\ J. Numer.\ Methods Eng.}
\def\IMAJNA{IMA J. Numer.\ Anal.}
\def\IPL{Inform.\ Process.\ Lett.}
\def\ITBE{IEEE Trans.\ Biomedical Engrg.}
\def\ITC{IEEE Trans.\ Computers}
\def\ITMI{IEEE Trans.\ Medical Imaging}
\def\ITPAMI{IEEE Trans.\ Pattern Anal.\ and Machine Intelligence}
\def\ITPASSP{IEEE Trans.\ Acoustic, Speech, and Signal Processing}
\def\ITSE{IEEE Trans.\ Software Engrg.}
\def\IUMJ{Indiana Univ.\ Math.\ J.}
\def\IsJM{Israel J. Math.}
\def\JACM{J. Assoc.\ Comput.\ Mach.}
\def\JAMP{J. Appl.\ Math.\ Phys.}
\def\JAMSA{J. Austral.\ Math.\ Soc.\ Ser.\ A}
\def\JAMSB{J. Austral.\ Math.\ Soc.\ Ser.\ B}
\def\JAM{J. Analyse Math.}
\def\JATA{\ATA}
\def\JAT{J. Approx.\ Theory}
\def\JCAM{J. Comput.\ Appl.\ Math.}
\def\JCP{J. Comput.\ Phys.}
\def\JFA{J. Funct.\ Anal.}
\def\JFAA{J. Fourier Anal.\ Appl.}
\def\JIMA{J. Inst.\ Math.\ Applics.}
\def\JLMS{J. London Math.\ Soc.}
\def\JMAA{J. Math.\ Anal.\ Appl.}
\def\JMM{J. Math.\ Mech.}
\def\JMP{J. Math.\ Phys.}
\def\JNMCA{J. Numer.\ Methods Comput.\ Appl.}
\def\JSC{J. Sci.\ Comput.}
\def\JSIAM{J. SIAM}
\def\LAA{Linear Algebra Appl.}
\def\LMJ{London Math.\ J.}
\def\MA{Math.\ Ann.}
\def\MAA{Math.\ Anal.\ Appl.}
\def\MCAD{Math.\ CAD}
\def\MC{Math.\ Comp.}
\def\MM{Manuscripta Math.}
\def\MN{Math.\ Notes}
\def\MNa{Math.\ Nachrichten}
\def\MUSSRS{Math.\ USSR-Sb.}
\def\MZ{Math.\ Z.}
\def\MaZ{Mat.\ Zametki}
\def\NA{Numer.\ Algorithms}
\def\NFAO{Numer.\  Func.\ Anal.\ Optim.}
\def\NM{Numer.\ Math.}
\def\PAMS{Proc.\ Amer.\ Math.\ Soc.}
\def\PEMS{Proc.\ Edinburgh Math.\ Soc.}
\def\PLMS{Proc.\ London Math.\ Soc.}
\def\PJM{Pacific J. Math.}
\def\PNAS{Proc.\ Nat.\ Acad.\ Sci.}
\def\PR{Pattern Recognition}
\def\PRSEA{Proc.\ Roy.\ Soc.\ Edinburgh Sect.\ A}
\def\RAIROAN{Rev.\ Fran\c caise Automat.\ Informat.\
     Rech.\ Op\'er., Anal.\ Numer.}
\def\RMA{Rev.\ Math.\ Apl.}
\def\RAN{\RAIROAN}
\def\RM{Resultate Math.}
\def\RMJM{Rocky Mountain J. Math.}
\def\SAM{Studies in Appl.\ Math.}
\def\SJAM{SIAM J. Appl.\ Math.}
\def\SJADM{SIAM J. Alg.\ Disc.\ Methods}
\def\SJDM{SIAM J. Discrete Math.}
\def\SJC{SIAM J. Comput.}
\def\SJMA{SIAM J. Math.\ Anal.}
\def\SJNA{SIAM J. Numer.\ Anal.}
\def\SJSC{SIAM J. Scient.\ Computing}
\def\SJSSC{SIAM J. Sci.\ Statist.\ Comput.}
\def\SM{Studia Math.}
\def\SMD{Soviet Math.\ Dokl.}
\def\SR{SIAM Review}
\def\SSMH{Stud.\ Sci.\ Math.\ Hung.}
\def\TAMS{Trans.\ Amer.\ Math.\ Soc.}
\def\TOG{\ACMTG}
\def\ZAMM{Z. Angew.\ Math.\ Mech.}
\def\ZAMP{Z. Angew.\ Math.\ Phys.}
\def\ZMAM{ZMAM}

%***************************** PROCEEDINGS **************************

\def\biri{%
{\sl Mathematical Methods in Computer Aided Geometric Design II},
T. Lyche and L.~L. Schumaker (eds.), Academic Press, New York, 1992}

\def\chamonixI{%
{\sl Curves and Surfaces},
P.-J. Laurent, A. Le M\'ehaut\'e, and L. L. Schumaker (eds.),
Academic Press, New York, 1991}

\def\chamonixIIa{%
{\sl Curves and Surfaces in Geometric Design},
P.-J. Laurent, A. Le M\'ehaut\'e, and L. L. Schumaker (eds.),
A. K. Peters, Wellesley MA, 1994}

\def\chamonixIIb{%
{\sl Wavelets, Images, and Surface Fitting},
P.-J. Laurent, A. Le M\'ehaut\'e, and L. L. Schumaker (eds.),
A. K. Peters, Wellesley MA, 1994}

\def\chamonixIIIa{%
{\sl Curves and Surfaces with Applications in CAGD},
A. Le M\'ehaut\'e, C. Rabut, and L. L. Schumaker (eds.),
Vanderbilt University Press, Nashville, 1997}

\def\chamonixIIIb{%
{\sl Surface Fitting and Multiresolution Methods},
A. Le M\'ehaut\'e, C. Rabut, and L. L. Schumaker (eds.),
Vanderbilt University Press, Nashville, 1997}

\def\lillehammer{%
{\sl Mathematical Methods for Curves and Surfaces II},
M. D{\ae}\-hlen, T. Lyche, and L. L. Schumaker (eds.),
Vanderbilt University Press, Nashville, 1998}

\def\oslo{%
{\sl Mathematical Methods in Computer Aided Geometric Design},
T. Lyche and L. L. Schumaker (eds.), Academic Press, New York, 1989}

\def\osloII{%
{\sl Mathematical Methods in CAGD: Oslo 2000},
T. Lyche and L. L. Schumaker (eds.), Vanderbilt University
Press, Nashville, 2001}

\def\stmalod{%
{\sl Curve and Surface Design: Saint-Malo 1999},
Pierre-Jean Laurent, Paul Sablonni\`ere,
and Larry L. Schumaker (eds.),
Vanderbilt University Press, Nashville, 2000}

\def\stmalof{%
{\sl Curve and Surface Fitting: Saint-Malo 1999},
Albert Cohen, Chris\-tophe Rabut,
and Larry L. Schumaker (eds.),
Vanderbilt University Press, Nashville, 2000}

\def\stmalotwo{%
{\sl Curve and Surface Fitting: Saint-Malo 2002},
XXX (eds.),
XXX Press, Nashville, 2003}

\def\texasI{%
{\sl Approximation Theory},
G. G. Lorentz {\it et al.} (eds.),
Academic Press, New York, 1973}

\def\texasII{%
{\sl Approximation Theory, II},
G. G. Lorentz, C. K. Chui,  and L. L. Schumaker (eds.),
Academic Press, New York, 1976}

\def\texasIII{%
{\sl Approximation Theory III},
E. W. Cheney (ed.),
Academic Press, New York, 1980}

\def\texasIV{%
{\sl Approximation Theory IV},
C. Chui, L. Schumaker, and J. Ward (eds.),
Academic Press, New York, 1983}

\def\texasV{%
{\sl Approximation Theory V},
C. Chui,  L. Schumaker, and J. Ward (eds.),
Academic Press, New York, 1986}

\def\texasVI{%
{\sl Approximation Theory VI},
C. Chui,  L. Schumaker, and J. Ward (eds.),
Academic Press, New York, 1989}

\def\texasVII{%
{\sl Approximation Theory VII},
E. W. Cheney, C. Chui,  and  L. Schumaker (eds.),
Academic Press, New York, 1992}

\def\texasVIIIa{%
{\sl Approximation Theory VIII, Vol.\ 1: Approximation and
Interpolation},
Charles K. Chui  and  Larry L. Schumaker (eds.),
World Scientific Publishing Co., Inc.{}, Singapore, 1995}

\def\texasVIIIw{%
{\sl Approximation Theory VIII, Vol.\ 2: Wavelets},
Charles K. Chui  and  Larry L. Schumaker (eds.),
World Scientific Publishing Co., Inc.{}, Singapore, 1995}

\def\texasIXt{%
{\sl Approximation Theory IX, Vol.~1: Theoretical Aspects},
Charles K. Chui  and  Larry L. Schumaker (eds.),
Vanderbilt University Press,  Nashville, 1998}

\def\texasIXc{%
{\sl Approximation Theory IX, Vol.~2: Computational Aspects},
Charles K. Chui  and  Larry L. Schumaker (eds.),
Vanderbilt University Press,  Nashville, 1998}

\def\texasXa{%
{\sl Approximation Theory X:
Abstract and Classical Analysis},
C. K. Chui,  L. L. Schumaker, and J. St\"ockler (eds.),
Vanderbilt University Press, Nash\-ville, 2002}

\def\texasXs{%
{\sl Approximation Theory X: Splines, Wavelets, and Applications},
C. K. Chui,  L. L. Schumaker, and J. St\"ockler (eds.),
Vanderbilt University Press, Nash\-ville, 2002}

\def\ulvik{%
{\sl Mathematical Methods for Curves and Surfaces},
M. D{\ae}\-hlen, T. Lyche, and L. L. Schumaker (eds.),
Vanderbilt University Press, Nashville, 1995}

\def\trends{%
{\sl Trends in Approximation Theory},
K.~Kopotun, T.~Lyche, and M.~Neamtu (eds.),
Vanderbilt University Press, Nashville, 2001}

\def\taormina{%
{\sl Wavelets: Theory, Algorithms, and Applications},
C. Chui, L. Montefusco, and L. Puccio (eds.),
Academic Press, New York, 1994}

\def\karlin{%
{\sl Studies in Spline  Functions and Approximation Theory},
S. Karlin, C. Micchelli, A. Pinkus, and I. Schoenberg (eds.),
Academic Press, New York, 1976}

\def\npaper{\nonfrenchspacing\vfill\eject}

\def\bpage{
\def\leftheadline{\hfill}%
\def\rightheadline{\hfill}%
\null\centerline{}
\supereject}


%%%%% END of at01.tex
