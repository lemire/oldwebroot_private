%*********************  STMALO2 MACROS *******************
% FILE NAME: autonumber.tex
% COPYRIGHT: 10/15/97, updated 6/1/2002
% BY: Larry Schumaker
% EMAIL: s@mars.cas.vanderbilt.edu
% WEB: http://www.math.vanderbilt.edu/~schumake/

\def\updated{6/1/02}
\count100= 1
\count101= 5
\input malo2
\input lls00
%\draft %%% remove initial percent if you want the labels to appear
        %%% in small print in the margins

%\let\proclaimnum = \eqnum %%% remove initial percent if you want to number
                           %%% all items with the same numbering system.

\def\back{{\tt \char'134}}
\def\tb#1{{\tt \back #1}}
\def\Tex{\TeX\ }
\def\lb{\char'173}
\def\rb{\char'175}

\title{Automatic Numbering Using the}
\titwo{\conf Macros}
\author{Larry L. Schumaker}
\def\shorttitle{Automatic Numbering in \Tex}
\def\shortauthor{Larry L. Schumaker}

\abstract
{The purpose of this note (which corresponds to the \Tex file
{\tt autonumber.tex}) is to discuss how to use
the automatic numbering macros contained in
{\tt lls00.tex}.}

\sect{1. Introduction}
In typesetting mathematics using \TeX,
it is very useful to dynamically label
\sm
\ritem{1)} equations,
\sm
\ritem{2)} proclamations (Lemmas, Theorems, Corollaries, etc.),
\sm
\ritem{3)} figures,
\sm
\ritem{4)} tables,
\sm
\ritem{5)} references.
\ms

The idea is to assign (easy-to-remember)
labels to each of these items, and to
refer to the items by label. We then rely on \Tex to
automatically associate numbers with the labels. If
labels are added or deleted, the numbers will be
automatically adjusted.

Many people who are familiar with automatic numbering in La\Tex
think that it cannot be done in plain \TeX.
Nothing could be further from the truth, and in fact
it can be achieved with a small set of macros.
Here we discuss how to do it using
macros created by Carl de\thinspace{}Boor and included in the macro files
{\tt lls00.tex}. Thus, to use these macros, you will have to input
both {\tt \macname.tex} and {\tt lls00.tex} into your \Tex file.

\sect{2. Equations}
In this section, we discuss how to label equations.
The idea is very simple.
Every time you introduce an equation into a manuscript
which you want to number,  use \tb{en}.
If you write \tb{en\lb\rb}, the equation will be given the
next number in sequence. If you write \tb{en\lb{}name\rb},
the equation will be given the next number, and also assigned
the label ``name". This label can be used
in referring to the equation. For example, if you type
$$ \hbox{\tt \$\$ x = y.  \tb{en}\lb{}xy\rb \$\$}$$
you get
$$ x= y.  \en{xy} $$
Here \tb{en} is the macro (``en" stands for equation name),
and {\tt xy} is the label.

The situation is slightly different
when you are using \tb{eqalignno}.  Then you have to use
\tb{enn\lb{}name\rb} as shown in the following example:
$$ \eqalignno{
   a &= b, & \enn{ab} \cr
   u &= v. & \enn{uv} \cr} $$
To get these equations, you have to type
$$ \eqalign{
  \hbox{\$\$ \tb{eqalignno\lb}}& \cr
  &\hbox{a \&= b, \& \tb{enn\lb ab\rb} \tb{cr}} \cr
  &\hbox{u \&= v. \& \tb{enn\lb{}uv\rb} \tb{cr\rb\$\$}}\cr} $$

To refer to a labeled equation,
use the macro \tb{er}, where ``er" stands for equation reference.
Thus, to refer to the above equation labeled xy, we write
\tb{er\lb xy\rb}. For example, if the \Tex file includes
the sentence
$$ \hbox{\tt The equations {\back}er\lb{}xy\rb{}, {\back}er\lb{}ab\rb{},
and {\back}er\lb{}uv\rb{} are important,} $$
we get
$$\hbox{The equations \er{xy}, \er{ab}, and \er{uv} are important.} $$

Here is how it works.
Suppose the name of your \Tex file is {\tt jobname.tex},
and that it inputs both {\tt \macname.tex} and {\tt lls00.tex}.
Then in processing {\tt jobname.tex},
a new file called {\tt jobname.aux} is automatically created.
Information about labels is written to this file.
Now every time you \Tex the file {\tt jobname.tex},
the file {\tt jobname.aux} is read in.
Thus, whatever labels you have defined the last
time {\tt jobname.aux} was written will be used.

As your file is processed, {\tt jobname.aux}
is rewritten to reflect any changes you have
made to the labels. Every time
the macro \tb{en} is encountered, a counter called \tb{eqnum}
is advanced, and a definition is created
and inserted in {\tt jobname.aux} which associates the label
with the number.

\eject
Note that if you try to cite an equation with a label which has
not been defined, you will get a message in your log file {\tt jobname.log}
telling you which labels are not defined, and reminding you to
\Tex your document again.  Moreover, it is easy to find the
references to undefined labels
in your \Tex file {\tt jobname.tex}, since
the name of each such undefined label will
appear in the \TeX'ed material between easy-to-locate black boxes.

Note that the citation process works even if the
citation of something with a given label
appears before the label itself has been defined. This
is called \eword{forward referencing}.
Here is an example of a
forward reference \er{forward} to the following equation:
$$  y = x.  \en{forward} $$
If you use forward referencing,
the first time you \Tex your document,
you will get a pair of black boxes marking each
forward citation, and
a warning message telling you which labels are not defined.
But if you \Tex your file again, the boxes and the warning
should disappear. If they do not, then probably you are trying
to cite something that you never defined.

\sect{3. Definitions, Lemmas, Theorems, and Corollaries}
We prefer that you number
definitions, lemmas, theorems, and corollaries (so-called proclamations)
consecutively throughout the paper.
This can be done automatically
using the macro \tb{pn} (here ``pn" stands for proclaim name).
This gives each proclamation a label which can then
be referred to using \tb{pr}.

\proclaim Lemma~\pn{prelim}. This is a lemma labeled ``prelim".
\nopf

\proclaim Theorem~\pn{main}. This is the theorem labeled ``main".
\nopf
\ms

In the \Tex file {\tt autonumber.tex} for this document,
we label this lemma by
writing \tb{proclaim Lemma$\sim$}\tb{pn\lb prelim\rb}.
To refer to this lemma, we write
{\tt Lemma$\sim${\back}pr\lb prelim\rb}. Thus, if
we insert the sentence
$$ \hbox{\tt Lemma$\sim$\back{pr}\lb prelim\rb{}
   immediately implies Theorem$\sim$\back{pr}\lb main\rb.} $$
into our \Tex file, we get
$$ \hbox{Lemma~\pr{prelim} immediately implies Theorem~\pr{main}.} $$

Note that in \eword{assigning} the label for a Lemma
and in \eword{referring} to it, we
insert the symbol $\sim$ between the word
Lemma and the macro \tb{pn} or \tb{pr}, respectively.
\Tex interprets the symbol $\sim$ as a space, but does not allow
a line break at this point.
This is a useful tool when you want to keep things together.
You should routinely use it with numbered items. Thus, for example,
in referring to Fig.~1 and Tab.~2, you should type
{\tt Fig.$\sim$1} and
{\tt Tab.$\sim$2}.

If you want to number all equations and proclamations consecutively,
you can add the command
{\tt \back{}let\back{}proclaimnum = \back{}eqnum}
to your file.

\sect{4. Labelling Figures and Tables}
Use \tb{fn} to label figures, and \tb{fr}
to refer to them.
Here ``fn" stands for ``figure name", and ``fr" stands
for ``figure reference".

Similarly, use \tb{tn} to label tables,
and \tb{tr} to refer to them.
Here ``tn" stands for ``table name", and ``tr" stands
for ``table reference".

\sect{5. Labelling References}
If you want to label references, use
\tb{rn} instead of \tb{ref} to introduce each reference.
Here ``rn" stands for ``reference name".
You write \tb{rn}
followed by the label (in curly brackets)
that you want to assign to the reference. Here is an example of
an entry with an assigned label:
\ms
{\tt \obeylines
\back{}rn\lb{}S1\rb{}
Schumaker, L. L., Inserting postscript files into \back{}TeX,
see the \back{}Tex file \lb\back{}tt psfigures.tex\rb{} which is
included in the \conf Macro Package and is also
available on my home page.}
\ms

This is the first entry in the reference list below.
To cite this reference (which is
labeled {\tt S1}), simply type \tb{cite}$\{$S1$\}$.
This gives \cite{S1}. Note that the macro \tb{cite} automatically
creates brackets around the number of the cited article.
To cite several papers at one time, use the macro \tb{cit} which
does not automatically create the brackets.  For example,
you type {\tt [\back{}cit\lb{}S1\rb{},\back{}cit\lb{}S2\rb{}]} to get
[\cit{S1},\cit{S2}].

\sect{6. How to Fix a Contaminated jobname.aux File}
If your file fails to \Tex and you get an error message
related to the file {\tt jobname.aux}, where {\tt jobname.tex} is
the name of your tex file, probably the file
{\tt jobname.aux} has somehow become contaminated.
Simply delete it
from your directory, and \Tex {\tt jobname.tex} twice.

\sect{7. Draft Mode}
If you use a lot of labels, you may find it hard to remember how
you labeled various equations, figures, tables, proclamations, and
references.
You can find out by inserting the command \tb{draft} into
your \Tex file right after \tb{input lls00}.  This causes the
labels to appear in small print just following the equation numbers.
For figures, tables, proclamations, and references, the label appears
in the left margin.
To see how this works, try it on the file {\tt autonumber.tex} which
generated this document.

\Acknowledgments{My thanks to Carl de\thinspace{}Boor for
creating these numbering macros,
and for comments on a draft of this note.}

\References

\rn{S1}
Schumaker, L. L., Inserting postscript files
using the \conf macros,
see the \Tex file {\tt psfigures.tex} which is
included in the {\it \conf Macro Package} and is also
available on the conference home page.

\rn{S2}
Schumaker, L. L.,
Instructions for using the \conf \Tex Macros,
see the \Tex file {\tt instruct.tex} which is
included in the {\it \conf Macro Package} and is also
available on the conference home page.

\Address{
Larry L. Schumaker
Dept. of Mathematics
Vanderbilt University
Nashville, TN 37240
{\tt s@mars.cas.vanderbilt.edu
http://www.math.vanderbilt.edu/$\sim$schumake/}
}

\bye
%%%  END of autonumber.tex

