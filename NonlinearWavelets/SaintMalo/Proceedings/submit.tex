%**************  STMALO2 MACROS *****
% FILE NAME: submit.tex
% COPYRIGHT: 10/15/97, updated 6/1/02
% BY: Larry Schumaker
% EMAIL: s@mars.cas.vanderbilt.edu
% WEB: http://www.math.vanderbilt.edu/~schumake/

%********************** INITIALIZATION *************************
\count100= 1
\count101= 3
\input malo2.tex

%************************ TITLE ******************************

\title{Submitting Papers to the}
\titwo{\conf Proceedings}

%*********************RUNNING HEAD ***************************

\def\shorttitle{Submitting Papers to \conf}
%********************* AUTHORS ***************************

\author{Larry Schumaker}

%********************* AUTHORS RUNNING HEAD **************

\def\shortauthor{Larry L. Schumaker}

%********************** MACROS  ******************************

\def\back{{\tt \char'134}}
\def\co{{\cal O}}
\def\cP{{\cal P}}
\def\ct{{\cal T}}
\def\ctil{\tilde{\cal T}}
\def\lb{\char'173}
\def\rb{\char'175}
\def\oln{\co(n \log(n))}
\def\tb#1{{\tt \back #1}}
\def\Tex{{\TeX~}}

%************************* ABSTRACT **********************************

\abstract
{The purpose of this document (which corresponds to the file
{\tt submit.tex}) is to discuss how
to submit papers for the \conf proceedings.
As a supplement, please read
{\sl Instructions for Using the \conf \Tex Macros} which
corresponds to the file {\tt instruct.tex}.
A careful reading of these two documents should
answer most questions, but if you need further help, please contact
us.}

%******************************* BODY *********************************

\sect{1. Introduction}
The proceedings of the \conf meeting will be published
in two volumes similar in style to the proceedings of the
first St.~Malo conference
[2,5], see also [3,4,6--9].
Contributions will be refereed
and edited, and must be original work. Abstracts and summaries
of work published elsewhere will not be accepted.

Papers to be included in the proceedings
must be prepared in \TeX, using a common macro file
(called {\tt \macname{}.tex}) which sets
up the style of the paper.
We recommend that you read both this document and
\eword{Instructions for Using the \conf \Tex Macros}
which can be generated from {\tt instruct.tex},
even if you are an experienced \Tex  user
and have used some of our macros before.
If you don't read anything else, please see Sect.~14
of that document on things you should \eword{not} do.

\sect{2. Preparing your Manuscript}
The easiest way to prepare your paper is to
fill in the blanks in the file\break
{\tt paper.tex}. The file {\tt paper.tex} inputs a macro file
called {\tt \macname.tex} which must be present in the same directory.
These two files are included in the
{\it \conf Macro Package}.
If you want to take advantage of some advanced features of \TeX,
including automatic numbering, insertion of postscript files, or
use the spline bibliography, you will also need the macro file
{\tt lls00.tex}.

\sect{3. Page Limits}
Due to page restrictions from the publisher and to give
as many attendees as possible an opportunity to publish
their paper in the proceedings, we regret that we
must enforce the following page limits:
\ms
\noindent{{\bf Invited Speakers:} 16 pages}

\noindent{{\bf Others:} 10 pages}
\ms
\noindent These limits apply to the manuscript produced
by our \TeX{} macros. Please do not try to
squeeze more into your paper
by using smaller fonts or changing spacing.

\sect{4. Submitting your Paper}
Papers are due by {\bf October 31, 2002}.
Please send 3 copies of your paper  to 
\ms
\leftline {Marie-Laurence Mazure}
\leftline {LMC-IMAG}
\leftline {Univ. Joseph Fourier,  BP 53}
\leftline {38041 GRENOBLE cedex 9, FRANCE}
\ms

\noin You should get an email within a day or two
acknowledging receipt of your paper. 

\sect{6. Refereeing and Editing}
The organizers will oversee the refereeing process for
all submitted papers, and you 
may be asked to revise your paper. 
Once it has been accepted, I will ask you for
tex files and figure files.  At that point I will
copy-edit your paper for English usage and layout, and I
may ask you to make further corrections.

%******************************* REFERENCES *************************

\References

\ref  de\thinspace{}Boor, C. and L. L. Schumaker, Using the
de\thinspace{}Boor/Schumaker
spline bibliography, see the \Tex file {\tt splinebib.tex},
included in the \conf Macro Package
available on the conference  home page.

\ref  Cohen, A., C. Rabut and L. L. Schumaker,
{\sl Curve and Surface Fitting: Saint-Malo 99},
Vanderbilt University Press, Nashville, 2000.

\ref D{\ae}hlen, M., T. Lyche, L. L. Schumaker (eds.),
{\sl Mathematical Methods for Curves and Surfaces},
Vanderbilt University Press, Nashville, 1995.

\ref D{\ae}hlen, M., T. Lyche, L. L. Schumaker (eds.),
{\sl Mathematical Methods for Curves and Surfaces II},
Vanderbilt University Press, Nashville, 1998.

\ref Laurent, P. J., P.  Sablonni\`ere,, and L. L. Schumaker,
{\sl Curve and Surface Design: Saint-Malo 99},
Vanderbilt University Press, Nashville, 2000.

\ref
A. Le M\'ehaut\'e, C. Rabut, and L. L. Schumaker (eds.),
{\sl Curves and Surfaces in Geometric Design},
Vanderbilt University Press, Nashville TN, 1997.

\ref Le M\'ehaut\'e, A, C. Rabut, and L. L. Schumaker (eds.),
{\sl Surface Fitting and Multiresolution Methods},
Vanderbilt University Press, Nashville, 1997.

\ref Lyche, T., and L. L. Schumaker,
{\sl Mathematical Methods for Curves and Surfaces: Oslo 2000},
Vanderbilt University Press, Nashville, 2001.

\ref  Schumaker, L. L., Instructions for using the \conf \Tex Macros,
see the \Tex file {\tt instruct.tex}
included in the \conf Macro Package
available on the conference home page.

\ref  Schumaker, L. L., Inserting postscript figures
with the \conf\break macros,
see the \Tex file {\tt psfigures.tex}
included in the \conf Macro Package
available on the conference home page.

\ref  Schumaker, L. L., Automatic numbering with the
\conf macros,
see the \Tex file {\tt autonumber.tex}
included in the \conf Macro Package
available on the conference  home page.

\Address{
Larry L. Schumaker
Dept. of Mathematics
Vanderbilt University
Nashville, TN 37240
{\tt s@mars.cas.vanderbilt.edu
http://www.math.vanderbilt.edu/$\sim$schumake/}
}

\bye

%%%%%%%%  END of submit.tex
