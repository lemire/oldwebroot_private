%%%%%%%%%%%%%%%%%%%%%%%%%%%%%%%%%%%%%%%%%%%%%%%%%%%%%%%%%%%%%%%%%%%%%%%%%%%
% This is the documentation of the fig4tex.tex macro-package, version 1.6,
% to be compiled with plain TeX.
% For example, on a UNIX system, if this file is named fig4texDOC.tex:
%  1. compilation             : tex fig4texDOC.tex
%  2. creation of  postscript : dvips -o fig4texDOC.ps fig4texDOC.dvi
%  3. print it                : lp -d Your_Printer fig4texDOC.ps
% Do not modify \vsize neither \hsize, otherwise the page numbers in the
% index will be wrong.
%%%%%%%%%%%%%%%%%%%%%%%%%%%%%%%%%%%%%%%%%%%%%%%%%%%%%%%%%%%%%%%%%%%%%%%%%%%
\input fig4tex.tex
\newcount\Figno\Figno=1
\newcount\sectno\newcount\subsectno
\newtoks\CONTEXTEtok\CONTEXTEtok={\rm}
%
\def\CtrlSq#1{{\tt \char`\\#1}}
\def\Mac#1{{\tt \char`\\#1}}
\def\abstract#1{\bigskip\bigskip\medskip%
    {\narrower \baselineskip 10pt \noindent {\bf Abstract.~~}%
    #1\bigskip}\medskip}
\def\boxit#1#2{\leavevmode\hbox{\vrule\vbox{\hrule\vglue#1%
    \vtop{\hbox{\kern#1{#2}\kern#1}\vglue#1\hrule}}\vrule}}
\def\eword#1{{\it #1}}
\def\figforTeX{fig4tex.tex}
\def\fmtlist#1{\catcode`\:=12\catcode`\&=12\obeyspaces\tt\openup#1\CONTEXTEtok={\it}}
\def\fmtprog#1{\catcode`\:=12\catcode`\&=12\obeyspaces\tt\openup#1\CONTEXTEtok={\bf}}
\def\NewPSfile#1{\advance\Figno1\def#1{Fig\the\Figno.ps}\ignorespaces}
\long\def\rightitem#1#2{{\parindent0pt\def\sujet{#1}\setbox20=\hbox{\ \ \sujet}%
{\leftskip\wd20\item{\sujet}#2\par}}}
\def\vect#1{\overrightarrow{\mkern -2mu#1\mkern 2mu}}
%********************* SECTIONS  ******************************
\def\sect#1{\advance\sectno1\subsectno=0%
    \goodbreak\bigskip\smallskip\centerline{\bf\S \the\sectno. #1}%
    \medskip\noindent\ignorespaces}
  %% DEFINE SECTION HEADING
\def\subsect#1{\advance\subsectno1\goodbreak\bigskip\leftline{\bf\the\subsectno. #1}%
  \medskip\noindent\ignorespaces}
  %% DEFINE SUBSECTION HEADING
\def\fsubsect#1{\advance\subsectno1\nobreak\leftline{\bf\the\subsectno. #1}%
   \medskip\noindent\ignorespaces}
  %% USE in place of \subsect if it immediately follows \sect
%%%%%%%%%%%%%%%%%%%%%%%%%%%%%%%%%%%%%%%%%%%%%%%%%%%%%%%%%%%%%%%%%%%%%%%%%%%
\def\sectC#1{\advance\sectno1\subsectno=0\medskip{\the\sectno. #1}\smallskip}
\def\subsectC#1{\advance\subsectno1\qquad{\the\subsectno. #1}\smallskip}
\let\fsubsectC=\subsectC
%%%%%%%%%%%%%%%%%%%%%%%%%%%%%%%%%%%%%%%%%%%%%%%%%%%%%%%%%%%%%%%%%%%%%%%%%%%
\def\IntroG{Introduction}
\def\FLook{A first look}
\def\DefPts{Definition of points}
 \def\UnScale{Unit and scale}
 \def\BasMac{The basic macro}
 \def\TextArg{Text argument}
 \def\SymbCst{Symbolic constants}
 \def\MacElGeo{Macros for vector and point generation}
\def\DrawFig{Drawing the figure}
 \def\Princip{Principle}
 \def\LineWS{Line width and style}
 \def\BasDrMac{Basic drawing macros}
 \def\BasicEx{Example}
\def\WTxtFig{Writing text on the figure}
\def\WTxtExFig{Writing text on an existing figure}
\def\DrawMac{Drawing macros}
 \def\PrelRem{Preliminary remarks}
 \def\Arc{Arc}
 \def\EllipseEx{Example}
 \def\Curve{Curve}
 \def\Arrow{Arrow}
 \def\Color{Using color}
 \def\Area{Filled area}
 \def\Triangle{Triangle related macros}
 \def\Grid{Grid}
\def\Helpers{Helpers}
\def\TriDim{Three-dimensional macros}
 \def\IntroTD{Introduction}
 \def\Projec{Projection}
 \def\MacSpec{Macro specifications}
 \def\ExTD{Examples}
\def\UsingMac{Using this macro-package}
\def\LstUMac{List of user macros}
\def\Index{Index}
%%%%%%%%%%%%%%%%%%%%%%%%%%%%%%%%%%%%%%%%%%%%%%%%%%%%%%%%%%%%%%%%%%%%%%%%%%%
\null\vfill
{\nopagenumbers\font\bigf=cmbx10 scaled\magstep1\bigf\baselineskip=13pt
\centerline{Documentation of the
 {\font\bigtt=cmtt10 scaled\magstep1\bigtt \figforTeX} macro-package}
\bigskip\bigskip
\sectC{\IntroG}
\sectC{\FLook}
\sectC{\DefPts}
 \fsubsectC{\UnScale}
 \subsectC{\BasMac}
 \subsectC{\TextArg}
 \subsectC{\SymbCst}
 \subsectC{\MacElGeo}
\sectC{\DrawFig}
 \fsubsectC{\Princip}
 \subsectC{\LineWS}
 \subsectC{\BasDrMac}
 \subsectC{\BasicEx}
\sectC{\WTxtFig}
\sectC{\WTxtExFig}
\sectC{\DrawMac}
 \fsubsectC{\PrelRem}
 \subsectC{\Arc}
 \subsectC{\EllipseEx}
 \subsectC{\Curve}
 \subsectC{\Arrow}
 \subsectC{\Color}
 \subsectC{\Area}
 \subsectC{\Triangle}
 \subsectC{\Grid}
\sectC{\Helpers}
\sectC{\TriDim}
 \fsubsectC{\IntroTD}
 \subsectC{\Projec}
 \subsectC{\MacSpec}
 \subsectC{\ExTD}
\sectC{\UsingMac}
\sectC{\LstUMac}
\sectC{\Index}
\vfill\eject
\null\vfill\eject
}
\pageno=1
%%%%%%%%%%%%%%%%%%%%%%%%%%%%%%%%%%%%%%%%%%%%%%%%%%%%%%%%%%%%%%%%%%%%%%%%%%%
%\null\vskip2mm
\centerline{\bf Documentation of the {\tt \figforTeX} macro-package}
\centerline{(version 1.6)}
\bigskip
\abstract{This document describes a set of \TeX\ macros designed to \eword{create}
a figure (as a \eword{short} postscript file) at compilation time of the document,
and  to \eword{write text} on it in a straightforward way, using the \TeX\ fonts,
under total control of the user.
These macros can also be used to write a legend on an existing figure, given
as an encapsulated postscript file created by any other software.
The macros have been designed from the user's point of view.
They can be helpful to design a figurative drawing as well as to build an
accurate geometric construction based on points in 2D or in 3D so that we can
even think of this macro-package as a kind of geometric modeller.}
%%%%%%%%%%%%%%%%%%%%%%%%%%%%%%%%%%%%%%%%%%%%%%%%%%%%%%%%%%%%%%%%%%%%%%%%%%%
\bigskip
\sect{\IntroG}
Inserting a figure given as a postscript file in a \TeX\ document has now
become very common. There exist several sets of macros allowing to do this
rather easily. With some other packages, it is also possible to create directly
a postscript file. In that case, the \TeX\ macros are more or less a translation
of postscript commands.
\smallskip
But, as far as we know, with these kinds of tools, there is no real \eword{link}
between the figure and the \TeX\ document, essentially because generally two
softwares are needed: one for the \TeX\ document, one for the figure.
In particular, a question that arises very often is: how can I write some text or
caption using \TeX's facilities (font, boxes, maths formul\ae,\dots) at a
\eword{precise} point on the figure ? An even worse case: if I have to modify my
geometry for any good reason or change the observation angles in 3D, shall I have
to make plenty of modifications to make everything fit together again ?
\par
The set of macros described below gives an answer to these questions. Yes, it is
possible to write anything anywhere on a figure. No, provided that we have followed
some elementary pieces of advice, any modification made at the top level of the
construction automatically propagates to lower levels, allowing to change nearly
anything at any time.
\medskip
The macros have been designed from a \eword{geometrical} point of view (i.e. so that
geometric constructions are allowed), in such a way that they are easy to use and give
an immediate result (only one compilation is necessary).
Moreover, since the result of the process is a \TeX\ box, every usual \TeX\ commands
can be applied to give the figure the rigth position inside the document.
\smallskip
The principle is the following. With the help of adequate macros, the user defines
some characteristic points. These points are used to build the figure
which is generated as a postscript file. Each of them can also be used as an attach
point of a text printed on the figure. Thus, an interesting
fact is to have the \eword{same} geometric description both to draw the figure and
to locate the text.
\medskip
In the general case, three steps are needed:\par
{\leftskip\parindent
\item{1)} definition of characteristic points necessary to build the geometrical
skeleton of the figure,
\item{2)} creation of the postscript file corresponding to the geometry,
\item{3)} if needed, writing text on the figure, generally using the
characteristic points, or other ones.
\par}
\smallskip
It must be emphasized that a figure already existing as an encapsulated
postscript file, i.e. a postscript file specifying a correct bounding box,
can be handled as well; step 2 has then to be skipped. By the way, a tool macro
(\Mac{figscan}) has been designed to determine accurately the bounding box: see
the {\bf \WTxtExFig} section for further information.
\smallskip
The set of macros consists of internal macros and user macros.
Internal macros deal with technical issues like arithmetic and vectorial
computation. They are not intended to be used directly from the \TeX\ document.
On the contrary, the so-called user macros are higher level macros designed in
such a way that most of the technical part of the work is hidden to the user. In
order to make them as easy to use as possible, general syntactic rules have
been settled:\par
{\leftskip\parindent
\item{a.} Except in text arguments, spaces are not significant.
\item{b.} Whenever it is possible, the arguments are separated with commas
(e.g. {\tt [3,5,7]}, {\tt (1.33, 0.57)}).
Colon ({\tt :}), semi-colon ({\tt ;}) and equal sign ({\tt =}) are also used
as separators.
\item{c.} Brakets ({\tt []}), parentheses ({\tt ()}) and slashes ({\tt //}) are
used to surround groups of arguments that are logically linked together. In most
of the cases, brakets are devoted to point numbers, parentheses are devoted to
numerical values. \TeX's braces ({\tt $\{\}$}) are obviously always available.
\item{d.} Numerical values are always given without any unit specification (there
is one exception described in the section {\bf \WTxtFig}).
\par}
\smallskip
The user macros are divided in two main groups according to the beginning of
their name: \Mac{fig} or \Mac{ps}. Macros beginning with \Mac{ps} are commands
used to create the postscript file. They are intended to be called during step 2,
they have no effect otherwise. Macros beginning with \Mac{fig} can themselves
be divided in two classes, so-called mute and non-mute macros.
Mute macros are mainly related to the definition of the geometry and can be
called at any step. Non-mute macros are the only ones that write something
on the figure. Logically, they have to be called during step 3 ; their names
begin with \Mac{figwrite}.
\smallskip
In the following sections, we describe the user macros. Then comes a list recalling
the syntax of each of them and at last an index, which can be quite helpful to find
examples where a given macro is used.
\smallskip
To catch the essential, the reader is advised to read this documentation at least
up to the section {\bf \WTxtFig}, and then jump to any other specific topic.
%%%%%%%%%%%%%%%%%%%%%%%%%%%%%%%%%%%%%%%%%%%%%%%%%%%%%%%%%%%%%%%%%%%%%%%%%%%
%
\sect{\FLook}
Here is a short example to show how it works. Consider the following file:\par
$$\vbox{\fmtprog{0pt}
\+{}\char`\\input \figforTeX\cr
\+{}\char`\\newbox\char`\\demo\cr
\+{}\% 1. Definition of characteristic points\cr
\+{}\Mac{figinit}$\{$pt$\}$\cr
\+{}\Mac{figpt} 1:(80.5, 120)\cr
\+{}\Mac{figpt} 2:(-10, -10)\cr
\+{}\Mac{figpt} 3:(130, 20)\cr
\+{}\Mac{figptbary}  5:c.g.[1,2,3 ; 1,1,1]\cr
\+{}\% 2. Creation of the postscript file\cr
\+{}\Mac{psbeginfig}$\{$Fig1.ps$\}$\cr
\+{}\Mac{psline}[1,2,3,1]\cr
\+{}\Mac{psendfig}\cr
\+{}\% 3. Writing text on the figure\cr
\+{}\Mac{figvisu}$\{$\char`\\demo$\}$$\{$Figure 1$\}$$\{$\cr
\+{}\Mac{figinsert}$\{$Fig1.ps$\}$\cr
\+{}\Mac{figwriten} 1:(4)\cr
\+{}\Mac{figwritew} 2:(2)\cr
\+{}\Mac{figwritee} 3:(2)\cr
\+{}\Mac{figsetmark}$\{$\$\char`\\bullet\$$\}$\Mac{figwrites} 5:\$G\$(4)\cr
\+{}$\}$\cr
\+{}\char`\\centerline$\{$\char`\\box\char`\\demo$\}$\cr
\+{}\char`\\bye\cr
}$$
\par\noindent
If we compile this file with \TeX, then we obtain the figure \the\Figno.
Some LA\TeX\ users should read specific instructions in the section
{\bf\UsingMac}. Now, let us comment these statements.
\def\Figone{Fig1.ps}
\topinsert
%\input \figforTeX
\newbox\demo
% 1. Definition of characteristic points
\figinit{pt}
\figpt 1:(80.5, 120)
\figpt 2:(-10, -10)
\figpt 3:(130, 20)
\figptbary  5:c.g.[1,2,3 ; 1,1,1]
% 2. Creation of the postscript file
\psbeginfig{\Figone}
\psline[1,2,3,1]
\psendfig
% 3. Writing text on the figure
\figvisu{\demo}{Figure \the\Figno}{
\figinsert{\Figone}
\figwriten 1:(4)
\figwritew 2:(2)
\figwritee 3:(2)
\figsetmark{$\bullet$}\figwrites 5:$G$(4)
}
\centerline{\box\demo}
%\bye
\endinsert
%
\smallskip
\item{0.} The first line loads the macro-package. Then we make \CtrlSq{demo}
a private box to be used later.
\smallskip
\item{1.} Definition of the three vertices of a triangle.
The system is first initialized and the unit to be used for the coordinates
is set to {\tt pt}.
Each point is associated to a number whose value is chosen
by the user. The text $A_i$ is implicitly joined to point $i$. At last, the
isobarycenter is defined as point 5, with {\tt c.g.} as joined text.
\smallskip
\item{2.} Creation of the postscript file that will contain the figure, here the three
edges of the triangle, drawn as a single closed broken line. The name of the file,
chosen by the user, usually has a {\tt .ps} or a {\tt .eps} suffix.
\smallskip
\item{3.} Writing text or comments on the figure. The macro \Mac{figvisu} builds a
box whose name (here \CtrlSq{demo}) is given as an argument, as well as the figure
caption.
The last argument consists of a set of commands that deal with the appearance of the
figure. We refer to an existing postscript file, {\tt \Figone}, the one created just
before, and ``put" it into the box.
The text associated to each point is printed at the place chosen by the user,
i.e. at 4$\,$pt towards the north for point \#1, at 2$\,$pt towards the west for point \#2,
at 2$\,$pt towards the east for point \#3, at 4$\,$pt towards the south for point \#5.
Note that the text associated to point \#5 is modified at this level. By default, the
attach point of the text is not marked. However, for the point \#5, it is necessary
and the symbol $\bullet$ ({\tt \$\char`\\bullet\$}) as been used.
\medskip\noindent
Remarks:
\item{$\bullet$} All the specifications, graphical and textual ones, are gathered into
\eword{one} \TeX\ file.
\item{$\bullet$} Textual specifications refer to attach points that have been previously
defined. It must be emphasized that the definition of a point can take place anywhere
in the process, that means even during step 2 or 3. For example, the point \#5 could
have been defined just before \Mac{figwrites 5:\$G\$(4)}.
Each attach point is referred to by its number.
\item{$\bullet$} A point can be redefined at any time, the new coordinates are then
valid until another redefinition occurs. This means that, except for the points defined
during step 1 which are usually kept unmodified until the completion of the figure,
many temporary points referring to the same number can be ``defined-and-used" as many
times as needed.
\item{$\bullet$} Once created, the postscript file is reused as is during any further
compilation of the source file. Indeed, the macros called between \Mac{psbeginfig} and
\Mac{psendfig} are executed only during the creation of a \eword{new} file. To re-create
the postscript file, one can simply delete it and compile the source file again. See the
{\bf\Princip} subsection for further information.
\item{$\bullet$} It is possible to create several postscript files from the same source
file. Each of them is given a name which is supplied first as an argument to the macro
\Mac{psbeginfig} for the creation and then to the macro \Mac{figinsert} to use it.
It must be emphasized that the \eword{same} box (here \CtrlSq{demo}) can be used for
several figures. For example, this documentation uses only two such boxes. Indeed, two
boxes must be processed into a single group in order to put two figures on the same line.
\item{$\bullet$} Use of symbolic constants is possible and can be quite helpful (see
the {\bf\SymbCst} subsection).
\item{$\bullet$} If used, the \CtrlSq{magnification} command would apply to the
whole document including the figure and its legends.
\item{$\bullet$} Except for points numbers, no control is made on the validity of the
arguments of the macros.
\item{$\bullet$} In addition to the postscript files explicitly wanted by the user and
the classical files produced by \TeX, this package creates a temporary file whose name
is \CtrlSq{jobname.anx}, which is empty at the end of the compilation. In fact, this is
true only if a postscript file is created.
%%%%%%%%%%%%%%%%%%%%%%%%%%%%%%%%%%%%%%%%%%%%%%%%%%%%%%%%%%%%%%%%%%%%%%%%%%%
\smallbreak
\sect{\DefPts}
\fsubsect{\UnScale}
First of all, the user has to choose the unit to be used for the figure among those
defined in The \TeX book:
\smallskip
{\leftskip2cm\obeylines
 {\tt pt } \TeX\ point,
 {\tt pc } pica (1$\,$pc = 12$\,$pt),
 {\tt in } inch (1$\,$in = 72.27$\,$pt),
 {\tt bp } big point, postscript unit (72$\,$bp = 1$\,$in),
 {\tt cm } centimeter (2.54$\,$cm = 1$\,$in),
 {\tt mm } millimeter (10$\,$mm = 1$\,$cm),
 {\tt dd } didot point (1157$\,$dd = 1238$\,$pt),
 {\tt cc } cicero (1$\,$cc = 12$\,$dd),
 {\tt sp } scaled point (65536$\,$sp = 1$\,$pt).
\par}
\smallskip
The unit is given as an argument to the macro \Mac{figinit} whose purpose is
double: 1) set the unit and the scale factor, 2) initialize everything to the default
state. If \Mac{figinit} is not called, the default unit is {\tt pt}. Because of 2),
it must be called if several figures are defined in the same document.
\smallskip
Moreover, it is also possible to put a scale factor before the unit. It may be useful
for example when the user wants to give every coordinates, say in {\tt cm}, but wants
the figure to appear twice bigger on the paper. In this case, he has to set a scale
factor of 2 with {\tt cm} as unit by saying {\tt \Mac{figinit}$\{$2cm$\}$}.
The default scale factor is 1, so for example {\tt \Mac{figinit}$\{$in$\}$} is
equivalent to {\tt \Mac{figinit}$\{$2.54cm$\}$}.
\smallskip
Notice that although 1$\,$pt is nearly equal to 1$\,$bp, it is not a good idea to use one
unit instead of the other because the difference is most of the time visible on the figure.
%%%%%%%%%%%%%%%%%%%%%%%%%%%%%%%%%%%%%%%
\subsect{\BasMac}
The following action is the definition of points. The basic macro is \Mac{figpt}
whose prototype is\smallskip
\centerline{\tt \Mac{figpt} NewPt :Text(X,Y)}
\smallskip
The first argument {\tt NewPt} is a positive integer chosen by the user. The theoretical
limit to this value is $2^{31}-1$ and is imposed by \TeX. This is not a problem since,
in practice, huge numbers are useless. Precisely, because of memory limitations, it is
not wise to use big numbers. On the contrary, the user is \eword{advised} to choose small
numbers first, but must feel free to define point \#400 or \#1500, which could be seen
as a wise limit. Notice that 3-digit numbers are especially useful to define points in 3D.
\smallskip
The second argument {\tt Text} is called ``joined text" to this point (see below).
The last arguments are the coordinates {\tt (X,Y)} of the point with respect to the
user Cartesian axes. We recall that the unit (see above) must be omitted.
%%%%%%%%%%%%%%%%%%%%%%%%%%%%%%%%%%%%%%%
\subsect{\TextArg}
To each point the user defines, a text is implicitly joined, namely $A_i$
for point $i$. The macro \Mac{figsetptname} sets this default text by saying
{\tt\Mac{figsetptname}$\{$\$A\_$\{$\#1$\}$\$$\}$}, where {\tt \#1} stands for the
point number, and it can be used to modify this default. For instance, the
command {\tt\Mac{figsetptname}$\{$\$X\^{}$\{$(\#1)$\}$\$$\}$}
changes it to $X^{(i)}$. To remove the automatic association of text, just say
{\tt\Mac{figsetptname}$\{$$\}$}. Since the effect of this macro is dynamic, from a
logical point of view, this macro has to be called during step 3. This allows for
example to write the names of some points, then call {\tt\Mac{figsetptname}$\{$$\}$}
which now hides the implicit name of every point.
\smallskip
If the text argument is not empty, it overrides the implicit text. More
generally, the last text associated with the point will be printed. This is often
done by the non-mute macros (whose name begin with \Mac{figwrite}).
\smallskip
A text argument can be any valid \TeX\ material, that is to say in practice:
\item{-} nothing. Implicit text is then used.
\item{-} free text where spaces are significant. To print one delimiter, like
{\tt :} or {\tt (}, just enclose the whole text into braces ({\tt $\{\}$}).
\item{-} a box. Inside a \CtrlSq{vbox}, use \CtrlSq{hfill}\CtrlSq{break} instead of
\CtrlSq{par} and think of \CtrlSq{parindent0pt} if needed.
\smallbreak
By default, the position of the point is not displayed. However, it is sometimes needed
and the user can choose anything he wants among \TeX 's possibilities to highlight the
point location.
For this purpose, marks like $\bullet$, $\circ$, ., $\times$, + are most commonly
chosen and can be set by the macro {\tt\Mac{figsetmark}$\{$Mark$\}$}. To get back to
the default state, just say {\tt\Mac{figsetmark}$\{$$\}$}.
\smallskip
In fact, the following marks are not quite vertically centered:
$\ast$ (\CtrlSq{ast}), $\bullet$ (\CtrlSq{bullet}), $\circ$~(\CtrlSq{circ}) and
$\diamond$~(\CtrlSq{diamond}). So corrected versions have been created, to be used
instead if a very accurate result is needed. Their names are \CtrlSq{Ast}, \CtrlSq{Bullet},
\CtrlSq{Circ}, \CtrlSq{Diamond}. We can observe the difference on this figure:
\bigskip
\NewPSfile{\MyPSfile}
% 1. Definition of characteristic points
\figinit{cm}
\figpt 0:(0,0)\figpt 10:(14,0)\figpt 1:(1,0)\figpt 2:(2.7,0)
\figvectC 3(3.5,0)
% 2. Creation of the postscript file
\psbeginfig{\MyPSfile}
\psline[0,10]
\psendfig
% 3. Writing text on the figure
\def\dist{0.2}
\figvisu{\demo}{}{
\figinsert{\MyPSfile}
\figsetmark{$\ast$}\figwrites 1:\CtrlSq{ast}(\dist)
\figsetmark{$\Ast$}\figwrites 2:\CtrlSq{Ast}(\dist)
\figptstra 1=1,2/1,3/
\figsetmark{$\bullet$}\figwrites 1:\CtrlSq{bullet}(\dist)
\figsetmark{$\Bullet$}\figwrites 2:\CtrlSq{Bullet}(\dist)
\figptstra 1=1,2/1,3/
\figsetmark{$\circ$}\figwrites 1:\CtrlSq{circ}(\dist)
\figsetmark{$\Circ$}\figwrites 2:\CtrlSq{Circ}(\dist)
\figptstra 1=1,2/1,3/
\figsetmark{$\diamond$}\figwrites 1:\CtrlSq{diamond}(\dist)
\figsetmark{$\Diamond$}\figwrites 2:\CtrlSq{Diamond}(\dist)
}
\centerline{\box\demo}
\medskip
\rightitem{Remark:}{By the way, for the same reason, do not use the macro \CtrlSq{cdot}:
simply say {\tt\Mac{figsetmark}$\{$.$\}$}.}
\medskip
At last, it is possible to print the coordinates of the point on the figure in the
unit chosen for the \Mac{figinit} call. To do that, just call the macro
{\tt\Mac{figcoord}$\{$Ndec$\}$} in the definition of the text argument.
In the previous example, one could have written during step 1:\medskip
\centerline{\fmtprog{0pt}%
\Mac{figpt} 3:\$A\_3\char`\\scriptstyle\Mac{figcoord}$\{$2$\}$\$(130, 20)}
\noindent
or during step 3:\par
\centerline{\fmtprog{0pt}%
\Mac{figwritee} 3:\$A\_3\char`\\scriptstyle\Mac{figcoord}$\{$2$\}$\$(2)}
\medskip\noindent
The argument {\tt Ndec} of \Mac{figcoord} is a positive integer setting the number
of decimals to be printed. Note that \TeX\ provides at most 5 decimals.
Missing decimals, for integer values for example, are printed as $0$.
By default, the printed values are rounded up to the integer ({\tt Ndec=0}),
or the first, second or third decimal, according to the value of the argument {\tt Ndec}.
Note that if {\tt Ndec=4} or {\tt Ndec=5}, the values are not rounded: we obtain
the internal raw value.
To suppress this rounding, one can say {\tt\Mac{figsetroundcoord}$\{$no$\}$},
and then {\tt\Mac{figsetroundcoord}$\{$yes$\}$} to enable it again for subsequent
printed values.
\smallskip
\rightitem{Remark:}{The two macros \Mac{figsetptname} and \Mac{figcoord} can be
combined to print together the point name and its coordinates by saying, for example,
{\tt\Mac{figsetptname}$\{$\$A\_$\{$\#1$\}$\$ \Mac{figcoord}$\{$1$\}$$\}$}.
This will automatically apply for any point, except for those bearing a specific
joined text, as just sayed.}
%%%%%%%%%%%%%%%%%%%%%%%%%%%%%%%%%%%%%%%
\subsect{\SymbCst}
As already mentionned in the remarks following the first example, a symbolic constant
can be used instead of any numerical value. This can be quite useful for example when
a dimension has to be passed as an argument to several macros. This is shown in the
following lines :
$$\vbox{\fmtprog{0pt}
\+{}\char`\\def\char`\\Absone$\{$12.3456$\}$\cr
\+{}\Mac{figpt} 10:(\char`\\Absone, 120)\cr
\+{}\Mac{figpt} 11:(\char`\\Absone, -10)\cr
}$$
The points \#10 and \#11 have the same abscissa which is define just before as a \TeX\
macro.
\smallskip
Moreover, the macro\par
\centerline{\tt \Mac{figgetdist}\CtrlSq{Value}[Pt1,Pt2]}
\smallskip\noindent
computes the euclidian distance between the two points {\tt Pt1} and {\tt Pt2},
and ``put it" in the macro \CtrlSq{Value} whose name is chosen by the user.
The macro can then be used as a symbolic numerical constant. A typical example is
the computation of the radius of a circle (see the macro \Mac{pscirc}). This feature
allows us to build geometrical constructions depending only on the coordinates of
characteristic points.
\smallskip
Let $C$, $P_1$ and $P_2$ stand for {\tt Center}, {\tt Pt1} and {\tt Pt2}.
The companion macro of the previous one is\smallskip
\centerline{\tt \Mac{figgetangle}\CtrlSq{Value}[Center,Pt1,Pt2]}
\smallskip\noindent\nobreak
which computes the value (in degrees) of the oriented angle $(\vect{CP_1},\,\vect{CP_2})$.
%\smallbreak
\goodbreak
\rightitem{Remark:}{
The name of the postscript file to be created can also be put in a macro definition.
The reference to the file is then done via this macro, \CtrlSq{MyPSfile}
in the following example:}
$$\vbox{\fmtprog{0pt}
\+{}\% 2. Creation of the postscript file\cr
\+{}\char`\\def\char`\\MyPSfile$\{$Figure.ps$\}$\cr
\+{}\Mac{psbeginfig}$\{$\char`\\MyPSfile$\}$\cr
\+{}\dots\cr
\+{}\% 3. Writing text on the figure\cr
\+{}\Mac{figvisu}$\{$\char`\\demo$\}$$\{$Caption$\}$$\{$\cr
\+{}\Mac{figinsert}$\{$\char`\\MyPSfile$\}$\cr
\+{}\dots\cr
}$$
%%%%%%%%%%%%%%%%%%%%%%%%%%%%%%%%%%%%%%%
\subsect{\MacElGeo}
The name given to each of the macros described below is governed by the following rule:
\par
\item{.}  a macro whose name begin with \Mac{figvect} produces a vector,
\item{.}  a macro whose name begin with \Mac{figpt} produces one point,
\item{.}  a macro whose name begin with \Mac{figpts} produces at least one point.
\bigskip\noindent
{\bf Vector creation}\smallskip
The definition of a vector can be done with one of the two macros\smallskip
\centerline{\vbox{\hbox{\tt\Mac{figvectC} NewVect (X,Y)}
\hbox{\tt\Mac{figvectP} NewVect [Pt1,Pt2]}}}
\smallskip
The first macro defines the vector by its components {\tt (X,Y)} and the second one
by the origin {\tt Pt1} and the end point {\tt Pt2}. The word {\tt NewVect} stands
for a number chosen by the user in the \eword{same} set as the one used for the
definition of a point. That means that vectors and points are internally stored in the
same way. The user must remember if say \#10 is a point or a vector ; no check is
performed if a point is used instead of a vector or vice-versa.
\smallskip
The macro\par
\leftline{\hskip5cm\tt\Mac{figvectN} NewVect [Pt1,Pt2]}
\par\noindent
defines a vector normal to the line ({\tt Pt1}, {\tt Pt2}). Precisely, the
vector is the image of $\vect{P_1P_2}$ by a $\pi / 2$ rotation, where $P_1$
stands for {\tt Pt1} and $P_2$ stands for {\tt Pt2}.
\smallskip
The macro\par
\leftline{\hskip5cm\tt\Mac{figvectNV} NewVect [Vector]}
\par\noindent
defines a vector normal to the vector {\tt Vector}. Precisely, {\tt NewVect}
is the image of {\tt Vector} by a $\pi / 2$ rotation.
\smallskip
The macro\par
\leftline{\hskip5cm\tt\Mac{figvectU} NewVect [Vector]}
\par\noindent
defines a unitary vector in the direction given by {\tt Vector}, taking into
account the unit and the scale factor chosen by the user. Thus, after
{\tt\Mac{figinit}$\{$2cm$\}$}, the length of {\tt NewVect} is 2cm.
This macro is useful when scaling is important.
\smallskip
The macro \Mac{figvectDBezier} computes the derivative at a point lying on a B\'ezier
curve. It is described at the end of this section.
\bigskip\noindent
{\bf Point transformation}\smallskip
We describe here macros that transform one point into another.
The simplest one corresponds to the identity transformation.
It may happen that it is convenient to copy points,
for example to save the result of a geometric transformation to be used later.
The macro\smallskip
\centerline{\tt \Mac{figptcopy} NewPt :Text/Pt/}
\smallskip\noindent
has been built for this purpose. It makes {\tt NewPt} be a copy of {\tt Pt},
changing the joined text to {\tt Text}.
\medskip
Now here is a set of macros corresponding to classical geometrical transformations,
namely homothety, rotation, symmetry and translation.
Roughly speaking, these macros use the following syntactic scheme:\smallskip
\centerline{\tt \Mac{macroname} Result = Data /{\it transformation definition}/}
\smallbreak
%--------
\NewPSfile{\MyPSfile}
% 1. Definition of characteristic points
\figinit{cm}
\figvectC 5(3,0)
\figpt 0:$O$(0,0)\figpt 1:$M$(2,2)\figptrot2:$M'$=1/0,45/        % rotation
\figptstra10=0,1/1,5/\figpthom12:$M'$=11/10,1.5/                 % homothety
\figpttra21:$M$=1/2,5/
\figpttraC23:$P_1$=21/0.5,1.7/\figpttraC24:$P_2$=21/1.5,-1.7/
\figptsym22:$M'$=21/23,24/                                       % symmetry
\figpttra31:$M$=0/4,5/\figvectC 30(2,2)\figpttra32:$M'$=31/1,30/ % translation
\figpttraC34:=31/0,2/\figpttra35:=34/1,30/                       %   vector u
% 2. Creation of the postscript file
\psbeginfig{\MyPSfile}
\psline[1,0,2]                 % rotation
\psline[10,11,12]              % homothety
\psline[21,22]\psline[23,24]   % symmetry
\psline[31,32]\psarrow[34,35]  % translation
\psendfig
% 3. Writing text on the figure
\figvisu{\demo}{Figure \the\Figno: Rotation ($\pi/4$), homothety (ratio 1.5),
symmetry / $[P_1, P_2]$ and translation of vector $\vec{u}$}{
\figinsert{\MyPSfile}\figsetmark{$\times$}
\figwrites 0:(0.15)\figwritee 1:(0.1)\figwritew 2:(0.1)            % rotation
\figwrites 10:$O$(0.15)\figwritee 11:$M$(0.1)\figwritew 12:(0.1)   % homothety
\figwritew 21:(0.1)\figwritee 22:(0.1)                             % symmetry
\figwriten 23:(0.1)\figwrites 24:(0.1)                             % symmetry
\figwritew 31:(0.1)\figwritee 32:(0.1)                             % translation
\figsetmark{}\figptbary36:$\vec{u}$[34,35;1,1]\figwritenw 36:(0.1) % translation
}
%\topinsert
\centerline{\box\demo}
\vskip-10pt
$$\vbox{\fmtprog{0pt}
\+{}\% 1. Definition of characteristic points\cr
\+{}\Mac{figinit}$\{$cm$\}$\cr
\+{}\Mac{figvectC} 5(3,0)\cr
\+{}\Mac{figpt} 0:\$O\$(0,0)\Mac{figpt} 1:\$M\$(2,2)\Mac{figptrot}2:\$M'\$=1/0,45/        \% rotation\cr
\+{}\Mac{figptstra}10=0,1/1,5/\Mac{figpthom}12:\$M'\$=11/10,1.5/                 \% homothety\cr
\+{}\Mac{figpttra}21:\$M\$=1/2,5/\cr
\+{}\Mac{figpttraC}23:\$P\_1\$=21/0.5,1.7/\Mac{figpttraC}24:\$P\_2\$=21/1.5,-1.7/\cr
\+{}\Mac{figptsym}22:\$M'\$=21/23,24/                                       \% symmetry\cr
\+{}\Mac{figpttra}31:\$M\$=0/4,5/\Mac{figvectC} 30(2,2)\Mac{figpttra}32:\$M'\$=31/1,30/ \% translation\cr
\+{}\Mac{figpttraC}34:=31/0,2/\Mac{figpttra}35:=34/1,30/                       \%   vector u\cr
\+{}\% 2. Creation of the postscript file\cr
\+{}\Mac{psbeginfig}$\{$\char`\\MyPSfile$\}$\cr
\+{}\Mac{psline}[1,0,2]                 \% rotation\cr
\+{}\Mac{psline}[10,11,12]              \% homothety\cr
\+{}\Mac{psline}[21,22]\Mac{psline}[23,24]   \% symmetry\cr
\+{}\Mac{psline}[31,32]\Mac{psarrow}[34,35]  \% translation\cr
\+{}\Mac{psendfig}\cr
\+{}\% 3. Writing text on the figure\cr
\+{}\Mac{figvisu}$\{$\char`\\demo$\}$$\{$Figure \char`\\the\char`\\Figno: Rotation (\$\char`\\pi/4\$), homothety (ratio 1.5),\cr
\+{}symmetry / \$[P\_1, P\_2]\$ and translation of vector \$\char`\\vec$\{$u$\}$\$$\}$$\{$\cr
\+{}\Mac{figinsert}$\{$\char`\\MyPSfile$\}$\Mac{figsetmark}$\{$\$\char`\\times\$$\}$\cr
\+{}\Mac{figwrites} 0:(0.15)\Mac{figwritee} 1:(0.1)\Mac{figwritew} 2:(0.1)            \% rotation\cr
\+{}\Mac{figwrites} 10:\$O\$(0.15)\Mac{figwritee} 11:\$M\$(0.1)\Mac{figwritew} 12:(0.1)   \% homothety\cr
\+{}\Mac{figwritew} 21:(0.1)\Mac{figwritee} 22:(0.1)                             \% symmetry\cr
\+{}\Mac{figwriten} 23:(0.1)\Mac{figwrites} 24:(0.1)                             \% symmetry\cr
\+{}\Mac{figwritew} 31:(0.1)\Mac{figwritee} 32:(0.1)                             \% translation\cr
\+{}\Mac{figsetmark}$\{$$\}$\Mac{figptbary}36:\$\char`\\vec$\{$u$\}$\$[34,35;1,1]\Mac{figwritenw} 36:(0.1) \% translation\cr
\+{}$\}$\cr
\+{}\char`\\centerline$\{$\char`\\box\char`\\demo$\}$\cr
}$$
%\endinsert
%--------
\par\noindent
Their prototypes along with their definition are given below.
They compute the image called {\tt NewPt} of a given point called {\tt Pt}.
A text can be joined to each {\tt NewPt}.
\smallskip
\leftline{\tt \Mac{figpthom} NewPt :Text= Pt /Center, Ratio/}
refers to the homothety of center {\tt Center} and of ratio {\tt Ratio},\smallskip
\leftline{\tt \Mac{figptrot} NewPt :Text= Pt /Center, Angle/}
refers to the rotation defined by the center {\tt Center} and the angle
{\tt Angle} given in degrees,\smallskip
\leftline{\tt \Mac{figptsym} NewPt :Text= Pt /LinePt1, LinePt2/}
refers to the orthogonal symmetry with respect to the line defined by the points
{\tt LinePt1} and {\tt LinePt2},\smallskip
\leftline{\tt \Mac{figpttra} NewPt :Text= Pt /Lambda, Vector/}
refers to the translation of vector {\tt Lambda * Vector}, where {\tt Lambda} is any
real value.
\medskip
A ``light" version of the translation macro exists. It is especially useful to build
a path joining several points whose relative distances are known. The components
{\tt (X,Y)} of the vector are directly given as arguments. Its prototype is:\par\nobreak
\centerline{\tt \Mac{figpttraC} NewPt :Text= Pt /X,Y/}
\medskip
The figure \the\Figno\ shows the image $M'$ of a point $M$ computed by these macros.
It is followed by the text of the program that produces it.
Previously, \CtrlSq{MyPSfile} has been defined as the name of the postscript file
and the box \CtrlSq{demo} has been made private. Also, the figures are automatically
numbered by the \CtrlSq{newcount} register \CtrlSq{Figno}. This applies to all the
examples given further.
\bigskip
To make some geometrical constructions easier, a ``set" version of each of these
macros have been written. They compute the image of a set of $N$ data points called
{\tt Pt1}, {\tt Pt2}, \dots, {\tt PtN} which can be given in any order.
The numbers associated with the result points are successive. Only the first number
{\tt NewPt1}, chosen by the user, is given as argument. So, if {\tt NewPt1}
is equal to $k$, then the points created have numbers $k, k+1,\dots, k+N-1$.
Text eventually previously associated with the result points is lost and moreover no
text can be joined to the result points.
\medskip
Due to the fact that the result points have successive numbers, the choice of the
first number must be done carefully. Let $I_r$ (resp. $I_d$) the set of the result
points numbers (resp. the set of the data points numbers). Let $J = I_r \cap I_d$.
\item{-} If $I_r = I_d$, or $J$ is empty, or {\tt NewPt1} does not belong to $J$
(which is generally the case in practice), then there is no problem.
\item{-} Otherwise, the result given by the macro \eword{may be wrong}, at least
partially: this is because, in this case, {\tt NewPt1} belongs to $J$, and the
data points are taken into account sequentially, beginning from the first element
in the list.
However, we can take advantage of this fact to compute, in a quite simple way,
 a sequence of points $P_{n+1}=f(P_n)$, $n=1,\dots N$ by just saying\par
\centerline{\tt \Mac{macroname} P2 = P1,P2,... PN /{\it transformation definition}/}
\smallskip\noindent
Since these macros are extensions of the previous ones \Mac{figpt{\it xxx}}$\;$ and can
produce more than one result, their names are \Mac{figpts{\it xxx}}.  The arguments
defining the transformation are the same as for the corresponding ``unary" version.
Here are their prototypes:
\medskip
\centerline{\vbox{
\hbox{\tt \Mac{figptsrot} NewPt1 = Pt1, Pt2, ..., PtN /Center, Angle/}
\hbox{\tt \Mac{figptshom} NewPt1 = Pt1, Pt2, ..., PtN /Center, Ratio/}
\hbox{\tt \Mac{figptssym} NewPt1 = Pt1, Pt2, ..., PtN /LinePt1, LinePt2/}
\hbox{\tt \Mac{figptstra} NewPt1 = Pt1, Pt2, ..., PtN /Lambda, Vector/}
}}
\medskip
The last one is the macro\par
\centerline{\tt \Mac{figptorthoprojline} NewPt :Text= Pt /LinePt1, LinePt2/}
\noindent that computes the orthogonal projection {\tt NewPt} of {\tt Pt} onto
the line defined by the points {\tt LinePt1} and {\tt LinePt2}.
Its corresponding ``set" version is then:\smallskip
\centerline{\tt
\Mac{figptsorthoprojline} NewPt1 = Pt1, Pt2, ..., PtN /LinePt1, LinePt2/}
\bigskip\noindent
{\bf Point computation}\smallskip
Once points and vectors are defined, we would like to make some elementary geometrical
computation.\par
First of all, it is very convenient to define a point as the barycenter of other ones.
The macro \Mac{figptbary} allows us to do that. Its prototype is\smallskip
\centerline{\tt \Mac{figptbary} NewPt :Text[Pt1,... ,PtN ; Coef1,... ,CoefN]}
\smallskip\noindent
The resulting point is the barycenter of a set of $N$ points whose numbers 
are given as a comma separated list, {\tt Pt1,... ,PtN}. Each point bears
an \eword{integer} coefficient. The coefficients are given in the corresponding
order, also as a comma separated list of $N$ elements, {\tt Coef1,... ,CoefN}.
The user has to choose the number {\tt NewPt} and can associate a text with it.
\smallskip
Another version of this macro, \Mac{figptbaryR}, exists. In this case, the coefficients
assigned to each point are \eword{real}. Both versions are useful, depending on the
context: \Mac{figptbaryR} is slightly less accurate and \Mac{figptbary}
is more convenient in loops.
\medbreak
Then, we often need to compute the intersection of two lines.
This can be done with the macro\smallskip
\centerline{\tt \Mac{figptinterlines} NewPt :Text[LinePt1,Vector1; LinePt2,Vector2]}
\smallskip\noindent
It computes the intersection {\tt NewPt} of the line defined by the point {\tt LinePt1}
and the vector {\tt Vector1}, and the line defined by the point {\tt LinePt2} and the
vector {\tt Vector2}. As usual now, the user has to choose the number of the result
point and can associate a text with it.
\medskip
In the same idea, the macro\par
\centerline{\tt \Mac{figptsintercirc} NewPt1 [Center1,Radius1 ; Center2,Radius2]}
\smallskip\noindent
computes the intersection of two circles. More precisely, we consider the two
circles defined by their center and radius, here respectively
({\tt Center1}, {\tt Radius1}) and ({\tt Center2}, {\tt Radius2}). Only one result
point number is to be given as argument ; if {\tt NewPt1} is equal to $k$, then the
second point {\tt NewPt2} is equal to $k+1$. They must obviously be different from
{\tt Center1} and {\tt Center2}. The points {\tt NewPt1} and {\tt NewPt2} are ordered
so that the angle ({\tt NewPt1}, {\tt Center1}, {\tt NewPt2}) is positive.
If the two circles do not intersect, then on return {\tt NewPt1} is set to {\tt Center1}
and {\tt NewPt2} is set to {\tt Center2}.\par\noindent
\rightitem{Remark:}{This macro can be used to compute the intersection of a line and a
circle by considering the symmetric circle of the given one with respect to the line.}
\medskip
As we will see in following sections, it is possible to draw circles and more
generally ellipses. So we need a macro to create points lying on an ellipse.
The macro\smallskip
\centerline{\tt \Mac{figptell} NewPt :Text: Center;XRad,YRad (Angle,Inclination)}
\smallskip\noindent
creates the point {\tt NewPt}, with an associated {\tt Text}, lying on the ellipse
defined by its {\tt Center}, its radius {\tt XRad} in the X direction, its radius
{\tt YRad} in the Y direction (in local axes) and its {\tt Inclination}, which is the
angle between the local axes of the ellipse and the absolute axes corresponding to the
paper sheet. The position of the point is set by the parametrization {\tt Angle}
measured with respect to the local axes.
The coordinates of the point are then classically computed as
$(X_{Rad} \cos\alpha, Y_{Rad} \sin\alpha)$ where $\alpha$ stands for {\tt Angle}.
The two angles, {\tt Angle} and {\tt Inclination}, are to be given in degrees.
\smallskip
Another version of this macro, \Mac{figptellP}, exists. The difference comes from the
definition of the ellipse with three points. Its prototype is\smallskip
\centerline{\tt \Mac{figptellP} NewPt :Text: Center,PtAxis1,PtAxis2 (Angle)}
\smallskip\noindent
Let $C, A_1, A_2$ stand for {\tt Center}, {\tt PtAxis1} and {\tt PtAxis2}.
The ellipse is defined by its center $C$ and the end points of its two axes $A_1$
and $A_2$. Thus, the local axes are defined by the orthogonal basis
$(\vect{CA_1},\,\vect{CA_2})$.
The macro creates the point {\tt NewPt}, with an associated {\tt Text}, lying on the
ellipse at the position set by the parametrization {\tt Angle} given in degrees,
starting from the half-line $(C,A_1)$ and measured counterclockwise around the vector
$\vect{CA_1}\times\,\vect{CA_2}$.
\medskip
Since we handle circles more often than ellipses, the macro\smallskip
\centerline{\tt \Mac{figptcirc} NewPt :Text: Center;Radius (Angle)}
\smallskip\noindent
has been written to obtain the result in a more straightforward way. This calling
sequence is equivalent to
\centerline{\tt \Mac{figptell} NewPt :Text: Center;Radius,Radius (Angle,0)}
\bigskip\noindent
{\bf Macros related to B\'ezier curves}\smallskip
Now, here are some more specific macros related to a cubic B\'ezier arc. Although they
do not seem quite necessary at first glance, they are used internally especially for
3D computations. Consequently, they have been made public.\par
Let us briefly recall that a cubic B\'ezier arc is a parametric curve
$B$ whose parameter value $t$ is taken in $[0,1]$ in order to benefit from some 
mathematical properties. Each point $B(t)$ can then be defined as the barycenter of
four points, called \eword{control points}, the corresponding weigths being the
Bernstein cubic polynomials evaluated at $t$.
\smallskip
For each of the three following macros, we consider a cubic B\'ezier curve $B$ defined
by the four control points {\tt Pt1}, {\tt Pt2}, {\tt Pt3} and {\tt Pt4}.
\smallbreak
The first macro\smallskip
\centerline{\tt \Mac{figptBezier} NewPt :Text: t [Pt1,Pt2,Pt3,Pt4]}
\smallskip\noindent
computes the point $B(t)$, namely {\tt NewPt}, lying on $B$ for the
parameter value {\tt t}. We recall that, as a consequence of the definition, for
{\tt t = 0}, {\tt NewPt = Pt1}, and for {\tt t = 1}, {\tt NewPt = Pt4}.
\smallskip
The next step is to compute the derivatives $B'(t)$ and $B"(t)$. The macro\smallskip
\centerline{\tt \Mac{figvectDBezier} NewVect : n, t [Pt1,Pt2,Pt3,Pt4]}
\smallskip\noindent
computes the vector {\tt NewVect} corresponding to the derivative of order {\tt n} of
$B$ for the parameter value {\tt t}. The order {\tt n} must be equal to 1 or 2.
\smallskip
We are thus ready to introduce the macro\smallskip
\centerline{\tt \Mac{figptcurvcenter} NewPt :Text: t [Pt1,Pt2,Pt3,Pt4]}
\smallskip\noindent
which computes the curvature center {\tt NewPt} at the point lying on $B$ 
for the parameter value {\tt t}.
\smallskip
The last macro\smallskip
\centerline{\tt \Mac{figptscontrol} NewPt1 [Pt1,Pt2,Pt3,Pt4]}
\smallskip\noindent
computes the two control points {\tt NewPt1} and {\tt NewPt2} so that the cubic
B\'ezier curve defined by the control points {\tt Pt1}, {\tt NewPt1}, {\tt NewPt2}
and {\tt Pt4} interpolates the four points {\tt Pt1}, {\tt Pt2}, {\tt Pt3}
and {\tt Pt4} with respective parameter values of 0, 1/3, 2/3 and 1. Applying
the rule setting the point numbers already seen, if {\tt NewPt1} is equal to $k$,
then the second point {\tt NewPt2} is equal to $k+1$.
%%%%%%%%%%%%%%%%%%%%%%%%%%%%%%%%%%%%%%%%%%%%%%%%%%%%%%%%%%%%%%%%%%%%%%%%%%%
%\vfill\eject
\sect{\DrawFig}
\fsubsect{\Princip}
The first macro to call is \Mac{psbeginfig} whose only argument is the name of the
postscript file to be created. It initializes all the settings related to the drawing
macros.\par
To command the end of the postscript file being created, just say \Mac{psendfig}.
Between these two statements, macros that generate the figure are called. Their name
begin with \Mac{ps}.
\smallskip
We recall that, if the file already exists, it is not updated. The user has
to delete it to get a new version.
However, during the tuning time, this can be somewhat cumbersome. So to enforce \TeX\ 
to update the postscript file at each compilation, the user can insert the command
{\tt \Mac{pssetupdate}$\{$yes$\}$} before step 2. When the file is satisfactory, this
command has to be removed, otherwise the file will be unusefully recreated at each
compilation.
\par
WARNING ! We must be aware of what we do : clearly, the contents
of the file are destroyed. So, if for some reason the filename given as argument to
\Mac{psbeginfig} is the one of a precious file while the update mode is active,
the file will be destroyed as well\dots
\smallskip
Between the two commands \Mac{psbeginfig} and \Mac{psendfig}, all the points created
up to now can be used, but they can be redefined, as well as new points can be created.
Notice that drawing macros and computing macros are executed \eword{only} if the
postscript file is created or updated. In other words, every drawing or computing
macro called between \Mac{psbeginfig} and \Mac{psendfig} is ignored if the postscript
file exists and no update is wanted. This ensures future compilations to run faster.
%%%%%%%%%%%%%%%%%%%%%%%%%%%%%%%%%%%%%%%
\subsect{\LineWS}
By default, the line width is set to \defaultwidth\ bp.
The macro {\tt \Mac{pssetwidth}$\{$Width$\}$}
modifies the line width by using the new value {\tt Width}. The unit used
here is always the postscript point and the argument must be given without any unit
specification. To get back to the default line width, just say
{\tt \Mac{pssetwidth}$\{$\Mac{defaultwidth}$\}$}~. For example, you get
\medskip
% 1. Definition of characteristic points
\figinit{cm}
\figvectC 0(0,-0.5)
\figpt 1:(0,0)
\figpt 2:(4,0)
\figptstra 3=1,2/1,0/
\figptstra 5=1,2/2,0/
\figptstra 7=1,2/3,0/
\NewPSfile{\MyPSfile}
% 2. Creation of the postscript file
\psbeginfig{\MyPSfile}
\pssetwidth{\defaultwidth}\psline[1,2]
\pssetwidth{1}\psline[3,4]
\pssetwidth{1.5}\psline[5,6]
\pssetwidth{2}\psline[7,8]
\psendfig
% 3. Writing text on the figure
\figvisu{\demo}{}{%
\figinsert{\MyPSfile}
\figwritee 2:with {\tt \Mac{pssetwidth}$\{$\Mac{defaultwidth}$\}$}(1)
\figwritee 4:with {\tt \Mac{pssetwidth}$\{$1$\}$}(1)
\figwritee 6:with {\tt \Mac{pssetwidth}$\{$1.5$\}$}(1)
\figwritee 8:with {\tt \Mac{pssetwidth}$\{$2$\}$}(1)
}
\centerline{\box\demo}
\smallskip
\rightitem{Remark:}{
If the filling mode is on (see \Mac{pssetfillmode}), the width of a straight
line cannot be changed ; think to check this in case of trouble.}
\medskip
The default line style is solid. The macro {\tt \Mac{pssetdash}} modifies the line style.
Two calling sequences are available:\smallskip
\centerline{\vbox{\hbox{\tt\Mac{pssetdash}$\{$Index$\}$}
\hbox{\tt\Mac{pssetdash}$\{$Pattern$\}$}}}
\smallskip
The first form refers to the value {\tt Index} which is an integer
taking its value between 1 to 10. The solid line corresponds to the value 1, which
can be set by saying {\tt \Mac{pssetdash}$\{$\Mac{defaultdash}$\}$}~.
The different line styles available are shown below, using the default line width.
\smallskip
% 1. Definition of characteristic points
\figinit{cm}
\figvectC 0(0,-0.5)
\figpt 1:(0,0)
\figpt 2:(4,0)
\figptstra 3=1,2/1,0/
\figptstra 5=1,2/2,0/
\figptstra 7=1,2/3,0/
\figptstra 9=1,2/4,0/
\figptstra 11=1,2/5,0/
\figptstra 13=1,2/6,0/
\figptstra 15=1,2/7,0/
\figptstra 17=1,2/8,0/
\figptstra 19=1,2/9,0/
\NewPSfile{\MyPSfile}
% 2. Creation of the postscript file
\psbeginfig{\MyPSfile}
\pssetdash{\defaultdash}\psline[1,2]
\pssetdash{2}\psline[3,4]
\pssetdash{3}\psline[5,6]
\pssetdash{4}\psline[7,8]
\pssetdash{5}\psline[9,10]
\pssetdash{6}\psline[11,12]
\pssetdash{7}\psline[13,14]
\pssetdash{8}\psline[15,16]
\pssetdash{9}\psline[17,18]
\pssetdash{10}\psline[19,20]
\psendfig
% 3. Writing text on the figure
\figvisu{\demo}{}{%
\figinsert{\MyPSfile}
\figwritee 2:with {\tt \Mac{pssetdash}$\{$\Mac{defaultdash}$\}$}(1)
\figwritee 4:with {\tt \Mac{pssetdash}$\{$2$\}$}(1)
\figwritee 6:with {\tt \Mac{pssetdash}$\{$3$\}$}(1)
\figwritee 8:with {\tt \Mac{pssetdash}$\{$4$\}$}(1)
\figwritee 10:with {\tt \Mac{pssetdash}$\{$5$\}$}(1)
\figwritee 12:with {\tt \Mac{pssetdash}$\{$6$\}$}(1)
\figwritee 14:with {\tt \Mac{pssetdash}$\{$7$\}$}(1)
\figwritee 16:with {\tt \Mac{pssetdash}$\{$8$\}$}(1)
\figwritee 18:with {\tt \Mac{pssetdash}$\{$9$\}$}(1)
\figwritee 20:with {\tt \Mac{pssetdash}$\{$10$\}$}(1)
}
\centerline{\box\demo}
\par
The second form refers to a pattern which is a sequence of integers separated with white
spaces. The general form is\quad {\it Dash$_1$ Space$_1$ \dots Dash$_n$ Space$_n$ Offset}
\quad
where each value is a dimension given in postscript points (bp), {\it Dash$_i$}
corresponding
to the length of a dash, {\it Space$_i$} to the length between two dashes. The last value
{\it Offset} is optional. It corresponds to the length by which the whole pattern is shifted
towards the beginning of the line. Its default value is $0$. It allows to adjust the
position of the pattern. For example, on the previous figure, we may want to ``center''
the pattern over the second segment. Since {\tt \Mac{pssetdash}$\{$2$\}$} is equivalent
to {\tt \Mac{pssetdash}$\{$6 2$\}$}, we obtain a better look using the pattern form of the
macro as shown below.
\medskip
\NewPSfile{\MyPSfile}
% 2. Creation of the postscript file
\psbeginfig{\MyPSfile}
\pssetdash{2}\psline[3,4]
\pssetdash{6 2}\psline[5,6]
\pssetdash{6 2 2}\psline[7,8]
\psendfig
% 3. Writing text on the figure
\figvisu{\demo}{}{%
\figinsert{\MyPSfile}
\figwritee 4:with {\tt \Mac{pssetdash}$\{$2$\}$}(1)
\figwritee 6:{with {\tt \Mac{pssetdash}$\{$6 2$\}$} (same as previous)}(1)
\figwritee 8:with {\tt \Mac{pssetdash}$\{$6 2 2$\}$}(1)
}
\centerline{\box\demo}
\medskip\noindent
%\par\noindent
{\bf Secondary settings}\smallskip
The macros {\tt \Mac{pssetsecondwidth}$\{$Width$\}$} and
{\tt \Mac{pssetseconddash}$\{$Index$\}$} define a secondary line width and style
that are used by some high level macros. Their arguments range in the same set as
the corresponding ``primary" macros. The default values are obtained by saying
{\tt \Mac{pssetsecondwidth}$\{$\Mac{defaultsecondwidth}$\}$} and
{\tt \Mac{pssetseconddash}$\{$\Mac{defaultseconddash}$\}$}, or more simply
{\tt \Mac{psresetsecondsettings}} to reset both. Currently, \Mac{defaultsecondwidth}
equals to \defaultsecondwidth\ and \Mac{defaultseconddash} equals to \defaultseconddash.
\smallskip
Moreover, we explain in the forthcoming section {\bf \Color} that it is possible to
give a line a specific color, referred to as primary color. So, in the same way as for
the width and style attributes, a secondary line color can be set by one of the macros
{\tt \Mac{pssetsecondgray}}, {\tt \Mac{pssetsecondcmyk}} or {\tt \Mac{pssetsecondrgb}}.
Here, the default secondary color is the primary color, so the macro
{\tt \Mac{psresetsecondsettings}} resets the secondary color to the last
primary color selected.
%%%%%%%%%%%%%%%%%%%%%%%%%%%%%%%%%%%%%%%
\subsect{\BasDrMac}
Then, the first thing we need to draw is a line. The macro\smallskip
\centerline{\tt\Mac{psline}[Pt1,Pt2,... ,PtN]}
\smallskip\noindent
draws a broken line joining every point from {\tt Pt1} to {\tt PtN} in
this order. To close the path, just let the last point number {\tt PtN} be equal
to the first one {\tt Pt1}. In this case, the meeting of the two segments at
{\tt Pt1} is properly handled.
\medskip
Another version of this macro exists. Its prototype is\smallskip
\centerline{\tt\Mac{pslineF}$\{$Filename$\}$}
\smallskip\noindent
Here, the coordinates of the points are read from a text file whose name {\tt Filename}
is passed as argument to the macro. Each line of the file must contain the coordinates
of one point according to the format$\tt \ x\ y\ $ in 2D, or $\tt\ x\ y\ z\ $ in 3D.
Notice that two coordinates are separated by a white space. The line is closed if the
last point is identical to the first one.
\medskip
Circles are among the geometrical entities which are the most often drawn. This is the
reason why a specific macro has been built. It draws a circle defined by its center
and its radius, and its prototype is\smallskip
\centerline{\tt \Mac{pscirc} Center (Radius)}
%%%%%%%%%%%%%%%%%%%%%%%%%%%%%%%%%%%%%%%
\subsect{\BasicEx}
Now, let us examine an example to illustrate some of the macros presented so far.
First, draw a circle centered on the origin $O$ passing through a given point $M$.
Note the use of the macro \Mac{figgetdist} to compute the radius. Then, create the
point $P$, intersection of the circle and the X axis, and the point $Q$ such that
$\vect{PQ} = 1.5\ \vect{OP}$. At last, draw the circle with center $Q$ and radius $PQ$.
%\par\noindent
\smallskip
\NewPSfile{\MyPSfile}
We can see the result on Figure \the\Figno.
The text of the program that produces it follows. Notice the scale and the unit.
Moreover, since $M$ is the only true data, if we modify its coordinates, we just have
to recompile this program to get the corresponding result (see the previous subsection
called {\bf \Princip} about the update process).
\bigskip
% 1. Definition of characteristic points
\figinit{0.5in}
\figpt 1:$O$(0,0)\figpt 2:$M$(-0.9354, 0.45467)
\figgetdist\DistOM[1,2]\figvectC 10(\DistOM,0)
\figpttra 3:$P$=1/1,10/\figpttra 4:$Q$=3/1.5,10/
% 2. Creation of the postscript file
\psbeginfig{\MyPSfile}
\pscirc 1(\DistOM)
\figgetdist\DistPQ[3,4]\pscirc 4(\DistPQ)
\psendfig
% 3. Writing text on the figure
\figvisu{\demo}{Figure \the\Figno}{
\figinsert{\MyPSfile}
\figsetmark{$\times$}\figwritenw 2:(0.04)\figwritee 3:(0.04)
\figsetmark{.}\figwriten 1,4:(0.08)}
\centerline{\box\demo}
%\medskip
$$\vbox{\fmtprog{0pt}
\+{}\% 1. Definition of characteristic points\cr
\+{}\Mac{figinit}$\{$0.5in$\}$\cr
\+{}\Mac{figpt} 1:\$O\$(0,0)\Mac{figpt} 2:\$M\$(-0.9354, 0.45467)\cr
\+{}\Mac{figgetdist}\char`\\DistOM[1,2]\Mac{figvectC} 10(\char`\\DistOM,0)\cr
\+{}\Mac{figpttra} 3:\$P\$=1/1,10/\Mac{figpttra} 4:\$Q\$=3/1.5,10/\cr
\+{}\% 2. Creation of the postscript file\cr
\+{}\Mac{psbeginfig}$\{$\char`\\MyPSfile$\}$\cr
\+{}\Mac{pscirc} 1(\char`\\DistOM)\cr
\+{}\Mac{figgetdist}\char`\\DistPQ[3,4]\Mac{pscirc} 4(\char`\\DistPQ)\cr
\+{}\Mac{psendfig}\cr
\+{}\% 3. Writing text on the figure\cr
\+{}\Mac{figvisu}$\{$\char`\\demo$\}$$\{$Figure \char`\\the\char`\\Figno$\}$$\{$\cr
\+{}\Mac{figinsert}$\{$\char`\\MyPSfile$\}$\cr
\+{}\Mac{figsetmark}$\{$\$\char`\\times\$$\}$\Mac{figwritenw} 2:(0.04)\Mac{figwritee} 3:(0.04)\cr
\+{}\Mac{figsetmark}$\{$.$\}$\Mac{figwriten} 1,4:(0.08)$\}$\cr
\+{}\char`\\centerline$\{$\char`\\box\char`\\demo$\}$\cr
}$$
%%%%%%%%%%%%%%%%%%%%%%%%%%%%%%%%%%%%%%%%%%%%%%%%%%%%%%%%%%%%%%%%%%%%%%%%%%%
\vfill\eject
\sect{\WTxtFig}
Since the postscript file is now prepared, it
remains to make it appear at the right place on the page along with the
appropriate legend. To do that, we strongly recommand to use the macro \Mac{figvisu},
although it is possible to do without it, but probably more painfully.
Its prototype is:
\smallskip
\centerline{\tt \Mac{figvisu}$\{$Vbox$\}$$\{$Caption$\}$$\{$Commands$\}$}
\smallskip
As we have already seen in the previous examples, the first argument is the name of
a box register that the user has to make private by the \CtrlSq{newbox} command.
This box is filled according to the third argument and is the result of this macro.
Note that the macro produces a \eword{vertical} box.
\smallskip
The second argument is the figure caption. It can be void, but can also be, for
example, a box containing several lines.
\smallskip
The third argument is a list of commands that make the legend and the drawing fit
together. In this part, extra blank lines or \CtrlSq{par}, are forbidden.
Moreover, any user provided macro should end with \CtrlSq{ignorespaces}.
\par
To show the drawing, one has to insert it in the box with the macro \Mac{figinsert}
whose first argument is the name of the postscript file containing the drawing
description. This is done by a separate macro rather than a fourth argument to
\Mac{figvisu} because it allows to insert several drawings in the same box. The
drawings will then overlap since the origin is common. The second argument of
\Mac{figinsert} is a scale factor. This argument is \eword{optional} ; it is described
in the section {\bf\WTxtExFig}.
\smallskip
Once this has been done, the last thing to do, before showing the contents of the box,
is to write text on the figure. Several macros have been designed for that. We recall
that a text is written on the figure by specifying its position with respect to an
attach point. This point can be made visible with the macro \Mac{figsetmark} already
seen.
\medskip
The simplest macro is {\tt \ \Mac{figwritep}[Pt1, Pt2, ..., PtN]\ } which writes the
point marker set by \Mac{figsetmark} at the locations defined by the points
{\tt Pt1}, {\tt Pt2}, \dots, {\tt PtN}.
\smallskip
Then come the macros:\smallskip
\centerline{\vbox{
\hbox{\tt\Mac{figwrite} [Pt1, Pt2, ..., PtN]$\{$Text$\}$}
\hbox{\tt\Mac{figwritec}[Pt1, Pt2, ..., PtN]$\{$Text$\}$}
}}
\smallskip
The first one writes a {\tt Text} after the points {\tt Pt1}, {\tt Pt2}, \dots, {\tt PtN}
according to \TeX 's alignment, while the second one writes the text \eword{centered}
on theses positions. These two macros do not print the point marker.
\par
At first glance, it may look strange that the same text is written at each point.
This is the case if the argument {\tt Text} is present and this may actually be useful.
However, if an empty text is given, recalling the definition of this argument (given in
the {\bf\TextArg} subsection), the macro writes the text given when the point was
defined, or the implicit text which can be modified by \Mac{figsetptname}.
\medskip
This remark also applies to the following macros which are the most useful because they
allow to put a text at a prescribed position with respect to the attach points. We refer
to a compass-card notation to specify the position of the text. To write {\tt Text} at a 
given {\tt Distance} to the west, the east, the north or the south of the point {\tt Pti},
we can use the macros:\smallskip
\centerline{\vbox{\hbox{\tt\Mac{figwritew} Pt1, Pt2, ..., PtN :Text(Distance)}
\hbox{\tt\Mac{figwritee} Pt1, Pt2, ..., PtN :Text(Distance)}
\hbox{\tt\Mac{figwriten} Pt1, Pt2, ..., PtN :Text(Distance)}
\hbox{\tt\Mac{figwrites} Pt1, Pt2, ..., PtN :Text(Distance)}
}}
\smallskip
To write to the north-west, the south-west, the north-east or the south-east
of the point {\tt Pti}, we can use the macros:\smallskip
\centerline{\vbox{\hbox{\tt\Mac{figwritenw} Pt1, Pt2, ..., PtN :Text(Distance)}
\hbox{\tt\Mac{figwritesw} Pt1, Pt2, ..., PtN :Text(Distance)}
\hbox{\tt\Mac{figwritene} Pt1, Pt2, ..., PtN :Text(Distance)}
\hbox{\tt\Mac{figwritese} Pt1, Pt2, ..., PtN :Text(Distance)}
}}
\medskip
\NewPSfile{\MyPSfile}
For the different cases, the figure \the\Figno\ shows the position of the
corresponding text with respect to the attach point marked with the + sign. To avoid
hiding the + sign, the text in central position is represented by an empty rectangle.
\bigskip
% 1. Definition of characteristic points
\figinit{cm}
\def\Dist{1.5}\figpt 0:(0, 0)
% 2. Creation of the postscript file
\psbeginfig{\MyPSfile}
\pssetdash{5}\pscirc 0(\Dist)
\psendfig
% 3. Writing text on the figure
\figvisu{\demo}{Figure \the\Figno}{
\figinsert{\MyPSfile}\figsetmark{+}
\figwritew 0:West(\Dist)\figwritee 0:East(\Dist)
\figwriten 0:North(\Dist)\figwrites 0:South(\Dist)
\figwritenw 0:North-west(\Dist)\figwritesw 0:South-west(\Dist)
\figwritene 0:North-east(\Dist)\figwritese 0:South-east(\Dist)
\figwritec[0]{\boxit{1pt}{\phantom{Center}}}
% or equivalently:
\figsetmark{\boxit{1pt}{\phantom{Center}}}\figwritep[0]
}
\centerline{\box\demo}
\bigskip
The text of the program that produces this figure is:
$$\vbox{\fmtprog{0pt}
\+{}\% 1. Definition of characteristic points\cr
\+{}\Mac{figinit}$\{$cm$\}$\cr
\+{}\char`\\def\char`\\Dist$\{$1.5$\}$\Mac{figpt} 0:(0, 0)\cr
\+{}\% 2. Creation of the postscript file\cr
\+{}\Mac{psbeginfig}$\{$\char`\\MyPSfile$\}$\cr
\+{}\Mac{pssetdash}$\{$5$\}$\Mac{pscirc} 0(\char`\\Dist)\cr
\+{}\Mac{psendfig}\cr
\+{}\% 3. Writing text on the figure\cr
\+{}\Mac{figvisu}$\{$\char`\\demo$\}$$\{$Figure \char`\\the\char`\\Figno$\}$$\{$\cr
\+{}\Mac{figinsert}$\{$\char`\\MyPSfile$\}$\Mac{figsetmark}$\{$+$\}$\cr
\+{}\Mac{figwritew} 0:West(\char`\\Dist)\Mac{figwritee} 0:East(\char`\\Dist)\cr
\+{}\Mac{figwriten} 0:North(\char`\\Dist)\Mac{figwrites} 0:South(\char`\\Dist)\cr
\+{}\Mac{figwritenw} 0:North-west(\char`\\Dist)\Mac{figwritesw} 0:South-west(\char`\\Dist)\cr
\+{}\Mac{figwritene} 0:North-east(\char`\\Dist)\Mac{figwritese} 0:South-east(\char`\\Dist)\cr
\+{}\Mac{figwritec}[0]$\{$\char`\\boxit$\{$1pt$\}$$\{$\char`\\phantom$\{$Center$\}$$\}$$\}$\cr
\+{}\% or equivalently:\cr
\+{}\Mac{figsetmark}$\{$\char`\\boxit$\{$1pt$\}$$\{$\char`\\phantom$\{$Center$\}$$\}$$\}$\Mac{figwritep}[0]\cr
\+{}$\}$\cr
\+{}\char`\\centerline$\{$\char`\\box\char`\\demo$\}$\cr
}$$
\medskip\noindent
Remarks:
\item{$\bullet$} Here, the 8 first calls to \Mac{figwrite*}
write the + sign 8 times at the same place.
\item{$\bullet$} The dashed circle shows the distance between the text and the attach
point. For the show, we have chosen a ``large" distance, but in practice of course the
text is much closer to the attach point.
\item{$\bullet$} Generally, we want the point-text distance to be independent of
the scale chosen by the \Mac{figinit} call. Thus, the dimension of the figure
can be tuned without acting on this distance. To obtain this result, just specify
a unit after the numerical value (see next example). This is the only case where a
dimension can be followed  by a unit specification. (This works in the same way as
with the \TeX\ \CtrlSq{magnification} command: to get a \eword{true} unit, you give
a dimension specifying a unit like {\tt truept}).
\medskip
These macros are sufficient most of the time, but for particular cases, they are not.
That is why two general macros have been designed. Their prototypes are:\smallskip
\centerline{\vbox{\hbox{\tt\Mac{figwritegw} Pt1, Pt2, ..., PtN :Text(DistanceX,DistanceY)}
\hbox{\tt\Mac{figwritege} Pt1, Pt2, ..., PtN :Text(DistanceX,DistanceY)}
}}
\smallskip
They write a {\tt Text} placed, with respect to the point {\tt Pti}, at a horizontal
distance {\tt DistanceX} (respectively west or east) and a vertical distance
{\tt DistanceY} from the bottom of the text if ${\tt DistanceY} > 0$, from the top
if ${\tt DistanceY} < 0$. If ${\tt DistanceY} = 0$, the text is vertically centered
with respect to {\tt Pti}.
\smallskip
The last two macros are internally used by some of the previous ones. They are made
public although they are maybe redundant. Their prototypes are:\smallskip
\centerline{\vbox{\hbox{\tt\Mac{figwritegcw} Pt1, Pt2, ..., PtN :Text(DistanceX,DistanceY)}
\hbox{\tt\Mac{figwritegce} Pt1, Pt2, ..., PtN :Text(DistanceX,DistanceY)}
}}
\smallskip
They do the same job as the two previous ones except that {\tt DistanceY} measures
the vertical distance between the point and the mid-height of the text.
\NewPSfile{\MyPSfile}
This is shown on figure \the\Figno\ where the vectors {\tt (DistanceX,DistanceY)} are
drawn from the attach point.
\medskip
% 1. Definition of characteristic points
\figinit{cm}
\figpt 0:(0, 0)
\figpt 1:(-2,1)\figpt 2:(-1.5,-1.5)
\figpt 3:(1.5,1.5)\figpt 4:(2,-1)
% 2. Creation of the postscript file
\psbeginfig{\MyPSfile}
\psline [0,1]\psline [0,2]\psline [0,3]\psline [0,4]
\psendfig
% 3. Writing text on the figure
\figvisu{\demo}{Figure \the\Figno}{
\figinsert{\MyPSfile}
\figsetmark{+}
\figwritegw 0:General West(2,10mm)
\figwritegcw 0:General Centered West(1.5,-1.5)
\figwritegce 0:General Centered East(1.5,1.5)
\figwritege 0:General East(2,-1)
}
\centerline{\box\demo}
\medskip
The attach point is the origin (point \#0), marked with the + sign.
The unit is {\tt cm}. Notice that in the first call the unit is specified.
The corresponding calls are:
$$\vbox{\fmtprog{0pt}
\+{}\Mac{figsetmark}$\{$+$\}$\cr
\+{}\Mac{figwritegw} 0:General West(2,10mm)\cr
\+{}\Mac{figwritegcw} 0:General Centered West(1.5,-1.5)\cr
\+{}\Mac{figwritegce} 0:General Centered East(1.5,1.5)\cr
\+{}\Mac{figwritege} 0:General East(2,-1)\cr
}$$
%%%%%%%%%%%%%%%%%%%%%%%%%%%%%%%%%%%%%%%%%%%%%%%%%%%%%%%%%%%%%%%%%%%%%%%%%%%
\vskip-\bigskipamount
\sect{\WTxtExFig}
\NewPSfile{\MyPSfile}
To write something on an existing figure, any unit can be chosen. For the
demonstration, we consider the triangle drawn in the file {\tt \Figone} used at
the beginning. If we want to write something at the right of the triangle,
we can choose {\tt cm} as unit and write the following commands:\par
$$\vbox{\fmtprog{0pt}
\+{}\Mac{figinit}$\{$cm$\}$\cr
\+{}\% 3. Writing text on the figure\cr
\+{}\Mac{figvisu}$\{$\char`\\demo$\}$$\{$Figure \char`\\the\char`\\Figno$\}$$\{$\cr
\+{}\Mac{figpt} 1:(4,3)\cr
\+{}\Mac{figwrite} [1]$\{$\char`\\vbox$\{$\char`\\hbox$\{$Something at the right$\}$\char`\\hbox$\{$of the triangle$\}$$\}$$\}$\cr
\+{}\Mac{figinsert}$\{$Fig1.ps$\}$\cr
\+{}$\}$\cr
\+{}\char`\\centerline$\{$\char`\\box\char`\\demo$\}$\cr
}$$
Thus, we get the result shown on the figure \the\Figno.
\smallskip
Notice that the macro \Mac{figinsert} needs an encapsulated postscript file to work
correctly, because it needs the bounding box of the figure. If the file does not
contain this information, a message is printed on the screen and in the log file
at compilation time, and a small default bounding box is used. In this case, the best
thing to do is to provide one by editing the file and adding such a line in the header
(after the first line which is a control line):\par
\centerline{\tt \%\%BoundingBox: Xmin Ymin Xmax Ymax}\smallskip\noindent
where {\tt Xmin} and {\tt Xmax} generally range between 0 and 540, and {\tt Ymin} and
{\tt Ymax} range between 0 and 720 (these values correspond to the whole page). It
must be noticed that some graphical softwares do not set a correct bounding box.
Obviously, the position of the figure on the page depends on these values, so it is
recommended to set a bounding box as close as possible to the real figure. This can
be done after a few trials with the help of the macro \Mac{figscan} presented below
or using {\tt ghostview}.
For example, the fifth line of the header of the file {\tt \Figone} is:\par
\centerline{\tt \%\%BoundingBox: -9.96262 -9.96262 129.514 119.55139}
\smallskip\noindent
The values of {\tt Xmin} and {\tt Ymin} are negative. This is not an error and is
due to the coordinates of the points, but the drawing must be translated to appear
entirely on the page, which is done by the macro \Mac{figvisu}.
\topinsert
\figinit{cm}
% 3. Writing text on the figure
\def\MyPsfile{\Figone}
\figvisu{\demo}{Figure \the\Figno}{
\figpt 1:(4,3)
\figwrite [1]{\vbox{\hbox{Something at the right}\hbox{of the triangle}}}
\figinsert{\MyPsfile}
}
\centerline{\box\demo}
\endinsert
\par
In the previous example, the coordinates of the attach point of the text need not
to be set very accurately and we used integers. However, it sometimes happens that
one wants to refer to a particular point on the figure. For that reason, we will have
to give the position in postscript units, so the unit to use must be {\tt bp}.
Except in the rare case where the user explicitly knows the postscript coordinates
of the point, the problem is precisely to obtain easily these coordinates.
\par
One solution may be to use {\tt ghostview}, the postscript previewer available on
most UNIX systems. When the postscript file is processed by this software, the drawing
appears on the screen and the position of the mouse is shown cross-shaped when moved
on it. At the same time, the postscript coordinates of the position of the mouse is
given at the top left corner of the window, varying in real time with each motion.
We can take advantage of this to pick the postscript coordinates of any point of the
drawing and put them into the \TeX\ document to define an attach point as shown above.
Note however that this feature is not available with every {\tt ghostview}
implementation.
\par
A second solution is to use the macro {\tt \Mac{figscan} FileName(HX,HY)} which
draws a rectangular grid with horizontal steps {\tt HX} and vertical steps {\tt HY}.
Numerical values corresponding to the size of the figure described in the file
{\tt FileName} are printed, at most every 20 bp horizontally and every 10 bp vertically
to avoid overprinting. The values of the steps are given in postscript units ({\tt bp})
and supposed to be integers, because the first value is rounded and no decimals are
printed. This macro is intended to be used temporarily together with the macro
\Mac{figinsert} to set the coordinates of the attach point or adjust the bounding box
of the figure.
\smallskip
For example, if we want to write the word ``Triangle" inside the triangle, we can do
the following.
\par
First, determine the coordinates of the attach point. We compile a file
containing the commands:
$$\vbox{\fmtprog{0pt}
\+{}\char`\\def\char`\\MyPsfile$\{$Fig1.ps$\}$\cr
\+{}\Mac{figvisu}$\{$\char`\\demo$\}$$\{$Figure \char`\\the\char`\\Figno$\}$$\{$\cr
\+{}\Mac{figinsert}$\{$\char`\\MyPsfile$\}$\cr
\+{}\Mac{figscan} \char`\\MyPsfile (10,10)$\}$\cr
\+{}\char`\\centerline$\{$\char`\\box\char`\\demo$\}$\cr
}$$
\par
\NewPSfile{\MyPSfile}
Then we use a \TeX\ previewer to look at the result shown on the figure \the\Figno.
\figinit{pt}
\def\MyPsfile{\Figone}
\figvisu{\demo}{Figure \the\Figno}{
\figinsert{\MyPsfile}
\figscan \MyPsfile (10,10)}
%\centerline{\box\demo}
\smallskip
If the \TeX\ previewer does not show the postscript files, it is necessary to create
the postscript version and use a postscript previewer such as {\tt ghostview} or
{\tt gs}. As a last resort, it is always possible to print it on a paper sheet.\par
The grid and the drawing are printed together. We can observe that horizontally, only
one value over two is printed. We are now able to fix the coordinates
of the point: we choose (70,50). Then, we set the unit to {\tt bp}, we remove the call
to \Mac{figscan}, we create the point and center the text on it, which leads to the
following commands.
\NewPSfile{\MyPSfile}
The result is shown on the figure \the\Figno.
$$\vbox{\fmtprog{0pt}
\+{}\Mac{figinit}$\{$bp$\}$\cr
\+{}\char`\\def\char`\\MyPsfile$\{$Fig1.ps$\}$\cr
\+{}\Mac{figvisu}$\{$\char`\\demo$\}$$\{$Figure \char`\\the\char`\\Figno$\}$$\{$\cr
\+{}\Mac{figinsert}$\{$\char`\\MyPsfile$\}$\cr
\+{}\Mac{figpt} 1:(70,50)\Mac{figwritec}$\{$1$\}$$\{$Triangle$\}$$\}$\cr
}$$
\smallskip
\newbox\demotwo
\figinit{bp}
\def\MyPsfile{\Figone}
\figvisu{\demotwo}{Figure \the\Figno}{
\figinsert{\MyPsfile}
\figpt 1:(70,50)\figwritec[1]{Triangle}}
\topinsert
\centerline{\box\demo\hfil\box\demotwo}
\endinsert
\smallskip
Another useful feature is to control the scale of the figure to be inserted in the
page. This can be done with the second argument of the macro \Mac{figinsert} whose
full prototype is:\smallskip
\centerline{\tt \Mac{figinsert}$\{$FileName, ScaleFactor$\}$}\smallskip\noindent
 The user has to provide the extra
argument {\tt ScaleFactor} which is a positive real number whose default value is 1,
so that {\tt \Mac{figinsert}$\{$FileName,1$\}$} is equivalent to
{\tt \Mac{figinsert}$\{$FileName$\}$}.
\smallskip\noindent
Remarks:
\item{$\bullet$} Although this macro works with postscript files created by the
{\tt \figforTeX} macro-package, it is essentially intended to be used with files
created by other softwares. We recall that {\tt \figforTeX} controls the scale with
the macro \Mac{figinit}.
\item{$\bullet$} When several scale factors are specified, the macro \Mac{figscan}
works correctly only if it appears \eword{immediately after} the corresponding call
to \Mac{figinsert}.
\medskip\noindent
\NewPSfile{\MyPSfile}
An example of its use is given on figure \the\Figno.
\smallskip
\figinit{bp}
\def\MyPsfile{\Figone}
\figvisu{\demo}{Figure \the\Figno}{
\figinsert{\MyPsfile}
\figpt 1:(90,60)\figwritec[1]{\bf T}
\figinsert{\MyPsfile, 0.5}
\figpt 1:(45,30)\figwritec[1]{\bf t}
}
\centerline{\box\demo}
\medskip
The same postscript file,
{\tt\Figone}, is used twice. We obtain two nested triangles and we observe that they
are drawn with respect to the same origin. We choose 0.5 as scale factor and we write
the letter {\bf T} on the big triangle, the letter {\bf t} at the same relative
position on the small one.\par
The program that produces it is :
$$\vbox{\fmtprog{0pt}
\+{}\Mac{figinit}$\{$bp$\}$\cr
\+{}\char`\\def\char`\\MyPsfile$\{$Fig1.ps$\}$\cr
\+{}\Mac{figvisu}$\{$\char`\\demo$\}$$\{$Figure \char`\\the\char`\\Figno$\}$$\{$\cr
\+{}\Mac{figinsert}$\{$\char`\\MyPsfile$\}$\cr
\+{}\Mac{figpt} 1:(90,60)\Mac{figwritec}$\{$1$\}$$\{$\char`\\bf T$\}$\cr
\+{}\Mac{figinsert}$\{$\char`\\MyPsfile,0.5$\}$\cr
\+{}\Mac{figpt} 1:(45,30)\Mac{figwritec}$\{$1$\}$$\{$\char`\\bf t$\}$\cr
\+{}$\}$\cr
\+{}\char`\\centerline$\{$\char`\\box\char`\\demo$\}$\cr
}$$
%%%%%%%%%%%%%%%%%%%%%%%%%%%%%%%%%%%%%%%%%%%%%%%%%%%%%%%%%%%%%%%%%%%%%%%%%%%
\sect{\DrawMac}
\fsubsect{\PrelRem}
We now present some macros we think useful in a variety of situations listed in the
following sections.
However, when we want to design a new figure, the available macros may not solve
directly the problem and it may be necessary to write a new macro. In this case, we
advise you to build it, as far as possible, using user macros. But sometimes, this
will imply to use internal macros: the best is to start from the structure of an
existing macro that looks similar to the new one.
\smallskip
Every system has limits of use and we must be aware of what can be expected
from this macro-package. We recall that its main purpose is to give the ability to
write some text on a figure, easily and under total control of the user. Whenever
complicated computations are necessary, they must be done elsewhere and used here,
or the figure must be entirely created by another software and then imported ; the
macro \Mac{figscan} has been created for this purpose.
%%%%%%%%%%%%%%%%%%%%%%%%%%%%%%%%%%%%%%%
\subsect{\Arc}
We already saw the macro \Mac{pscirc} which draws an entire circle. Here are
macros that deal with a portion of arc. There are several versions of them so
that one can choose the simplest solution depending on the available data.
\smallskip
If data consists mainly in angles, one can use the macros whose prototypes are:
\smallskip
\centerline{\vbox{\hbox{\tt\Mac{psarccirc} Center ; Radius (Ang1,Ang2)}
\hbox{\tt\Mac{psarcell} Center ; XRad,YRad (Ang1,Ang2, Inclination)}
}}
\smallskip
The first one draws a circular arc, the second one draws an arc of ellipse.
Although a circle is a particular case of ellipse, the macro \Mac{psarccirc}
has been built because a circular arc is needed more often than an arc of
ellipse, so a shorter list of arguments makes its use more straightforward.
\smallskip
Let us detail the arguments of the macros. {\tt Ang1}, {\tt Ang2} and
{\tt Inclination} are angles to be given in degrees. {\tt Ang1} and {\tt Ang2}
are measured counterclockwise from the local $X$ axis, also called major axis.
\smallskip
The circular arc is the part of the circle of center {\tt Center} and radius
{\tt Radius} delimited by the angles {\tt Ang1} and {\tt Ang2}.
\smallskip
Except the limiting angles, the arguments of \Mac{psarcell} have the same definitions
as those of the macro \Mac{figptell}. The arc of ellipse is the part of the
ellipse defined by the center {\tt Center}, the two radii {\tt XRad} and {\tt YRad},
delimited by the parametrization angles {\tt Ang1} and {\tt Ang2}. The ellipse is
rotated by {\tt Inclination} on the paper sheet.
\medskip
If data consists mainly in points, these macros are not very handy. This is the case
for example when the arc we would like to draw is limited by two crossing lines.
We know the intersecting point (that can be computed by \Mac{figptinterlines})
and one other point on each line.
So ``point versions" have been created whose prototypes are:\smallskip
\centerline{\vbox{\hbox{\tt\Mac{psarccircP} Center ; Radius [Pt1,Pt2]}
\hbox{\tt\Mac{psarcellPP} Center,PtAxis1,PtAxis2 [Pt1,Pt2]}
}}
\smallskip\noindent
The first macro draws a circular arc of center {\tt Center} and of radius {\tt Radius}
limited by the two half-lines ({\tt Center}, {\tt Pt1}) and
({\tt Center}, {\tt Pt2}). The arc is drawn from {\tt Pt1} towards {\tt Pt2}
turning counterclockwise around {\tt Center}.
\smallskip
Let $C, A_1, A_2$ stand for {\tt Center}, {\tt PtAxis1} and {\tt PtAxis2}. These
arguments have the same definition as those of the macro \Mac{figptellP}:
$C$ is the center of the ellipse, $A_1$ is the end point of the major axis and 
$A_2$ is the end point of the minor axis.
The macro \Mac{psarcellPP} draws an arc of ellipse of center $C$ limited by the
two half-lines ({\tt Center}, {\tt Pt1}) and ({\tt Center}, {\tt Pt2}).
The arc is drawn counterclockwise around the vector $\vect{CA_1}\times\,\vect{CA_2}$.
\medskip
Another version of the latter one exists. Its prototype is:\smallskip
\centerline{\tt \Mac{psarcellPA} Center,PtAxis1,PtAxis2 (Ang1, Ang2)}
\smallskip\noindent
It behaves exactly like \Mac{psarcellPP} except that it allows to specify the
portion of the arc using the parametrization angles {\tt Ang1} and {\tt Ang2}.
\smallskip
\NewPSfile{\MyPSfile}
\edef\FigArc{\the\Figno}
% 1. Definition of characteristic points
\figinit{pt}
\def\haxis{100}\def\vaxis{40}
\def\startangle{-20}\def\stopangle{90}\def\inclin{45}
\figpt 1:(-10,10)\figpttraC 2:=1/0,-80/\figpttraC 3:=1/200,0/
\figptell 4:: 3 ; \haxis,\vaxis( 0, \inclin) % End point of major axis
\figptell 5:: 3 ; \haxis,\vaxis(90, \inclin) % End point of minor axis
% 2. Creation of the postscript file
\psbeginfig{\MyPSfile}
\psarccirc 1;\vaxis(\startangle,\stopangle)\psarccircP 2;\vaxis[1,3]
\pssetdash{3}\psarcell 3 ; \haxis,\vaxis(0,360, \inclin)
\pssetdash{\defaultdash}\pssetwidth{1}
\psarcell 3 ; \haxis,\vaxis(\startangle,\stopangle, \inclin)
\psarcellPP 3,4,5[1,2]\psarcellPA 3,4,5(-180,-90)
\pssetwidth{\defaultwidth}\pssetdash{5}\psline[1,2,3,1]
\psendfig
% 3. Writing text on the figure
\figvisu{\demo}{Figure \the\Figno}{
\figinsert{\MyPSfile}
\figsetmark{.}\figwriten 1,3:(2)\figwritesw 2:(2)
}
\centerline{\box\demo}
\medskip
Figure \FigArc\ shows an example of their use.
The text of the program that produces it follows. Notice that the points \#4 and \#5
are created only to have the opportunity to call the ``point" versions of
\Mac{psarcell}.
$$\vbox{\fmtprog{0pt}
\+{}\% 1. Definition of characteristic points\cr
\+{}\Mac{figinit}$\{$pt$\}$\cr
\+{}\char`\\def\char`\\haxis$\{$100$\}$\char`\\def\char`\\vaxis$\{$40$\}$\cr
\+{}\char`\\def\char`\\startangle$\{$-20$\}$\char`\\def\char`\\stopangle$\{$90$\}$\char`\\def\char`\\inclin$\{$45$\}$\cr
\+{}\Mac{figpt} 1:(-10,10)\Mac{figpttraC} 2:=1/0,-80/\Mac{figpttraC} 3:=1/200,0/\cr
\+{}\Mac{figptell} 4:: 3 ; \char`\\haxis,\char`\\vaxis( 0, \char`\\inclin) \% End point of major axis\cr
\+{}\Mac{figptell} 5:: 3 ; \char`\\haxis,\char`\\vaxis(90, \char`\\inclin) \% End point of minor axis\cr
\+{}\% 2. Creation of the postscript file\cr
\+{}\Mac{psbeginfig}$\{$\char`\\MyPSfile$\}$\cr
\+{}\Mac{psarccirc} 1;\char`\\vaxis(\char`\\startangle,\char`\\stopangle)\Mac{psarccircP} 2;\char`\\vaxis[1,3]\cr
\+{}\Mac{pssetdash}$\{$3$\}$\Mac{psarcell} 3 ; \char`\\haxis,\char`\\vaxis(0,360, \char`\\inclin)\cr
\+{}\Mac{pssetdash}$\{$\Mac{defaultdash}$\}$\Mac{pssetwidth}$\{$1$\}$\cr
\+{}\Mac{psarcell} 3 ; \char`\\haxis,\char`\\vaxis(\char`\\startangle,\char`\\stopangle, \char`\\inclin)\cr
\+{}\Mac{psarcellPP} 3,4,5[1,2]\Mac{psarcellPA} 3,4,5(-180,-90)\cr
\+{}\Mac{pssetwidth}$\{$\Mac{defaultwidth}$\}$\Mac{pssetdash}$\{$5$\}$\Mac{psline}[1,2,3,1]\cr
\+{}\Mac{psendfig}\cr
\+{}\% 3. Writing text on the figure\cr
\+{}\Mac{figvisu}$\{$\char`\\demo$\}$$\{$Figure \char`\\the\char`\\Figno$\}$$\{$\cr
\+{}\Mac{figinsert}$\{$\char`\\MyPSfile$\}$\cr
\+{}\Mac{figsetmark}$\{$.$\}$\Mac{figwriten} 1,3:(2)\Mac{figwritesw} 2:(2)\cr
\+{}$\}$\cr
\+{}\char`\\centerline$\{$\char`\\box\char`\\demo$\}$\cr
}$$
%%%%%%%%%%%%%%%%%%%%%%%%%%%%%%%%%%%%%%%
\subsect{\EllipseEx}
Here is a more elaborated example involving a small geometric construction.
The data consist of an ellipse of center $C$ and a point $O$ lying on its major
axis. The aim is to draw the tangent lines to the ellipse passing through $O$.
\smallskip
Thus, we have to compute the two points $I$ and $J$ lying on the ellipse such that
$(O, I)$ and $(O, J)$ are tangent to the ellipse.
\smallskip
The two radii of the ellipse are denoted $R_x$ and $R_y$, and are such that
$R_x > R_y$. Let $A$ be the middle point of $[OC]$ and $\rho=||\vect{AC}||$.
The two circles $(A ; \rho)$ and $(C ; R_x)$ intersect at points $I'$ and $J'$.
The lines $(O, I')$ and $(O, J')$ are the tangent lines to the circle $(C ; R_x)$
passing through $O$.
Moreover, if $B$ denotes a point lying on $(O, C)$ with $x_B > x_C$,
the angles $\theta_I=(\vect{CB},\vect{CI'})$ and $\theta_J=(\vect{CB},\vect{CJ'})$
are the parametrization angles of the points $I$ and $J$ which solves the problem.
\smallskip
The corresponding figure is given below, along with the program that produces it.
Some additional points are computed in order to draw the lines.
%
\bigskip
\NewPSfile{\MyPSfile}
% 1. Definition of characteristic points
\figinit{in}
% Data
\figpt 0:$O$(-0.5,0)\figpt 1:$C$(2,0)
\def\Rx{1}\def\Ry{0.7}
% Computations
\figptbary2:$A$[0,1;1,1]\figgetdist\RHO[2,1]
\figptsintercirc 3[1,\Rx;2,\RHO] % Computes points 3 (I') and 4 (J')
\figptbary 10:[0,1;6,-1]\figptbaryR 11:$B$[0,1;-1,2.5]
\figgetangle\ThetaI[1,11,3]
\figgetangle\ThetaJ[1,11,4]
\figptell 3:$I$: 1;\Rx,\Ry (\ThetaI,0)
\figptell 4:$J$: 1;\Rx,\Ry (\ThetaJ,0)
\figptbary 13:[0,3;-1,3]
\figptbary 14:[0,4;-1,3]
% 2. Creation of the postscript file
\psbeginfig{\MyPSfile}
\psarcell 1;\Rx,\Ry (0,360,0)
\psline[10,11]\psline[0,13]\psline[0,14]
\psendfig
% 3. Writing text on the figure
\def\dist{4pt}
\figvisu{\demo}{Figure \the\Figno}{
\figinsert{\MyPSfile}
\figwriten 0:(\dist)\figwriten 11:(\dist)
\figsetmark{+}\figwriten 2:(\dist)
\figsetmark{$\Bullet$}\figwriten 1:(\dist)
\figsetmark{$\times$}\figwriten 3,4:(\dist)
}
\centerline{\box\demo}
%
\medskip
$$\vbox{\fmtprog{0pt}
\+{}\% 1. Definition of characteristic points\cr
\+{}\Mac{figinit}$\{$in$\}$\cr
\+{}\% Data\cr
\+{}\Mac{figpt} 0:\$O\$(-0.5,0)\Mac{figpt} 1:\$C\$(2,0)\cr
\+{}\char`\\def\char`\\Rx$\{$1$\}$\char`\\def\char`\\Ry$\{$0.7$\}$\cr
\+{}\% Computations\cr
\+{}\Mac{figptbary}2:\$A\$[0,1;1,1]\Mac{figgetdist}\char`\\RHO[2,1]\cr
\+{}\Mac{figptsintercirc} 3[1,\char`\\Rx;2,\char`\\RHO] \% Computes points 3 (I') and 4 (J')\cr
\+{}\Mac{figptbary} 10:[0,1;6,-1]\Mac{figptbaryR} 11:\$B\$[0,1;-1,2.5]\cr
\+{}\Mac{figgetangle}\char`\\ThetaI[1,11,3]\cr
\+{}\Mac{figgetangle}\char`\\ThetaJ[1,11,4]\cr
\+{}\Mac{figptell} 3:\$I\$: 1;\char`\\Rx,\char`\\Ry (\char`\\ThetaI,0)\cr
\+{}\Mac{figptell} 4:\$J\$: 1;\char`\\Rx,\char`\\Ry (\char`\\ThetaJ,0)\cr
\+{}\Mac{figptbary} 13:[0,3;-1,3]\cr
\+{}\Mac{figptbary} 14:[0,4;-1,3]\cr
\+{}\% 2. Creation of the postscript file\cr
\+{}\Mac{psbeginfig}$\{$\char`\\MyPSfile$\}$\cr
\+{}\Mac{psarcell} 1;\char`\\Rx,\char`\\Ry (0,360,0)\cr
\+{}\Mac{psline}[10,11]\Mac{psline}[0,13]\Mac{psline}[0,14]\cr
\+{}\Mac{psendfig}\cr
\+{}\% 3. Writing text on the figure\cr
\+{}\char`\\def\char`\\dist$\{$4pt$\}$\cr
\+{}\Mac{figvisu}$\{$\char`\\demo$\}$$\{$Figure \char`\\the\char`\\Figno$\}$$\{$\cr
\+{}\Mac{figinsert}$\{$\char`\\MyPSfile$\}$\cr
\+{}\Mac{figwriten} 0:(\char`\\dist)\Mac{figwriten} 11:(\char`\\dist)\cr
\+{}\Mac{figsetmark}$\{$+$\}$\Mac{figwriten} 2:(\char`\\dist)\cr
\+{}\Mac{figsetmark}$\{$\$\char`\\Bullet\$$\}$\Mac{figwriten} 1:(\char`\\dist)\cr
\+{}\Mac{figsetmark}$\{$\$\char`\\times\$$\}$\Mac{figwriten} 3,4:(\char`\\dist)\cr
\+{}$\}$\cr
\+{}\char`\\centerline$\{$\char`\\box\char`\\demo$\}$\cr
}$$
%%%%%%%%%%%%%%%%%%%%%%%%%%%%%%%%%%%%%%%
\vfill\eject
\subsect{\Curve}
From the postscript capabilities, we derived a basic macro to draw a portion of curve.
Its prototype is\smallskip
\centerline{\tt \Mac{psBezier} N [Pt\_1, ..., Pt\_$\{$3N+1$\}$]}
\smallskip\noindent
It allows to draw polynomial curves of any shape using the B\'ezier representation,
that is to say by giving a set of control points. Since the postscript language
only provides cubic polynomials, this macro draws a curve formed
by a succession of cubic arcs, each of them defined by four control points.
The fourth control point of an arc is the first of the following arc. Thus, if
the curve is composed of $N$ arcs, the user must provide $3N+1$ control points.
This is not checked.
\smallskip
As a consequence of the B\'ezier representation, the points $P_1$,
$P_4$,\dots,$P_{3k+1}$,\dots,$P_{3N+1}$ lie on the curve. Moreover, the segment
[$P_{3k}, P_{3k+1}$] (respectively [$P_{3k+1}, P_{3k+2}$]) is tangent to the
arc number $k$ (respectively $k+1$) at point $P_{3k+1}$.
\medskip\noindent
Remarks:
\item{$\bullet$} Two successive points may coincide.
\item{$\bullet$} We recall that the coordinates of a point lying on a cubic
B\'ezier arc can be computed by the macro \Mac{figptBezier} and the derivatives
by the macro \Mac{figvectDBezier}.
\bigskip
Here follows an example showing a curve consisting of three cubic arcs. The 
dashed polygon shows the control points. The curve is $C^2$ except at point $P_4$
where it is only $C^0$ and at point $P_7$ where it is $C^1$ since $P_8$ is the
symmetric of $P_6$ with respect to $P_7$.
\NewPSfile{\MyPSfile}
\edef\FigCur{\the\Figno}
\bigskip
% 1. Definition of characteristic points
\figinit{pt}
\figpt 1:(-30,-10)\figpt 2:(0,-70)\figpt 3:(60,-60)\figpt 4:(80,10)
\figpt 0:(80,100) % Only used to define the line (0,4) for the symmetry
\figptssym 5=3,2,1/0,4/\figptsrot 8=6,5,4/7,180/
% 2. Creation of the postscript file
\psbeginfig{\MyPSfile}
\psBezier 3[1,2,3,4,5,6,7,8,9,10]
\pssetdash{8} \psline[1,2,3,4,5,6,7,8,9,10]
\psendfig
% 3. Writing text on the figure
\figvisu{\demo}{Figure \the\Figno}{
\figinsert{\MyPSfile}\figsetptname{$P_{#1}$}
\figwritew 1:(2)\figwritesw 2,5:(1)\figwritese 3,6:(1)\figwriten 4:(2)
\figwritee 7,10:(2)\figwritenw 8:(1)\figwritene 9:(1)
}
\centerline{\box\demo}
\bigskip
The text of the program that produces it is:
$$\vbox{\fmtprog{0pt}
\+{}\% 1. Definition of characteristic points\cr
\+{}\Mac{figinit}$\{$pt$\}$\cr
\+{}\Mac{figpt} 1:(-30,-10)\Mac{figpt} 2:(0,-70)\Mac{figpt} 3:(60,-60)\Mac{figpt} 4:(80,10)\cr
\+{}\Mac{figpt} 0:(80,100) \% Only used to define the line (0,4) for the symmetry\cr
\+{}\Mac{figptssym} 5=3,2,1/0,4/\Mac{figptsrot} 8=6,5,4/7,180/\cr
\+{}\% 2. Creation of the postscript file\cr
\+{}\Mac{psbeginfig}$\{$\char`\\MyPSfile$\}$\cr
\+{}\Mac{psBezier} 3[1,2,3,4,5,6,7,8,9,10]\cr
\+{}\Mac{pssetdash}$\{$8$\}$ \Mac{psline}[1,2,3,4,5,6,7,8,9,10]\cr
\+{}\Mac{psendfig}\cr
\+{}\% 3. Writing text on the figure\cr
\+{}\Mac{figvisu}$\{$\char`\\demo$\}$$\{$Figure \char`\\the\char`\\Figno$\}$$\{$\cr
\+{}\Mac{figinsert}$\{$\char`\\MyPSfile$\}$\Mac{figsetptname}$\{$\$P\_$\{$\#1$\}$\$$\}$\cr
\+{}\Mac{figwritew} 1:(2)\Mac{figwritesw} 2,5:(1)\Mac{figwritese} 3,6:(1)\Mac{figwriten} 4:(2)\cr
\+{}\Mac{figwritee} 7,10:(2)\Mac{figwritenw} 8:(1)\Mac{figwritene} 9:(1)\cr
\+{}$\}$\cr
\+{}\char`\\centerline$\{$\char`\\box\char`\\demo$\}$\cr
}$$
\vfill\eject
It turns out that the macro \Mac{psBezier} is not very handy when one wants to draw a
smooth curve. So, a new macro has been designed for that purpose whose prototype is
\smallskip
\centerline{\tt\Mac{pscurve} [Pt0,Pt1,... ,PtN,PtN+1]}
\smallskip
It draws a $C^1$ curve that \eword{interpolates} the points $P_1,
P_2,\dots ,P_N$. The direction of the tangent at $P_1$ is given by the vector
\vect{P_0P_2}, and at $P_N$ by the vector \vect{P_{N-1}P_{N+1}}. The curve consists of
$N-1$ B\'ezier cubic arcs. To get a $C^1$ closed curve, just let the last three points
be the same as the first three ones.
\smallskip
The shape of the curve can be modified by a ``roundness" parameter whose value is set
by the macro {\tt \Mac{pssetroundness}$\{$Roundness$\}$}.
Its argument {\tt Roundness} is a real
number whose value is to be chosen between $0.15$ and $0.3$ in order to obtain the best
results. Its default value is \defaultroundness, which can be set by saying
{\tt \Mac{pssetroundness}$\{$\Mac{defaultroundness}$\}$}.
The aim is to influence the roundness of the curve, and even its
smoothness, since setting this value to $0$ produces the polygonal line defined by the
given points. Other values greater than $0.5$ (or negative values) lead to (generally
unwanted) strange shapes.
\medskip\noindent
Remarks:
\item{$\bullet$} Wittingly, no check is performed on the ``roundness" parameter,
so a large value may produce an arithmetic error.
\item{$\bullet$} The macro \Mac{pscurve} must obviously be called with at least
$4$ points. This is not checked.
\item{$\bullet$} Two successive points may coincide: this feature may probably only be
useful at the end points of the line by setting, for example, $P_0=P_1$ so that the line
starts at $P_1$ with the tangent \vect{P_1P_2}.
\medskip\noindent
The two following figures show what can be done with this macro.
\bigskip
\NewPSfile{\MyPSfile}
% Left-hand side figure
% 1. Definition of characteristic points (Interpolation points)
\figinit{pt}
\figpt 1:(-100, 20)\figpt 2:(-30, 70)
\figpt 3:(50, 50)\figpt 4:(80,0)
\figpt 5:(30, -40)\figpt 6:(-30, 0)\figpt 7:(-80, -20)
% 2. Creation of the postscript file
\psbeginfig{\MyPSfile}
\pscurve[1,2,3,4,5,6,7,1,2,3]
\psendfig
% 3. Writing text on the figure
\figvisu{\demo}{Figure \the\Figno}{
\figinsert{\MyPSfile}
\figsetmark{$\Bullet$}
\figwritep[1,2,3,4,5,6,7]
}
\NewPSfile{\MyPSfile}
% Right-hand side figure
% 1. Definition of characteristic points (Interpolation points)
\figinit{pt}
\figpt 1:(0, 30)\figpt 2:(0, -30)
\figpt 10:(-50, 50)\figpt 11:(-50,-50)\figpt 12:(-90,0)
\figpt 20:(50, 60)\figpt 21:(50,-60)\figpt 22:(90,0)
% 2. Creation of the postscript file
\psbeginfig{\MyPSfile}
\pssetroundness{0.23} % More roundness
\pssetfillmode{yes}\pssetgray{0.5}
\pscurve[11,2,1,10]\pscurve[20,1,2,21] % Better to do this first
\pssetfillmode{no}\pssetgray{0}
\pscurve[11,2,1,10,12,11,2,1]\pscurve[20,1,2,21,22,20,1,2]
\psendfig
% 3. Writing text on the figure
%\newbox\demotwo
\figvisu{\demotwo}{Figure \the\Figno}{
\figinsert{\MyPSfile}
}
\centerline{\box\demo\hfil\box\demotwo}
\bigskip
The text of the program that produces the left-hand side figure is:
$$\vbox{\fmtprog{0pt}
\+{}\% Left-hand side figure\cr
\+{}\% 1. Definition of characteristic points (Interpolation points)\cr
\+{}\Mac{figinit}$\{$pt$\}$\cr
\+{}\Mac{figpt} 1:(-100, 20)\Mac{figpt} 2:(-30, 70)\cr
\+{}\Mac{figpt} 3:(50, 50)\Mac{figpt} 4:(80,0)\cr
\+{}\Mac{figpt} 5:(30, -40)\Mac{figpt} 6:(-30, 0)\Mac{figpt} 7:(-80, -20)\cr
\+{}\% 2. Creation of the postscript file\cr
\+{}\Mac{psbeginfig}$\{$\char`\\MyPSfile$\}$\cr
\+{}\Mac{pscurve}[1,2,3,4,5,6,7,1,2,3]\cr
\+{}\Mac{psendfig}\cr
\+{}\% 3. Writing text on the figure\cr
\+{}\Mac{figvisu}$\{$\char`\\demo$\}$$\{$Figure \char`\\the\char`\\Figno$\}$$\{$\cr
\+{}\Mac{figinsert}$\{$\char`\\MyPSfile$\}$\cr
\+{}\Mac{figsetmark}$\{$\$\char`\\Bullet\$$\}$\cr
\+{}\Mac{figwritep}[1,2,3,4,5,6,7]\cr
\+{}$\}$\cr
\+{}\cr
}$$
\smallskip
The text of the program that produces the right-hand side figure is given below.
Here, we are anticipating on some features described in the {\bf \Area} section.
$$\vbox{\fmtprog{0pt}
\+{}\% Right-hand side figure\cr
\+{}\% 1. Definition of characteristic points (Interpolation points)\cr
\+{}\Mac{figinit}$\{$pt$\}$\cr
\+{}\Mac{figpt} 1:(0, 30)\Mac{figpt} 2:(0, -30)\cr
\+{}\Mac{figpt} 10:(-50, 50)\Mac{figpt} 11:(-50,-50)\Mac{figpt} 12:(-90,0)\cr
\+{}\Mac{figpt} 20:(50, 60)\Mac{figpt} 21:(50,-60)\Mac{figpt} 22:(90,0)\cr
\+{}\% 2. Creation of the postscript file\cr
\+{}\Mac{psbeginfig}$\{$\char`\\MyPSfile$\}$\cr
\+{}\Mac{pssetroundness}$\{$0.23$\}$ \% More roundness\cr
\+{}\Mac{pssetfillmode}$\{$yes$\}$\Mac{pssetgray}$\{$0.5$\}$\cr
\+{}\Mac{pscurve}[11,2,1,10]\Mac{pscurve}[20,1,2,21] \% Better to do this first\cr
\+{}\Mac{pssetfillmode}$\{$no$\}$\Mac{pssetgray}$\{$0$\}$\cr
\+{}\Mac{pscurve}[11,2,1,10,12,11,2,1]\Mac{pscurve}[20,1,2,21,22,20,1,2]\cr
\+{}\Mac{psendfig}\cr
\+{}\% 3. Writing text on the figure\cr
\+{}\char`\\newbox\char`\\demotwo\cr
\+{}\Mac{figvisu}$\{$\char`\\demotwo$\}$$\{$Figure \char`\\the\char`\\Figno$\}$$\{$\cr
\+{}\Mac{figinsert}$\{$\char`\\MyPSfile$\}$\cr
\+{}$\}$\cr
}$$
%%%%%%%%%%%%%%%%%%%%%%%%%%%%%%%%%%%%%%%
\subsect{\Arrow}
The two main macros are:\par
\centerline{\vbox{\hbox{\tt\Mac{psarrow} [Pt1,Pt2]}
\hbox{\tt\Mac{psarrowhead} [Pt1,Pt2]}
}}
\smallskip
The first one draws a straight arrow from {\tt Pt1} to {\tt Pt2}. The arrow-head
is drawn according to the {\tt \Mac{psarrowhead}} macro settings. This macro can be
used whenever it is needed to draw an arrow-head alone (the body of the arrow is then
not drawn).
\smallskip
The appearance of the arrow-head depends on four attributes: the opening angle, the
length, the drawing mode and the position. Each of them can be modified by specific
macros given below. To reset at once all the attributes to their default values,
the best is to say just \Mac{psresetarrowhead}.
\smallskip\noindent
$\bullet$ opening angle: {\tt \Mac{pssetarrowheadangle}$\{$Angle$\}$}
\item{} sets the arrow-head half-angle to {\tt Angle} (in degrees). The default value
is \defaultarrowheadangle\ degrees, that can be set by saying
{\tt \Mac{pssetarrowheadangle}$\{$\Mac{defaultarrowheadangle}$\}$}.
\smallskip\noindent
$\bullet$ length: Precisely, we speak of the length of each of the two edges of the
arrow-head. This attribute can be set by one of the two following macros which
are mutually exclusive ; the default is to use the length.\smallskip
\leftline{\tt \Mac{pssetarrowheadlength}$\{$Length$\}$}
\item{} sets the arrow-head length to {\tt Length} (in user coordinates).
\figinit{pt}\psresetarrowhead
The default value is equivalent to a \defaultarrowheadlength$\,$pt length, that can be
set by saying {\tt \Mac{pssetarrowheadlength}$\{$\Mac{defaultarrowheadlength}$\}$}.
\smallskip
\leftline{\tt \Mac{pssetarrowheadratio}$\{$Ratio$\}$}
\item{} sets the arrow-head ratio to {\tt Ratio}, real number usually in (0,1).
The default value is \defaultarrowheadratio, that can be set by saying
{\tt \Mac{pssetarrowheadratio}$\{$\Mac{defaultarrowheadratio}$\}$}. The arrow-head
length is then equal to {\tt Ratio} times the length of the body of the arrow.
\smallskip\noindent
$\bullet$ drawing mode: {\tt \Mac{pssetarrowheadfill}$\{${\it switch}$\}$}
\item{} sets the arrow-head filling switch to {\tt yes} or {\tt no}. The default
value is {\tt no}. If {\tt yes} is chosen, the effect is to fill the triangle with
the current color.
\smallskip\noindent
$\bullet$ position: {\tt \Mac{pssetarrowheadout}$\{${\it switch}$\}$}
\item{} sets the arrow-head "outside" switch to {\tt yes} or {\tt no}. The default
value is {\tt no}. If {\tt yes} is chosen, the effect is to draw the arrow-head
outside the segment [{\tt Pt1}, {\tt Pt2}].
\smallskip
\NewPSfile{\MyPSfile}
Figure \the\Figno\ shows how to use these macros.
\medbreak
% 1. Definition of characteristic points
\figinit{cm}
\figpt 0:(-1,0)\figpt 1:(1,0)\figpt 2:(1,1)\figpt 3:(1,-1)
\figvectC 20(0.5,0)\def\VTrans{20}
\figptstra 4=1,2,3/1,\VTrans/
\figptcopy10:/0/
\def\LTrans{10}\figpttra 11:=0/\LTrans,\VTrans/
% 2. Creation of the postscript file
\psbeginfig{\MyPSfile}
\psline[2,3]\psline[5,6]
\pssetarrowheadratio{0.2}\psarrow [0,1]
\pssetarrowheadout{yes}\psarrowhead [1,4]
\figptstra 0=0,1,2,3,4,5,6/\LTrans,\VTrans/\figptstra 4=4,5,6/1,\VTrans/
\psline[2,3]\psline[5,6]
\psresetarrowhead\pssetarrowheadfill{yes}
\pssetarrowheadlength{0.2}\psarrow[0,1]
\pssetarrowheadout{yes}\psarrowhead[1,4]
\psendfig
% 3. Writing text on the figure
\figvisu{\demo}{Figure \the\Figno}{
\figinsert{\MyPSfile}
\figwriten 10:Narrow(0.2)\figwriten 11:Wide(0.2)
}
\centerline{\box\demo}
\medskip
The text of the program that produces it is:
$$\vbox{\fmtprog{0pt}
\+{}\% 1. Definition of characteristic points\cr
\+{}\Mac{figinit}$\{$cm$\}$\cr
\+{}\Mac{figpt} 0:(-1,0)\Mac{figpt} 1:(1,0)\Mac{figpt} 2:(1,1)\Mac{figpt} 3:(1,-1)\cr
\+{}\Mac{figvectC} 20(0.5,0)\char`\\def\char`\\VTrans$\{$20$\}$\cr
\+{}\Mac{figptstra} 4=1,2,3/1,\char`\\VTrans/\cr
\+{}\Mac{figptcopy}10:/0/\cr
\+{}\char`\\def\char`\\LTrans$\{$10$\}$\Mac{figpttra} 11:=0/\char`\\LTrans,\char`\\VTrans/\cr
\+{}\% 2. Creation of the postscript file\cr
\+{}\Mac{psbeginfig}$\{$\char`\\MyPSfile$\}$\cr
\+{}\Mac{psline}[2,3]\Mac{psline}[5,6]\cr
\+{}\Mac{pssetarrowheadratio}$\{$0.2$\}$\Mac{psarrow} [0,1]\cr
\+{}\Mac{pssetarrowheadout}$\{$yes$\}$\Mac{psarrowhead} [1,4]\cr
\+{}\Mac{figptstra} 0=0,1,2,3,4,5,6/\char`\\LTrans,\char`\\VTrans/\Mac{figptstra} 4=4,5,6/1,\char`\\VTrans/\cr
\+{}\Mac{psline}[2,3]\Mac{psline}[5,6]\cr
\+{}\Mac{psresetarrowhead}\Mac{pssetarrowheadfill}$\{$yes$\}$\cr
\+{}\Mac{pssetarrowheadlength}$\{$0.2$\}$\Mac{psarrow}[0,1]\cr
\+{}\Mac{pssetarrowheadout}$\{$yes$\}$\Mac{psarrowhead}[1,4]\cr
\+{}\Mac{psendfig}\cr
\+{}\% 3. Writing text on the figure\cr
\+{}\Mac{figvisu}$\{$\char`\\demo$\}$$\{$Figure \char`\\the\char`\\Figno$\}$$\{$\cr
\+{}\Mac{figinsert}$\{$\char`\\MyPSfile$\}$\cr
\+{}\Mac{figwriten} 10:Narrow(0.2)\Mac{figwriten} 11:Wide(0.2)\cr
\+{}$\}$\cr
\+{}\char`\\centerline$\{$\char`\\box\char`\\demo$\}$\cr
}$$
\par
It is also very convenient to draw circular arrows. Two macros are available
for that:\smallskip
\centerline{\vbox{
\hbox{\tt\Mac{psarrowcirc} Center ; Radius (Ang1,Ang2)}
\hbox{\tt\Mac{psarrowcircP} Center ; Radius [Pt1,Pt2]}
}}
\smallskip
The macro \Mac{psarrowcirc} draws a circular arrow such that the circular arc
is centered at {\tt Center}, has the radius {\tt Radius} and is limited by the
angles {\tt Ang1} and {\tt Ang2} given in degrees. If {\tt Ang2} $>$ {\tt Ang1},
the arrow is drawn counterclockwise, else it is drawn clockwise.
The arrow-head is drawn according to the \Mac{psarrowhead} macro settings.
\smallskip
The macro \Mac{psarrowcircP} is the ``point version" of the previous one. It draws
a circular arrow such that the circular arc is centered at {\tt Center}, has the
radius {\tt |Radius|} and is limited by the two half-lines ({\tt Center}, {\tt Pt1})
and ({\tt Center}, {\tt Pt2}). By default, the arrow is drawn from {\tt Pt1}
towards {\tt Pt2} turning counterclockwise around {\tt Center}. Inserting a minus
sign before the radius reverts the turning direction. In other words, the arrow is
drawn counterclockwise if {\tt Radius} $>$ 0, clockwise if {\tt Radius} $<$ 0.
\smallskip
At last, a free form arrow can be drawn with the help of the macro\smallskip
\centerline{\tt \Mac{psarrowBezier} [Pt1,Pt2,Pt3,Pt4]}
\smallskip\noindent
It draws the arrow that consists of the cubic B\'ezier curve defined by the four
control points {\tt Pt1}, {\tt Pt2}, {\tt Pt3} and {\tt Pt4}, and the
arrow-head at point {\tt Pt4}.
\smallskip
\NewPSfile{\MyPSfile}
Figure \the\Figno\ shows how to use these macros.
\bigskip
% 1. Definition of characteristic points
\figinit{cm}
\def\rotangle{38}
\figpt 1:(-3,0)\figpt 2:(3,0)\figpt 0:(0,0)
\figptsrot 3=1,2/0,\rotangle/\figpt 5:(-3,1)
% 2. Creation of the postscript file
\psbeginfig{\MyPSfile}
\psline[1,2]\psline[3,4]
\psarrowcirc 0;1.7(0,\rotangle)\pssetarrowheadangle{30}
\psarrowcircP 0;-1[2,3]\pssetarrowheadangle{\defaultarrowheadangle}
\pssetwidth{2}\pssetarrowheadfill{yes}
\psarrowcircP 0;1[4,1]\psarrowcircP 0;-2[3,1]
\pssetwidth{\defaultwidth}\pssetarrowheadlength{0.3}
\psarrowcircP 0;-1.4[2,3]\psarrowBezier[4,5,1,3]
\psendfig
% 3. Writing text on the figure
\figvisu{\demo}{Figure \the\Figno}{
\figinsert{\MyPSfile}
\figwritew 1:(0.2)\figwritee 2:(0.2)\figwritesw 3:(0.2)\figwritene 4:(0.2)}
\centerline{\box\demo}
\bigskip
The text of the program that produces it is:
$$\vbox{\fmtprog{0pt}
\+{}\% 1. Definition of characteristic points\cr
\+{}\Mac{figinit}$\{$cm$\}$\cr
\+{}\char`\\def\char`\\rotangle$\{$38$\}$\cr
\+{}\Mac{figpt} 1:(-3,0)\Mac{figpt} 2:(3,0)\Mac{figpt} 0:(0,0)\cr
\+{}\Mac{figptsrot} 3=1,2/0,\char`\\rotangle/\Mac{figpt} 5:(-3,1)\cr
\+{}\% 2. Creation of the postscript file\cr
\+{}\Mac{psbeginfig}$\{$\char`\\MyPSfile$\}$\cr
\+{}\Mac{psline}[1,2]\Mac{psline}[3,4]\cr
\+{}\Mac{psarrowcirc} 0;1.7(0,\char`\\rotangle)\Mac{pssetarrowheadangle}$\{$30$\}$\cr
\+{}\Mac{psarrowcircP} 0;-1[2,3]\Mac{pssetarrowheadangle}$\{$\Mac{defaultarrowheadangle}$\}$\cr
\+{}\Mac{pssetwidth}$\{$2$\}$\Mac{pssetarrowheadfill}$\{$yes$\}$\cr
\+{}\Mac{psarrowcircP} 0;1[4,1]\Mac{psarrowcircP} 0;-2[3,1]\cr
\+{}\Mac{pssetwidth}$\{$\Mac{defaultwidth}$\}$\Mac{pssetarrowheadlength}$\{$0.3$\}$\cr
\+{}\Mac{psarrowcircP} 0;-1.4[2,3]\Mac{psarrowBezier}[4,5,1,3]\cr
\+{}\Mac{psendfig}\cr
\+{}\% 3. Writing text on the figure\cr
\+{}\Mac{figvisu}$\{$\char`\\demo$\}$$\{$Figure \char`\\the\char`\\Figno$\}$$\{$\cr
\+{}\Mac{figinsert}$\{$\char`\\MyPSfile$\}$\cr
\+{}\Mac{figwritew} 1:(0.2)\Mac{figwritee} 2:(0.2)\Mac{figwritesw} 3:(0.2)\Mac{figwritene} 4:(0.2)$\}$\cr
\+{}\char`\\centerline$\{$\char`\\box\char`\\demo$\}$\cr
}$$
%%%%%%%%%%%%%%%%%%%%%%%%%%%%%%%%%%%%%%%
\vskip-\bigskipamount
\subsect{\Color}
Highlighting some part of a figure can be achieved by changing the line style, but
also, often more efficiently, by adding color. This can be done while creating the
figure with one of the following macros:\smallskip
\centerline{\vbox{%
\hbox{\tt\Mac{pssetgray}$\{$GrayLevel$\}$}
\hbox{\tt\Mac{pssetcmyk}$\{$cmykColor$\}$}
\hbox{\tt\Mac{pssetrgb}$\{$rgbColor$\}$}
}}
\smallskip
Once a color is set, every subsequent graphic will be drawn in this color,
until another color setting occurs. This apply to lines as well as to filled areas.
The default color is black.
\medskip
The macro \Mac{pssetgray} sets a \eword{gray} shade whose intensity is controlled by
a real number {\tt GrayLevel} taking its value between 0 and 1, with
0 corresponding to black, 1 corresponding to white and intermediate values
corresponding to intermediate shades of gray.
\smallskip
Each of the two macros, \Mac{pssetcmyk} or \Mac{pssetrgb}, sets the color to be used,
referring to the \eword{cmyk} or to the \eword{rgb} color model. They both have one
argument that can be :\par
\item{-} either numerical values (or coordinates) corresponding to the
color definition in the color model (note there are 4 coordinates for \eword{cmyk}
and 3 for \eword{rgb}),
\item{-} or a color name, which is a macro containing the color definition.
\par\noindent
For example, one obtains the same result by saying \Mac{pssetrgb}\Mac{Redrgb} or
\Mac{pssetrgb}{\tt $\{$1 0 0$\}$}, provided the macro \Mac{Redrgb} has been defined.
Note that the coordinates are separated with spaces and range between 0 and 1.
The basic colors have been predefined in each color model, along with some other ones :\par
\item{-} in \eword{cmyk} : \Mac{Blackcmyk}, \Mac{Whitecmyk}, \Mac{Cyancmyk},
\Mac{Magentacmyk}, \Mac{Yellowcmyk}, \Mac{Redcmyk}, \Mac{Greencmyk}, \Mac{Bluecmyk},
\Mac{Graycmyk}, \Mac{BrickRedcmyk}, \Mac{Browncmyk}, \Mac{ForestGreencmyk},
\Mac{Goldenrodcmyk}, \Mac{Marooncmyk}, \Mac{Orangecmyk}, \Mac{Purplecmyk},
\Mac{RoyalBluecmyk}, \Mac{Violetcmyk}, 
\item{-} in \eword{rgb} :\Mac{Blackrgb}, \Mac{Whitergb}, \Mac{Redrgb}, \Mac{Greenrgb}, 
\Mac{Bluergb}, \Mac{Cyanrgb}, \Mac{Magentargb}, \Mac{Yellowrgb}, \Mac{Grayrgb}, 
\Mac{Chocolatergb}, \Mac{DarkGoldenrodrgb}, \Mac{DarkOrangergb}, \Mac{Firebrickrgb}, 
\Mac{ForestGreenrgb}, \Mac{Goldrgb}, \Mac{HotPinkrgb}, \Mac{Maroonrgb}, \Mac{Pinkrgb}, 
\Mac{RoyalBluergb}.
\smallskip
\rightitem{Remark:}{
The color coordinates are taken from different sources and the associated
names have been preserved. So a same color may render differently in the two color
models: this is the case for {\tt ForestGreen} for example.}
\smallskip\noindent
To define a color, one has to associate its coordinates with the name of a macro.
For example, orange is defined in \eword{cmyk} model by saying 
$$\vbox{\fmtprog{0pt}
\+{}\char`\\def\char`\\Orangecmyk$\{$0 0.61 0.87 0$\}$\cr
}$$
so it is easy to enrich this small list.
Obviously, \eword{cmyk} colors must be used with \Mac{pssetcmyk} and \eword{rgb}
colors must be used with \Mac{pssetrgb} ; no control is performed. 
\bigskip
Finally, as already mentionned in the section {\bf \LineWS}, some high level macros
use secondary settings, including color. Setting a secondary color in \eword{gray}
tone, \eword{cmyk} or \eword{rgb} color code can be achieved using one of the
following macros:
\smallskip
\centerline{\vbox{%
\hbox{{\tt \Mac{pssetsecondcmyk}$\{$\char`\\ColorName$\}$}
 or {\tt \Mac{pssetsecondcmyk}$\{$cmyk color code$\}$}}
\hbox{{\tt \Mac{pssetsecondrgb}$\{$\char`\\ColorName$\}$}
 or {\tt \Mac{pssetsecondrgb}$\{$rgb color code$\}$}}
\hbox{\tt \Mac{pssetsecondgray}$\{$GrayLevel$\}$}
}}
%%%%%%%%%%%%%%%%%%%%%%%%%%%%%%%%%%%%%%%
\subsect{\Area}
Except arrow drawing macros, every drawing macro dealing with lines, arcs and curves
presented so far can fill the area delimited by the line, the arc or the curve with
colored ink, according to the current setting made with \Mac{pssetgray}, \Mac{pssetcmyk}
or \Mac{pssetrgb}. The filling algorithm is governed by postscript own rules. In particular,
for circular or elliptic arcs, the area is limited by the arc and the corresponding chord.
\smallskip
The macro {\tt\Mac{pssetfillmode}$\{${\it switch}$\}$} allows to activate the filling
algorithm if {\it switch} equals {\tt yes} or not if {\it switch} equals {\tt no}.
The default is {\tt no}.
\smallskip
To show how it works, we start from the two figures \FigArc\ and \FigCur\ and
we replace the postscript file definition by
\medskip
{\leftskip1cm\leavevmode
\vbox{\fmtprog{0pt}
\+{}\Mac{psbeginfig}$\{$\char`\\MyPSfile$\}$\cr
\+{}\Mac{pssetfillmode}$\{$yes$\}$\cr
\+{}\Mac{psarccirc} 1;\char`\\vaxis(\char`\\startangle,\char`\\stopangle) \% Use default color\cr
\+{}\Mac{pssetrgb}\char`\\Redrgb\Mac{psarcell} 3 ; \char`\\haxis,\char`\\vaxis(\char`\\startangle,\char`\\stopangle, \char`\\inclin)\cr
\+{}\Mac{pssetrgb}\char`\\Bluergb\Mac{psarcellPA} 3,4,5(-180,-90)\cr
\+{}\Mac{psendfig}\cr
}
\medskip}
\noindent
in the first one and by
\medskip
{\leftskip1cm\leavevmode
\vbox{\fmtprog{0pt}
\+{}\Mac{psbeginfig}$\{$\char`\\MyPSfile$\}$\cr
\+{}\Mac{pssetfillmode}$\{$yes$\}$\Mac{pssetgray}$\{$0.7$\}$\cr
\+{}\Mac{psBezier} 3[1,2,3,4,5,6,7,8,9,10]\cr
\+{}\Mac{psendfig}\cr
}
\medskip}
\noindent
in the second one. We then get the two new figures:
\bigskip
\NewPSfile{\MyPSfile}
% 1. Definition of characteristic points
\figinit{0.9pt}
\def\haxis{100}\def\vaxis{40}
\def\startangle{-20}\def\stopangle{90}\def\inclin{45}
\figpt 1:(-10,10)\figpttraC 3:=1/200,0/
\figptell 4:: 3 ; \haxis,\vaxis( 0, \inclin) % End point of major axis
\figptell 5:: 3 ; \haxis,\vaxis(90, \inclin) % End point of minor axis
% 2. Creation of the postscript file
\psbeginfig{\MyPSfile}
\pssetfillmode{yes}
\psarccirc 1;\vaxis(\startangle,\stopangle) % Use default color
\pssetrgb\Redrgb\psarcell 3 ; \haxis,\vaxis(\startangle,\stopangle, \inclin)
\pssetrgb\Bluergb\psarcellPA 3,4,5(-180,-90)
\psendfig
% 3. Writing text on the figure
\figvisu{\demo}{Figure \the\Figno}{
\figinsert{\MyPSfile}
\figsetmark{.}\figwriten 1:(2)\figwriten 3:(2)
}
\NewPSfile{\MyPSfile}
% 1. Definition of characteristic points
\figinit{0.5pt}
\figpt 1:(-30,-10)\figpt 2:(0,-70)\figpt 3:(60,-60)\figpt 4:(80,10)
\figpt 0:(80,100)
\figptssym 5=3,2,1/0,4/\figptsrot 8=6,5,4/7,180/
% 2. Creation of the postscript file
\psbeginfig{\MyPSfile}
\pssetfillmode{yes}\pssetgray{0.7}
\psBezier 3[1,2,3,4,5,6,7,8,9,10]
\psendfig
% 3. Writing text on the figure
\figvisu{\demotwo}{Figure \the\Figno}{
\figinsert{\MyPSfile}\figsetptname{$P_{#1}$}
\figwritew 1:(2)\figwriten 4:(2)\figwritee 10:(2)
}
\centerline{\box\demo\hfil\box\demotwo}
\bigskip
Note that to shorten the presentation some modifications have been made: 1) the dashed
lines have been removed, 2) the point $A_2$ has been removed along with one of the arcs
on the ellipse, 3) the dimensions of the figures have been shrinked to 90\% and 50\%
respectively.
\smallskip
To put the two figures on the same line, we just created a new box \CtrlSq{demotwo} for
the curve, gave it as first argument to \Mac{figvisu}, and made the classical assembly:
$$\vbox{\fmtprog{0pt}
\+{}\char`\\centerline$\{$\char`\\box\char`\\demo\char`\\hfil\char`\\box\char`\\demotwo$\}$\cr
}$$
\par\noindent
Remarks:
\item{$\bullet$} When the filling mode is active, the macro \Mac{psline} give the same
result if the path is closed or not.
\item{$\bullet$} Filling an area with white ink can be used to erase some existing
drawing and then write some text at this ``cleaned" place. Note that this feature may
not be handled properly by some screen previewers although the result on the paper is
correct.
\bigskip
\NewPSfile{\MyPSfile}
The following example is a reproduction of the well known RGB color disks shown
on figure \the\Figno.
The data are the centers $(A, B, C)$ of the three disks and the radius. The points
$(A, B, C)$ are the vertices of an equilateral triangle, $A$ and $B$ lying on the
X-axis. In order to fill the different areas, we need to compute their limits given
by the intersections of two circles, which is required by the macro \Mac{psarccircP}.
\medskip
The program that produces the figure \the\Figno\ is given below it. Note that the
intersections of two disks are filled with two symmetric areas. It may also be useful
to specify that since postscript images are opaque, the order in which the different
areas are painted is essential to get the expected result.
\topinsert
%
% 1. Definition of characteristic points
\figinit{0.6cm}
\def\xB{2} % Abscissa of B
\def\R{3}  % Radius of the 3 circles, r < R < r*sqrt(3),
%           r radius of the circumscribed circle (r=2*\xB/sqrt(3)=2.3094)
\figpt 1:A(-\xB, 0) \figpt 2:B(\xB, 0) \figptrot 3:C=2/1,-60/
\figptsintercirc 4[1,\R;2,\R] \figptsintercirc 14[3,\R;1,\R]
\figptsintercirc 24[2,\R;3,\R]
% 2. Creation of the postscript file
\psbeginfig{\MyPSfile}
\pssetfillmode{yes}
\pssetrgb\Redrgb  \pscirc 1(\R)
\pssetrgb\Greenrgb\pscirc 2(\R)
\pssetrgb\Bluergb \pscirc 3(\R)
\pssetrgb\Yellowrgb \psarccircP1;\R[4,5]  \psarccircP2;\R[5,4]
\pssetrgb\Magentargb\psarccircP1;\R[15,14]\psarccircP3;\R[14,15]
\pssetrgb\Cyanrgb   \psarccircP3;\R[25,24]\psarccircP2;\R[24,25]
\pssetrgb\Whitergb
\psarccircP2;\R[24,4]\psarccircP1;\R[4,14]\psarccircP3;\R[14,24]
\psline[4,14,24]
\psendfig
% 3. Inserting the figure
\figvisu{\demo}{Figure \the\Figno}{\figinsert{\MyPSfile}}
\centerline{\box\demo}
$$\vbox{\fmtprog{0pt}
\+{}\% 1. Definition of characteristic points\cr
\+{}\Mac{figinit}$\{$0.6cm$\}$\cr
\+{}\char`\\def\char`\\xB$\{$2$\}$ \% Abscissa of B\cr
\+{}\char`\\def\char`\\R$\{$3$\}$  \% Radius of the 3 circles, r < R < r*sqrt(3),\cr
\+{}\%           r radius of the circumscribed circle (r=2*\char`\\xB/sqrt(3)=2.3094)\cr
\+{}\Mac{figpt} 1:A(-\char`\\xB, 0) \Mac{figpt} 2:B(\char`\\xB, 0) \Mac{figptrot} 3:C=2/1,-60/\cr
\+{}\Mac{figptsintercirc} 4[1,\char`\\R;2,\char`\\R] \Mac{figptsintercirc} 14[3,\char`\\R;1,\char`\\R]\cr
\+{}\Mac{figptsintercirc} 24[2,\char`\\R;3,\char`\\R]\cr
\+{}\% 2. Creation of the postscript file\cr
\+{}\Mac{psbeginfig}$\{$\char`\\MyPSfile$\}$\cr
\+{}\Mac{pssetfillmode}$\{$yes$\}$\cr
\+{}\Mac{pssetrgb}\char`\\Redrgb  \Mac{pscirc} 1(\char`\\R)\cr
\+{}\Mac{pssetrgb}\char`\\Greenrgb\Mac{pscirc} 2(\char`\\R)\cr
\+{}\Mac{pssetrgb}\char`\\Bluergb \Mac{pscirc} 3(\char`\\R)\cr
\+{}\Mac{pssetrgb}\char`\\Yellowrgb \Mac{psarccircP}1;\char`\\R[4,5]  \Mac{psarccircP}2;\char`\\R[5,4]\cr
\+{}\Mac{pssetrgb}\char`\\Magentargb\Mac{psarccircP}1;\char`\\R[15,14]\Mac{psarccircP}3;\char`\\R[14,15]\cr
\+{}\Mac{pssetrgb}\char`\\Cyanrgb   \Mac{psarccircP}3;\char`\\R[25,24]\Mac{psarccircP}2;\char`\\R[24,25]\cr
\+{}\Mac{pssetrgb}\char`\\Whitergb\cr
\+{}\Mac{psarccircP}2;\char`\\R[24,4]\Mac{psarccircP}1;\char`\\R[4,14]\Mac{psarccircP}3;\char`\\R[14,24]\cr
\+{}\Mac{psline}[4,14,24]\cr
\+{}\Mac{psendfig}\cr
\+{}\% 3. Inserting the figure\cr
\+{}\Mac{figvisu}$\{$\char`\\demo$\}$$\{$Figure \char`\\the\char`\\Figno$\}$$\{$\Mac{figinsert}$\{$\char`\\MyPSfile$\}$$\}$\cr
\+{}\char`\\centerline$\{$\char`\\box\char`\\demo$\}$\cr
}$$
\endinsert
%%%%%%%%%%%%%%%%%%%%%%%%%%%%%%%%%%%%%%%
\subsect{\Triangle}
We now present a set of macros related to the geometry of the triangle.
\medskip
The macro, whose prototype is\smallskip
\centerline{\tt \Mac{psaltitude} Dim [Pt1,Pt2,Pt3]}
\smallskip\noindent
builds the altitude drawn from {\tt Pt1} in the triangle
({\tt Pt1}, {\tt Pt2}, {\tt Pt3}). {\tt Dim} is the dimension of the square
at the end point of the altitude over the segment [{\tt Pt2},~{\tt Pt3}].
If this point is outside the segment [{\tt Pt2},~{\tt Pt3}], a line is
drawn to support the square. The attributes of this line are controlled by
the secondary settings described in the {\bf \LineWS} section: the line
width and style can be modified with the macros \Mac{pssetsecondwidth} and
\Mac{pssetseconddash}, its color can be modified with one the macros
\Mac{pssetsecondgray}, \Mac{pssetsecondcmyk} or \Mac{pssetsecondrgb}.
\NewPSfile{\MyPSfile}
On Figure \the\Figno, we can see the two altitudes drawn from $P_2$ and $P_3$, with
respective end points $H_2$ and $H_3$.
\medskip
The macro, whose prototype is\smallskip
\centerline{\tt \Mac{psnormal} Length,Lambda [Pt1,Pt2]}
\smallskip\noindent
draws the so-called exterior normal, i.e. the vector of length {\tt Length} which
is normal to the segment $[{\tt Pt1}, {\tt Pt2}]=[P_1, P_2]$.
The direction of the vector is defined by saying that if one goes along the segment
from $P_1$ towards $P_2$, then the normal vector is directed towards
the righthand side. The word ``exterior" implicitly means exterior to a domain:
if the segment is part of the boundary of a polygonal domain, the domain is on the
left when one goes along the boundary from $P_1$ towards $P_2$.
The origin of the vector is the barycenter of $\{(P_1 ;\lambda), (P_2 ; 1-\lambda)\}$
where $\lambda$ is a real number whose value is given by {\tt Lambda}. Since
this macro draws an arrow, we have to use the macros presented in the section
{\bf\Arrow} if we want to modify the appearance of the arrow-head.
\medskip
Generally used in connection with \Mac{psnormal}, the macro\par
\centerline{\tt \Mac{figptendnormal} NewPt :Text: Length,Lambda [Pt1,Pt2]}
\noindent computes the end point {\tt NewPt} of the ``exterior normal" to the
segment $[{\tt Pt1}, {\tt Pt2}]$. The last four arguments have the same meaning
as those of the macro \Mac{psnormal}.
On Figure \the\Figno, we can see the normal vector $\vec{n}_2$ to the segment
$[P_1, P_3]$.
\medskip
The text of the program that produces Figure \the\Figno\ is:
$$\vbox{\fmtprog{0pt}
\+{}\% 1. Definition of characteristic points\cr
\+{}\Mac{figinit}$\{$0.9pt$\}$\cr
\+{}\Mac{figpt} 1:(-30,0)\cr
\+{}\Mac{figpt} 2:(170,60)\cr
\+{}\Mac{figpt} 3:(60,-40)\cr
\+{}\Mac{figptorthoprojline} 12:\$H\_3\$=3/1,2/\cr
\+{}\Mac{figptorthoprojline} 13:\$H\_2\$=2/1,3/\cr
\+{}\% 2. Creation of the postscript file\cr
\+{}\Mac{psbeginfig}$\{$\char`\\MyPSfile$\}$\cr
\+{}\Mac{psline}[1,2,3,1]\cr
\+{}\Mac{psaltitude} 5[3,1,2]\Mac{psaltitude} 5[2,1,3]\cr
\+{}\Mac{pssetarrowheadratio}$\{$0.2$\}$\Mac{psnormal} 20,0.5 [1,3]\cr
\+{}\Mac{psarrowcircP} 1;20 [3,2]\cr
\+{}\Mac{psendfig}\cr
\+{}\% 3. Writing text on the figure\cr
\+{}\Mac{figvisu}$\{$\char`\\demo$\}$$\{$Figure \char`\\the\char`\\Figno$\}$$\{$\cr
\+{}\Mac{figinsert}$\{$\char`\\MyPSfile$\}$\cr
\+{}\Mac{figsetptname}$\{$\$P\_$\{$\#1$\}$\$$\}$\cr
\+{}\Mac{figwriten} 1,2,12:(4)\cr
\+{}\Mac{figwrites} 3,13:(4)\cr
\+{}\Mac{figptendnormal} 4 :: 20,0.5 [1,3]\Mac{figwritew} 4:\$\char`\\vec$\{$n$\}$\_2\$(4)\cr
\+{}$\}$\cr
\+{}\char`\\centerline$\{$\char`\\box\char`\\demo$\}$\cr
}$$
%
\topinsert
% 1. Definition of characteristic points
\figinit{0.9pt}
\figpt 1:(-30,0)
\figpt 2:(170,60)
\figpt 3:(60,-40)
\figptorthoprojline 12:$H_3$=3/1,2/
\figptorthoprojline 13:$H_2$=2/1,3/
% 2. Creation of the postscript file
\psbeginfig{\MyPSfile}
\psline[1,2,3,1]
\psaltitude 5[3,1,2]\psaltitude 5[2,1,3]
\pssetarrowheadratio{0.2}\psnormal 20,0.5 [1,3]
\psarrowcircP 1;20 [3,2]
\psendfig
% 3. Writing text on the figure
\figvisu{\demo}{Figure \the\Figno}{
\figinsert{\MyPSfile}
\figsetptname{$P_{#1}$}
\figwriten 1,2,12:(4)
\figwrites 3,13:(4)
\figptendnormal 4 :: 20,0.5 [1,3]\figwritew 4:$\vec{n}_2$(4)
}
\centerline{\box\demo}
\endinsert
\bigbreak
\NewPSfile{\MyPSfile}
We may be interested in locating the four following characteristic points of a
triangle $(P_1,P_2,P_3)$:\par
- the intersection of the medians, which is the center of gravity,\par
- the intersection of the bisectors, which is the center of the inscribed circle,\par
- the intersection of the mid-perpendiculars, which is the center of the
circumscribed circle,\par
- the intersection of the altitudes, which is the orthocenter.
\smallskip
The computation of the gravity center is straigthforward since it is the barycenter
of $(P_1,P_2,P_3)$ bearing each the coefficient 1/3. The center of the inscribed circle
is also a barycenter of $(P_1,P_2,P_3)$ (bearing the length of the edges) and the last
two points are obtained by lines intersection. This lead to the following macros:
\smallskip
\centerline{\vbox{%
\hbox{\tt\Mac{figptinscribedcenter} NewPt :Text[Pt1,Pt2,Pt3]}
\hbox{\tt\Mac{figptcircumcenter} NewPt :Text[Pt1,Pt2,Pt3]}
\hbox{\tt\Mac{figptorthocenter} NewPt :Text[Pt1,Pt2,Pt3]}
}}
\bigskip\noindent
The figure \the\Figno\ shows these four points along with Euler's line.
The text of the program that produces it is:
$$\vbox{\fmtprog{0pt}
\+{}\% 1. Definition of characteristic points\cr
\+{}\Mac{figinit}$\{$0.85pt$\}$\cr
\+{}\Mac{figpt} 1:(80, 120)\cr
\+{}\Mac{figpt} 2:(-10, -10)\cr
\+{}\Mac{figpt} 3:(250, 50)\cr
\+{}\Mac{figptbary}  5:\$G\$[1,2,3 ; 1,1,1]\cr
\+{}\Mac{figptinscribedcenter} 6:\$\char`\\Omega\_i\$[1,2,3]\cr
\+{}\Mac{figptcircumcenter} 7:\$\char`\\Omega\_c\$[1,2,3]\cr
\+{}\Mac{figptorthocenter} 8:\$O\$[1,2,3]\cr
\+{}\Mac{figptbary}10:[7,8;-11,1]\Mac{figptbary}11:[7,8;1,-11] \% Euler's line\cr
\+{}\% 2. Creation of the postscript file\cr
\+{}\Mac{psbeginfig}$\{$\char`\\MyPSfile$\}$\cr
\+{}\Mac{psline}[1,2,3,1]\Mac{psline}[10,11]\cr
\+{}\Mac{psendfig}\cr
\+{}\% 3. Writing text on the figure\cr
\+{}\Mac{figvisu}$\{$\char`\\demo$\}$$\{$Figure \char`\\the\char`\\Figno: Euler's line in a triangle.$\}$$\{$\cr
\+{}\Mac{figinsert}$\{$\char`\\MyPSfile$\}$\cr
\+{}\Mac{figwritenw} 1:(2)\Mac{figwritew} 2:(2)\Mac{figwritee} 3:(2)\cr
\+{}\Mac{figsetmark}$\{$\$\char`\\times\$$\}$\cr
\+{}\Mac{figwritee} 5,8:(4)\Mac{figwritew} 6,7:(4)\cr
\+{}$\}$\cr
\+{}\char`\\centerline$\{$\char`\\box\char`\\demo$\}$\cr
}$$
\topinsert
% 1. Definition of characteristic points
\figinit{0.85pt}
\figpt 1:(80, 120)
\figpt 2:(-10, -10)
\figpt 3:(250, 50)
\figptbary  5:$G$[1,2,3 ; 1,1,1]
\figptinscribedcenter 6:$\Omega_i$[1,2,3]
\figptcircumcenter 7:$\Omega_c$[1,2,3]
\figptorthocenter 8:$O$[1,2,3]
\figptbary10:[7,8;-11,1]\figptbary11:[7,8;1,-11] % Euler's line
% 2. Creation of the postscript file
\psbeginfig{\MyPSfile}
\psline[1,2,3,1]\psline[10,11]
\psendfig
% 3. Writing text on the figure
\figvisu{\demo}{Figure \the\Figno: Euler's line in a triangle.}{
\figinsert{\MyPSfile}
\figwritenw 1:(2)\figwritew 2:(2)\figwritee 3:(2)
\figsetmark{$\times$}
\figwritee 5,8:(4)\figwritew 6,7:(4)
}
\centerline{\box\demo}
\endinsert
%%%%%%%%%%%%%%%%%%%%%%%%%%%%%%%%%%%%%%%
\subsect{\Grid}
The following macro can be used in the domain of finite element methods
or in approximation theory. Its prototype is\smallskip
\centerline{\tt \Mac{pstrimesh} Type [Pt1,Pt2,Pt3]}
\smallskip\noindent
It draws a triangle of type {\tt Type}, according to finite element terminology,
which is related to the degree of approximation, represented by the so-called B\'ezier
mesh inside the triangle. The attributes of the line used to draw this mesh are
controlled by the secondary settings described in the {\bf \LineWS} section:
the line width and style can be modified with the macros \Mac{pssetsecondwidth} and
\Mac{pssetseconddash}, its color can be modified with one of the macros
\Mac{pssetsecondgray}, \Mac{pssetsecondcmyk} or \Mac{pssetsecondrgb}.
\smallskip
\NewPSfile{\MyPSfile}
Figure \the\Figno\ shows some examples.
\bigskip
% 1. Definition of characteristic points
\figinit{pt}
\figpt 1:(-10, 10)
\figpt 2:(110, 40)
\figpt 3:(50, -40)
\figptbary  11:[1,2,3 ; 1,1,1]\figpttraC 11:=11/0,40/
\figvectC 10(150,0)\figptstra 12=11,12/1,10/
% 2. Creation of the postscript file
\psbeginfig{\MyPSfile}
\pstrimesh 1 [1,2,3]
\figptstra 1=1,2,3,4/1,10/\pstrimesh 2 [1,2,3]
\pssetseconddash{5}
\figptstra 1=1,2,3,4/1,10/\pstrimesh 3 [1,2,3]
\psendfig
% 3. Writing text on the figure
\figvisu{\demo}{Figure \the\Figno}{
\figinsert{\MyPSfile}
\figwritec[11]{Type 1}\figwritec[12]{Type 2}\figwritec[13]{Type 3}
}
\centerline{\box\demo}
\bigskip
The text of the program that produces this figure is:
$$\vbox{\fmtprog{0pt}
\+{}\% 1. Definition of characteristic points\cr
\+{}\Mac{figinit}$\{$pt$\}$\cr
\+{}\Mac{figpt} 1:(-10, 10)\cr
\+{}\Mac{figpt} 2:(110, 40)\cr
\+{}\Mac{figpt} 3:(50, -40)\cr
\+{}\Mac{figptbary}  11:[1,2,3 ; 1,1,1]\Mac{figpttraC} 11:=11/0,40/\cr
\+{}\Mac{figvectC} 10(150,0)\Mac{figptstra} 12=11,12/1,10/\cr
\+{}\% 2. Creation of the postscript file\cr
\+{}\Mac{psbeginfig}$\{$\char`\\MyPSfile$\}$\cr
\+{}\Mac{pstrimesh} 1 [1,2,3]\cr
\+{}\Mac{figptstra} 1=1,2,3,4/1,10/\Mac{pstrimesh} 2 [1,2,3]\cr
\+{}\Mac{pssetseconddash}$\{$5$\}$\cr
\+{}\Mac{figptstra} 1=1,2,3,4/1,10/\Mac{pstrimesh} 3 [1,2,3]\cr
\+{}\Mac{psendfig}\cr
\+{}\% 3. Writing text on the figure\cr
\+{}\Mac{figvisu}$\{$\char`\\demo$\}$$\{$Figure \char`\\the\char`\\Figno$\}$$\{$\cr
\+{}\Mac{figinsert}$\{$\char`\\MyPSfile$\}$\cr
\+{}\Mac{figwritec}$\{$11$\}$$\{$Type 1$\}$\Mac{figwritec}$\{$12$\}$$\{$Type 2$\}$\Mac{figwritec}$\{$13$\}$$\{$Type 3$\}$\cr
\+{}$\}$\cr
\+{}\char`\\centerline$\{$\char`\\box\char`\\demo$\}$\cr
}$$
\par
Drawing grids is also often necessary. The macro\smallskip
\centerline{\tt \Mac{psmesh} N1,N2 [Pt1,Pt2,Pt3,Pt4]}
\smallskip\noindent
draws a mesh of {\tt N1}$\times${\tt N2} intervals in the quadrangle
{\tt (Pt1, Pt2, Pt3, Pt4)}. More precisely, the two segments
{\tt [Pt1, Pt2]} and {\tt [Pt3, Pt4]} are divided into {\tt N1} equal
parts, while the two other segments are divided into {\tt N2} equal parts. The four
points, in the order they are given, are logically assumed to define a convex polygon
(although the macro works as well if this condition is not satisfied).
As in the previous macro, the attributes of the line used to draw the internal mesh
are controlled by the secondary settings described in the {\bf \LineWS} section.
\par
Moreover, with the help of the macro\smallskip
\centerline{\tt \Mac{pssetmeshdiag}$\{$Index$\}$}
\smallskip\noindent
it is possible to draw a triangulation. This macro sets a flag that controls the
drawing of a grid mesh:\par
\item{.} if {\tt Index = ~1}, the SW-NE diagonal is drawn inside each cell,
\item{.} if {\tt Index = -1}, the NW-SE diagonal is drawn inside each cell,
\item{.} otherwise, each cell is empty ; this is the default, which can
be set by saying:\par
\centerline{\tt \Mac{pssetmeshdiag}$\{$\Mac{defaultmeshdiag}$\}$}
\smallskip
\NewPSfile{\MyPSfile}
Notice that the orientation of the diagonal depends on the order of the four given
points. Figure \the\Figno\ shows an example where the four points are the vertices
of a rectangle and of a general quadrangle. It is followed by the program that
produces it.
\medskip
% 1. Definition of characteristic points
\figinit{pt}
\figpt 1:(0,0)
\figpttraC 2:=1/130,0/\figpttraC 3:=2/0,80/\figpttraC 4:=3/-130,0/
\figvectC 0(250,0)
\figptstra 11=2,3,4,1/1,0/\figpttraC 14:=14/30,15/\figpttraC 12:=12/-20,25/
% 2. Creation of the postscript file
\psbeginfig{\MyPSfile}
\psmesh 10,5 [1,2,3,4]
\pssetseconddash{\defaultdash}\pssetmeshdiag{1}\psmesh 10,5 [11,12,13,14]
\psendfig
% 3. Writing text on the figure
\figvisu{\demo}{Figure \the\Figno}{\figinsert{\MyPSfile}
\figwritesw 1:$P_1$(2)\figwritese 2:$P_2$(2)
\figwritene 3:$P_3$(2)\figwritenw 4:$P_4$(2)
\figwritese 11:$P_1$(2)\figwritene 12:$P_2$(2)
\figwritenw 13:$P_3$(2)\figwritesw 14:$P_4$(2)
}
\centerline{\box\demo}
%
\medskip\noindent
The text of the program that produces this figure is:
$$\vbox{\fmtprog{0pt}
\+{}\% 1. Definition of characteristic points\cr
\+{}\Mac{figinit}$\{$pt$\}$\cr
\+{}\Mac{figpt} 1:(0,0)\cr
\+{}\Mac{figpttraC} 2:=1/130,0/\Mac{figpttraC} 3:=2/0,80/\Mac{figpttraC} 4:=3/-130,0/\cr
\+{}\Mac{figvectC} 0(250,0)\cr
\+{}\Mac{figptstra} 11=2,3,4,1/1,0/\Mac{figpttraC} 14:=14/30,15/\Mac{figpttraC} 12:=12/-20,25/\cr
\+{}\% 2. Creation of the postscript file\cr
\+{}\Mac{psbeginfig}$\{$\char`\\MyPSfile$\}$\cr
\+{}\Mac{psmesh} 10,5 [1,2,3,4]\cr
\+{}\Mac{pssetseconddash}$\{$\Mac{defaultdash}$\}$\Mac{pssetmeshdiag}$\{$1$\}$\Mac{psmesh} 10,5 [11,12,13,14]\cr
\+{}\Mac{psendfig}\cr
\+{}\% 3. Writing text on the figure\cr
\+{}\Mac{figvisu}$\{$\char`\\demo$\}$$\{$Figure \char`\\the\char`\\Figno$\}$$\{$\Mac{figinsert}$\{$\char`\\MyPSfile$\}$\cr
\+{}\Mac{figwritesw} 1:\$P\_1\$(2)\Mac{figwritese} 2:\$P\_2\$(2)\cr
\+{}\Mac{figwritene} 3:\$P\_3\$(2)\Mac{figwritenw} 4:\$P\_4\$(2)\cr
\+{}\Mac{figwritese} 11:\$P\_1\$(2)\Mac{figwritene} 12:\$P\_2\$(2)\cr
\+{}\Mac{figwritenw} 13:\$P\_3\$(2)\Mac{figwritesw} 14:\$P\_4\$(2)\cr
\+{}$\}$\cr
\+{}\char`\\centerline$\{$\char`\\box\char`\\demo$\}$\cr
}$$
%%%%%%%%%%%%%%%%%%%%%%%%%%%%%%%%%%%%%%%%%%%%%%%%%%%%%%%%%%%%%%%%%%%%%%%%%%%
\sect{\Helpers}
In this section, we present some tools designed to help the user while he is
creating a figure.
\smallskip
We first recall the existence of the macro \Mac{pssetupdate} that can be used
before \Mac{psbeginfig} in order to enforce the postscript file to be updated
at next compilation (see subsection {\bf \Princip}).
\smallskip
It often happens that points are defined, but will never be displayed on the
figure because they are only used for the geometric construction.
However, it is also often useful to show these points temporarily on the figure.
For example, when defining a curve, this helps 
to adjust the position of the interpolating points.
The macro \Mac{figshowpts[Nmin, Nmax]} does this job: it displays on the figure
every point defined since the the beginning of the session, whose number lies in
the interval [{\tt Nmin}, {\tt Nmax}] (which must not be too large, otherwise,
an error message "{\tt ! TeX capacity exceeded}" may occur).
The location of each point is marked by a bullet along with its joined text or,
by default, its number. Obviously, this macro must be called during step 3.
\smallskip
We have seen that the user can modify some parameters such as those concerning
the arrow-head, the line style, the color, etc\dots The macro \Mac{figshowsettings}
prints on the terminal the \eword{current} value of each of these parameters,
ordered by subject. This macro can be called at any place in the process.
\smallskip
We also recall that one can easily insert an existing encapsulated postscript file
with the macro \Mac{figinsert}. It is then possible to add any text on it: the
macro \Mac{figscan} is specially helpful to locate the position of the attach point.
Refer to the section {\bf \WTxtFig} for more information.
%%%%%%%%%%%%%%%%%%%%%%%%%%%%%%%%%%%%%%%%%%%%%%%%%%%%%%%%%%%%%%%%%%%%%%%%%%%
\bigskip
\sect{\TriDim}
\fsubsect{\IntroTD}
The explanations given so far concerned two-dimensional geometry. It so happens
that, in their major part, they also hold for three-dimensional geometry. We can
switch the 3D-mode on with the help of the macro \Mac{figinit}. In the following
sections, we describe how to use the macros in this mode.
There are three sets of macros:\par
\item{$\bullet$} macros that work only in the plane $z=0$, which are: \Mac{figptell},
\Mac{figptendnormal}, \Mac{figptsintercirc}, \Mac{psarcell}, \Mac{psnormal}.\hfill\break
They exist because either the list of arguments is specific to the 2D
(which is the case for \Mac{figptell} and \Mac{psarcell}, but other versions of
these macros are available), or the authors judged that the macro is very rarely
used in 3D (\Mac{figptendnormal}, \Mac{figptsintercirc}, \Mac{psnormal}).
\hfill\break
They can however be used in 3D-mode: they simply do not take the third component $z$
into account during the computation. The macro \Mac{figptsintercirc} is still useful
since we can compute the intersection of two circles in the plane $z=0$ and then give
them the rigth position in the space with the help of a rotation and a translation for
example.
%
\item{$\bullet$} macros that are specific to the 3D, which are:
\Mac{figptinterlineplane}, \Mac{figptorthoprojplane}, \Mac{figptsorthoprojplane},
\Mac{figsetobdist}, \Mac{figsettarget}, \Mac{figsetview}.
\hfill\break
Their existence is simply due to the fact that they have no equivalent in 2D.
Thus, any attempt to use them in 2D-mode brings on the message\hfill\break
\hbox to 2cm{\hfill}
{\tt *** The macro {\it macro name} is not available in the current context.}
\hfill\break
to be printed on the terminal.
%
\item{$\bullet$} macros that can be used in both dimensions, which are all the other
ones. These macros have the same name in 2D and in 3D, although one additional
argument may be needed in 3D. Thus, a macro can have a slightly different calling
sequence in 2D and in 3D. This is detailed in a forthcoming section.
\smallskip
The macros are still used in the same three steps scheme: 1)~definition of
characteristic points, 2)~creation of the postscript file, 3)~writing on the figure
if needed. However, we need here one more tool, a projection, which is used during
steps 2) and 3).
%%%%%%%%%%%%%%%%%%%%%%%%%%%%%%%%%%%%%%%
\subsect{\Projec}
Three projections are available : cavalier, orthogonal and realistic.
The latter is a perspective (or conical) projection, the two first are
parallel (or cylindrical) projections.
\smallskip
The choice is made with the help of the macro \Mac{figinit} (see next section).
\par
Then, we can modify the parameters governing the projection with the macro
\Mac{figsetview}, and for the realistic projection only, \Mac{figsettarget}
and \Mac{figsetobdist}.
\smallskip
\NewPSfile{\MyPSfile}
The {\bf cavalier projection} is the default one. It provides a false
perspective
view and is generally used to show objects with parallel faces. Its characteristics are
(see figure \the\Figno):
\item{-} the plane $(x,z)$ always corresponds to the paper sheet plane,
\item{-} the three-dimensional effect can be tuned by the angle $\psi = (Ox, Oy)$
and a real number $\lambda$ usually chosen between 0 and 1, which is the reduction
coefficient of the perspective.
\smallskip
The angle $\psi$ can be changed by saying {\tt \Mac{figsetview}(Psi)}.
The default value is 40 degrees. Values of $\psi$ such that $0 \le \psi \le 180$
correspond to a view from above and to a view from beneath if $-180 \le \psi \le 0$.
\smallskip
In fact, the genuine {\it cavalier} projection is obtained with $\lambda=1$.
In this projection, all edges of a cube have the same length. This property may be
interesting since measurements can be made directly. However, we may find that the
objects seem too elongated. That is why the default value is $\lambda=0.5$, which
corresponds to the so-called {\it cabinet} projection and lead to a more realistic
result. The coefficient $\lambda$ can be changed by saying
{\tt \Mac{figsetview}(Psi, Lambda)}.
\bigskip
% 1. Definition of characteristic points
\figinit{5cm, 3D}
\figpt 0:$O$(0,0)
\figpttraC 1:$x$=0/1,0,0/
\figpttraC 2:$y$=0/0,1,0/
\figpttraC 3:$z$=0/0,0,1/
\def\Rpsi{0.3}\figptcirc 4:$\psi$:0,1,3;\Rpsi(20)
% 2. Creation of the postscript file
\psbeginfig{\MyPSfile}
\psarrow[0,1]\psarrow[0,2]\psarrow[0,3]
\pssetarrowheadlength{0.05}\pssetarrowheadfill{yes}
\psarrowcirc 0,1,3;\Rpsi(0,40)
\psendfig
% 3. Writing text on the figure
\def\dist{3pt}
\figvisu{\demo}{Figure \the\Figno: Definition of the cavalier projection.}{
\figinsert{\MyPSfile}
\figwritesw 0:(\dist)\figwritee 1,2,4:(\dist)\figwriten 3:(\dist)}
\centerline{\box\demo}
\bigskip
The text of the program that produces this figure is given below, although we are
anticipating on the description of some of the macros used there.
Just notice that the default settings are used, so no call to \Mac{figsetview} is done.
$$\vbox{\fmtprog{0pt}
\+{}\% 1. Definition of characteristic points\cr
\+{}\Mac{figinit}$\{$5cm, 3D$\}$\cr
\+{}\Mac{figpt} 0:\$O\$(0,0)\cr
\+{}\Mac{figpttraC} 1:\$x\$=0/1,0,0/\cr
\+{}\Mac{figpttraC} 2:\$y\$=0/0,1,0/\cr
\+{}\Mac{figpttraC} 3:\$z\$=0/0,0,1/\cr
\+{}\char`\\def\char`\\Rpsi$\{$0.3$\}$\Mac{figptcirc} 4:\$\char`\\psi\$:0,1,3;\char`\\Rpsi(20)\cr
\+{}\% 2. Creation of the postscript file\cr
\+{}\Mac{psbeginfig}$\{$\char`\\MyPSfile$\}$\cr
\+{}\Mac{psarrow}[0,1]\Mac{psarrow}[0,2]\Mac{psarrow}[0,3]\cr
\+{}\Mac{pssetarrowheadlength}$\{$0.05$\}$\Mac{pssetarrowheadfill}$\{$yes$\}$\cr
\+{}\Mac{psarrowcirc} 0,1,3;\char`\\Rpsi(0,40)\cr
\+{}\Mac{psendfig}\cr
\+{}\% 3. Writing text on the figure\cr
\+{}\char`\\def\char`\\dist$\{$3pt$\}$\cr
\+{}\Mac{figvisu}$\{$\char`\\demo$\}$$\{$Figure \char`\\the\char`\\Figno: Definition of the cavalier projection.$\}$$\{$\cr
\+{}\Mac{figinsert}$\{$\char`\\MyPSfile$\}$\cr
\+{}\Mac{figwritesw} 0:(\char`\\dist)\Mac{figwritee} 1,2,4:(\char`\\dist)\Mac{figwriten} 3:(\char`\\dist)$\}$\cr
\+{}\char`\\centerline$\{$\char`\\box\char`\\demo$\}$\cr
}$$
\par
\NewPSfile{\MyPSfile}
The {\bf realistic projection} is defined by a target point $T$ and an observation point $E$,
also called eye point (see figure \the\Figno). The coordinates of the target point are
those of a point {\tt Pt} already defined or computed. They are set with the help
of the macro: {\tt \Mac{figsettarget}[Pt]}
\par\noindent
Once the position of the target point is set, {\tt Pt} can be modified. By default,
the target point is the center of the bounding box of the points defined so far.
Then, the eye point position is defined by:
\item{-} the observation distance $TE$, which can be set by the macro:
{\tt \Mac{figsetobdist}(Dist)}\hfill\break
By default, this distance is equal to 5 times the
largest dimension of the bounding box of the points defined so far.
\item{-} the longitude $\psi = (\vec{x},\,\vect{TH})$ and the latitude
$\theta = (\vect{TH},\,\vect{TE})$, angles to be given in degrees, that can be set
by saying: {\tt \Mac{figsetview}(Psi, Theta)}
\hfill\break
$H$ is the orthogonal projection of $E$ onto the plane $(T,x,y)$.
\bigskip
The text of the program that produces the figure \the\Figno\ is given below it.
\par
%\topinsert
% 1. Definition of characteristic points
\figinit{5cm, realistic}
\figpt 0:$T$(0,0)
\figpttraC 1:$x$=0/1,0,0/\figpttraC 2:$y$=0/0,1,0/\figpttraC 3:$z$=0/0,0,1/
\figpt 4:$E$(0.4,0.7,0.9)
\figvectP10[0,3]\figptorthoprojplane 5:$H$=4/0,10/
\def\Rpsi{0.3}  \figptcirc 6:$\psi$:0,1,2;\Rpsi(30)
\def\Rtheta{0.4}\figptcirc 7:$\theta$:0,5,3;\Rtheta(25)
% 2. Creation of the postscript file
\figsetview(30,30)
\psbeginfig{\MyPSfile}
\psarrow[0,1]\psarrow[0,2]\psarrow[0,3]
\psline[0,4]
\pssetarrowheadlength{0.05}\pssetarrowheadfill{yes}
\psarrowcircP 0;\Rpsi[1,5,2]\psarrowcircP 0;\Rtheta[5,4,3]
\pssetdash{5}\psline[0,5,4]
\psendfig
% 3. Writing text on the figure
\def\dist{3pt}
\figvisu{\demo}{Figure \the\Figno: Definition of the perspective projection.}{
\figinsert{\MyPSfile}
\figwritenw 0:(\dist)\figwritew 1:(\dist)\figwritee 2:(\dist)\figwriten 3:(\dist)
\figwrites 6:(\dist)\figwritene 7:(\dist)
\figsetmark{$\bullet$}\figwritene 4:(\dist)\figwritee 5:(\dist)}
\centerline{\box\demo}
%\endinsert
$$\vbox{\fmtprog{0pt}
\+{}\% 1. Definition of characteristic points\cr
\+{}\Mac{figinit}$\{$5cm, realistic$\}$\cr
\+{}\Mac{figpt} 0:\$T\$(0,0)\cr
\+{}\Mac{figpttraC} 1:\$x\$=0/1,0,0/\Mac{figpttraC} 2:\$y\$=0/0,1,0/\Mac{figpttraC} 3:\$z\$=0/0,0,1/\cr
\+{}\Mac{figpt} 4:\$E\$(0.4,0.7,0.9)\cr
\+{}\Mac{figvectP}10[0,3]\Mac{figptorthoprojplane} 5:\$H\$=4/0,10/\cr
\+{}\char`\\def\char`\\Rpsi$\{$0.3$\}$  \Mac{figptcirc} 6:\$\char`\\psi\$:0,1,2;\char`\\Rpsi(30)\cr
\+{}\char`\\def\char`\\Rtheta$\{$0.4$\}$\Mac{figptcirc} 7:\$\char`\\theta\$:0,5,3;\char`\\Rtheta(25)\cr
\+{}\% 2. Creation of the postscript file\cr
\+{}\Mac{figsetview}(30,30)\cr
\+{}\Mac{psbeginfig}$\{$\char`\\MyPSfile$\}$\cr
\+{}\Mac{psarrow}[0,1]\Mac{psarrow}[0,2]\Mac{psarrow}[0,3]\cr
\+{}\Mac{psline}[0,4]\cr
\+{}\Mac{pssetarrowheadlength}$\{$0.05$\}$\Mac{pssetarrowheadfill}$\{$yes$\}$\cr
\+{}\Mac{psarrowcircP} 0;\char`\\Rpsi[1,5,2]\Mac{psarrowcircP} 0;\char`\\Rtheta[5,4,3]\cr
\+{}\Mac{pssetdash}$\{$5$\}$\Mac{psline}[0,5,4]\cr
\+{}\Mac{psendfig}\cr
\+{}\% 3. Writing text on the figure\cr
\+{}\char`\\def\char`\\dist$\{$3pt$\}$\cr
\+{}\Mac{figvisu}$\{$\char`\\demo$\}$$\{$Figure \char`\\the\char`\\Figno: Definition of the perspective projection.$\}$$\{$\cr
\+{}\Mac{figinsert}$\{$\char`\\MyPSfile$\}$\cr
\+{}\Mac{figwritenw} 0:(\char`\\dist)\Mac{figwritew} 1:(\char`\\dist)\Mac{figwritee} 2:(\char`\\dist)\Mac{figwriten} 3:(\char`\\dist)\cr
\+{}\Mac{figwrites} 6:(\char`\\dist)\Mac{figwritene} 7:(\char`\\dist)\cr
\+{}\Mac{figsetmark}$\{$\$\char`\\bullet\$$\}$\Mac{figwritene} 4:(\char`\\dist)\Mac{figwritee} 5:(\char`\\dist)$\}$\cr
\+{}\char`\\centerline$\{$\char`\\box\char`\\demo$\}$\cr
}$$
\smallskip
Notice that if the distance $TE$ tends to infinity, the projection tends to be
the orthogonal projection onto the plane, i.e. the paper sheet. No control is made on the
value given as argument ; a too big one may bring on \TeX\ to print the error message
``{\tt ! Dimension too large.}" and stop the compilation process.
\bigskip
The {\bf orthogonal projection} is a simplified version of the previous one since
the eye point $E$ is placed at infinity. It is an orthogonal projection onto a plane which
is itself orthogonal to the observation direction. As a consequence, this projection is only
defined by the observation direction, i.e. by the two angles $\psi$ and $\theta$
(see figure \the\Figno), which are set with the macro \Mac{figsetview}.
%%%%%%%%%%%%%%%%%%%%%%%%%%%%%%%%%%%%%%%
\subsect{\MacSpec}
In this section, we focus on macros that are specific to the 3D and macros that
do not have the same calling sequence in both dimensions. Notice that for those last
macros, an error occur if we attempt to use them with the 2D calling sequence.
The other macros are not mentionned here since their description, which is valid in 3D,
can be found in the previous sections.
\bigskip\noindent
{\bf Control}\smallskip
The full prototype of \Mac{figinit} is:\smallskip
\centerline{\tt\Mac{figinit} $\{$ScaleFactorUnit, {\it Dim/Proj}$\,\}$}
\smallskip\noindent
The first argument {\tt ScaleFactorUnit} has been described in the {\bf\UnScale}
subsection. The second argument {\it Dim/Proj} is optional. It sets the dimension of
the space to be used until next call to \Mac{figinit}. In 3D, it also sets the projection
type. If this argument is absent or is given the value ``{\tt 2D}", then geometry in the
plane is assumed, otherwise geometry in three dimensions is performed.
Moreover, if this argument is equal to:\par
\item{.} {\tt orthogonal}, then the orthogonal projection is used,
\item{.} {\tt realistic}, then the realistic projection is used,
\item{.} {\tt 3D} or {\tt cavalier} (or anything else), then the cavalier projection is used.
\bigskip\noindent
{\bf Vector creation}\smallskip
The definition of a vector by its components {\tt (X,Y,Z)} can be done with the help
of the macro whose prototype is:\par
\centerline{\tt\Mac{figvectC} NewVect (X,Y,Z)}
\smallskip\noindent
Notice that the 2D calling sequence {\tt\ \Mac{figvectC} NewVect (X,Y) } is still
allowed, implying {\tt Z} to be 0.
\medskip
The macro\par
\leftline{\hskip3cm\tt\Mac{figvectN} NewVect [Pt1,Pt2,Pt3]}
\smallskip\noindent
defines a vector normal to the plane defined by the 3 points {\tt Pt1},
{\tt Pt2} and {\tt Pt3}.
\par\noindent
More precisely, let $P_1$, $P_2$, $P_3$ and $\vect{V}$ stand for
{\tt Pt1}, {\tt Pt2}, {\tt Pt3} and {\tt NewVect}.
\par\noindent
This macro computes
$\displaystyle \vect{V}={\vect{P_1P_2}\over ||\vect{P_1P_2}||}\times
{\vect{P_1P_3}\over ||\vect{P_1P_3}||}$.
Thus, $(\vect{P_1P_2},\,\vect{P_1P_3},\,\vect{V})$ is a positively oriented basis.
\bigskip
The macro\par
\leftline{\hskip3cm\tt\Mac{figvectNV} NewVect [Vector1, Vector2]}
\smallskip\noindent
defines a vector normal to the two vectors {\tt Vector1} and {\tt Vector2}.
\par\noindent
More precisely, let $\vect{V_1}$, $\vect{V_2}$ and $\vect{V}$ stand for
{\tt Vector1}, {\tt Vector2} and {\tt NewVect}.
\par\noindent
This macro computes
$\displaystyle \vect{V} = {\vect{V_1}\over ||\vect{V_1}||}\times
{\vect{V_2}\over ||\vect{V_2}||}$.
Thus, $(\vect{V_1},\,\vect{V_2},\,\vect{V})$ is a positively oriented basis.
\bigskip\noindent
{\bf Basic macros}\smallskip
The macro\par
\centerline{\tt \Mac{figpt} NewPt :Text(X,Y,Z)}
\smallskip\noindent
creates a point with coordinates {\tt (X,Y,Z)}.
The third coordinate {\tt Z} can be omitted, in which case the value {\tt Z}~=~0 is
assumed. In other words, the 2D calling sequence {\tt \Mac{figpt} NewPt :Text(X,Y)}
is valid in 3D-mode as well.
\medskip
Let $C$, $P_1$, $P_2$ and $P_3$ stand for {\tt Center}, {\tt Pt1}, {\tt Pt2}
and {\tt Pt3}. The macro\smallskip
\centerline{\tt \Mac{figgetangle}\CtrlSq{Value}[Center,Pt1,Pt2,Pt3]}
\smallskip\noindent
computes the value (in degrees) of the oriented angle $(\vect{CP_1},\,\vect{CP_2})$
and stores it in the macro \CtrlSq{Value} whose name is chosen
by the user. The angle is measured counterclockwise around the vector
$\vect{CP_1} \times \vect{CP_3}$.
The points $C$, $P_1$, $P_2$ and $P_3$ must be coplanar, but $P_3$ must not lie
on the line $(C,P1)$.
\bigskip\noindent
{\bf Point transformation}\smallskip
Somme transformation macros have been adapted for the 3D-mode. They each compute the
image {\tt NewPt} of the point {\tt Pt}.
Let $P$, $P'$, $C$ and $\vect{V}$ stand for {\tt Pt}, {\tt NewPt}, {\tt Center}
and {\tt Vector}.
\medskip
The macro\par
\leftline{\hskip3cm\tt\Mac{figptrot} NewPt :Text= Pt /Center, Angle, Vector/}
\smallskip\noindent
refers to the rotation of angle {\tt Angle} (given in degrees) around the axis
$(C, \vect{V})$. The sense of rotation is such that $(\vect{CP},\,\vect{CP'},\,\vect{V})$
is a positively oriented basis.
\medskip
The macro\par
\leftline{\hskip3cm\tt \Mac{figptsym} NewPt :Text= Pt /PlanePt, NormalVector/}
\smallskip\noindent
refers to the orthogonal symmetry with respect to the plane defined by the point
{\tt PlanePt} and the vector {\tt NormalVector} which is normal to the plane.
This vector may be computed with the help of \Mac{figvectN} or \Mac{figvectNV}.
\medskip
The macro\par
\leftline{\hskip3cm\tt \Mac{figpttraC} NewPt :Text= Pt /X,Y,Z/}
\smallskip\noindent
refers to the translation of vector {\tt (X,Y,Z)}.
\medskip
The macro\par
\leftline{\hskip3cm\tt \Mac{figptorthoprojplane} NewPt :Text= Pt /PlanePt, NormalVector/}
\smallskip\noindent
refers to the orthogonal projection onto the plane defined by the point
{\tt PlanePt} and the vector {\tt NormalVector} which is normal to the plane.
\bigskip
The corresponding available ``set" versions have then the following prototypes:
\smallskip
\centerline{\vbox{
\hbox{\tt \Mac{figptsrot} NewPt1 = Pt1, Pt2, ..., PtN /Center, Angle, Vector/}
\hbox{\tt \Mac{figptssym} NewPt1 = Pt1, Pt2, ..., PtN /PlanePt, NormalVector/}
\hbox{\tt \Mac{figptsorthoprojplane} NewPt1 = Pt1, Pt2, ..., PtN /PlanePt, NormalVector/}
}}
\bigbreak\noindent
{\bf Point computation}\smallskip
The macro\par
\leftline{\hskip3cm\tt \Mac{figptcirc} NewPt :Text: Center,Pt1,Pt2;Radius (Angle)}
\smallskip\noindent
computes the coordinates of a point lying on a circle. Let $C$, $P_1$ and $P_2$ stand
for {\tt Center}, {\tt Pt1} and {\tt Pt2}. The circle, of center $C$ and radius
{\tt Radius}, is lying in the plane $(C,P_1,P_2)$. The two points $P_1$ and $P_2$ do
not need to belong to the circle. The position of the point is defined by the angle
{\tt Angle} (given in degrees), measured counterclockwise around the vector
$\vect{CP_1}\times\,\vect{CP_2}$, starting from the half-line $(C,P_1)$.
\medskip
The macro\par
\leftline{\hskip3cm\tt \Mac{figptinterlineplane} NewPt :Text[LinePt,Vector; PlanePt,NormalVector]}
\smallskip\noindent
computes the intersection of the line defined by the point {\tt LinePt} and the vector
{\tt Vector}, and the plane defined by the point {\tt PlanePt} and the vector
{\tt NormalVector} which is normal to the plane.
\bigskip\noindent
{\bf Drawing macros}\smallskip
Let $C$, $P_1$ and $P_2$ stand for {\tt Center}, {\tt Pt1} and {\tt Pt2}.
In the three following macros, we consider a circle (or circular arc) of center
$C$ and radius {\tt Radius}, lying in the plane $(C,P_1,P_2)$.
The two points $P_1$ and $P_2$ do not need to belong to the circle.
The point $P_2$ must not lie on the line $(C,P_1)$.
\smallskip
These macros are:
\smallskip
\leftline{\hskip3cm\tt \Mac{pscirc} Center,Pt1,Pt2 (Radius)}
\smallskip\noindent
which draws the entire circle.
\medskip
\leftline{\hskip3cm\tt \Mac{psarccirc} Center,Pt1,Pt2 ; Radius (Ang1,Ang2)}
\smallskip\noindent
which draws the circular arc limited by the angles {\tt Ang1} and {\tt Ang2} given
in degrees and measured counterclockwise around the vector
$\vect{CP_1}\times\,\vect{CP_2}$, starting from the half-line $(C,P_1)$.
\medskip
\leftline{\hskip3cm\tt \Mac{psarrowcirc} Center,Pt1,Pt2 ; Radius (Ang1,Ang2)}
\smallskip\noindent
which draws the circular arrow beared by the circular arc defined just before.
If {\tt Ang2} $>$ {\tt Ang1}, the arrow is drawn counterclockwise, else it is drawn
clockwise. The arrow-head is drawn according to the \Mac{psarrowhead} macro settings.
\bigskip
The two last macros have a so-called ``point" version.
Let $P_3$ stand for {\tt Pt3} and $\vect{N} = \vect{CP_1} \times \vect{CP_3}$. 
In the two following macros, we consider a circular arc of center $C$,
lying in the plane $(C,P_1,P_3)$ and limited by the two
half-lines $(C,P_1)$ and $(C,P_2)$.
The three points $P_1$, $P_2$ and $P_3$ do not need to belong to the circle.
The points $C$, $P_1$, $P_2$ and $P_3$ must be coplanar, but $P_3$ must not lie
on the line $(C,P1)$.
\smallskip
These macros are:
\smallskip
\leftline{\hskip3cm\tt \Mac{psarccircP} Center ; Radius [Pt1,Pt2,Pt3]}
\smallskip\noindent
which draws the circular arc of radius {\tt Radius}, from $P_1$ towards $P_2$
turning counterclockwise around the axis $(C, \vect{N})$.
\medskip
\leftline{\hskip3cm\tt \Mac{psarrowcircP} Center ; Radius [Pt1,Pt2,Pt3]}
\smallskip\noindent
which draws the circular arrow of radius {\tt |Radius|}, from $P_1$ towards $P_2$
turning around the axis $(C, \vect{N})$:
\item{.} counterclockwise if the radius is positive,
\item{.} clockwise if the radius is negative.
\bigskip\noindent
Remarks:
\item{$\bullet$} An arrow (or an arrow-head) is always two-dimensional because the
arrow which is really drawn is the two-dimensional arrow whose body is the projection
of the true 3D arrow body.
\item{$\bullet$} In the same order of idea, non-mute macros (whose name begin with
\Mac{figwrite}) write their text in the plane of the sheet with respect of the
position of the projection of the attach point.
%%%%%%%%%%%%%%%%%%%%%%%%%%%%%%%%%%%%%%%
\subsect{\ExTD}
{\bf $\bullet$ Cube}\smallskip
We begin with the representation of the unit cube defined by its 8 vertices $P_1$,
$P_2$, \dots $P_8$ as follows:\par
% 1. Definition of characteristic points
\figinit{3cm, 3D}
\figpt 1:(0,0)
\figpttraC 2:=1/1,0,0/
\figpttraC 3:=1/1,1,0/
\figpttraC 4:=1/0,1,0/
\figvectC 10(0,0,1)\figptstra 5=1,2,3,4/1,10/
\figsetptname{$P_#1$}
%
{\medskip\leftskip2cm\leavevmode
\vbox{\fmtprog{0pt}
\+{}\% 1. Definition of characteristic points\cr
\+{}\Mac{figinit}$\{$3cm, 3D$\}$\cr
\+{}\Mac{figpt} 1:(0,0)\cr
\+{}\Mac{figpttraC} 2:=1/1,0,0/\cr
\+{}\Mac{figpttraC} 3:=1/1,1,0/\cr
\+{}\Mac{figpttraC} 4:=1/0,1,0/\cr
\+{}\Mac{figvectC} 10(0,0,1)\Mac{figptstra} 5=1,2,3,4/1,10/\cr
\+{}\Mac{figsetptname}$\{$\$P\_\#1\$$\}$\cr
}
\medskip}
%
As we will need to draw the cube several times, we define the macro \CtrlSq{DrawCube}
to create the postscript file whose name is contained in the macro \CtrlSq{MyPSfile}.
The background lines are dashed ; obviously, they depend on the observation direction.
\par
\def\DrawCube{
% 2. Creation of the postscript file
\psbeginfig{\MyPSfile}
\psline[1,2,3]\psline[5,6,7,8,5]
\psline[1,5]\psline[2,6]\psline[3,7]
\pssetdash{4}\psline[1,4,3]\psline[4,8]
\psendfig}
%
{\medskip\leftskip2cm\leavevmode
\vbox{\fmtprog{0pt}
\+{}\char`\\def\char`\\DrawCube$\{$\cr
\+{}\% 2. Creation of the postscript file\cr
\+{}\Mac{psbeginfig}$\{$\char`\\MyPSfile$\}$\cr
\+{}\Mac{psline}[1,2,3]\Mac{psline}[5,6,7,8,5]\cr
\+{}\Mac{psline}[1,5]\Mac{psline}[2,6]\Mac{psline}[3,7]\cr
\+{}\Mac{pssetdash}$\{$4$\}$\Mac{psline}[1,4,3]\Mac{psline}[4,8]\cr
\+{}\Mac{psendfig}$\}$\cr
}
\medskip}
%
In the same way, to create the box and write the text on the figure, we define the
macro \CtrlSq{WriteOnCube} which has the name of the box register as first argument
and the legend to be printed under the figure as second argument.
\par
\def\WriteOnCube#1#2{
% 3. Writing text on the figure
\def\dist{2pt}
\figvisu{#1}{Figure \the\Figno\ - #2}{\figinsert{\MyPSfile}
\figwritesw 1:(\dist)\figwritese 2:(\dist)
\figwritee 3:(\dist)\figwritenw 4:(\dist)
\figwritenw 5:(\dist)\figwritese 6:(\dist)
\figwritene 7:(\dist)\figwritenw 8:(\dist)
}}
%
{\medskip\leftskip2cm\leavevmode
\vbox{\fmtprog{0pt}
\+{}\char`\\def\char`\\WriteOnCube\#1\#2$\{$\cr
\+{}\% 3. Writing text on the figure\cr
\+{}\char`\\def\char`\\dist$\{$2pt$\}$\cr
\+{}\Mac{figvisu}$\{$\#1$\}$$\{$Figure \char`\\the\char`\\Figno\char`\\ - \#2$\}$$\{$\Mac{figinsert}$\{$\char`\\MyPSfile$\}$\cr
\+{}\Mac{figwritesw} 1:(\char`\\dist)\Mac{figwritese} 2:(\char`\\dist)\cr
\+{}\Mac{figwritee} 3:(\char`\\dist)\Mac{figwritenw} 4:(\char`\\dist)\cr
\+{}\Mac{figwritenw} 5:(\char`\\dist)\Mac{figwritese} 6:(\char`\\dist)\cr
\+{}\Mac{figwritene} 7:(\char`\\dist)\Mac{figwritenw} 8:(\char`\\dist)\cr
\+{}$\}$$\}$\cr
}
\medskip}
%
\NewPSfile{\MyPSfile}\edef\FigLeft{\the\Figno}
\DrawCube\WriteOnCube{\demo}{Default settings}
\NewPSfile{\MyPSfile}\edef\FigRight{\the\Figno}
\def\Valpsi{60} \figsetview(\Valpsi)
\DrawCube\WriteOnCube{\demotwo}{$\psi = \Valpsi$}
\topinsert
\centerline{\box\demo\hfil\box\demotwo}
\endinsert
%
Thus, the figures \FigLeft\ and \FigRight\ are obtained by simply writing:\par
{\medskip\leftskip2cm\leavevmode
\vbox{\fmtprog{0pt}
\+{}\char`\\newbox\char`\\demo \char`\\newbox\char`\\demotwo \% if not already done\cr
\+{}\char`\\NewPSfile$\{$\char`\\MyPSfile$\}$\cr
\+{}\char`\\DrawCube\char`\\WriteOnCube$\{$\char`\\demo$\}$$\{$Default settings$\}$\cr
\+{}\char`\\NewPSfile$\{$\char`\\MyPSfile$\}$\cr
\+{}\char`\\def\char`\\Valpsi$\{$60$\}$ \Mac{figsetview}(\char`\\Valpsi)\cr
\+{}\char`\\DrawCube\char`\\WriteOnCube$\{$\char`\\demotwo$\}$$\{$\$\char`\\psi = \char`\\Valpsi\$$\}$\cr
\+{}\char`\\centerline$\{$\char`\\box\char`\\demo\char`\\hfil\char`\\box\char`\\demotwo$\}$\cr
}
\medskip}
%
\bigskip
\def\WriteOnCubeB#1#2{
% 3. Writing text on the figure
\def\dist{2pt}
\figvisu{#1}{Figure \the\Figno\ - #2}{\figinsert{\MyPSfile}
\figwritene 1:(\dist)\figwritenw 2:(\dist)
\figwritesw 3:(\dist)\figwritese 4:(\dist)
\figwritesw 5:(\dist)\figwritese 6:(\dist)
\figwritene 7:(\dist)\figwritenw 8:(\dist)
}}
These figures are drawn with the cavalier projection ; we can observe the influence
of the angle $\psi$. The four following figures use the realistic
projection. We recall that the observation direction is defined by the angles
$(\psi, \theta)$. The right-hand side figures correspond to the ordinary 2D view,
for which the position of the text has been adapted (macro \CtrlSq{WriteOnCubeB}).
The program that produces the first two figures is:\par
$$\vbox{\fmtprog{0pt}
\+{}\Mac{figinit}$\{$3cm, realistic$\}$\cr
\+{}...\cr
\+{}\char`\\NewPSfile$\{$\char`\\MyPSfile$\}$\cr
\+{}\char`\\def\char`\\Valpsi$\{$-60$\}$\char`\\def\char`\\Valtheta$\{$30$\}$ \Mac{figsetview}(\char`\\Valpsi, \char`\\Valtheta)\cr
\+{}\char`\\DrawCube\char`\\WriteOnCube$\{$\char`\\demo$\}$$\{$\$(\char`\\psi, \char`\\theta)=(\char`\\Valpsi, \char`\\Valtheta)\$$\}$\cr
\+{}\char`\\NewPSfile$\{$\char`\\MyPSfile$\}$\cr
\+{}\char`\\def\char`\\Valpsi$\{$-90$\}$\char`\\def\char`\\Valtheta$\{$90$\}$ \Mac{figsetview}(\char`\\Valpsi, \char`\\Valtheta)\cr
\+{}\char`\\DrawCube\char`\\WriteOnCubeB$\{$\char`\\demotwo$\}$$\{$\$(\char`\\psi, \char`\\theta)=(\char`\\Valpsi, \char`\\Valtheta)\$$\}$\cr
\+{}\char`\\centerline$\{$\char`\\box\char`\\demo\char`\\hfil\char`\\box\char`\\demotwo$\}$\cr
}$$
\bigskip
\figinit{3cm, realistic}
\figpt 1:(0,0)
\figpttraC 2:=1/1,0,0/
\figpttraC 3:=1/1,1,0/
\figpttraC 4:=1/0,1,0/
\figvectC 10(0,0,1)\figptstra 5=1,2,3,4/1,10/
\figsetptname{$P_#1$}
\NewPSfile{\MyPSfile}
\def\Valpsi{-60}\def\Valtheta{30} \figsetview(\Valpsi, \Valtheta)
\DrawCube\WriteOnCube{\demo}{$(\psi, \theta)=(\Valpsi, \Valtheta)$}
\NewPSfile{\MyPSfile}
\def\Valpsi{-90}\def\Valtheta{90} \figsetview(\Valpsi, \Valtheta)
\DrawCube\WriteOnCubeB{\demotwo}{$(\psi, \theta)=(\Valpsi, \Valtheta)$}
\centerline{\box\demo\hfil\box\demotwo}
\bigskip
%\figshowsettings
\figsettarget[7]
\NewPSfile{\MyPSfile}\edef\FigLeft{\the\Figno}
\def\Valpsi{-60}\def\Valtheta{30} \figsetview(\Valpsi, \Valtheta)
\DrawCube\WriteOnCube{\demo}{$(\psi, \theta)=(\Valpsi, \Valtheta),\ T=(1,1,1)$}
\NewPSfile{\MyPSfile}\edef\FigRight{\the\Figno}
\def\Valpsi{-90}\def\Valtheta{90} \figsetview(\Valpsi, \Valtheta)
\DrawCube\WriteOnCubeB{\demotwo}{$(\psi, \theta)=(\Valpsi, \Valtheta),\ T=(1,1,1)$}
\centerline{\box\demo\hfil\box\demotwo}
\bigskip
After this, if we add the command \Mac{figshowsettings}, we get the following
information to be printed on the terminal at compilation time:
$$\vbox{\fmtprog{-0.5pt}
\+{}====================================================================\cr
\+{} Current settings about:\cr
\+{} --- GENERAL ---\cr
\+{}Scale factor and Unit = 3cm (85.35826pt)  -> \Mac{figinit}$\{$ScaleFactorUnit$\}$\cr
\+{}Update mode = no -> \Mac{pssetupdate}$\{$yes/no$\}$\cr
\+{} --- PRINTING ---\cr
\+{}Implicit point name = \$P\_i\$  -> \Mac{figsetptname}$\{$Name:  $\}$\cr
\+{}Point marker =  -> \Mac{figsetmark}$\{$Mark$\}$\cr
\+{}Print rounded coordinates = yes -> \Mac{figsetroundcoord}$\{$yes/no$\}$\cr
\+{} --- POSTSCRIPT ---\cr
\+{}Arrow-head:\cr
\+{} (half-)Angle = 20 -> \Mac{pssetarrowheadangle}$\{$Angle$\}$\cr
\+{} Filling mode = no -> \Mac{pssetarrowheadfill}$\{$yes/no$\}$\cr
\+{} "Outside" = no -> \Mac{pssetarrowheadout}$\{$yes/no$\}$\cr
\+{} Length = 0.11703  (active) -> \Mac{pssetarrowheadlength}$\{$Length$\}$\cr
\+{} Ratio = 0.1 (not active) -> \Mac{pssetarrowheadratio}$\{$Ratio$\}$\cr
\+{}Color = 0 -> \Mac{pssetcmyk}$\{$\char`\\ColorName$\}$ or\cr
\+{}    \Mac{pssetrgb}$\{$\char`\\ColorName$\}$ or \Mac{pssetgray}$\{$GrayLevel$\}$\cr
\+{}Curve roundness = 0.2 -> \Mac{pssetroundness}$\{$Roundness$\}$\cr
\+{}Filling mode = no -> \Mac{pssetfillmode}$\{$yes/no$\}$\cr
\+{}Line style = 1 -> \Mac{pssetdash}$\{$Index/Pattern$\}$\cr
\+{}Line width = 0.4 -> \Mac{pssetwidth}$\{$Width$\}$\cr
}$$
$$\vbox{\fmtprog{-0.5pt}
\+{}Secondary settings:\cr
\+{} Color = 0 -> \Mac{pssetsecondcmyk}$\{$\char`\\ColorName$\}$ or\cr
\+{}    \Mac{pssetsecondrgb}$\{$\char`\\ColorName$\}$ or \Mac{pssetsecondgray}$\{$GrayLevel$\}$\cr
\+{} Line style = 4 -> \Mac{pssetseconddash}$\{$Index/Pattern$\}$\cr
\+{} Line width = 0.4 -> \Mac{pssetsecondwidth}$\{$Width$\}$\cr
\+{} --- 3D to 2D PROJECTION ---\cr
\+{}Projection : realistic -> \Mac{figinit}$\{$ScaleFactorUnit, ProjType$\}$\cr
\+{}Psi = -90 -> \Mac{figsetview}(Psi)\cr
\+{}Theta =  90 -> \Mac{figsetview}(Psi, Theta)\cr
\+{}Observation distance = 4.99493  -> \Mac{figsetobdist} (Dist)\cr
\+{}Target point = (0.49948 , 0.49948 , 0.49948 )  -> \Mac{figsettarget} [Pt]\cr
\+{}====================================================================\cr
}$$
\par
Remark: In 2D-mode the last section ({\tt 3D to 2D PROJECTION}) is not printed.
\smallskip
We can see that the target point is the center of the cube as expected (numerical
values are not rounded here). If we change it to the point $P_7$ by saying
{\tt \Mac{figsettarget}[7]} and then execute the same program, then we obtain the two
last figures \FigLeft\ and \FigRight.
\smallskip
The observation distance can be modified in the same way, for instance by saying
{\tt \Mac{figsetobdist}(10)} which doubles the current distance.
%%%%%%%%%%%%%%%%%
\medskip\noindent
{\bf $\bullet$ Intersecting planes}\smallskip
We now consider two intersecting planes, represented by two rectangles (vertices
{\tt 1}, {\tt 2}, {\tt 3}, {\tt 4} and {\tt 11}, {\tt 12}, {\tt 13}, {\tt 14}).
The intersection line is defined by the points {\tt 5} and {\tt 6}. All the points
are defined only once for both figures. The corresponding part of the program is:
\medskip
{\leftskip1cm\leavevmode
\vbox{\fmtprog{0pt}
\+{}\% 1. Definition of characteristic points\cr
\+{}\Mac{figinit}$\{$3cm, realistic$\}$ \Mac{figpt} 1:(0,0)\cr
\+{}\Mac{figpttraC} 2:=1/0,2,0/\Mac{figpttraC} 3:=2/1.2,0,0/\Mac{figpttraC} 4:=3/0,-2,0/\cr
\+{}\Mac{figpt} 5:(0,0.9)\Mac{figpt} 6:(1.2,1.3)\cr
\+{}\Mac{figvectC} 10(1,0,0)\Mac{figptsrot} 11 = 1,2 /5,60,10/\cr
\+{}\Mac{figvectP} 10[5,6]\Mac{figpttra} 13:=12/1,10/\Mac{figpttra} 14:=11/1,10/\cr
\+{}\char`\\def\char`\\Valpsi$\{$-30$\}$\char`\\def\char`\\Valtheta$\{$30$\}$\Mac{figsetview}(\char`\\Valpsi, \char`\\Valtheta)\cr
}
\par}
\bigskip
%--------------
% 1. Definition of characteristic points
\figinit{3cm, realistic}
\figpt 1:(0,0)
\figpttraC 2:=1/0,2,0/
\figpttraC 3:=2/1.2,0,0/
\figpttraC 4:=3/0,-2,0/
\figpt 5:(0,0.9)\figpt 6:(1.2,1.3)
\figvectC 10(1,0,0)\figptsrot 11 = 1,2 /5,60,10/
\figvectP 10[5,6]\figpttra 13:=12/1,10/\figpttra 14:=11/1,10/
\def\Valpsi{-30}\def\Valtheta{30}\figsetview(\Valpsi, \Valtheta)
%
% 2a. Creation of the first postscript file
\NewPSfile{\MyPSfile}
\psbeginfig{\MyPSfile}
\psline[1,2,3,4,1]\psline[11,12,13,14,11]\psline[5,6]
\psendfig
% 3a. Writing text on the figure
\figvisu{\demo}{Figure \the\Figno\ - $(\psi, \theta)=(\Valpsi, \Valtheta)$}
{\figinsert{\MyPSfile}}
%
% 2b. Creation of the second postscript file
\NewPSfile{\MyPSfile}
\psbeginfig{\MyPSfile}
\def\White{1}\def\Black{0}
\pssetfillmode{yes}\pssetgray{\White}\psline[11,14,6,5,11]
\pssetfillmode{no} \pssetgray{\Black}\psline[11,14,6,5,11]
\pssetfillmode{yes}\pssetgray{\White}\psline[1,2,3,4,1]
\pssetfillmode{no} \pssetgray{\Black}\psline[1,2,3,4,1]
\pssetfillmode{yes}\pssetgray{\White}\psline[5,6,13,12,5]
\pssetfillmode{no} \pssetgray{\Black}\psline[5,6,13,12,5]
\psendfig
% 3b. Writing text on the figure
\figvisu{\demotwo}{Figure \the\Figno\ - $(\psi, \theta)=(\Valpsi, \Valtheta)$}
{\figinsert{\MyPSfile}}
%\topinsert
\centerline{\box\demo\hfil\box\demotwo}
%\endinsert
%--------------
\medskip
The left-hand side figure is created by the sections {\tt 2a} and {\tt 3a} of the
program given below. It provides a straightforward representation, which looks somewhat
ambiguous because the hidden parts of the scene are not handled. They could be
represented by dashed lines but their limits depend on the observation direction
which involves some technical computation that it is better to avoid.
{\medskip\leftskip1cm\leavevmode
\vbox{\fmtprog{0pt}
\+{}\% 2a. Creation of the first postscript file\cr
\+{}\char`\\NewPSfile$\{$\char`\\MyPSfile$\}$\cr
\+{}\Mac{psbeginfig}$\{$\char`\\MyPSfile$\}$\cr
\+{}\Mac{psline}[1,2,3,4,1]\Mac{psline}[11,12,13,14,11]\Mac{psline}[5,6]\cr
\+{}\Mac{psendfig}\cr
\+{}\% 3a. Writing text on the figure\cr
\+{}\Mac{figvisu}$\{$\char`\\demo$\}$$\{$Figure \char`\\the\char`\\Figno\char`\\ - \$(\char`\\psi, \char`\\theta)=(\char`\\Valpsi, \char`\\Valtheta)\$$\}$\cr
\+{}$\{$\Mac{figinsert}$\{$\char`\\MyPSfile$\}$$\}$\cr
}
\medskip}
A solution to this is proposed on the right-hand side figure (created by the sections
{\tt 2b} and {\tt 3b} of the program) where the planes are shown using three filled
areas drawn from bottom to top. Notice that this algorithm holds as long as the
observation angle $\theta$ is positive. With this kind of representation, we could
of course choose the color of each plane. Here, since they are both white, a black
border is necessary, which is drawn at the same time without forgetting to switch
to the correct filling mode.
{\medskip\leftskip1cm\leavevmode
\vbox{\fmtprog{0pt}
\+{}\% 2b. Creation of the second postscript file\cr
\+{}\char`\\NewPSfile$\{$\char`\\MyPSfile$\}$\cr
\+{}\Mac{psbeginfig}$\{$\char`\\MyPSfile$\}$\cr
\+{}\char`\\def\char`\\White$\{$1$\}$\char`\\def\char`\\Black$\{$0$\}$\cr
\+{}\Mac{pssetfillmode}$\{$yes$\}$\Mac{pssetgray}$\{$\char`\\White$\}$\Mac{psline}[11,14,6,5,11]\cr
\+{}\Mac{pssetfillmode}$\{$no$\}$ \Mac{pssetgray}$\{$\char`\\Black$\}$\Mac{psline}[11,14,6,5,11]\cr
\+{}\Mac{pssetfillmode}$\{$yes$\}$\Mac{pssetgray}$\{$\char`\\White$\}$\Mac{psline}[1,2,3,4,1]\cr
\+{}\Mac{pssetfillmode}$\{$no$\}$ \Mac{pssetgray}$\{$\char`\\Black$\}$\Mac{psline}[1,2,3,4,1]\cr
\+{}\Mac{pssetfillmode}$\{$yes$\}$\Mac{pssetgray}$\{$\char`\\White$\}$\Mac{psline}[5,6,13,12,5]\cr
\+{}\Mac{pssetfillmode}$\{$no$\}$ \Mac{pssetgray}$\{$\char`\\Black$\}$\Mac{psline}[5,6,13,12,5]\cr
\+{}\Mac{psendfig}\cr
\+{}\% 3b. Writing text on the figure\cr
\+{}\Mac{figvisu}$\{$\char`\\demotwo$\}$$\{$Figure \char`\\the\char`\\Figno\char`\\ - \$(\char`\\psi, \char`\\theta)=(\char`\\Valpsi, \char`\\Valtheta)\$$\}$\cr
\+{}$\{$\Mac{figinsert}$\{$\char`\\MyPSfile$\}$$\}$\cr
}
\medskip}
The two figures are then displayed by saying:
\smallskip\nobreak
{\leftskip1cm\leavevmode
\vbox{\fmtprog{0pt}
\+{}\char`\\centerline$\{$\char`\\box\char`\\demo\char`\\hfil\char`\\box\char`\\demotwo$\}$\cr
}\par}
%%%%%%%%%%%%%%%%%
\bigskip\noindent
{\bf $\bullet$ Fun}\smallskip
Here is a recreative example. It shows a movement, called ``jibe", made
by a funboard turning around a floating mark. This will be the opportunity to
illustrate how a jointed object can be handled.
\smallskip
The funboard is made of two main rigid parts: the board, bearing the fin (macro
\Mac{board}) and the rigging set: sail, mast and wishbone (macro \Mac{sail}).
The bottom point of the mast is attached to the middle part of the board
according to a spherical link, allowing the sail to rotate in any direction
over the board.
\par
Three orthogonal axes are linked to the board (macro \Mac{axes}).
Thus, the geometric description is entirely done with respect to the basis
$(\vec{u}_1,\vec{u}_2,\vec{u}_3)$
and the projection $P$ of the prow point of the board onto the
trajectory along the $\vec{u}_3$ direction.
This also gives the sail a neutral position in the plane $(P ; \vec{u}_1, \vec{u}_3)$.
The vector $\vec{u}_1$ is oriented from the front to the rear and the vector
$\vec{u}_3$ from the bottom to the top. In the program, the vectors
$\vec{u}_1$, $\vec{u}_2$ and $\vec{u}_3$ bear respectively the numbers 11, 12
and 13, and the prow point $P$ number $0$.
\par
In its neutral position, the funboard is symmetric with respect to the plane
$(P ; \vec{u}_1, \vec{u}_3)$. To give the funboard the rigth position, we just
have to translate the point $P$ and rotate the basis (macro \Mac{boardposition}).
Then, the position of the sail is tuned by rotating the basis again (macro
\Mac{sailposition}). The corresponding calls are made in the macro \Mac{funboard}.
\smallskip
The funboard is shown in three characteristic positions along its trajectory
which is represented here as a B\'ezier curve lying in the plane $z=0$.
To obtain this result, the point $P$ occupies three positions on the trajectory at
the abscissae $0.2$, $0.55$ and $0.95$. At each of these positions, $\vec{u}_1$
is made colinear with the tangent vector to the curve (macro \Mac{jibe}).
\par
%=================================================================================
%--------------
% Definition of macros
%--------------
\def\board{% Draws the board
% Main points
\figpttra 1:=0/0.22,13/
\figpttra 2:=0/0.3,11/\figpttra 3:=0/2.7,11/\figpttra 4:=0/2.7,11/
\figpttra 2:=2/0.5,12/\figpttra 3:=3/0.3,12/
\figpttra 5:=0/0.3,11/\figpttra 6:=0/2.5,11/\figpttra 7:=0/2.5,11/
\figpttra 5:=5/0.4,12/\figpttra 6:=6/0.2,12/
\figpttra 5:=5/0.07,13/\figpttra 6:=6/0.06,13/\figpttra 7:=7/0.04,13/
% Bottom point of the mast
\figptBezier 8::0.45[1,5,6,7]\figptorthoprojplane 8:=8/1,12/
% Define the fin and draw it
\figpttra 20:=4/-0.30,11/\figpttra 21:=4/-0.25,11/
\figpttra 22:=4/-0.05,11/\figpttra 23:=4/-0.15,11/
\figpttra 21:=21/-0.15,13/\figpttra 22:=22/-0.5,13/
\pssetfillmode{yes}\pssetrgb{\Blackrgb}\psBezier 1[20,21,22,23]
% Compute the symmetric part of the board and draw it
\figptssym 20=2,3/1,12/
\pssetwidth{\defaultwidth}\pssetrgb{\Redrgb}\psBezier 2[1,2,3,4,21,20,1]
\pssetfillmode{no}\pssetrgb{\Blackrgb}\psBezier 2[1,2,3,4,21,20,1]
\figptssym 20=5,6/1,12/
\pssetfillmode{yes}\pssetrgb{\DarkOrangergb}\psBezier 2[1,5,6,7,21,20,1]
\pssetfillmode{no}\pssetrgb{\Blackrgb}\psBezier 2[1,5,6,7,21,20,1]}
%--------------
\def\sail{% Draws the rigging set
% Mast
\figpttra 21:=8/1.5,13/\figpttra 22:=8/3,13/\figpttra 23:=8/4.5,13/
\figpttra 21:=21/-0.6,11/\figpttra 22:=22/-0.6,11/
% Sail
\figptBezier 40::0.04[8,21,22,23]
\figptBezier 31::0.230[8,21,22,23]\figpttra 41:=31/1.56,11/\figptrot41:=41/31,2,12/
\figptBezier 32::0.333[8,21,22,23]\figpttra 42:=32/2.08,11/\figptrot42:=42/32,0,12/
\figptBezier 33::0.423[8,21,22,23]\figpttra 43:=33/1.85,11/\figptrot43:=43/33,-10,12/
\figptBezier 34::0.576[8,21,22,23]\figpttra 44:=34/1.58,11/\figptrot44:=44/34,-15,12/
\figptBezier 35::0.736[8,21,22,23]\figpttra 45:=35/1.25,11/\figptrot45:=45/35,-15,12/
\figptBezier 36::0.888[8,21,22,23]\figpttra 46:=36/0.85,11/\figptrot46:=46/36,-20,12/
% Wishbone
\figpttra 37:=32/0.2,11/\figpttra 38:=32/1.6,11/
\figpttra 37:=37/0.5,12/\figpttra 38:=38/0.05,12/
% Draw the sail
\pssetfillmode{yes}\pssetrgb{\Yellowrgb}
\psline[40,31,41]\psline[33,34,44,43]\psline[35,36,46,45]
\pssetfillmode{no}\pssetwidth{1.5}\pssetrgb{\Blackrgb}
\psline[31,41]\psline[33,43]\psline[34,44]\psline[35,45]\psline[36,46]
\psline[40,41,42,43,44,45,46,23]
% Draw the wishbone
\psBezier 1[32,37,38,42]\figptssym 37=37,38/8,12/\psBezier 1[32,37,38,42]
% Draw the mast
\pssetwidth{2}\psBezier 1[8,21,22,23]}
%--------------
\def\funboard{\boardposition\board\sailposition\sail}
%--------------
\def\axes{% Defines the local axes linked to the board
\figptBezier 0:$P$:\abscissa[15,16,17,18]
\figvectDBezier 21:1,\abscissa[15,16,17,18]\figvectU 21[21]\figpttra 1:=0/-1,21/
\figvectP 11[0,1]\figvectC 13(0,0,1)\figptrot 2:=1/0,90,13/\figvectP 12[0,2]}
%--------------
\def\jibe{% Draws the funboard at 3 successive positions
% First position
\def\boardposition{\def\abscissa{0.2}\axes}
\def\sailposition{\figpt 20:(0,0)\figptrot 11:=11/20,20,12/\figptrot 13:=13/20,20,12/
\figptsrot 12=12,13/20,20,11/}
\funboard
% Second position
\def\boardposition{\def\abscissa{0.55}\axes\figpt20:(0,0)\figptsrot12=12,13/20,-40,11/}
\def\sailposition{\figpt 20:(0,0)\figptsrot 11=11,12/20,70,13/
\figptrot 11:=11/20,-35,12/\figptrot 13:=13/20,-35,12/}
\funboard
% Third position
\def\boardposition{\def\abscissa{0.95}\axes}
\def\sailposition{\figpt 20:(0,0)\figptrot 11:=11/20,20,12/\figptrot 13:=13/20,20,12/
\figptsrot 12=12,13/20,-20,11/}
\funboard}
%--------------
\def\trajectory{% Draws the trajectory and the floating mark
\pssetwidth{2}\pssetarrowheadlength{0.6}\psarrowBezier [15,16,17,18]
\figpttraC 1:=19/1,0,0/\figpttraC 2:=19/0,1,0/
\pssetfillmode{yes}\pssetrgb{\Blackrgb}\pscirc 19,1,2(0.2)}
%--------------
\def\newfig{% Creates a new postscript file and a new figure
% 2. Creation of the postscript file
\NewPSfile{\MyPSfile}
\psbeginfig{\MyPSfile}
\trajectory\jibe
\psendfig
% 3. Writing text on the figure
\figvisu{\demo}{Figure \the\Figno\ - $(\psi, \theta)=(\Valpsi, \Valtheta)$}
{\figinsert{\MyPSfile}}
\bigskip
\centerline{\box\demo}}
%--------------
% Program
%--------------
\figinit{cm, realistic}
% 1. Trajectory control points
\figpt 15:(7,3)\figpt 16:(2,-6)\figpt 17:(-8,-5)\figpt 18:(-2,6)
%    Floating mark position
\figptcurvcenter 19::0.5[15,16,17,18]
% First figure
\figsettarget[19]\figsetobdist(20)
\def\Valpsi{-35}\def\Valtheta{20}\figsetview(\Valpsi, \Valtheta)
\newfig
% Second figure
\def\Valtheta{90}\figsetview(\Valpsi, \Valtheta)
\newfig
%=================================================================================
\vfill\eject
The only data are the four control points of the trajectory (points number 15, 16,
17 and 18). The floating mark (point 19) is the center of curvature computed
at the abscissa $t=0.5$. There is no good reason for that except that this shows
how to use the macro \Mac{figptcurvcenter}. These points are the only ones that
keep their values all along the process. The trajectory and the floating mark are
drawn by the macro \Mac{trajectory}.
\par
Two angles of observation are chosen ; the second one is a view from the top.
The numerical values are given in meters in true dimensions. The scale of the model
is $1\over 100$ since the first argument of \Mac{figinit} is {\tt cm}. Notice
the use the macro \Mac{figvectU} that gives the tangent vector the right length.
The target point is set to the floating mark and the observation distance is set
to 20 m (from this point). Since the eye point is rather close to the scenery,
we can observe the depth effect on the first figure: the funboard at the first
position is on the foreground and is bigger than the other ones.
\smallskip
The program that produces the two figures is given below. Each task corresponds
to a macro which is self explanatory ; the main program at the end is then very
short.
\par
$$\vbox{\fmtprog{-0.5pt}
\+{}\%--------------\cr
\+{}\% Definition of macros\cr
\+{}\%--------------\cr
\+{}\char`\\def\char`\\board$\{$\% Draws the board\cr
\+{}\% Main points\cr
\+{}\Mac{figpttra} 1:=0/0.22,13/\cr
\+{}\Mac{figpttra} 2:=0/0.3,11/\Mac{figpttra} 3:=0/2.7,11/\Mac{figpttra} 4:=0/2.7,11/\cr
\+{}\Mac{figpttra} 2:=2/0.5,12/\Mac{figpttra} 3:=3/0.3,12/\cr
\+{}\Mac{figpttra} 5:=0/0.3,11/\Mac{figpttra} 6:=0/2.5,11/\Mac{figpttra} 7:=0/2.5,11/\cr
\+{}\Mac{figpttra} 5:=5/0.4,12/\Mac{figpttra} 6:=6/0.2,12/\cr
\+{}\Mac{figpttra} 5:=5/0.07,13/\Mac{figpttra} 6:=6/0.06,13/\Mac{figpttra} 7:=7/0.04,13/\cr
\+{}\% Bottom point of the mast\cr
\+{}\Mac{figptBezier} 8::0.45[1,5,6,7]\Mac{figptorthoprojplane} 8:=8/1,12/\cr
\+{}\% Define the fin and draw it\cr
\+{}\Mac{figpttra} 20:=4/-0.30,11/\Mac{figpttra} 21:=4/-0.25,11/\cr
\+{}\Mac{figpttra} 22:=4/-0.05,11/\Mac{figpttra} 23:=4/-0.15,11/\cr
\+{}\Mac{figpttra} 21:=21/-0.15,13/\Mac{figpttra} 22:=22/-0.5,13/\cr
\+{}\Mac{pssetfillmode}$\{$yes$\}$\Mac{pssetrgb}$\{$\char`\\Blackrgb$\}$\Mac{psBezier} 1[20,21,22,23]\cr
\+{}\% Compute the symmetric part of the board and draw it\cr
\+{}\Mac{figptssym} 20=2,3/1,12/\cr
\+{}\Mac{pssetwidth}$\{$\Mac{defaultwidth}$\}$\Mac{pssetrgb}$\{$\char`\\Redrgb$\}$\Mac{psBezier} 2[1,2,3,4,21,20,1]\cr
\+{}\Mac{pssetfillmode}$\{$no$\}$\Mac{pssetrgb}$\{$\char`\\Blackrgb$\}$\Mac{psBezier} 2[1,2,3,4,21,20,1]\cr
\+{}\Mac{figptssym} 20=5,6/1,12/\cr
\+{}\Mac{pssetfillmode}$\{$yes$\}$\Mac{pssetrgb}$\{$\char`\\DarkOrangergb$\}$\Mac{psBezier} 2[1,5,6,7,21,20,1]\cr
\+{}\Mac{pssetfillmode}$\{$no$\}$\Mac{pssetrgb}$\{$\char`\\Blackrgb$\}$\Mac{psBezier} 2[1,5,6,7,21,20,1]$\}$\cr
\+{}\%--------------\cr
\+{}\char`\\def\char`\\sail$\{$\% Draws the rigging set\cr
\+{}\% Mast\cr
\+{}\Mac{figpttra} 21:=8/1.5,13/\Mac{figpttra} 22:=8/3,13/\Mac{figpttra} 23:=8/4.5,13/\cr
\+{}\Mac{figpttra} 21:=21/-0.6,11/\Mac{figpttra} 22:=22/-0.6,11/\cr
\+{}\% Sail\cr
\+{}\Mac{figptBezier} 40::0.04[8,21,22,23]\cr
\+{}\Mac{figptBezier} 31::0.230[8,21,22,23]\Mac{figpttra} 41:=31/1.56,11/\Mac{figptrot}41:=41/31,2,12/\cr
\+{}\Mac{figptBezier} 32::0.333[8,21,22,23]\Mac{figpttra} 42:=32/2.08,11/\Mac{figptrot}42:=42/32,0,12/\cr
\+{}\Mac{figptBezier} 33::0.423[8,21,22,23]\Mac{figpttra} 43:=33/1.85,11/\Mac{figptrot}43:=43/33,-10,12/\cr
\+{}\Mac{figptBezier} 34::0.576[8,21,22,23]\Mac{figpttra} 44:=34/1.58,11/\Mac{figptrot}44:=44/34,-15,12/\cr
\+{}\Mac{figptBezier} 35::0.736[8,21,22,23]\Mac{figpttra} 45:=35/1.25,11/\Mac{figptrot}45:=45/35,-15,12/\cr
\+{}\Mac{figptBezier} 36::0.888[8,21,22,23]\Mac{figpttra} 46:=36/0.85,11/\Mac{figptrot}46:=46/36,-20,12/\cr
}$$
\null\vfill
$$\vbox{\fmtprog{-0.5pt}
\+{}\% Wishbone\cr
\+{}\Mac{figpttra} 37:=32/0.2,11/\Mac{figpttra} 38:=32/1.6,11/\cr
\+{}\Mac{figpttra} 37:=37/0.5,12/\Mac{figpttra} 38:=38/0.05,12/\cr
\+{}\% Draw the sail\cr
\+{}\Mac{pssetfillmode}$\{$yes$\}$\Mac{pssetrgb}$\{$\char`\\Yellowrgb$\}$\cr
\+{}\Mac{psline}[40,31,41]\Mac{psline}[33,34,44,43]\Mac{psline}[35,36,46,45]\cr
\+{}\Mac{pssetfillmode}$\{$no$\}$\Mac{pssetwidth}$\{$1.5$\}$\Mac{pssetrgb}$\{$\char`\\Blackrgb$\}$\cr
\+{}\Mac{psline}[31,41]\Mac{psline}[33,43]\Mac{psline}[34,44]\Mac{psline}[35,45]\Mac{psline}[36,46]\cr
\+{}\Mac{psline}[40,41,42,43,44,45,46,23]\cr
\+{}\% Draw the wishbone\cr
\+{}\Mac{psBezier} 1[32,37,38,42]\Mac{figptssym} 37=37,38/8,12/\Mac{psBezier} 1[32,37,38,42]\cr
\+{}\% Draw the mast\cr
\+{}\Mac{pssetwidth}$\{$2$\}$\Mac{psBezier} 1[8,21,22,23]$\}$\cr
\+{}\%--------------\cr
\+{}\char`\\def\char`\\funboard$\{$\char`\\boardposition\char`\\board\char`\\sailposition\char`\\sail$\}$\cr
\+{}\%--------------\cr
\+{}\char`\\def\char`\\axes$\{$\% Defines the local axes linked to the board\cr
\+{}\Mac{figptBezier} 0:\$P\$:\char`\\abscissa[15,16,17,18]\cr
\+{}\Mac{figvectDBezier} 21:1,\char`\\abscissa[15,16,17,18]\Mac{figvectU} 21[21]\Mac{figpttra} 1:=0/-1,21/\cr
\+{}\Mac{figvectP} 11[0,1]\Mac{figvectC} 13(0,0,1)\Mac{figptrot} 2:=1/0,90,13/\Mac{figvectP} 12[0,2]$\}$\cr
\+{}\%--------------\cr
\+{}\char`\\def\char`\\jibe$\{$\% Draws the funboard at 3 successive positions\cr
\+{}\% First position\cr
\+{}\char`\\def\char`\\boardposition$\{$\char`\\def\char`\\abscissa$\{$0.2$\}$\char`\\axes$\}$\cr
\+{}\char`\\def\char`\\sailposition$\{$\Mac{figpt} 20:(0,0)\Mac{figptrot} 11:=11/20,20,12/\Mac{figptrot} 13:=13/20,20,12/\cr
\+{}\Mac{figptsrot} 12=12,13/20,20,11/$\}$\cr
\+{}\char`\\funboard\cr
\+{}\% Second position\cr
\+{}\char`\\def\char`\\boardposition$\{$\char`\\def\char`\\abscissa$\{$0.55$\}$\char`\\axes\Mac{figpt}20:(0,0)\Mac{figptsrot}12=12,13/20,-40,11/$\}$\cr
\+{}\char`\\def\char`\\sailposition$\{$\Mac{figpt} 20:(0,0)\Mac{figptsrot} 11=11,12/20,70,13/\cr
\+{}\Mac{figptrot} 11:=11/20,-35,12/\Mac{figptrot} 13:=13/20,-35,12/$\}$\cr
\+{}\char`\\funboard\cr
\+{}\% Third position\cr
\+{}\char`\\def\char`\\boardposition$\{$\char`\\def\char`\\abscissa$\{$0.95$\}$\char`\\axes$\}$\cr
\+{}\char`\\def\char`\\sailposition$\{$\Mac{figpt} 20:(0,0)\Mac{figptrot} 11:=11/20,20,12/\Mac{figptrot} 13:=13/20,20,12/\cr
\+{}\Mac{figptsrot} 12=12,13/20,-20,11/$\}$\cr
\+{}\char`\\funboard$\}$\cr
\+{}\%--------------\cr
\+{}\char`\\def\char`\\trajectory$\{$\% Draws the trajectory and the floating mark\cr
\+{}\Mac{pssetwidth}$\{$2$\}$\Mac{pssetarrowheadlength}$\{$0.6$\}$\Mac{psarrowBezier} [15,16,17,18]\cr
\+{}\Mac{figpttraC} 1:=19/1,0,0/\Mac{figpttraC} 2:=19/0,1,0/\cr
\+{}\Mac{pssetfillmode}$\{$yes$\}$\Mac{pssetrgb}$\{$\char`\\Blackrgb$\}$\Mac{pscirc} 19,1,2(0.2)$\}$\cr
\+{}\%--------------\cr
\+{}\char`\\def\char`\\newfig$\{$\% Creates a new postscript file and a new figure\cr
\+{}\% 2. Creation of the postscript file\cr
\+{}\char`\\NewPSfile$\{$\char`\\MyPSfile$\}$\cr
\+{}\Mac{psbeginfig}$\{$\char`\\MyPSfile$\}$\cr
\+{}\char`\\trajectory\char`\\jibe\cr
\+{}\Mac{psendfig}\cr
\+{}\% 3. Writing text on the figure\cr
\+{}\Mac{figvisu}$\{$\char`\\demo$\}$$\{$Figure \char`\\the\char`\\Figno\char`\\ - \$(\char`\\psi, \char`\\theta)=(\char`\\Valpsi, \char`\\Valtheta)\$$\}$\cr
\+{}$\{$\Mac{figinsert}$\{$\char`\\MyPSfile$\}$$\}$\cr
\+{}\char`\\bigskip\cr
\+{}\char`\\centerline$\{$\char`\\box\char`\\demo$\}$$\}$\cr
}$$
\null\vfill
$$\vbox{\fmtprog{-0.5pt}
\+{}\%--------------\cr
\+{}\% Program\cr
\+{}\%--------------\cr
\+{}\Mac{figinit}$\{$cm, realistic$\}$\cr
\+{}\% 1. Trajectory control points\cr
\+{}\Mac{figpt} 15:(7,3)\Mac{figpt} 16:(2,-6)\Mac{figpt} 17:(-8,-5)\Mac{figpt} 18:(-2,6)\cr
\+{}\%    Floating mark position\cr
\+{}\Mac{figptcurvcenter} 19::0.5[15,16,17,18]\cr
\+{}\% First figure\cr
\+{}\Mac{figsettarget}[19]\Mac{figsetobdist}(20)\cr
\+{}\char`\\def\char`\\Valpsi$\{$-35$\}$\char`\\def\char`\\Valtheta$\{$20$\}$\Mac{figsetview}(\char`\\Valpsi, \char`\\Valtheta)\cr
\+{}\char`\\newfig\cr
\+{}\% Second figure\cr
\+{}\char`\\def\char`\\Valtheta$\{$90$\}$\Mac{figsetview}(\char`\\Valpsi, \char`\\Valtheta)\cr
\+{}\char`\\newfig\cr
}$$
\par\noindent
Remark: In the macros \Mac{boardposition} and \Mac{sailposition}, in order to rotate
the basis  $(\vec{u}_1,\vec{u}_2,\vec{u}_3)$, we have used the macro \Mac{figptrot},
which acts on a point, to rotate a vector, after having the rotation axis through
the origin. This has the advantage that the result is obtained in a straightforward
way. However, this is not mathematically consistent since the object created is
a point and not a vector. In this macro package, the only consequence would be on
the macro \Mac{figshowpts} which show every point created so far, but not the vectors.
If we are purist, we can generate vectors by creating intermediate points.
For example, in the first call to \Mac{sailposition}, we can replace
{\medskip\leftskip5mm\leavevmode
\vbox{\fmtprog{0pt}
\+{}\Mac{figpt} 20:(0,0)\Mac{figptrot} 11:=11/20,20,12/\cr
}\medskip}
\par\noindent
by
{\medskip\leftskip5mm\leavevmode
\vbox{\fmtprog{0pt}
\+{}\Mac{figpt} 20:(0,0)\Mac{figpttra} 21:=20/1,11/\Mac{figptrot} 21:=21/20,20,12/\Mac{figvectP} 11[20,21]\cr
}\medskip}
%%%%%%%%%%%%%%%%%%%%%%%%%%%%%%%%%%%%%%%%%%%%%%%%%%%%%%%%%%%%%%%%%%%%%%%%%%%
\sect{\UsingMac}
This macro-package is ready to be used with plain \TeX\ and LA\TeX\ under any
Unix system. With TeXtures on a MacIntosh running MacOS 9 or less, the macro file
{\tt \figforTeX}\ must be modified as explained in its header, in order
to handle postscript files correctly.
\smallskip\noindent
Remark concerning LA\TeX :\par
\item{} If a \eword{french} package is used, then the macro file
{\tt \figforTeX}\ must be loaded \eword{after} the statement
{\tt\CtrlSq{begin}$\{$document$\}$}. In case of trouble with another package,
please try this as a first workaround. This gives the following structure:
$$\vbox{\fmtprog{0pt}
\+{}\char`\\documentclass[12pt]$\{$article$\}$   \% for example\cr
\+{}\char`\\usepackage$\{$french$\}$             \% or \char`\\usepackage[french]$\{$babel$\}$\cr
\+{}\cr
\+{}\char`\\begin$\{$document$\}$\cr
\+{}\char`\\input$\{$fig4tex.tex$\}$\cr
\+{}       .\cr
\+{}       .\cr
\+{}       .\cr
\+{}\char`\\end$\{$document$\}$\cr
}$$
\smallskip\noindent
{\bf Bug report}\par
The authors are interested to receive back any information about trouble in
using this macro-package. Please mail your message to
{\tt Yvon.Lafranche@univ-rennes1.fr} or {\tt Daniel.Martin@univ-rennes1.fr}.
%%%%%%%%%%%%%%%%%%%%%%%%%%%%%%%%%%%%%%%%%%%%%%%%%%%%%%%%%%%%%%%%%%%%%%%%%%%
\vfill\eject
\ifodd\pageno\else\null\vfill\eject\fi
\sect{\LstUMac}
\medskip
An HTML version of this list is available at:\hfill\break
{\tt http://www.maths.univ-rennes1.fr/$\sim$lafranch/fig4tex/ReferenceGuide.html}
\bigskip
\raggedbottom
{\fmtlist{-0.1pt}%\font\fmac=cmitt10\def\Mac#1{{\tt \char`\\\fmac#1}}
\def\soul#1{$\underline{\hbox{#1}}$}\def\MacS#1{{\tt\char`\\\soul{#1}}}
% Pour generer ce qui suit : csh crlstmac 1_6 1_6_1
%=================================================================================
\+{}                       Geometrical macros - Mute macros\cr
\+{}                       --------------------------------\cr
\+{} \cr
\+{}  --------- Control macros\cr
\+{}\MacS{figinit}$\{$ScaleFactorUnit$\}$ or \Mac{figinit}$\{$ScaleFactorUnit, 2D$\}$\cr
\+{}\MacS{figinit}$\{$ScaleFactorUnit, X$\}$ with X in $\{$3D, cavalier, orthogonal, realistic$\}$\cr
\+{}  Initialization before the creation of a new figure. It is necessary to call this\cr
\+{}  macro in case of successive figures. No check is performed on the arguments.\cr
\+{}  First argument : Choice of the unit and the scale factor.\cr
\+{}   The unit can be one those defined in the TeX Book, namely: pt (TeX point),\cr
\+{}   pc (pica, 1pc = 12pt), in (inch, 1in = 72.27pt), bp (big point, 72bp = 1in),\cr
\+{}   cm (centimeter, 2.54cm = 1in), mm (millimeter, 10mm = 1cm), dd (point didot,\cr
\+{}   1157dd = 1238pt), cc (cicero, 1cc = 12dd), sp (scaled point, 65536sp = 1pt)\cr
\+{}   By default, pt is assumed and the scale factor is 1. For example, \Mac{figinit}$\{$in$\}$\cr
\+{}   is equivalent to \Mac{figinit}$\{$2.54cm$\}$.\cr
\+{}  Second argument (optional) : Choice of the space dimension and, in 3D, the\cr
\+{}   projection type.\cr
\+{}   If this argument is equal to 2D or absent, then geometry in the plane is assumed,\cr
\+{}   otherwise geometry in three dimensions is performed.\cr
\+{}   Moreover, if this argument is equal to:\cr
\+{}    . orthogonal, then the orthogonal projection is used,\cr
\+{}    . realistic, then the realistic projection is used,\cr
\+{}    . 3D or cavalier (or anything else), then the cavalier projection is used.\cr
\+{}\MacS{figinsert}$\{$FileName$\}$ or \Mac{figinsert}$\{$FileName, ScaleFactor$\}$\cr
\+{}  Insertion of the postscript file FileName in the page, scaled by ScaleFactor\cr
\+{}  which  by default equals to 1.\cr
\+{}\MacS{figscan} FileName(HX,HY)\cr
\+{}  Draws a rectangular grid with horizontal and vertical steps HX and HY, with\cr
\+{}  numerical values corresponding to the bounding box read in the file FileName.\cr
\+{}\MacS{figsetobdist} (Dist)\cr
\+{}  Definition of the observation distance Dist for the realistic projection.\cr
\+{}  Must be called before step 2 (\Mac{psbeginfig}) and 3 (\Mac{figvisu}).\cr
\+{}\MacS{figsettarget} [Pt]\cr
\+{}  Definition of the target point Pt for the realistic projection.\cr
\+{}  Must be called before step 2 (\Mac{psbeginfig}) and 3 (\Mac{figvisu}).\cr
\+{}\MacS{figsetview}(Psi) or \Mac{figsetview}(Psi, Theta) or \Mac{figsetview}(Psi, Lambda)\cr
\+{}  Setting the parameters of the projection.\cr
\+{}  Must be called before step 2 (\Mac{psbeginfig}) and 3 (\Mac{figvisu}).\cr
\+{}  Psi and Theta denote angles to be given in degrees, Lambda denotes a real number.\cr
\+{}  The cavalier projection is defined by the angle Psi and Lambda which is a depth reduction\cr
\+{}  coefficient usually lying between 0 and 1. In this case, values of Psi lying in [0,180]\cr
\+{}  correspond to a view from above and to a view from beneath for values lying in [-180,0].\cr
\+{}  For the other projections, the observation direction is defined by the longitude Psi and\cr
\+{}  the latitude Theta, with (Psi, Theta) = (0, 0) corresponding to the Ox line (see also\cr
\+{}  \Mac{figsettarget} and \Mac{figsetobdist}).\cr
\+{}  The second argument (Theta or Lambda) is optional.\cr
\+{}  The default values are Psi = 40 degrees, Theta = 25 degrees, Lambda = 0.5.\cr
\+{}\MacS{figshowsettings}\cr
\+{}  Prints to the terminal the current settings.\cr
\+{} \cr
\+{}\MacS{figvisu}$\{$Vbox$\}$$\{$Caption$\}$$\{$Commands$\}$\cr
\+{}  Creation of a Vbox containing the figure and the legends defined by the Commands,\cr
\+{}  with a Caption centered below. The box must be previously allocated by the user.\cr
\+{}  Commands and Caption can be void.\cr
\+{} \cr
\+{}  --------- Basic macros\cr
\+{}\MacS{figpt} NewPt :Text(X,Y)\cr
\+{}\MacS{figpt} NewPt :Text(X,Y,Z)\cr
\+{}  Definition of the point NewPt whose coordinates are (X,Y) or (X,Y,Z), with a joined Text.\cr
\+{}  If Z is missing, then Z=0 is assumed.\cr
\+{}  The default Text is \$A\_i\$ where i is the number of the point (see \Mac{figsetptname}).\cr
\+{}\MacS{figptbary} NewPt :Text[Pt1,... ,PtN ; Coef1,... ,CoefN]\cr
\+{}  Barycenter or centroid (with a joined Text) of N points bearing integer coefficients\cr
\+{}\MacS{figptbaryR} NewPt :Text[Pt1,... ,PtN ; Coef1,... ,CoefN]\cr
\+{}  Barycenter or centroid (with a joined Text) of N points bearing real coefficients\cr
\+{}\MacS{figptcopy} NewPt :Text/Pt/\cr
\+{}  Definition of NewPt, with an associated Text, with the same coordinates as Pt.\cr
\+{}\MacS{figgetangle} \char`\\Value [Center,Pt1,Pt2]\cr
\+{}\MacS{figgetangle} \char`\\Value [Center,Pt1,Pt2,Pt3]\cr
\+{}  Computation of the value \char`\\Value (in degrees) of the oriented angle (CP1, CP2), where\cr
\+{}  C=Center, P1=Pt1, P2=Pt2. \char`\\Value is a macro whose name is chosen by the user and can\cr
\+{}  then be used as a symbolic numerical constant.\cr
\+{}  In 3D, the plane is oriented so that the angle is measured counterclockwise around the\cr
\+{}  vector CP1 x CP3, where P3=Pt3. Notice that C, P1, P2 and P3 must be coplanar, but P3\cr
\+{}  must not lie on the line (C,P1).\cr
\+{}\MacS{figgetdist}\char`\\Value[Pt1,Pt2]\cr
\+{}  Computation of the euclidian distance \char`\\Value between the points Pt1 and Pt2.\cr
\+{}  \char`\\Value is a macro whose name is chosen by the user and can then be used as a symbolic\cr
\+{}  numerical constant.\cr
\+{}\MacS{figvectC} NewVect (X,Y)\cr
\+{}\MacS{figvectC} NewVect (X,Y,Z)\cr
\+{}  Definition of the vector NewVect with components (X,Y) or (X,Y,Z).\cr
\+{}  If Z is missing, then Z=0 is assumed.\cr
\+{}\MacS{figvectN} NewVect [Pt1,Pt2]\cr
\+{}\MacS{figvectN} NewVect [Pt1,Pt2,Pt3]\cr
\+{}  Definition of the vector NewVect which is:\cr
\+{}  . in 2D, the vector with origin point Pt1 and end point Pt2 rotated by pi/2\cr
\+{}    (so normal to [Pt1,Pt2]),\cr
\+{}  . in 3D, the vector product P1P2 / ||P1P2|| x P1P3 / ||P1P3||, ie a vector normal to\cr
\+{}    the plane defined by the 3 points and so that (P1P2, P1P3, NewVect) is a positively\cr
\+{}    oriented basis.\cr
\+{}\MacS{figvectNV} NewVect [Vector]\cr
\+{}\MacS{figvectNV} NewVect [Vector1, Vector2]\cr
\+{}  Definition of the vector NewVect which is:\cr
\+{}  . in 2D, the vector Vector rotated by pi/2\cr
\+{}  . in 3D, the vector product Vector1 / ||Vector1|| x Vector2 / ||Vector2|| so that\cr
\+{}  (Vector1, Vector2, NewVect) is a positively oriented basis.\cr
\+{}\MacS{figvectP} NewVect [Pt1,Pt2]\cr
\+{}  Definition of the vector NewVect with origin point Pt1 and end point Pt2.\cr
\+{}\MacS{figvectU} NewVect [Vector]\cr
\+{}  Definition of the unitary vector NewVect which is Vector normalized according to\cr
\+{}  the unit and the scale factor chosen by the user (see \Mac{figinit}).\cr
\+{} \cr
\+{}  --------- Macros for elementary geometry\cr
\+{}  --- Transformation macros (one result) ---\cr
\+{}\MacS{figpthom} NewPt :Text= Pt /Center, Ratio/\cr
\+{}  Image NewPt of point Pt by the homothety of center Center and of ratio Ratio\cr
\+{}  with a joined Text\cr
\+{}\MacS{figptrot} NewPt :Text= Pt /Center, Angle/\cr
\+{}\MacS{figptrot} NewPt :Text= Pt /Center, Angle, Vector/\cr
\+{}  Image NewPt (with a joined Text) of point Pt by the rotation defined:\cr
\+{}  . in 2D, by the center Center and the angle Angle (in degrees),\cr
\+{}  . in 3D, by the axis (Center, Vector) and the angle Angle (in degrees).\cr
\+{}    Pt is rotated by Angle around the axis. The sense of rotation is such that\cr
\+{}    (CP, CP', Vector) is a positively oriented basis, where C=Center, P=Pt, P'=NewPt.\cr
\+{}\MacS{figptsym} NewPt :Text= Pt /LinePt1, LinePt2/\cr
\+{}\MacS{figptsym} NewPt :Text= Pt /PlanePt, NormalVector/\cr
\+{}  Image NewPt (with a joined Text) of point Pt by the orthogonal symmetry\cr
\+{}  with respect to:\cr
\+{}  . in 2D, the line defined by the points LinePt1 and LinePt2,\cr
\+{}  . in 3D, the plane defined by the point PlanePt and the vector NormalVector normal\cr
\+{}  to the plane.\cr
\+{}\MacS{figpttra} NewPt :Text= Pt /Lambda, Vector/\cr
\+{}  Image NewPt of point Pt by the translation of vector Lambda * Vector\cr
\+{}  with a joined Text.\cr
\+{}\MacS{figpttraC} NewPt :Text= Pt /X,Y/\cr
\+{}\MacS{figpttraC} NewPt :Text= Pt /X,Y,Z/\cr
\+{}  Image NewPt of point Pt by the translation of vector (X,Y) in 2D, (X,Y,Z) in 3D\cr
\+{}  with a joined Text.\cr
\+{}\MacS{figptorthoprojline} NewPt :Text= Pt /LinePt1, LinePt2/\cr
\+{}  Image NewPt of point Pt by the orthogonal projection onto the line (LinePt1,LinePt2),\cr
\+{}  with a joined Text.\cr
\+{}\MacS{figptorthoprojplane} NewPt :Text= Pt /PlanePt, NormalVector/\cr
\+{}  Image NewPt of point Pt by the orthogonal projection onto the plane defined by\cr
\+{}  the point PlanePt and the normal vector NormalVector, with a joined Text.\cr
\+{} \cr
\+{}  --- Transformation macros (multiple result) ---\cr
\+{}\MacS{figptshom} NewPt1 = Pt1, Pt2, ..., PtN /Center, Ratio/\cr
\+{}  Images NewPt1, NewPt1+1,..., NewPt1+(N-1) of points Pt1, Pt2, ..., PtN\cr
\+{}  by the homothety of center Center and of ratio Ratio.\cr
\+{}  See following information at \Mac{figptstra}.\cr
\+{}\MacS{figptsrot} NewPt1 = Pt1, Pt2, ..., PtN /Center, Angle/\cr
\+{}\MacS{figptsrot} NewPt1 = Pt1, Pt2, ..., PtN /Center, Angle, Vector/\cr
\+{}  Images NewPt1, NewPt1+1,..., NewPt1+(N-1) of points Pt1, Pt2, ..., PtN\cr
\+{}  by the rotation defined:\cr
\+{}  . in 2D, by the center Center and the angle Angle (in degrees),\cr
\+{}  . in 3D, by the axis (Center, Vector) and the angle Angle (in degrees). Each point\cr
\+{}    Pti is rotated by Angle around the axis. The sense of rotation is such that\cr
\+{}    (CP, CP', Vector) is a positively oriented basis, where C=Center, P=Pti, P'=NewPti.\cr
\+{}  See following information at \Mac{figptstra}.\cr
\+{}\MacS{figptssym} NewPt1 = Pt1, Pt2, ..., PtN /LinePt1, LinePt2/\cr
\+{}\MacS{figptssym} NewPt1 = Pt1, Pt2, ..., PtN /PlanePt, NormalVector/\cr
\+{}  Images NewPt1, NewPt1+1,..., NewPt1+(N-1) of points Pt1, Pt2, ..., PtN\cr
\+{}  by the orthogonal symmetry with respect to:\cr
\+{}  . in 2D, the line defined by the points LinePt1 and LinePt2,\cr
\+{}  . in 3D, the plane defined by the point PlanePt and the vector NormalVector normal\cr
\+{}  to the plane. See following information at \Mac{figptstra}.\cr
\+{}\MacS{figptstra} NewPt1 = Pt1, Pt2, ..., PtN /Lambda, Vector/\cr
\+{}  Images NewPt1, NewPt1+1,..., NewPt1+(N-1) of points Pt1, Pt2, ..., PtN\cr
\+{}  by the translation of vector Lambda * Vector\cr
\+{}  Text eventually previously associated with the result points is lost.\cr
\+{}  The result points NewPti have successive numbers.\cr
\+{}  The data points Pti can be given in any order.\cr
\+{}  WARNING : Let Ir (resp. Id) the set of the result points numbers (resp. the set\cr
\+{}  of the given points numbers). Let J the intersection of Ir and Id.\cr
\+{}  1) Ir = Id, or J is empty, or NewPt1 does not belong to J: no problem.\cr
\+{}  2) Otherwise, the result given by the macro MAY BE WRONG, at least partially:\cr
\+{}     this is because, in this case, NewPt1 belongs to J, and the given points are\cr
\+{}     taken into account sequentially, beginning from the first element in the list.\cr
\+{}\MacS{figptsorthoprojline} NewPt1 = Pt1, Pt2, ..., PtN /LinePt1, LinePt2/\cr
\+{}  Images NewPt1, NewPt1+1,..., NewPt1+(N-1) of points Pt1, Pt2, ..., PtN\cr
\+{}  by the orthogonal projection onto the line (LinePt1, LinePt2).\cr
\+{}\MacS{figptsorthoprojplane} NewPt1 = Pt1, Pt2, ..., PtN /PlanePt, NormalVector/\cr
\+{}  Images NewPt1, NewPt1+1,..., NewPt1+(N-1) of points Pt1, Pt2, ..., PtN\cr
\+{}  by the orthogonal projection onto the plane defined by the point PlanePt and the\cr
\+{}  normal vector NormalVector.\cr
\+{} \cr
\+{}  --- Geometrical construction macros ---\cr
\+{}\MacS{figptinterlines} NewPt :Text[LinePt1,Vector1; LinePt2,Vector2]\cr
\+{}  Intersection of the line defined by the point LinePt1 and the vector Vector1\cr
\+{}              and the line defined by the point LinePt2 and the vector Vector2\cr
\+{}  with a joined Text\cr
\+{}\MacS{figptinterlineplane} NewPt :Text[LinePt,Vector; PlanePt,NormalVector]\cr
\+{}  Intersection of the line defined by the point LinePt and the vector Vector\cr
\+{}              and the plane defined by the point PlanePt and the normal vector NormalVector\cr
\+{}  with a joined Text.\cr
\+{}\MacS{figptendnormal} NewPt :Text: Length,Lambda [Pt1,Pt2]\cr
\+{}  Endpoint (with a joined Text) of the "exterior normal" to the segment [Pt1, Pt2].\cr
\+{}  The length of the normal vector is Length. Lambda is the barycentric coordinate\cr
\+{}  of the origin of the normal with respect to the segment [Pt1, Pt2], which sets the\cr
\+{}  position of the vector along [Pt1, Pt2]. In 3D-mode, it works only in the plane Z=0.\cr
\+{}\MacS{figptsintercirc} NewPt1 [Center1,Radius1 ; Center2,Radius2]\cr
\+{}  Intersections NewPt1 and NewPt2 (=NewPt1+1) of the two circles defined by their\cr
\+{}  center and radius, (Center1,Radius1) and (Center2,Radius2).\cr
\+{}  NewPt1 and NewPt2 must be different from Center1 and Center2.\cr
\+{}  NewPt1 and NewPt2 are ordered so that the angle (NewPt1, Center1, NewPt2) is positive.\cr
\+{}  If the two circles do not intersect, then NewPt1=Center1 and NewPt2=Center2.\cr
\+{}  In 3D-mode, it works only in the plane Z=0.\cr
\+{} \cr
\+{}  --- Macros related to the triangle ---\cr
\+{}\MacS{figptcircumcenter} NewPt :Text[Pt1,Pt2,Pt3]\cr
\+{}  Center NewPt of the circumscribed circle to the triangle (Pt1,Pt2,Pt3) with a joined Text.\cr
\+{}\MacS{figptinscribedcenter} NewPt :Text[Pt1,Pt2,Pt3]\cr
\+{}  Center NewPt of the inscribed circle to the triangle (Pt1,Pt2,Pt3) with a joined Text.\cr
\+{}\MacS{figptorthocenter} NewPt :Text[Pt1,Pt2,Pt3]\cr
\+{}  Orthocenter NewPt of the triangle (Pt1,Pt2,Pt3) with a joined Text.\cr
\+{} \cr
\+{}  --------- Macros related to arcs and curves\cr
\+{}\MacS{figptcirc} NewPt :Text: Center;Radius (Angle)\cr
\+{}\MacS{figptcirc} NewPt :Text: Center,Pt1,Pt2;Radius (Angle)\cr
\+{}  Creation of the point NewPt, with an associated Text, on the circle defined by its Center,\cr
\+{}  and its Radius. The position of the point is set by the Angle given in degrees.\cr
\+{}  In 3D, the circle is lying in the plane (Center,Pt1,Pt2) = (C,P1,P2) and\cr
\+{}  the Angle is measured counterclockwise around the vector CP1 x CP2, starting from\cr
\+{}  the half-line (C,P1).\cr
\+{}\MacS{figptell} NewPt :Text: Center;XRad,YRad (Angle,Inclination)\cr
\+{}  Creation of the point NewPt, with an associated Text, on the ellipse defined by Center,\cr
\+{}  XRad, YRad and Inclination. Inclination is the rotation angle of the local axes with\cr
\+{}  respect to the paper sheet. The position of the point is set by the parametrization\cr
\+{}  Angle given in local axes. The two angles are given in degrees.\cr
\+{}  In 3D-mode, it works only in the plane Z=0.\cr
\+{}\MacS{figptellP} NewPt :Text: Center,PtAxis1,PtAxis2 (Angle)\cr
\+{}  Creation of the point NewPt, with an associated Text, on the ellipse defined by its Center,\cr
\+{}  and the end points of its axes, A1=PtAxis1 and A2=PtAxis2. The local axes are then\cr
\+{}  defined by the basis (CA1, CA2). The position of the point is set by the parametrization\cr
\+{}  Angle given in degrees, starting from the half-line (C,A1) and measured counterclockwise\cr
\+{}  around the vector CA1 x CA2.\cr
\+{}\MacS{figptBezier} NewPt :Text: t [Pt1,Pt2,Pt3,Pt4]\cr
\+{}  Computes the point NewPt (with a joined Text) lying on the cubic Bezier curve defined\cr
\+{}  by the four control points Pt1, Pt2, Pt3 and Pt4, for the parameter value t.\cr
\+{}  We recall that if t=0, NewPt=Pt1 and if t=1, NewPt=Pt4.\cr
\+{}\MacS{figptscontrol} NewPt1 [Pt1,Pt2,Pt3,Pt4]\cr
\+{}  Computes the two control points NewPt1 and NewPt2 (=NewPt1+1) so that the cubic Bezier\cr
\+{}  curve defined by the control points Pt1, NewPt1, NewPt2 and Pt4 interpolates the\cr
\+{}  four points Pt1, Pt2, Pt3 and Pt4 with respective parameter values of 0, 1/3, 2/3 and 1.\cr
\+{}\MacS{figptcurvcenter} NewPt :Text: t [Pt1,Pt2,Pt3,Pt4]\cr
\+{}  Curvature center NewPt (with a joined Text) at the point lying on the cubic Bezier curve\cr
\+{}  defined by the four control points Pt1, Pt2, Pt3 and Pt4, for the parameter value t.\cr
\+{}\MacS{figvectDBezier} NewVect : n, t [Pt1,Pt2,Pt3,Pt4]\cr
\+{}  Computes the vector NewVect corresponding to the derivative of order n of the cubic Bezier\cr
\+{}  curve defined by the four control points Pt1, Pt2, Pt3 and Pt4, for the parameter value t.\cr
\+{}  The order n must be equal to 1 or 2.\cr
\+{} \cr
\+{}                       Writing macros - Non-mute macros\cr
\+{}                       --------------------------------\cr
\+{} \cr
\+{}  --------- Control macros\cr
\+{}\MacS{figcoord}$\{$NDec$\}$\cr
\+{}  To write the coordinates of a point. To be used in the joined text argument\cr
\+{}  of the macros that create points and the non-mute macros \char`\\figwrit* .\cr
\+{}  Only NDec decimals are printed.\cr
\+{}\MacS{figsetmark}$\{$Mark$\}$\cr
\+{}  Definition of the point marker to be written, for example a point (.) or \$\char`\\bullet\$.\cr
\+{}  By default, nothing is written.\cr
\+{}\MacS{figsetptname}$\{$Name:  $\}$\cr
\+{}  Sets the new default name for the points created.\cr
\+{}  The default name for the point i is \$A\_i\$: \Mac{figsetptname}$\{$\$X\^{}$\{$(\#1)$\}$\$$\}$ changes it\cr
\+{}  to \$X\^{}$\{$(i)$\}$\$.\cr
\+{}\MacS{figsetroundcoord}$\{$yes$\}$ or \Mac{figsetroundcoord}$\{$no$\}$\cr
\+{}  Switches on rounding of decimals printed with \Mac{figcoord}.\cr
\+{}\MacS{figshowpts}[Nmin, Nmax]\cr
\+{}  Shows on the figure the location of every point defined since the beginning\cr
\+{}  of the session, whose number lies in the interval [Nmin, Nmax].\cr
\+{}  Writes a bullet at each location point along with the number of the point or\cr
\+{}  the joined text if any.\cr
\+{}  CAUTION :\cr
\+{}  If Nmax-Nmin is too large, an error message "! TeX capacity exceeded" may occur.\cr
\+{} \cr
\+{}  --------- Writing macros\cr
\+{}\MacS{figwrite}[Pt1, Pt2, ..., PtN]$\{$Text$\}$\cr
\+{}  Writing a Text after Pt1, Pt2, ..., PtN according to TeX's alignment.\cr
\+{}\MacS{figwritec}[Pt1, Pt2, ..., PtN]$\{$Text$\}$\cr
\+{}  Writing a Text vertically and horizontally centered at Pt1, Pt2, ..., PtN.\cr
\+{}\MacS{figwritep}[Pt1, Pt2, ..., PtN]\cr
\+{}  Writing the point marker at the locations defined by Pt1, Pt2, ..., PtN.\cr
\+{}\MacS{figwritew} Pt1, Pt2, ..., PtN :Text(Distance)\cr
\+{}\MacS{figwritee} Pt1, Pt2, ..., PtN :Text(Distance)\cr
\+{}\MacS{figwriten} Pt1, Pt2, ..., PtN :Text(Distance)\cr
\+{}\MacS{figwrites} Pt1, Pt2, ..., PtN :Text(Distance)\cr
\+{}  Writing a point marker at the locations defined by Pt1, Pt2, ..., PtN, with a Text\cr
\+{}  placed, with respect to each point, at a given Distance from each point towards\cr
\+{}  the west, the east, the north or the south.\cr
\+{}\MacS{figwritenw} Pt1, Pt2, ..., PtN :Text(Distance)\cr
\+{}\MacS{figwritesw} Pt1, Pt2, ..., PtN :Text(Distance)\cr
\+{}\MacS{figwritene} Pt1, Pt2, ..., PtN :Text(Distance)\cr
\+{}\MacS{figwritese} Pt1, Pt2, ..., PtN :Text(Distance)\cr
\+{}  Writing a point marker at the locations defined by Pt1, Pt2, ..., PtN, with a Text\cr
\+{}  placed, with respect to each point, at a given Distance from each point towards\cr
\+{}  the north-west, the south-west, the north-east or the south-east.\cr
\+{}\MacS{figwritegcw} Pt1, Pt2, ..., PtN :Text(DistanceX,DistanceY)\cr
\+{}\MacS{figwritegce} Pt1, Pt2, ..., PtN :Text(DistanceX,DistanceY)\cr
\+{}  Writing a point marker at the locations defined by Pt1, Pt2, ..., PtN, with a Text\cr
\+{}  placed, with respect to each point, at a horizontal distance DistanceX (west or east)\cr
\+{}  and a vertical distance DistanceY between the point and the mid-height of the text.\cr
\+{}\MacS{figwritegw} Pt1, Pt2, ..., PtN :Text(DistanceX,DistanceY)\cr
\+{}\MacS{figwritege} Pt1, Pt2, ..., PtN :Text(DistanceX,DistanceY)\cr
\+{}  Writing a point marker at the locations defined by Pt1, Pt2, ..., PtN, with a Text\cr
\+{}  placed, with respect to each point, at a horizontal distance DistanceX (west or east)\cr
\+{}  and a vertical distance DistanceY from the bottom of the text if DistanceY > 0, from\cr
\+{}  the top if DistanceY < 0. If DistanceY = 0, the text is vertically centered.\cr
\+{} \cr
\+{}                       Macros for postscript generation\cr
\+{}                       --------------------------------\cr
\+{} \cr
\+{}  --------- Control macros\cr
\+{} \cr
\+{}  --- General ---\cr
\+{}\MacS{psbeginfig}$\{$FileName$\}$\cr
\+{}  Starting the creation of a PostScript file whose name is FileName.\cr
\+{}\MacS{psendfig}\cr
\+{}  End of the current PostScript file.\cr
\+{}\MacS{pssetupdate}$\{$no$\}$ or \Mac{pssetupdate}$\{$yes$\}$\cr
\+{}  Setting the update mode to "yes" before \Mac{psbeginfig} enforces the postscript\cr
\+{}  file to be recreated at each compilation. The default is "no".\cr
\+{} \cr
\+{}  --- Line ---\cr
\+{}\MacS{pssetdash}$\{$Index$\}$ or \Mac{pssetdash}$\{$Pattern$\}$\cr
\+{}  Setting the line style by Index (Index = 1 (solid) to 10) or by Pattern.\cr
\+{}  To switch back to the default value : \Mac{pssetdash}$\{$\Mac{defaultdash}$\}$\cr
\+{}\MacS{pssetwidth}$\{$Width$\}$\cr
\+{}  Setting the line width (in postscript points).\cr
\+{}  To switch back to the default value : \Mac{pssetwidth}$\{$\Mac{defaultwidth}$\}$\cr
\+{}  Note : has no effect on a straight line after \Mac{pssetfillmode}$\{$yes$\}$.\cr
\+{}\MacS{pssetseconddash}$\{$Index$\}$ or \Mac{pssetseconddash}$\{$Pattern$\}$\cr
\+{}  Setting the secondary line style by Index (Index = 1 (solid) to 10) or by Pattern.\cr
\+{}  To switch back to the default value : \Mac{pssetseconddash}$\{$\Mac{defaultseconddash}$\}$\cr
\+{}\MacS{pssetsecondwidth}$\{$Width$\}$\cr
\+{}  Setting the secondary line width (in postscript points).\cr
\+{}  To switch back to the default value : \Mac{pssetsecondwidth}$\{$\Mac{defaultsecondwidth}$\}$\cr
\+{}\MacS{psresetsecondsettings}\cr
\+{}  To reset the secondary line settings.\cr
\+{} \cr
\+{}  --- Color ---\cr
\+{}\MacS{pssetcmyk}$\{$\char`\\ColorName$\}$ or \Mac{pssetcmyk}$\{$cmyk color code$\}$\cr
\+{}  Setting the color in cmyk code.\cr
\+{}\MacS{pssetrgb}$\{$\char`\\ColorName$\}$  or \Mac{pssetrgb}$\{$rgb color code$\}$\cr
\+{}  Setting the color in rgb code.\cr
\+{}\MacS{pssetgray}$\{$GrayLevel$\}$\cr
\+{}  Setting the gray level, real number between 0 (black) and 1 (white).\cr
\+{}\MacS{pssetfillmode}$\{$no$\}$ or \Mac{pssetfillmode}$\{$yes$\}$\cr
\+{}  Setting the filling mode to "no" tells the concerned macros to draw lines ;\cr
\+{}  this is the default. Setting the filling mode to "yes" tells the macros to\cr
\+{}  fill the appropriate area using the current color or gray shade.\cr
\+{}\MacS{pssetsecondcmyk}$\{$\char`\\ColorName$\}$ or \Mac{pssetsecondcmyk}$\{$cmyk color code$\}$\cr
\+{}\MacS{pssetsecondrgb}$\{$\char`\\ColorName$\}$  or \Mac{pssetsecondrgb}$\{$rgb color code$\}$\cr
\+{}\MacS{pssetsecondgray}$\{$GrayLevel$\}$\cr
\+{}  Setting the secondary color in cmyk or rgb color code, or in gray tone.\cr
\+{} \cr
\+{}  --- Arrow ---\cr
\+{}\MacS{pssetarrowheadangle}$\{$Angle$\}$\cr
\+{}  Setting the arrow-head half-angle to Angle (in degrees).\cr
\+{}  To switch back to the default value : \Mac{pssetarrowheadangle}$\{$\Mac{defaultarrowheadangle}$\}$\cr
\+{}\MacS{pssetarrowheadfill}$\{$no$\}$ or \Mac{pssetarrowheadfill}$\{$yes$\}$\cr
\+{}  Setting the arrow-head filling switch. The default value is "no".\cr
\+{}\MacS{pssetarrowheadout}$\{$no$\}$ or \Mac{pssetarrowheadout}$\{$yes$\}$\cr
\+{}  Setting the arrow-head "outside" switch. The default value is "no".\cr
\+{}\MacS{pssetarrowheadlength}$\{$Length$\}$\cr
\+{}  Setting the length of each edge of the arrow-head to Length (in user coordinates).\cr
\+{}  To switch back to the default value : \Mac{pssetarrowheadlength}$\{$\Mac{defaultarrowheadlength}$\}$\cr
\+{}\MacS{pssetarrowheadratio}$\{$Ratio$\}$\cr
\+{}  Setting the arrow-head ratio to Ratio (real number usually in (0,1)).\cr
\+{}  To switch back to the default value : \Mac{pssetarrowheadratio}$\{$\Mac{defaultarrowheadratio}$\}$\cr
\+{}\MacS{psresetarrowhead}\cr
\+{}  To reset the arrow-head attributes to the default values.\cr
\+{} \cr
\+{}  --- Curve ---\cr
\+{}\MacS{pssetroundness}$\{$Roundness$\}$\cr
\+{}  Setting the roundness of a curve to Roundness (real number).\cr
\+{}  To switch back to the default value : \Mac{pssetroundness}$\{$\Mac{defaultroundness}$\}$\cr
\+{} \cr
\+{}  --- Mesh ---\cr
\+{}\MacS{pssetmeshdiag}$\{$Index$\}$\cr
\+{}  Setting the flag that controls the drawing of a grid mesh:\cr
\+{}   . if Index =  1, the SW-NE diagonal is drawn inside each cell,\cr
\+{}   . if Index = -1, the NW-SE diagonal is drawn inside each cell,\cr
\+{}   . otherwise, each cell is empty ; this is the default (Index = 0).\cr
\+{}  To switch back to the default value : \Mac{pssetmeshdiag}$\{$\Mac{defaultmeshdiag}$\}$\cr
\+{} \cr
\+{}  --------- Basic drawing macros\cr
\+{}\MacS{pscirc} Center (Radius)\cr
\+{}\MacS{pscirc} Center,Pt1,Pt2 (Radius)\cr
\+{}  Circle of center Center and of radius Radius.\cr
\+{}  In 3D, the circle is in the plane defined the 3 points Center, Pt1 and Pt2 ;\cr
\+{}  Pt1 and Pt2 do not need to lie on the circle.\cr
\+{}\MacS{psline}[Pt1,Pt2,... ,PtN]\cr
\+{}  Line defined by N points, closed if the last point number equals the first one.\cr
\+{}\MacS{pslineF}$\{$Filename$\}$\cr
\+{}  Line defined by points read from the text file Filename, which contains the\cr
\+{}  coordinates of the points, one point per line, given as  X Y  in 2D and as\cr
\+{}  X Y Z  in 3D (the values must be separated by a white space and nothing else).\cr
\+{}  The line is closed if the last point equals the first one.\cr
\+{} \cr
\+{}  --------- Other drawing macros\cr
\+{} \cr
\+{}  --- Arc ---\cr
\+{}\MacS{psarccirc} Center ; Radius (Ang1,Ang2)\cr
\+{}\MacS{psarccirc} Center,Pt1,Pt2 ; Radius (Ang1,Ang2)\cr
\+{}  Circular arc of center Center and radius Radius limited by the angles\cr
\+{}  Ang1 and Ang2 given in degrees.\cr
\+{}  In 2D, the angles are measured counterclockwise.\cr
\+{}  In 3D, the arc lies in the plane defined by the 3 points Center, Pt1 and Pt2.\cr
\+{}  The angles are measured counterclockwise around the vector CP1 x CP2, starting\cr
\+{}  from the half-line (C, P1), where C=Center, P1=Pt1, P2=Pt2.\cr
\+{}\MacS{psarccircP} Center ; Radius [Pt1,Pt2]\cr
\+{}\MacS{psarccircP} Center ; Radius [Pt1,Pt2,Pt3]\cr
\+{}  Point version of \Mac{psarccirc}\cr
\+{}  Circular arc of center Center and of radius Radius limited by the two half-lines\cr
\+{}  (Center, Pt1) and (Center, Pt2).\cr
\+{}  Let N be the normal vector that orients the plane in which lies the arc.\cr
\+{}  In 2D, N is defined as usual by N = Z = X x Y.\cr
\+{}  In 3D, N = CP1 x CP3, where C=Center, P1=Pt1, P3=Pt3.\cr
\+{}  The arc is drawn from Pt1 towards Pt2 turning counterclockwise around the axis (C; N).\cr
\+{}  Notice that C, P1, P2 and P3 must be coplanar, but P3 must not lie on the line (C,P1).\cr
\+{}\MacS{psarcell} Center ; XRad,YRad (Ang1,Ang2, Inclination)\cr
\+{}  Arc of ellipse of center Center, of axes XRad and YRad, limited by the parametrization\cr
\+{}  angles Ang1 and Ang2. The major axis is turned by an angle Inclination from the\cr
\+{}  horizontal line. The angles are given in degrees and measured counterclockwise.\cr
\+{}  In 3D-mode, it works only in the plane Z=0.\cr
\+{}\MacS{psarcellPA} Center,PtAxis1,PtAxis2 (Ang1, Ang2)\cr
\+{}  Arc of ellipse of center Center limited by the parametrization angles Ang1 and Ang2.\cr
\+{}  The end point of the major axis is PtAxis1 and the end point of the minor axis is PtAxis2.\cr
\+{}  The angles are given in degrees and measured counterclockwise around the vector CA1 x CA2.\cr
\+{}\MacS{psarcellPP} Center,PtAxis1,PtAxis2 [Pt1,Pt2]\cr
\+{}  Arc of ellipse of center Center limited by the two half-lines (Center,Pt1)\cr
\+{}  and (Center,Pt2). The end point of the major axis is PtAxis1 and the end point\cr
\+{}  of the minor axis is PtAxis2. The arc is drawn counterclockwise around the vector\cr
\+{}  CA1 x CA2, where C=Center, A1=PtAxis1, A2=PtAxis2.\cr
\+{} \cr
\+{}  --- Arrow ---\cr
\+{}\MacS{psarrow} [Pt1,Pt2]\cr
\+{}  Arrow from Pt1 to Pt2.\cr
\+{}  The arrow-head is drawn according to the \Mac{psarrowhead} macro settings.\cr
\+{}\MacS{psarrowBezier} [Pt1,Pt2,Pt3,Pt4]\cr
\+{}  Arrow that consists of the cubic Bezier curve defined by the four control points\cr
\+{}  Pt1, Pt2, Pt3 and Pt4, and the arrow-head at point Pt4.\cr
\+{}\MacS{psarrowcirc} Center ; Radius (Ang1,Ang2)\cr
\+{}\MacS{psarrowcirc} Center,Pt1,Pt2 ; Radius (Ang1,Ang2)\cr
\+{}  Circular arrow such that the circular arc is centered at Center, has the radius Radius\cr
\+{}  and is limited by the angles Ang1 and Ang2 given in degrees.\cr
\+{}  If Ang2 > Ang1, the arrow is drawn counterclockwise, else it is drawn clockwise.\cr
\+{}  The arrow-head is drawn according to the \Mac{psarrowhead} macro settings.\cr
\+{}  In 3D, the arc lies in the plane defined the 3 points Center, Pt1 and Pt2.\cr
\+{}  The angles are measured counterclockwise around the vector CP1 x CP2, starting\cr
\+{}  from the half-line (C, P1), where C=Center, P1=Pt1, P2=Pt2.\cr
\+{}\MacS{psarrowcircP} Center ; Radius [Pt1,Pt2]\cr
\+{}\MacS{psarrowcircP} Center ; Radius [Pt1,Pt2,Pt3]\cr
\+{}  Point version of \Mac{psarrowcirc}\cr
\+{}  Circular arrow such that the circular arc is centered at Center, has the radius\cr
\+{}  |Radius| and is limited by the two half-lines (Center, Pt1) and (Center, Pt2).\cr
\+{}  Let N be the normal vector that orients the plane in which lies the arc.\cr
\+{}  In 2D, N is defined as usual by N = Z = X x Y.\cr
\+{}  In 3D, N = CP1 x CP3, where C=Center, P1=Pt1, P3=Pt3.\cr
\+{}  The arrow is drawn from Pt1 towards Pt2 turning around the axis (Center; N):\cr
\+{}   . counterclockwise if Radius > 0,\cr
\+{}   . clockwise if Radius < 0.\cr
\+{}  Notice that C, P1, P2 and P3 must be coplanar, but P3 must not lie on the line (C,P1).\cr
\+{}\MacS{psarrowhead} [Pt1,Pt2]\cr
\+{}  Arrow-head of the arrow from Pt1 to Pt2. The segment [Pt1, Pt2] is not drawn.\cr
\+{}  The appearance of the arrow-head can be modified with the help of the macros :\cr
\+{}   . \Mac{pssetarrowheadangle} which sets the opening angle,\cr
\+{}   . \Mac{pssetarrowheadfill} which tells whether the interior of the arrow-head\cr
\+{}     must be filled or not,\cr
\+{}   . \Mac{pssetarrowheadout} which tells whether the arrow-head must be drawn outside\cr
\+{}     the segment [Pt1, Pt2] or not,\cr
\+{}   . \Mac{pssetarrowheadlength} which sets the length of the arrow-head in user coordinates,\cr
\+{}   . \Mac{pssetarrowheadratio} which sets the length of the arrow-head as a ratio of the\cr
\+{}     length of the segment [Pt1, Pt2].\cr
\+{}  Nota : The last two macros are mutually exclusive ; the default is to use the length.\cr
\+{}         Default values exist for each attribute.\cr
\+{} \cr
\+{}  --- Curve ---\cr
\+{}\MacS{psBezier} N [Pt\_1, ..., Pt\_$\{$3N+1$\}$]\cr
\+{}  Bezier curve defined by N cubic arcs. The arc number i is defined by the four\cr
\+{}  control points P\_$\{$j$\}$, P\_$\{$j+1$\}$, P\_$\{$j+2$\}$, P\_$\{$j+3$\}$ with j=3i-2.\cr
\+{}  The curve interpolates each 3 points beginning with the first, i.e. at P\_$\{$3i-2$\}$,\cr
\+{}  i=1,...N+1. At these points, the curve is only C0. To obtain G1 continuity at P\_j,\cr
\+{}  the points P\_$\{$j-1$\}$, P\_j and P\_$\{$j+1$\}$ must be aligned.\cr
\+{}  The total number of points must be 3N+1 and is not checked.\cr
\+{}\MacS{pscurve} [Pt0,Pt1,... ,PtN,PtN+1]\cr
\+{}  C1 curve that interpolates the points P1, P2,... ,Pn. The direction of the\cr
\+{}  tangent at P1 is given by P0P2, and at Pn by Pn-1Pn+1. The curve consists of\cr
\+{}  n-1 Bezier cubic arcs. To get a C1 closed curve, the last three points must\cr
\+{}  be the same as the first three ones.\cr
\+{}  The shape of the curve can be modified by a roundness coefficient to be\cr
\+{}  set by the macro \Mac{pssetroundness}. The best values for this coefficient are\cr
\+{}  in the interval [0.15, 0.3] (0 gives a polygonal line).\cr
\+{} \cr
\+{}  --- Triangle related macros ---\cr
\+{}\MacS{psaltitude} Dim [Pt1,Pt2,Pt3]\cr
\+{}  Altitude from Pt1 in the triangle (Pt1, Pt2, Pt3).\cr
\+{}  Dim is the dimension of the square at the end point of the altitude.\cr
\+{}\MacS{psnormal} Length,Lambda [Pt1,Pt2]\cr
\+{}  Exterior normal of length Length to the segment [Pt1, Pt2]. Lambda is the barycentric\cr
\+{}  coordinate of the origin of the normal with respect to the segment [Pt1, Pt2], which\cr
\+{}  sets the position of the vector along [Pt1, Pt2].\cr
\+{}  In 3D-mode, it works only in the plane Z=0.\cr
\+{} \cr
\+{}  --- Grid ---\cr
\+{}\MacS{psmesh} N1,N2 [Pt1,Pt2,Pt3,Pt4]\cr
\+{}  Mesh of N1 x N2 intervals on the quadrangle (Pt1, Pt2, Pt3, Pt4)\cr
\+{}  (N1 along [Pt1, Pt2] and [Pt3, Pt4], N2 along the 2 other segments).\cr
\+{}  If the flag set by \Mac{pssetmeshdiag} is equal to 1 (resp. -1), the SW-NE (resp. NW-SE)\cr
\+{}  diagonal is drawn inside each cell. By default, no diagonal is drawn.\cr
\+{}\MacS{pstrimesh} Type [Pt1,Pt2,Pt3]\cr
\+{}  Mesh of the type Type triangle on the triangle (Pt1, Pt2, Pt3).\cr
%=================================================================================
}
%----------------------------------------------
\vfill\eject
\ifodd\pageno\else\null\vfill\eject\fi
\null\vfill
\centerline{\bf INDEX}
\bigskip
To help the user to locate the information, page numbers are printed according to the
context in which each word of the index appears.
When a word appears in the text, the corresponding page numbers are printed
in {\rm roman} font. 
When a word appears in an {\bf example}, the corresponding page numbers are printed
in {\bf bold face} font.
When a word appears in the previous {\it list}, the corresponding page numbers are
printed in {\it italic} font.
\vfill
%\input \jobname.idx
{\parindent 0mm
\newdimen\marge  \marge=0.3\hsize
\def\ligneindex#1#2{\leavevmode\setbox88=\hbox{#1}\hskip -\marge%
\leftskip\marge\ifdim\wd88<\marge%
\hbox to\marge{\unhbox88\leaders\hbox to 2mm{\hfil.\hfil}\hfil}%
\else\unhbox88\ \ \fi#2\par}
\ligneindex{\Mac {defaultarrowheadangle} }{{ \rm   23}, { \bf   25}, { \it   53}}
\ligneindex{\Mac {defaultarrowheadlength} }{{ \rm   23}, { \it   53}}
\ligneindex{\Mac {defaultarrowheadratio} }{{ \rm   23}, { \it   53}}
\ligneindex{\Mac {defaultdash} }{{ \rm   11}, { \bf   19}, { \bf   32}, { \it   52}}
\ligneindex{\Mac {defaultmeshdiag} }{{ \rm   31}, { \it   54}}
\ligneindex{\Mac {defaultroundness} }{{ \rm   22}, { \it   53}}
\ligneindex{\Mac {defaultseconddash} }{{ \rm   11}, { \it   53}}
\ligneindex{\Mac {defaultsecondwidth} }{{ \rm   11}, { \it   53}}
\ligneindex{\Mac {defaultwidth} }{{ \rm   10}, { \bf   19}, { \bf   25}, { \bf   44}, { \it   53}}
\ligneindex{\Mac {figcoord} }{{ \bf   5}, { \rm   5}, { \it   51}}
\ligneindex{\Mac {figgetangle} }{{ \rm   5}, { \bf   20}, { \rm   36}, { \it   48}}
\ligneindex{\Mac {figgetdist} }{{ \rm   5}, { \bf   12}, { \rm   12}, { \bf   20}, { \it   48}}
\ligneindex{\Mac {figinit} }{{ \bf   2}, { \rm   4}, { \rm   5}, { \rm   6}, { \bf   7}, { \bf   12}, { \bf   14}, { \rm   14}, { \bf   15}, { \bf   16}, { \bf   17}, { \rm   17}, { \bf   19}, { \bf   20}, { \bf   21}, { \bf   22}, { \bf   23}, { \bf   24}, { \bf   25}, { \bf   28}, { \bf   29}, { \bf   30}, { \bf   31}, { \bf   32}, { \rm   33}, { \bf   34}, { \bf   35}, { \rm   36}, { \bf   38}, { \bf   39}, { \bf   40}, { \bf   41}, { \rm   44}, { \bf   46}, { \it   47}, { \it   48}}
\ligneindex{\Mac {figinsert} }{{ \bf   2}, { \rm   3}, { \bf   6}, { \bf   7}, { \bf   12}, { \rm   13}, { \bf   14}, { \bf   15}, { \rm   15}, { \bf   16}, { \rm   16}, { \bf   17}, { \rm   17}, { \bf   19}, { \bf   20}, { \bf   21}, { \bf   22}, { \bf   23}, { \bf   24}, { \bf   25}, { \bf   28}, { \bf   29}, { \bf   30}, { \bf   31}, { \bf   32}, { \rm   32}, { \bf   34}, { \bf   35}, { \bf   39}, { \bf   42}, { \bf   45}, { \it   47}}
\ligneindex{\Mac {figpt} }{{ \bf   2}, { \rm   4}, { \bf   5}, { \rm   6}, { \bf   7}, { \bf   12}, { \bf   14}, { \bf   15}, { \bf   16}, { \bf   17}, { \bf   19}, { \bf   20}, { \bf   21}, { \bf   22}, { \bf   23}, { \bf   24}, { \bf   25}, { \bf   28}, { \bf   29}, { \bf   30}, { \bf   31}, { \bf   32}, { \bf   34}, { \bf   35}, { \rm   36}, { \bf   38}, { \bf   41}, { \bf   45}, { \bf   46}, { \it   48}}
\ligneindex{\Mac {figptbary} }{{ \bf   2}, { \bf   7}, { \rm   8}, { \bf   20}, { \bf   30}, { \bf   31}, { \it   48}}
\ligneindex{\Mac {figptbaryR} }{{ \rm   8}, { \bf   20}, { \it   48}}
\ligneindex{\Mac {figptBezier} }{{ \rm   10}, { \rm   21}, { \bf   44}, { \bf   45}, { \it   51}}
\ligneindex{\Mac {figptcirc} }{{ \rm   9}, { \bf   34}, { \bf   35}, { \rm   37}, { \it   50}}
\ligneindex{\Mac {figptcircumcenter} }{{ \rm   29}, { \bf   30}, { \it   50}}
\ligneindex{\Mac {figptcopy} }{{ \rm   6}, { \bf   24}, { \it   48}}
\ligneindex{\Mac {figptcurvcenter} }{{ \rm   10}, { \rm   44}, { \bf   46}, { \it   51}}
\ligneindex{\Mac {figptell} }{{ \rm   9}, { \rm   18}, { \bf   19}, { \bf   20}, { \rm   33}, { \it   51}}
\ligneindex{\Mac {figptellP} }{{ \rm   9}, { \rm   18}, { \it   51}}
\ligneindex{\Mac {figptendnormal} }{{ \rm   28}, { \bf   29}, { \rm   33}, { \it   50}}
\ligneindex{\Mac {figpthom} }{{ \bf   7}, { \rm   7}, { \it   49}}
\ligneindex{\Mac {figptinscribedcenter} }{{ \rm   29}, { \bf   30}, { \it   50}}
\ligneindex{\Mac {figptinterlineplane} }{{ \rm   33}, { \rm   37}, { \it   50}}
\ligneindex{\Mac {figptinterlines} }{{ \rm   9}, { \rm   18}, { \it   50}}
\ligneindex{\Mac {figptorthocenter} }{{ \rm   29}, { \bf   30}, { \it   50}}
\ligneindex{\Mac {figptorthoprojline} }{{ \rm   8}, { \bf   29}, { \it   49}}
\ligneindex{\Mac {figptorthoprojplane} }{{ \rm   33}, { \bf   35}, { \rm   37}, { \bf   44}, { \it   49}}
\ligneindex{\Mac {figptrot} }{{ \bf   7}, { \rm   7}, { \bf   28}, { \rm   37}, { \bf   44}, { \bf   45}, { \bf   46}, { \rm   46}, { \it   49}}
\ligneindex{\Mac {figptscontrol} }{{ \rm   10}, { \it   51}}
\ligneindex{\Mac {figptshom} }{{ \rm   8}, { \it   49}}
\ligneindex{\Mac {figptsintercirc} }{{ \rm   9}, { \bf   20}, { \bf   28}, { \rm   33}, { \it   50}}
\ligneindex{\Mac {figptsorthoprojline} }{{ \rm   8}, { \it   50}}
\ligneindex{\Mac {figptsorthoprojplane} }{{ \rm   33}, { \rm   37}, { \it   50}}
\ligneindex{\Mac {figptsrot} }{{ \rm   8}, { \bf   21}, { \bf   25}, { \rm   37}, { \bf   41}, { \bf   45}, { \it   49}}
\ligneindex{\Mac {figptssym} }{{ \rm   8}, { \bf   21}, { \rm   37}, { \bf   44}, { \bf   45}, { \it   49}}
\ligneindex{\Mac {figptstra} }{{ \bf   7}, { \rm   8}, { \bf   24}, { \bf   31}, { \bf   32}, { \bf   38}, { \it   49}, { \it   50}}
\ligneindex{\Mac {figptsym} }{{ \bf   7}, { \rm   7}, { \rm   37}, { \it   49}}
\ligneindex{\Mac {figpttra} }{{ \bf   7}, { \rm   7}, { \bf   12}, { \bf   24}, { \bf   41}, { \bf   44}, { \bf   45}, { \bf   46}, { \it   49}}
\ligneindex{\Mac {figpttraC} }{{ \bf   7}, { \rm   8}, { \bf   19}, { \bf   31}, { \bf   32}, { \bf   34}, { \bf   35}, { \rm   37}, { \bf   38}, { \bf   41}, { \bf   45}, { \it   49}}
\ligneindex{\Mac {figscan} }{{ \rm   1}, { \rm   15}, { \bf   16}, { \rm   16}, { \rm   17}, { \rm   18}, { \rm   32}, { \it   47}}
\ligneindex{\Mac {figsetmark} }{{ \bf   2}, { \rm   5}, { \bf   7}, { \bf   12}, { \rm   13}, { \bf   14}, { \bf   15}, { \bf   19}, { \bf   20}, { \bf   22}, { \bf   30}, { \bf   35}, { \bf   40}, { \it   51}}
\ligneindex{\Mac {figsetobdist} }{{ \rm   33}, { \rm   34}, { \bf   41}, { \rm   41}, { \bf   46}, { \it   47}}
\ligneindex{\Mac {figsetptname} }{{ \rm   4}, { \rm   5}, { \rm   13}, { \bf   21}, { \bf   29}, { \bf   38}, { \bf   40}, { \it   48}, { \it   51}}
\ligneindex{\Mac {figsetroundcoord} }{{ \rm   5}, { \bf   40}, { \it   51}}
\ligneindex{\Mac {figsettarget} }{{ \rm   33}, { \rm   34}, { \bf   41}, { \rm   41}, { \bf   46}, { \it   47}}
\ligneindex{\Mac {figsetview} }{{ \rm   33}, { \rm   34}, { \bf   35}, { \rm   35}, { \bf   39}, { \bf   41}, { \bf   46}, { \it   47}}
\ligneindex{\Mac {figshowpts} }{{ \rm   46}, { \it   51}}
\ligneindex{\Mac {figshowsettings} }{{ \rm   32}, { \rm   40}, { \it   47}}
\ligneindex{\Mac {figvectC} }{{ \rm   6}, { \bf   7}, { \bf   12}, { \bf   24}, { \bf   31}, { \bf   32}, { \rm   36}, { \bf   38}, { \bf   41}, { \bf   45}, { \it   48}}
\ligneindex{\Mac {figvectDBezier} }{{ \rm   6}, { \rm   10}, { \rm   21}, { \bf   45}, { \it   51}}
\ligneindex{\Mac {figvectN} }{{ \rm   6}, { \rm   36}, { \rm   37}, { \it   48}}
\ligneindex{\Mac {figvectNV} }{{ \rm   6}, { \rm   36}, { \rm   37}, { \it   48}}
\ligneindex{\Mac {figvectP} }{{ \rm   6}, { \bf   35}, { \bf   41}, { \bf   45}, { \bf   46}, { \it   48}}
\ligneindex{\Mac {figvectU} }{{ \rm   6}, { \rm   44}, { \bf   45}, { \it   48}}
\ligneindex{\Mac {figvisu} }{{ \bf   2}, { \rm   3}, { \bf   6}, { \bf   7}, { \bf   12}, { \rm   13}, { \bf   14}, { \bf   15}, { \rm   15}, { \bf   16}, { \bf   17}, { \bf   19}, { \bf   20}, { \bf   21}, { \bf   22}, { \bf   23}, { \bf   24}, { \bf   25}, { \rm   27}, { \bf   28}, { \bf   29}, { \bf   30}, { \bf   31}, { \bf   32}, { \bf   34}, { \bf   35}, { \bf   39}, { \bf   42}, { \bf   45}, { \it   47}, { \it   48}}
\ligneindex{\Mac {figwrite} }{{ \rm   2}, { \rm   4}, { \rm   13}, { \bf   15}, { \rm   38}, { \it   52}}
\ligneindex{\Mac {figwritec} }{{ \rm   13}, { \bf   14}, { \bf   16}, { \bf   17}, { \bf   31}, { \it   52}}
\ligneindex{\Mac {figwritee} }{{ \bf   2}, { \bf   5}, { \bf   7}, { \bf   12}, { \rm   13}, { \bf   14}, { \bf   21}, { \bf   25}, { \bf   30}, { \bf   34}, { \bf   35}, { \bf   39}, { \it   52}}
\ligneindex{\Mac {figwritegce} }{{ \rm   14}, { \bf   15}, { \it   52}}
\ligneindex{\Mac {figwritegcw} }{{ \rm   14}, { \bf   15}, { \it   52}}
\ligneindex{\Mac {figwritege} }{{ \rm   14}, { \bf   15}, { \it   52}}
\ligneindex{\Mac {figwritegw} }{{ \rm   14}, { \bf   15}, { \it   52}}
\ligneindex{\Mac {figwriten} }{{ \bf   2}, { \bf   7}, { \bf   12}, { \rm   13}, { \bf   14}, { \bf   19}, { \bf   20}, { \bf   21}, { \bf   24}, { \bf   29}, { \bf   34}, { \bf   35}, { \it   52}}
\ligneindex{\Mac {figwritene} }{{ \rm   13}, { \bf   14}, { \bf   21}, { \bf   25}, { \bf   32}, { \bf   35}, { \bf   39}, { \it   52}}
\ligneindex{\Mac {figwritenw} }{{ \bf   7}, { \bf   12}, { \rm   13}, { \bf   14}, { \bf   21}, { \bf   30}, { \bf   32}, { \bf   35}, { \bf   39}, { \it   52}}
\ligneindex{\Mac {figwritep} }{{ \rm   13}, { \bf   14}, { \bf   22}, { \it   52}}
\ligneindex{\Mac {figwrites} }{{ \bf   2}, { \bf   7}, { \rm   13}, { \bf   14}, { \bf   29}, { \bf   35}, { \it   52}}
\ligneindex{\Mac {figwritese} }{{ \rm   13}, { \bf   14}, { \bf   21}, { \bf   32}, { \bf   39}, { \it   52}}
\ligneindex{\Mac {figwritesw} }{{ \rm   13}, { \bf   14}, { \bf   19}, { \bf   21}, { \bf   25}, { \bf   32}, { \bf   34}, { \bf   39}, { \it   52}}
\ligneindex{\Mac {figwritew} }{{ \bf   2}, { \bf   7}, { \rm   13}, { \bf   14}, { \bf   21}, { \bf   25}, { \bf   29}, { \bf   30}, { \bf   35}, { \it   52}}
\ligneindex{\Mac {psaltitude} }{{ \bf   27}, { \bf   29}, { \it   56}}
\ligneindex{\Mac {psarccirc} }{{ \rm   18}, { \bf   19}, { \bf   26}, { \rm   37}, { \it   54}}
\ligneindex{\Mac {psarccircP} }{{ \rm   18}, { \bf   19}, { \rm   27}, { \bf   28}, { \rm   38}, { \it   54}}
\ligneindex{\Mac {psarcell} }{{ \rm   18}, { \bf   19}, { \rm   19}, { \bf   20}, { \bf   26}, { \rm   33}, { \it   54}}
\ligneindex{\Mac {psarcellPA} }{{ \rm   18}, { \bf   19}, { \bf   26}, { \it   54}}
\ligneindex{\Mac {psarcellPP} }{{ \rm   18}, { \bf   19}, { \it   54}}
\ligneindex{\Mac {psarrow} }{{ \bf   7}, { \rm   23}, { \bf   24}, { \bf   34}, { \bf   35}, { \it   55}}
\ligneindex{\Mac {psarrowBezier} }{{ \rm   24}, { \bf   25}, { \bf   45}, { \it   55}}
\ligneindex{\Mac {psarrowcirc} }{{ \rm   24}, { \bf   25}, { \bf   34}, { \rm   38}, { \it   55}}
\ligneindex{\Mac {psarrowcircP} }{{ \rm   24}, { \bf   25}, { \bf   29}, { \bf   35}, { \rm   38}, { \it   55}}
\ligneindex{\Mac {psarrowhead} }{{ \rm   23}, { \bf   24}, { \rm   24}, { \rm   38}, { \it   55}}
\ligneindex{\Mac {psbeginfig} }{{ \bf   2}, { \rm   3}, { \bf   6}, { \bf   7}, { \rm   10}, { \bf   12}, { \bf   14}, { \bf   19}, { \bf   20}, { \bf   21}, { \bf   22}, { \bf   23}, { \bf   24}, { \bf   25}, { \bf   26}, { \bf   28}, { \bf   29}, { \bf   30}, { \bf   31}, { \bf   32}, { \rm   32}, { \bf   34}, { \bf   35}, { \bf   38}, { \bf   42}, { \bf   45}, { \it   47}, { \it   52}}
\ligneindex{\Mac {psBezier} }{{ \bf   21}, { \rm   21}, { \rm   22}, { \bf   26}, { \bf   44}, { \bf   45}, { \it   55}}
\ligneindex{\Mac {pscirc} }{{ \rm   5}, { \bf   12}, { \rm   12}, { \bf   14}, { \rm   18}, { \bf   28}, { \rm   37}, { \bf   45}, { \it   54}}
\ligneindex{\Mac {pscurve} }{{ \bf   22}, { \rm   22}, { \bf   23}, { \it   55}}
\ligneindex{\Mac {psendfig} }{{ \bf   2}, { \rm   3}, { \bf   7}, { \rm   10}, { \bf   12}, { \bf   14}, { \bf   19}, { \bf   20}, { \bf   21}, { \bf   22}, { \bf   23}, { \bf   24}, { \bf   25}, { \bf   26}, { \bf   28}, { \bf   29}, { \bf   30}, { \bf   31}, { \bf   32}, { \bf   34}, { \bf   35}, { \bf   38}, { \bf   42}, { \bf   45}, { \it   52}}
\ligneindex{\Mac {psline} }{{ \bf   2}, { \bf   7}, { \rm   11}, { \bf   19}, { \bf   20}, { \bf   21}, { \bf   24}, { \bf   25}, { \rm   27}, { \bf   28}, { \bf   29}, { \bf   30}, { \bf   35}, { \bf   38}, { \bf   42}, { \bf   45}, { \it   54}}
\ligneindex{\Mac {pslineF} }{{ \rm   12}, { \it   54}}
\ligneindex{\Mac {psmesh} }{{ \rm   31}, { \bf   32}, { \it   56}}
\ligneindex{\Mac {psnormal} }{{ \bf   27}, { \rm   28}, { \bf   29}, { \rm   33}, { \it   56}}
\ligneindex{\Mac {psresetarrowhead} }{{ \rm   23}, { \bf   24}, { \it   53}}
\ligneindex{\Mac {psresetsecondsettings} }{{ \rm   11}, { \it   53}}
\ligneindex{\Mac {pssetarrowheadangle} }{{ \rm   23}, { \bf   25}, { \bf   40}, { \it   53}, { \it   55}}
\ligneindex{\Mac {pssetarrowheadfill} }{{ \rm   23}, { \bf   24}, { \bf   25}, { \bf   34}, { \bf   35}, { \bf   40}, { \it   53}, { \it   55}}
\ligneindex{\Mac {pssetarrowheadlength} }{{ \rm   23}, { \bf   24}, { \bf   25}, { \bf   34}, { \bf   35}, { \bf   40}, { \bf   45}, { \it   53}, { \it   55}}
\ligneindex{\Mac {pssetarrowheadout} }{{ \rm   23}, { \bf   24}, { \bf   40}, { \it   53}, { \it   55}}
\ligneindex{\Mac {pssetarrowheadratio} }{{ \rm   23}, { \bf   24}, { \bf   29}, { \bf   40}, { \it   53}, { \it   55}}
\ligneindex{\Mac {pssetcmyk} }{{ \rm   25}, { \rm   26}, { \bf   40}, { \it   53}}
\ligneindex{\Mac {pssetdash} }{{ \rm   11}, { \bf   14}, { \bf   19}, { \bf   21}, { \bf   35}, { \bf   38}, { \bf   40}, { \it   52}}
\ligneindex{\Mac {pssetfillmode} }{{ \rm   11}, { \bf   23}, { \bf   26}, { \rm   26}, { \bf   28}, { \bf   40}, { \bf   42}, { \bf   44}, { \bf   45}, { \it   53}}
\ligneindex{\Mac {pssetgray} }{{ \bf   23}, { \rm   25}, { \bf   26}, { \rm   26}, { \bf   40}, { \bf   42}, { \it   53}}
\ligneindex{\Mac {pssetmeshdiag} }{{ \rm   31}, { \bf   32}, { \it   53}, { \it   54}, { \it   56}}
\ligneindex{\Mac {pssetrgb} }{{ \rm   25}, { \bf   26}, { \rm   26}, { \bf   28}, { \bf   40}, { \bf   44}, { \bf   45}, { \it   53}}
\ligneindex{\Mac {pssetroundness} }{{ \rm   22}, { \bf   23}, { \bf   40}, { \it   53}, { \it   56}}
\ligneindex{\Mac {pssetsecondcmyk} }{{ \rm   11}, { \rm   26}, { \bf   27}, { \rm   30}, { \bf   41}, { \it   53}}
\ligneindex{\Mac {pssetseconddash} }{{ \rm   11}, { \bf   27}, { \rm   30}, { \bf   31}, { \bf   32}, { \bf   41}, { \it   53}}
\ligneindex{\Mac {pssetsecondgray} }{{ \rm   11}, { \rm   26}, { \bf   27}, { \rm   30}, { \bf   41}, { \it   53}}
\ligneindex{\Mac {pssetsecondrgb} }{{ \rm   11}, { \rm   26}, { \bf   27}, { \rm   30}, { \bf   41}, { \it   53}}
\ligneindex{\Mac {pssetsecondwidth} }{{ \rm   11}, { \bf   27}, { \rm   30}, { \bf   41}, { \it   53}}
\ligneindex{\Mac {pssetupdate} }{{ \rm   10}, { \rm   32}, { \bf   40}, { \it   52}}
\ligneindex{\Mac {pssetwidth} }{{ \rm   10}, { \bf   19}, { \bf   25}, { \bf   40}, { \bf   44}, { \bf   45}, { \it   53}}
\ligneindex{\Mac {pstrimesh} }{{ \rm   30}, { \bf   31}, { \it   56}}
\par}
\vfill\eject
%----------------------------------------------
\bye
