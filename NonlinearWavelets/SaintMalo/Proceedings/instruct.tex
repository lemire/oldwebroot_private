
%************************* AT01 MACROS ******************************
% FILE NAME: instruct.tex
% Copyright: 10/15/97, updated 4/20/99, 6/1/02
% CREATED BY: Larry Schumaker
% EMAIL: s@mars.cas.vanderbilt.edu
% WEB: http://www.math.vanderbilt.edu/~schumake/
% ******************************************************************
%
%		For use in preparing papers for
%
%		     ST MALO Proceedings, 2002
%
%		 Larry Schumaker (s@mars.cas.vanderbilt.edu)

%********************  INSTRUCTIONS  **********************************

%1. REPLACE xxx by correct information
%2. INSERT body of text in the sections
%3. INSERT references
%4. REMOVE all lines starting with %
%5. RENAME this file "yourname.tex"

%************************ INITIALIZATION ************************

%	 In TEXing your file remember that TEX will be asking for the
%	 file at01.tex, so have it available in the
%	 same directory as this file.
%

\count100= 1
\count101= 7
   %replace 3 by the number of pages of your paper.
\input malo2.tex

\title{Instructions for Using the}
  % Replace xxx by your title.
\titwo{\conf \TeX{} Macros}
   % If need two lines for title, replace xxx.
\author{Larry L. Schumaker}
   % Replace xxx by author names.
%\autwo{xxx}
   % If you need two lines for authors, replace xxx.

\def\shorttitle{TEX Instructions}
   % Replace xxx by a short title for the running head.
\def\shortauthor{Larry L. Schumaker}
   % Replace xxx by author names (initials only) for running head

%   ********************** MACROS  ******************************

\def\back{{\tt \char'134}}
\def\co{{\cal O}}
\def\cP{{\cal P}}
\def\ct{{\cal T}}
\def\ctil{\tilde{\cal T}}
\def\lb{\char'173}
\def\rb{\char'175}
\def\oln{\co(n \log(n))}
\def\tb#1{{\tt \back #1}}
\def\Tex{{\TeX~}}

%************************* ABSTRACT **********************************

\abstract {This document is based on the \Tex file {\tt instruct.tex}.
Here is a sample abstract.  It should consist of a single
paragraph describing the main results of your paper.  It would be better
not to put too many symbols in the abstract.}
  % Replace xxx by your abstract

%******************************* BODY *********************************

\sect{1. Introduction}
The purpose of this sample paper is to illustrate how to use the
macro file {\tt \macname.tex} to create a \TeX{} file for your contribution
to the \conf{}proceedings, and to give
detailed instructions on style.
This sample paper was created by
filling in the blanks in the file {\tt paper.tex}.

I recommend that you read through this document,
even if you are an experienced \Tex  user
and have used some of my macros before. And
if you have not used plain \TeX{} macro files before,
I think you will be able to learn almost everything you need
to know by comparing
what is in this document with the \Tex file {\tt instruct.tex}
which generated it. If you don't read anything else, please
look at Sect.~14 on things \eword{not} to do.

For information on how to submit your paper, see the document
{\it Submitting Papers to the \conf{}Proceedings}
which you can create by processing the \Tex file {\tt submit.tex}.

\sect{2. Organization}
We recommend that you organize your paper into sections
which should be numbered starting with 1.  To create section
headings, use the \tb{sect} macro.  If your section heading
does not fit on one line, use \tb{sectlong}.
Do not leave a blank line between the section
heading and the first line of text.
Each major word (other than words like {\it a, and,
the, to, with}, etc.) in the section title
should be capitalized. Do not include a period at the end.

Subsections can be created with \tb{subsect} and \tb{fsubsect}.
The latter is for use immediately after the start of a section
and simply leaves a little less space.

\sect{3. Proclamations}
Definitions, Lemmas, Theorems, and Propositions are examples
of proclamations.  You create them using the macro
\tb{proclaim}.	This sets the body of the object in a
slanted font.

\proclaim Definition 1.  A set ${\cal T} =
\{(\alpha_i,\beta_i,\gamma_i)\}_1^{N}$
of triples of integers chosen from $\{1,\ldots,n\}$ is called a
\dword{triangulation} of $\cP$ provided that
\sm
\ritem{1)} for each $1 \le i \le N$, the points $P_{\alpha_i},
P_{\beta_i},
P_{\gamma_i}$ are the vertices of  $T_i$;
\sm
\ritem{2)} each triangle contains exactly three points of $\ct$, and
these are precisely the vertices of the triangle.
\nopf

\proclaim Theorem 2.  Given any point set $\cal P$, there always
exists at least one optimal triangulation.  Moreover, every optimal
triangulation is locally optimal provided
$$ n = \cases{3,& if $m$ is odd, \cr
	      0,& otherwise. \cr}$$

\pf
Here we have used the \tb{proclaim} macro to state a theorem.
If you are not giving a proof, end your theorem with \tb{nopf}.
Otherwise, leave a blank line and use
\tb{pf} to begin the proof.  We
have also illustrated the use of the macro \tb{cases}.
The end of the proof is marked with a square box by using the
\tb{eop} macro. If you need to put a black box in a mathematical
display, use \tb{meop}. \eop

\sect{4. Lists}
The usual \Tex macro for creating lists is called \tb{item}.
However, if you are creating lists within a proclamation,
as was done in Def.~1 above, it looks better to use the
macro \tb{ritem} which is defined in {\tt \macname.tex}
since it sets the numbers in roman rather than slant.
We prefer that you use a right parenthesis
after each number rather than a period.

If you don't want to number the items in a list, you can
use our macro \tb{bull} in place of \tb{item} or \tb{ritem}.
For an example, see the list in Sect.~14 below.

\sect{5. Open Face Letters}
Use the macros \tb{RR} and \tb{CC} for the sets $\RR$ and $\CC$
of real and complex numbers.
You may also want to use the macros \tb{NN}, \tb{PP}, and \tb{ZZ} to get
$\NN$, $\PP$,  and $\ZZ$.
If you have to use $\RR$ in the title, use the macro \tb{RRt}.
You may also use the macros
\tb{Cbb},
\tb{Dbb},
\tb{Kbb},
\tb{Nbb},
\tb{Pbb},
\tb{Qbb},
\tb{Rbb},
\tb{Tbb},
and \tb{Zbb},
to get $\Cbb, \Dbb, \Kbb,\Nbb,\Pbb,\Qbb,\Rbb,\Tbb,\Zbb$.


\sect{6. Bold Faced Characters}
To create bold-faced mathematical symbols, you can
use \tb{bfm}.
Thus to get $\bfm{n}$ you type 
{\tt \back bfm\lb{}n\rb}.
For non-mathematical letters, it is better to use the usual
bold face command, i.e., to get {\bf n} you type
{\tt\lb{}\back bf n\rb}.
If you need a bold-faced symbol in a subscript or superscript,
use \tb{bfs}.	For a subsubscript or supersuperscript,
use \tb{bfss}. For example to get $n_\bfs{b}$ you type
{\tt \$n\_\back{}bfs\lb{}b\rb\$}.

\sect{7. Displayed Equations}
Equations can be numbered or not numbered as you like.	However,
if you use them,
we want equation numbers to appear on the right. The appropriate
macro is  \tb{eqno}. For these short papers, we suggest numbering
sequentially starting with (1).

If a sentence ends within a displayed formula,
the period should be inserted in the formula after the last symbol.
If the sentence continues, but a comma is appropriate at the
end of the displayed formula, put it there.  {\it Do not} insert any
extra space in front of such punctuation.

To line up displayed equations you can use \tb{eqalign}
or \tb{eqalignno}. Here is an example
$$\eqalign{N& =  2n-\bfm{n}_b-2, \qquad 1 \le n \le 16, \cr
	    e & =  3n-\bfm{n}_b-3, \qquad 1 \le n \le 4. \cr}\eqno(1)$$
To see how this was achieved, study the \Tex file {\tt instruct.tex}
which generated this document. The idea is simple -- you use
the symbol $\&$ to mark the alignment points, and you end each
line with \tb{cr}. This is one of the few things that
La\TeX{} users have to learn to become plain \Tex users.

To line up equations while
giving each of them its own number, you use \tb{eqalignno}.
Here is an example:
$$\eqalignno{N& =  2n-\bfm{$n$}_b-2, \qquad 1 \le n \le 16, & (2), \cr
	    e & =  3n-\bfm{$n$}_b-3, \qquad 1 \le n \le 4.& (3) \cr}$$

\sect{8. Figures}
Almost everyone these days creates figures in postscript.  These
can be integrated into your manuscript by inserting some
simple commands into your \Tex files.  For details, see the
document {\sl Inserting Postscript Figures Using
the \conf Macros}
which corresponds to the \Tex file {\tt psfigures.tex} included
with this macro package.

All postscript files corresponding to figures should have names
which end with the suffix
{\tt fig}.  I suggest
{\tt name1.fig}, {\tt name2a.fig}, {\tt name2b.fig},
etc., where {\tt name}
is your name or some abbreviation of it.

We have no problem dealing with postscript files whose sizes
are a few megabytes.  But if you are creating postscript
files which are 10MB or more, you probably should be working with
photographs.

Figure captions should be created with the macro \tb{figcap}.
If your caption does not fit on one line, use the
macro \tb{figcaplong}.

If you need to use hardcopy figures, contact us for instructions.

\sect{9. Photos}
Photographs can be included in the book, but
due to exorbitant printing costs, they cannot be reproduced in
color.  All photos will be reduced to black and white halftones.
Photos can be submitted either as
prints (any size is acceptable) or slides, and can be either
color or black and white.
If you have high-resolution images on a computer screen, it is
usually better to photograph them as opposed to taking a digital screen
snapshot.  This gives higher resolution, and avoids working with
huge postscript files (often several megabytes).

\sect{10. Tables}
Tables can be created with standard \Tex commands.  If you prefer,
you can also create your table with La\TeX, convert it to
postscript, and then treat it like a figure.
Table captions should be created with the macro  \tb{tabcap}.
If your caption does not fit on one line, use
the macro \tb{tabcaplong}.

\sect{11. Using Labels and Automatic Numbering}
Many people seem to think that \Tex is inferior to La\Tex because
La\Tex can do automatic numbering.  Nothing could be further
from the truth.  Automatic numbering and labelling of
equations, formal statements (Lemmas, Theorems, etc.), figures,
Tables, and references is easy to do in \Tex using
the numbering macros (due to Carl de\thinspace{}Boor) which
are included in the macro file {\tt lls00.tex}.
For more details, see the document \eword{Automatic Numbering Using
the \conf \Tex Macros} which corresponds to the \Tex file
{\tt autonumber.tex}.

\sect{12. Emphasizing Words}
In order to make the book more useful, we are considering adding an
index.  To make this work, we need your help.  When you prepare your
paper, consider which words people are likely to want to find in the
index (this would include all words which you are defining for the
first time, but also other important concepts you may be making use of).
These are the words you would most likely be emphasizing with a special
font (people often use italic).  Thus, instead of writing
{\tt \lb{}\tb{it} triangulation\rb}
to distinguish the word being defined in Def.~1 above, please write
{\tb{dword} \tt \lb{}triangulation\rb}. 

The macro \tb{dword} should \eword{not} be used for words that you merely
want to emphasize, such as the word ``not" in this sentence.
For this we have another macro called \tb{eword}.  To emphasize the
word not, you simply type \tb{eword}{\tt\lb{}not\rb}.
Mnemonically, the d stands for ``define", and the e stands for
``emphasize".

\sect{13. A Few Other Macros}
The macros in {\tt \macname.tex} are grouped according to function,
and are in alphabetical order within the groups.
Besides the ones mentioned above, I have included a few others
which you may find useful.  For example, you may want to use
\tb{sm} for \tb{smallskip}, \tb{ms} for \tb{medskip}, and \tb{bs}
for \tb{bigskip}.  Also, \tb{noin} for \tb{noindent}.
To get $\frac{a}{b}$,
you can type \$\tb{frac\lb{}a\rb\lb{}b\rb\$}.
I have added a macro \tb{boxit} which can be used to draw a box
around some block of material.

\sect{14. Some Things Not to Do}
Here is a short list of some things you should {\bf not} do:
\ms
\bull Do not use La\Tex or AMS \Tex.
\sm
\bull Do not create a \Tex file by using some conversion software
to change a ``word" document or TROFF document to \Tex.
\sm
\bull Do not type your \Tex file using any special characters
you may have on your keyboard or in your editor.  This applies
especially to editors that have keys for letters with diacritical
marks like \"a, \"u, \'a, \`a, \~n, etc.
\sm
\bull Do not change any of the macro files.
\sm
\bull Do not redefine any of the standard macros of \Tex
or any of the macros in {\tt \macname.tex} or {\tt lls00.tex}
(\ie{} before you define a macro for yourself, 
check to make sure that it has not already been defined).
\sm
\bull Do not try to squeeze more into your paper by changing
\tb{hsize}, \tb{vsize}, \tb{baselineskip}, \tb{magstep},
or by using different fonts.
\sm
\bull Do not add extra space between paragraphs.
\sm
\bull Do not add spaces before punctuation.  \Tex does a very
  good job of spacing, and we just have to edit all such extra
  space out.
\sm
\bull Do not use footnotes.  Simply put what you want to say in
  the text at the appropriate place (possibly in parentheses).
\sm
\bull Do not put a colon before every displayed equation.  The
  only correct time to use a colon there is if you are saying
  something like {\sl ... the following equation:}
\sm
\bull Do not simply copy references from some file you may have
  on hand.  We have a specific reference style which is spelled
  out in Sect.~16 below. Please follow it.
\sm
\bull Do not use \tb{sl}, \tb{it}, or \tb{bf} to emphasize words.
Instead, use \tb{dword} and \tb{eword} as explained in
Sect.~12.
\sm
\bull Do not type long lines. \Tex doesn't care if you cut sentences
up into short pieces (just look at the file {\tt submit.tex} which
generated this document).  Many editors automatically break words
at around 80 characters, and this means that if you have lines with
more than 80 characters in them, my editor may break your words in
a bad place.  Just hit the \eword{Return} or \eword{Enter} key
whenever you reach 60 or 70 characters on a line.  I sometimes get
files where whole paragraphs are typed on a single line -- they are
extremely hard to edit.

\sect{15. The de Boor/Schumaker Bibliography}
I have been maintaining a bibliography of papers in approximation
theory and CAGD (with emphasis on splines) for many years.
Carl de\thinspace{}Boor joined
in this effort several years ago, and has contributed greatly
to making the bibliography a more useful tool.  Our idea is
to have an (authenticated) file of references
and a very flexible set of
\Tex macros for inserting them into \Tex documents.
The bibliography is available on the web at
$$\hbox{\tt http://www.math.ohio-state.edu/JAT/DATA/TOC/toc.html}$$

The macro file {\tt lls00.tex} includes macros which
can be used to insert entries into your \Tex file. While
at first glance the use of this bibliography may seem complicated,
we believe it is quite easy to use -- for details, see [2].
We would be very happy if you decide to use these macros
for your paper, and even more pleased if you would help us
with the bibliography by submitting additions and
corrections (especially to your own work).

\sect{16. References}
As in all good books and journals,
we would like to have a uniform way of listing references.
We want different styles for journal papers, proceedings papers,
books, unpublished reports or preprints, and dissertations.
You can save us all a lot of trouble if you prepare your
reference list carefully according to the following rules:
\ms

\bull  Arrange your references in alphabetical order.
\sm
\bull  Use the \tb{ref} macro; it
automatically numbers your references.
To see how to use it,
look at the file {\tt instruct.tex} which generated this document).
\sm
\bull  For uniformity, please follow the style for journals,
proceedings, books, and preprints illustrated in the references
listed below.
\sm
\bull  Insert spaces between the initials of author
names.	 Thus, you should type I. J. Schoenberg and not I.J. Schoenberg.
\sm
\bull The references below show how to handle multiple
authors.
\sm
\bull  Capitalize each word in the title of a book, but
only the first word in the titles of articles, etc.
\sm
\bull Use regular roman fonts for everything except for
the titles of books, which should be set in slant font using \tb{sl}.
\sm
\bull Type two dashes between page numbers to get 55--89.
\sm
\bull We would like to have the volume numbers of journals typeset
 in bold face as illustrated in the first item in the references below.
\sm
\bull  You can save typing while using the official AMS
abbreviations for journals by using macros included in
{\tt \macname.tex}.	For example, for the journal Computer Aided
Geometric Design, you type \tb{CAGD}; see the first reference
below.	Note, to insert a space after such a macro, type
the symbol $\sim$.  Thus, the reference [1] below is typed as
follows:
$$ \hbox{\tt \back CAGD$\sim$\lb\back bf 4\rb~(1987), 105--123.} $$
\sm
\bull Similarly, we have included macros for several proceedings
you might like to refer to. They are called  
\tb{ulvik}, \tb{lillehammer}, \tb{osloII},
\tb{chamonixI}, \tb{chamonixIIa}, \tb{chamonixIIb},
\tb{chamonixIIIa}, \tb{chamonixIIIb}, 
\tb{stmalod}, \tb{stmalof},
\tb{texasI}, \tb{texasII}, \tb{texasIII}, \tb{texasIV}, \tb{texasV},
\tb{texasVI}, \tb{texasVII}, \tb{texasVIIIa}, \tb{texasVIIIw},
\tb{texasIXt}, \tb{texasIXc}, \tb{texasXa}, and \tb{texasXs}.
As an example, reference [2] below was typed using \tb{oslo}.
We have also created
\tb{stmalotwo} to use for cross references to
papers in this proceedings, see reference [3] below.
\ms

\noin To cite references listed in the References section,
write their numbers in square brackets as [4] and
[1,2,3].

%******************************* REFERENCES *************************

\References

\ref Alfeld, P., B. Piper, and L. L. Schumaker, Minimally supported
bases
for spaces of bivariate piecewise polynomials of smoothness $r$ and degree
$d \ge 4r+1$, \CAGD~{\bf 4} (1987), 105--123.

\ref D{\ae}hlen, M., On the evaluation of box splines, in \oslo,
167--179.

\ref Author, A., and B. Author, Some paper to be cross-referenced
in this proceedings, in \stmalotwo, xxx--xxx.

\ref  Schumaker, L. L., {\sl Spline Functions: Basic Theory}, Wiley, New
York, 1981.


%******************************* ADDRESS *************************

\Address{
Larry L. Schumaker
Dept. of Mathematics
Vanderbilt University
Nashville, TN 37240
{\tt s@mars.cas.vanderbilt.edu
http://www.math.vanderbilt.edu/$\sim$schumake/}
}

\bye
%% END of file instruct.tex

