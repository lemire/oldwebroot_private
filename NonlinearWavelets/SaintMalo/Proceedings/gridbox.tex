%************************* SAINT-MALO MACROS **************************
% FILE NAME: gridbox.tex
% DATE: 4/20/99
% CREATED BY: Carl deBoor
% EMAIL: deboor@cs.wisc.edu
% WEB: http://www.cs.wisc.edu/$\sim$deboor/
% ******************************************************************

% The macro
%                         \gridbox#1/#2/#3
% builds a box vertically to #1 truecm and horizontally to #2 truecm, 
%overlays that box with gridlines exactly 1truecm apart (with every fifth 
%line heavier).
% Grid is turned off by saying \showgridfalse.  #3 should consist of 
%one or more statements of the form
%   \point(d,r){...}  or \cpoint(d,r){...} or  \ppoint ij(d,r){...}
% which have the effect of putting the material {...} onto the page in such
% a way that its upper left corner, respectively its center, respectively its
% i,j-point (with, e.g., 1,0 the left midedge and 2,2 the lower right corner)
% ends up  d truecm  down and  r truecm  to the right of the upper left
% corner of the gridbox.
%    Note that {...} might be the local way to include a picture file. E.g.,
%         \point(#1,0){\special{ps: plotfile <name>.ps}}
% would include the file <name>.ps (presumably a postscript file), giving it
% the lower left corner of the gridbox as the  currentpoint  (and thus
% controlling the placement of the figure in  <name>.ps  on the page).
%
% TeX should be in vertical mode when calling on \gridbox. E.g.,
%    \centerline{\vbox{\gridbox3/4.4/{\cpoint(1,1){$0$}\ppoint21(2,1){$0$}}}}
% would generate a centered gridbox and print a zero centered at the (1,1)
% point of that box and also one centered atop the (2,1) point.
%     See \gridbox uses in  \Figpls, \Figinbox, etc. in ./ps/figure.tex
%
% The macro
%                         \ptgridbox#1/#2/#3
% differs from gridbox only in that the units are pt rather than truecm and,
% correspondingly, the spacing of gridlines is by 20 units rather than 1.
% This macro is particularly useful in conjunction with recent versions of
% ghostview which supply the current pointer position in pt (albeit as
% (pt from left, pt from below) rather than the (pt from above, pt from left)
% used here). Be sure to use \magnification1000 while checking placements.

\def\gridunits{truecm}\newcount\tickskip\tickskip1\newcount\majortick\majortick5

\def\ptgridbox#1/#2/#3{
{\def\gridunits{pt}\tickskip20\multiply\majortick by\tickskip%
 \gridbox#1/#2/{#3}}}

\def\gridbox#1/#2/#3{
\vbox to #1\gridunits{#3
\ifshowgrid\tickcount=0
  \loop\cgridw%
   \vbox to 0pt{\kern\tickcount \gridunits\hrule width#2\gridunits
height\gridwidth\vss}
   \nointerlineskip \advance\tickcount by \tickskip
   \ifdim\tickcount pt<#1pt\repeat % Note: #1, #2 need not be integers
  \hbox to 0pt{\tickcount=0\hbox to 0pt{\tick#1/\hss}\advance\tickcount by
\tickskip%
 \loop\ifdim\tickcount pt<#2pt\nexttick\tickskip#1/\advance\tickcount by
\tickskip \repeat\hss}
\else \vbox to 0pt{\hrule width#2\gridunits height0pt\vss}
\fi\vfil}\vfil}

% The next macro is used in the macros for placing material within the grid.
\def\dpoint(#1,#2)#3{\vbox to 0pt{\kern#1
   \hbox{\kern#2{#3}}\vss}\nointerlineskip}

% The following places the upper left corner of {#3} at gridpoint (#1,#2)
\def\point(#1,#2)#3{\dpoint(#1\gridunits,#2\gridunits){#3}}
% The following places the center of {#3} at gridpoint (#1,#2)
\def\cpoint(#1,#2)#3{\ppoint11(#1,#2){#3}}
% The following places the #1,#2-point of {#5} at gridpoint (#3,#4), with
% 0,0 the upper left corner, 1,1 the center, and 2,2 the lower right corner,
% and, correspondingly, 0,1 the center of the upper edge, etc.
\def\ppoint#1#2(#3,#4)#5{\setbox0=\hbox{#5}
   \dimen0=\ht0\advance\dimen0 by\dp0\divide\dimen0 by-2
   \multiply\dimen0 by#1\advance\dimen0 by#3\gridunits
   \dimen1=\wd0\divide\dimen1 by-2\multiply\dimen1 by#2
   \advance\dimen1 by#4\gridunits\dpoint(\dimen0,\dimen1){#5}}

% The next macros are used in generating the grid:
\newcount\rmndr   % this macro returns the remainder of #1 mod #2
\def\rem#1#2{\rmndr=#1{}\divide\rmndr by#2{}%
\multiply\rmndr by-#2{}\advance\rmndr by #1}
\def\cgridw{\gridwidth\finegridw{}\rem\tickcount\majortick% every fifth line,
\ifnum\rmndr=0{}\gridwidth\roughgridw\fi}          % ... use double-thick line.

\def\tick#1/{\cgridw\vrule width\gridwidth height0pt depth#1\gridunits}
\def\nexttick#1#2/{\hbox to#1\gridunits{\hfil\tick#2/}}
\newcount\tickcount
\newdimen\finegridw\finegridw0.4pt\newdimen\roughgridw\roughgridw1.6pt
\newdimen\gridwidth
\newif\ifshowgrid \showgridtrue

% %Test \gridbox with the following code (only the end crosses should show):
% \newcount\hor\newcount\ver
% \centerline{\vbox{\gridbox20/16/{
% \hor0\loop\ver0
% {\loop\cpoint(\hor,\ver){$+$}
% \advance\ver by 1\relax\ifnum\ver<17\repeat}
% \advance\hor by 1\relax\ifnum\hor<21\repeat
% }}}

%END of gridbox.tex

