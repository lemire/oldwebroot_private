%*****************  STMALO2 MACROS ******************
% FILE NAME: splinebib.tex
% COPYRIGHT: 6/9/96 -- updated 6/1/2002
% CREATED BY: Carl de Boor and Larry Schumaker
% EMAIL: s@mars.cas.vanderbilt.edu
% WEB: http://www.math.vanderbilt.edu/~schumake/

\count100= 1
\count101= 7
\input malo2
\input lls00

\title{Using the de\thinspace{}Boor/Schumaker}
\titwo{Spline Bibliography}
\author{Carl de\thinspace{}Boor and Larry L. Schumaker}
\def\shorttitle{Using the Spline Bibliography}
\def\shortauthor{C. de\thinspace{}Boor and L. L. Schumaker}

\def\back{{\tt\char'134}}
\def\lb{\char'173}
\def\lc{$\{$}
\def\rb{\char'175}
\def\rc{$\}$}
\def\tb#1{{\tt\back #1}}
\def\Tex{{\TeX~}}
\def\per{\kern .03em \% \kern .02em}

\abstract{In this note we explain how to access and
use the spline bibliography
of de\thinspace{}Boor/Schumaker. We also discuss how to
use the macro file {\tt lls00.tex} contained in the \eword{\conf
Macro Package} to
incorporate bibliographic entries into your \Tex file.}

\sect{1. Introduction}
The second author has
been maintaining a bibliography of papers in approximation
theory and CAGD (with emphasis on splines) for many years.
The first author joined in this effort several years ago,
and recently the entire project has been made public domain.
The idea is to maintain an (authenticated) set of references
along with a very flexible set of
\Tex macros for inserting them into \Tex documents.
The bibliography is available on the world wide web at
$$ \eqalign{
& \hbox{\tt http://www.cs.wisc.edu/$\sim$deboor/bib}\cr
&\hbox{\tt http://www.math.ohio-state.edu/JAT/DATA/SPECIALS/sbd.html} \cr
&\hbox{\tt http://www.math.technion.ac.il/JAT/DATA/SPECIALS/sbd.html} \cr}$$
All sites include search engines which allow you to search
the bibliography in various ways.

This document is divided into several sections.
Sect.~2 is meant for the casual user who simply wants to find
references.  Sect.~3 is meant for the user who wants to
copy references from the bibliography into a \Tex file.
In Sect.~4 we discuss how users can contribute
to this project by submitting additions and corrections.
Labelling and some examples are discussed in Sects.~5 and 6.
\eject

\sect{2. Organization of the Bibliography}
Entries in the bibliography are stored in 26 files, one for
each letter of the alphabet. Here is a typical entry in
the file A.
\ms {\narrower \obeylines \tt
\%AhlbergNilson66
\per{} carl
\back{}refJ Ahlberg,  J. H., Nilson, E. N.;
The approximation of linear functionals;
\back{}SJNA; 3; 1966; 173--182;
\medskip}

Within each file, the entries are
sorted alphabetically according to a \dword{handle} which has been
assigned to each paper.  The handle appears in the first
line of each entry, and is
derived from the last names of the authors, plus the year of
publication, followed optionally by a letter if needed to make the
handle unique.  The handle assigned to the above paper
is {\tt AhlbergNilson66}.

Many (but unfortunately not all)
entries in the bibliography have been
authenticated. This means that one of us (or
one of our graduate students or relatives) has actually checked the
entry by looking at a physical copy of the article.
If an entry has been
authenticated, the person who did it is identified in
the second line of the entry with a line starting with {\tt \%}
followed by at least one blank.
The example entry above was authenticated by Carl.
If this line is missing in an entry, that means that the
entry has not been authenticated. However, the user should be
aware that there are still some errors
even in the authenticated entries.

The forms of the entries vary according to their nature.  We
have classified entries as 5 different types:
\sm
\ritem{1)} articles in proceedings
\ritem{2)} books
\ritem{3)} dissertations
\ritem{4)} journal articles
\ritem{5)} preprints and reports.
\ms
The type of an entry is distinguished by the \Tex macro name which
introduces the entry.  It can be \tb{refB} (books),
\tb{refD}  (dissertations), \tb{refP} (articles in a proceedings listed
in \tb{proceed.tex}),
\tb{refQ} (articles in a proceedings not listed in \tb{proceed.tex}),
\tb{refR} (reports), or \tb{refJ} (journal articles).

The body of each entry contains the usual pieces of information:
author(s), title, place and date of publication, and page
numbers.  Each of the different types of entries requires slightly
different information --- for a more detailed explanation, see
Sect.~4. For a complete formal description, see the last part of the file
{\tt journal.tex} which is available at the web site.

In creating the forms for these entries, we had two goals in mind.
First, we wanted to partition the information about a reference
into identifiable pieces for access by \Tex macros.
The idea is to achieve enough flexibility so that with
properly designed \Tex macros, one can match
the reference style of any given journal or proceedings volume.
The second goal was to minimize the amount of typing needed to
create an entry in the bibliography.

\sect{3. Inserting the Entries Directly in Your \Tex File}
One of our purposes in creating this spline bibliography was to
simplify the process of adding references to a \Tex document.
To add a given reference to a paper (assuming that you
are using the {\tt lls00.tex} macros), proceed as follows:
\sm
\ritem{1)} Insert the command \tb{startbib} in front of your
first reference.
\sm
\ritem{2)} Copy each reference into your \Tex file.
\sm
\ritem{3)} If an entry contains any \Tex commands (i.e.,
anything starting with a backslash), insert the word
\tb{noexpand} in front of the backslash.
\sm
\ritem{4)} Make sure that you have a copy of {\tt proceed.tex}
in the directory.
\sm
\ritem{5)} If a given reference causes an overfull hbox, you
need to find a good place to break the line, and insert the
macro \tb{noexpand}\tb{break}.
\sm
\ritem{6)} If you want to insert references which are not
contained in our bibliography, you have two choices.  Ideally,
you should prepare your references in our style,
following the instructions in the next two sections,
and using the appropriate macro \tb{refB}, \tb{refD}, \tb{refJ},
\tb{refP}, \tb{refQ}, or \tb{refR}.
As an alternative, you can
write your reference in the way that you would if you did
not use these macros. In this case you should begin your
reference with \tb{refnew} and end it with a semicolon.
\sm
\ritem{7)} If you want to label the reference, add a line
as described in Sect.~5 below.
\ritem{8)} Insert the command \tb{endbib} after your last
reference.
\ms

Without going into details, here is what happens when you
\Tex your file.   When \Tex encounters the command \tb{startbib},
it opens a file called {\tt jobname.bib}, where {\tt jobname.tex}
is the name of your \Tex file.	As your references are processed,
the entries are automatically translated into plain \Tex commands
for typesetting the entries in the \conf style.
Then the file {\tt jobname.bib} is read back in
to actually create the entries in your \Tex file.
To see how this works in practice, take a look at the file
{\tt splinebib.tex} which generates this document.

Reference [\cit{S1}] is a case where we	had to insert
a \tb{noexpand}.  The use of the macro \tb{refnew}
is also illustrated in reference [\cit{S1}].

If you decide to use the bibliography with other \Tex files
which do not input the macro file {\tt lls00.tex},
you will need to get copies of
{\tt refmac.tex}, {\tt journal.tex}, and {\tt proceed.tex}
from the web site. To use them, you input {\tt refmac.tex}
into your \Tex file with the command \tb{input refmac}.
This file is set up for a certain default style of references, but
there are instructions in the file on how to achieve other styles.

\sect{4. Creating New Entries for the Bibliography}
We would be very pleased if you would not only use
the bibliography, but would also contribute to the project
by pointing out corrections and by submitting new entries
for it.  To make this an orderly process, we are keeping
one master copy of the bibliography (from which the version on JAT's home
page is generated automatically, thanks to Paul Nevai).
It can be modified only by us. This means that to
suggest changes, you need to send a message to one of us
with your changes, or else make use of the convenient form created and
maintained by Shayne Waldron for submitting such information;
click on the appropriate word on any of the web pages listed in
Sect.~1.

To assist you in correcting existing entries and in
creating new ones, we now discuss the form
of each type of entry.	Each entry consists of from 4 to 6
main pieces of information.  Each such piece of information
is terminated with a semicolon followed by a blank in case there
is more on the same line.  (This means that we
do not allow semicolons to be part of the text of the entries.
If you need a semicolon, use the macro \tb{semicolon}, included in
{\tt lls00.tex}.)

Basically, each entry includes 1) author name(s), 2) title,
3) publication information (journal, proceedings, etc.),
4) year, 5) page numbers (if appropriate).  To see what
is required for each of the entry types, look at the
examples in Sect.~6. For a complete formal description, see the last part
of the file {\tt journal.tex} which is available at the web site.

\ms
\noin {\bf Author Names:} If there is just one author,	the list
has the form {\tt lastname, firstname}.
The first name could be replaced by  an initial or initials.
If there are two or more authors,
this pattern is repeated, using commas to separate the
various authors. Terminate the whole thing with one semicolon.
The above example entry of {\tt AhlbergNilson66} illustrates
the algorithm.
\bs

\noin {\bf Titles:} We follow the following convention for titles:
\sm
\ritem{1)} Each `nontrivial' word in the title of a book is capitalized
\sm
\ritem{2)} For all other entries, only the first word in
the title is capitalized (except for proper names).
\sm
\ritem{3)} Mathematics enclosed in dollar signs is allowed.
\bs

\noin {\bf Journals:} Names of journals are written in abbreviated
form following the conventions of the AMS. Thus, for example,
\eword{Journal of Approximation Theory} is written as \eword{J.~Approx.~Th.}.
The macro file {\tt lls00.tex} includes the macros for journal
names. You can simply look at this
file to see what the macro names are for various journals.
The general rule is that the macro name consists of the
capital letters which begin each major word of the name of the journal, with
additional lower case letters only used when needed to get a unique label.
Thus, the macro for \eword{J.~Approx.~Th.} is \tb{JAT}.
\bs

%\eject
\noin {\bf Proceedings:} Information on proceedings consists of several
parts, including
\sm
\ritem{1)} title of the proceedings
\sm
\ritem{2)} editors
\sm
\ritem{3)} publisher name and location
\sm
\ritem{4)} year of publication.
\ms
\noin As with journals, we have created a special file
{\tt proceed.tex} which contains macros for most of the proceedings
appearing in the bibliography.	Again, the user should look
at this file to see the names of the various proceedings macros.
Here the general rule is that the name reflects the \eword{place}
where the corresponding conference took place.	Thus, for example,
there are macros entitled \tb{Oslo}, \tb{Biri}, \tb{Ulvik}, \tb{Lillehammer},
\tb{ChamonixI}, \tb{ChamonixIIa},
\tb{ChamonixIIb}, \tb{ChamonixIIIa}, and \tb{ChamonixIIIb}.
Using the macro \tb{refP} in {\tt lls00.tex},
the macro \tb{Ulvik} expands out to
\ms
\noin {\sl Mathematical Methods for Curves and Surfaces},
Morten D{\ae}hlen, Tom Lyche, Larry L. Schumaker (eds.),
Vanderbilt University Press, Nashville, 1995
\ms

If you want to refer to a paper which you expect to appear in
the proceedings of this conference, use
the macro \tb{Stmalotwo}. The information in that macro may be
currently incomplete, but will be updated later when
the final camera-ready copy is prepared.

If you want to refer to a proceedings for which
there is no macro in {\tt proceed.tex}, you should use
\tb{refQ} instead of \tb{refP}.  In this case you must supply
the various parts of the information as illustrated in the
\tb{refQ} example in Sect.~5.

\sect{5. Labelling References}
To label your references (so that you can refer to
them by label), use the macro \tb{rhl}.
To create a label for a reference, simply
insert a line of the form {\tt \back{}rhl\lb{}name\rb{}},
where ``name" is the name you want to assign
to that particular reference.  This line should be inserted after
the handle and authentication lines, but before the \tb{ref..}
or \tb{refnew} macros.

The assigned labels can then be
used to cite references as described in \cite{S2}.
In particular, To cite the reference which is
labelled {\tt S1}, simply type \tb{cite}{\tt \lb{}S1\rb}.
This gives \cite{S1}. Note that the macro \tb{cite} automatically
creates brackets around the number of the cited article.
To cite several papers at one time, use the macro \tb{cit} which
does not automatically create the brackets.  For example,
you type {\tt [\back{}cit\lb{}S1\rb{},\back{}cit\lb{}S2\rb{}]} to get
[\cit{S1},\cit{S2}].


\sect{6. Examples}
In this section we list one example of each type of entry.
These entries are included in the set of references
below as references [1--7].

{\obeylines \tt
\ms
\%AhlbergNilson66
\% carl
\back{}rhl\lb{}A\rb{}
\back{}refJ Ahlberg,  J. H., Nilson, E. N.;
The approximation of linear functionals;
\back{}SJNA; 3; 1966; 173--182;
\ms

\%Bartels94
\% larry
\back{}rhl\lb{}B\rb{}
\back{}refP Bartels, R.;
Object oriented spline software;
\back{}ChamonixIIa; 27--34;
\ms

\%Cheney66
\back{}rhl\lb{}C\rb{}
\back{}refB Cheney,  E. W.;
Introduction to Approximation Theory;
McGraw-Hill, (New York); 1966;
\ms

\%DeRose85
\back{}rhl\lb{}D\rb{}
\back{}refD DeRose,  T.;
Continuity issues in computer-aided geometric design;
Univ. Cal. Berkeley; 1985;
\ms

\%McPherson82
\back{}rhl\lb{}M\rb{}
\back{}refQ McPherson,	R. D.;
Optimum interpolation\back{}semicolon practical aspects of
operational application;
(The Interaction Between Objective Analysis
and Initialization),
D. Williamson (ed.),  NCAR (Boulder); 1982; 7--12;
\ms

\%Resnikoff90
\back{}rhl\lb{}R\rb{}
\back{}refR Resnikoff,	H. L.;
Weierstrass functions and compactly supported wavelets;
Aware, Inc Report AD900513; 1990;
\ms

\back{}rhl\lb{}S1\rb{}
\back{}refnew Schumaker, L.,
Instructions for using the \conf \back{}noexpand\back{}Tex Macros,
see the \back{}noexpand\back{}Tex file
\lb{}\back{}noexpand\back{}tt instruct.tex\rb{} included in the
\conf{}Macro Package available on the conference
home page;
\ms

\back{}rhl\lb{}S2\rb{}
\back{}refnew Schumaker, L.,
Automatic numbering with the \conf macros,
see the \back{}noexpand\back{}Tex file
\lb{}\back{}noexpand\back{}tt autonumber.tex\rb{} included in the
\conf{}Macro Package available on the conference
home page;
\ms
}

\Acknowledgments{Supported in part by NSF Grant DMS-9500643.}

%******************************* REFERENCES *************************

\References
\startbib

%AhlbergNilson66
% carl
\rhl{A}
\refJ Ahlberg,	J. H., Nilson, E. N.;
The approximation of linear functionals;
\SJNA; 3; 1966; 173--182;

%Bartels94
% larry
\rhl{B}
\refP Bartels, R.;
Object oriented spline software;
\ChamonixIIa; 27--34;

%Cheney66
\rhl{C}
\refB Cheney,  E. W.;
Introduction to Approximation Theory;
McGraw-Hill (New York); 1966;

%DeRose85
\rhl{D}
\refD DeRose,  T.;
Continuity issues in computer-aided geometric design;
Univ.~Cal.~Berkeley; 1985;

%McPherson82
\rhl{M}
\refQ McPherson,  R. D.;
Optimum interpolation\semicolon practical aspects of
operational application;
(The Interaction Between Objective Analysis and Initialization),
D. Williamson (ed.),  NCAR (Boulder); 1982; 7--12;

%Resnikoff90
\rhl{R}
\refR Resnikoff,  H. L.;
Weierstrass functions and compactly supported wavelets;
Aware, Inc Report AD900513; 1990;

\rhl{S1}
\refnew Schumaker, L.,
Instructions for using the \conf \noexpand\Tex macros,
see the \noexpand\Tex \noexpand\break file {\noexpand\tt instruct.tex}
included in the \conf Macro Package
available on the conference home page;

\rhl{S2}
\refnew  Schumaker, L. L.,
Automatic numbering with the \conf macros,
see the \noexpand\Tex file {\noexpand\tt autonumber.tex}
included in the \conf Macro Package
available on the conference home page;

\endbib

\Address{
Carl de\thinspace{}Boor
Computer Sciences Dept.
1210 West Dayton St.
University of Wisconsin
Madison, WI 53706
{\tt deboor@cs.wisc.edu
http://www.cs.wisc.edu/$\sim$deboor/}}

\Address{
Larry L. Schumaker
Dept. of Mathematics
Vanderbilt University
Nashville, TN 37240
{\tt s@mars.cas.vanderbilt.edu
http://www.math.vanderbilt.edu/$\sim$schumake/}
}

\bye

%%%%%%%  END of splinebib.tex
