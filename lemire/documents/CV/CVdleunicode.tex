%%% LyX 1.3 created this file.  For more info, see http://www.lyx.org/.
%% Do not edit unless you really know what you are doing.
\documentclass[american,titleabove,10pt,contbibnum]{cv}
\clubpenalty=10000
\widowpenalty=10000
\brokenpenalty=10000
  \hyphenpenalty=5000
  \tolerance=1000
\sloppy
\usepackage[utf8]{inputenc}
\setlength{\parindent}{0in}
\usepackage{enumitem}
\setlist{nolistsep}
\usepackage{verbatim}
% \newcommand{\begin{itemize}}{
%    \begin{list}{$\bullet$}
%     { \setlength{\itemsep}{0pt}      \setlength{\parsep}{3pt}
%       \setlength{\topsep}{3pt}       \setlength{\partopsep}{0pt}
%       \setlength{\leftmargin}{1.5em} \setlength{\labelwidth}{1em}
%       \setlength{\labelsep}{0.5em} } }
% \newcommand{\begin{itemize}two}{
%    \begin{list}{$\bullet$}
%     { \setlength{\itemsep}{0pt}    \setlength{\parsep}{0pt}
%       \setlength{\topsep}{0pt}     \setlength{\partopsep}{0pt}
%       \setlength{\leftmargin}{2em} \setlength{\labelwidth}{1.5em}
%       \setlength{\labelsep}{0.5em} } }


\usepackage[american]{babel}
\addto\captionsamerican{\renewcommand{\refname}{\vspace{-14pt}}}% can be bibname

\makeatletter

%%%%%%%%%%%%%%%%%%%%%%%%%%%%%% LyX specific LaTeX commands.
%% Because html converters don't know tabularnewline


%%%%%%%%%%%%%%%%%%%%%%%%%%%%%% User specified LaTeX commands.
%% You can modify the fonts used in the document be using the
%% following macros. They take one parameter which is the font
%% changing command.
%% \headerfont: the font used in both headers.
%%              Defaults to sans serif.
%% \titlefont:  the font used for the title.
%%              Defaults to \LARGE sans-serif semi bold condensed.
%% \sectionfont: the font used by \section when beginning a new topic.
%%              Defaults to sans-serif semi bold condensed.
%% \itemfont:   the font used in descriptions of items.
%%              Defaults to sans-serif slanted.
% to make your name even bigger, uncomment the following line:
% \titlefont{\Huge}
%%
%% You can modify the following parameters using \renewcommand:
%% \topicmargin: the left margin inside topics.
%%               Defaults to 20% of the text width (0.20\textwidth).
% To get more room for left column of Topic layouts, uncomment following line:
% \setlength{\topicmargin}{0.3\textwidth}
%\usepackage{epsfig}
%\usepackage{amsmath,amssymb}
%\usepackage{pslatex}
\usepackage{color}


 \usepackage[pdftex,urlcolor=black,colorlinks=false]{hyperref}
 \pdfinfo{
            /Title       ( Daniel Lemire's c.v.)
            /Author     (Daniel Lemire, Ph.D.)
            /Subject  (Daniel Lemire's research, contributions and accomplishments .)
            /Keywords   ()
          }

%\usepackage[authoryear,sort&compress]{natbib}%

%\addto\captionsamerican{\renewcommand{\refname}{\vspace{-14pt}}}% can be bibname

\makeatother
\begin{document}

\leftheader{
%UER Sciences et Technologies\\
%T��universit�\
Centre de recherche LICEF\\
TELUQ, Université du Québec\\
Pavillon Saint-Urbain\\
%Université du Québec à Montréal (UQAM)\\
5800 Saint-Denis\\
Office 1105\\
Montreal (Quebec)\\
H2S 3L5 Canada}
\rightheader{
lemire@gmail.com\\
\url{http://lemire.me/en/}\\
~\\
Canadian citizen
}


\title{Daniel Lemire, Ph.D.}

\maketitle

%\textbf{Vision statement}: insure that technology continues to make individuals and communities smarter.
%\textbf{Keywords}: Collaborative Data Management, Database Theory, Data Warehousing, Multidimensional
%Databases (OLAP), Data Mining, Time Series, Collaborative Filtering, Information Retrieval.



\section{Experience}

\begin{topic}
\item [2004--\ldots] Computer Science Professor\\
LICEF Research Center, TELUQ, Université du Québec\\
Promoted to Full Professor in 2009%\\
%TELUQ%, \textbf{Université du Québec à Montréal (UQAM)}
%\url{http://www.professeurs.uqam.ca/pages/lemire.daniel.htm}

\item [2002--2004] Research Officer in eBusiness\\
%Institute for Information Technology, E-business\\
National Research Council of Canada (NRC)\\
Team Leader of the e-Health Research Group in 2002--2003


\item [2001--2002] Assistant Professor\\
%Mathematics and Statistics Department\\
Acadia University

\item [1999--2001] Entrepreneur\\
%Digital Signal Processing (Medical Images and Geophysics)\\
TechElements Inc. and Ondelette.com

\item [1998--1999] Post-doctoral fellow \\
%Signal processing (ECG, EEG) \\
Institut de génie biomédial

\end{topic}
\section{Education}

\begin{topic}
\item[1995--1998] Ph.D.---{}``Iterative Schemas''\\ %in Engineering Mathematics\\
%Mathematics and Industrial Engineering\\ % Department\\
%``Iterative Schemes and Wavelets''\\
\textbf{École Polytechnique de Montréal} and \textbf{Université de Montréal}\\
Supervisors: Gilles Deslauriers and Serge Dubuc

\item [1994--1995] M.Sc.---{}``A Priori Approximations for Nonlinear Systems''\\
\textbf{University of Toronto}\\
Supervisor: Catherine Sulem

\item [1990--1994] B.Sc. with ``High Distinction''\\
\textbf{University of Toronto}
\end{topic}

\section{Competitive scholarships and prizes as a student}
\begin{itemize}
%\item
%\end{itemize}
%\begin{itemize}
\item FCAR scholarship (Ph.D.) and NSERC scholarship (M.Sc. and Ph.D.);
\item C. D. Howe Memorial for 4 years ($\approx$\$50,000);
\item NSERC Canada Scholarship for 4 years (\$4,000);
\item St.Michael's College Dean's medal (class of 1994);
\item 3T0 Scholarship (UofT). % (\$1,000).
%\end{itemize}
\end{itemize}
\section{Secondary affiliations}

\begin{itemize}
\item I am an adjunct professor at the University of New Brunswick
and at the Université du Québec à Montréal
where I supervise graduate students. % in the Computer Science and Applied Statistics Department.
\end{itemize}

\section{Recent Teaching Experience}


%\subsection{Academic Programs}
%\begin{itemize}
%\item Chair of the graduate certificate in Computer Science and Knowledge Management~(2009-\ldots);
%\item Chair of the certificate in Applied Computer Science (2007-\ldots);
%\item Chair of the distance learning Science B.Sc. (2005-2007);
%\item Chair of the science and technology certificate (2005-2007).
%%\item Chair of the ``comité d'orientation et de développement des programmes en sciences'' (Science Programs Strategic Committee) in 2004-2005.
%\end{itemize}

Undergraduate (in French):
\begin{itemize}
\item INF 1250 - Introduction aux bases de données;
\item INF 6460 - Recherche et filtrage d'informations;
\item INF 6450 - Gestion de l'information avec XML;
\item INF 9002 - Évaluation et analyse des systèmes d'information;
\item INF 9004 - Informatique des entrepôts de données.
\end{itemize}

Graduate (in French):
\begin{itemize}
\item INF 6104 - Recherche d'informations et web;
\item INF 6107 - Le Web social;
\item INF 6408 - Informatique de l'analyse multidimensionnelle.
\end{itemize}

Moreover,  I have chaired and cochaired several programs: two~certificates,
one bachelor degree, and four~graduate programs including the master in Information Technology.


\section{Professional Membership}

\begin{itemize}
\item Association for Computing Machinery (ACM).
%\item ACM Special Interest Group on Electronic Commerce.
%\item ACM Special Interest Group on Information Technology Education.
%\item Institute of Electrical and Electronics Engineers, Inc. (IEEE).
%\item Society for Industrial and Applied Mathematics (SIAM).
%\item W3Québec
%\item Canadian Semantic Web Interest Group (SWIG), first founding member.
%\item Canadian Society for the Computational Studies of Intelligence (CSCSI)
\end{itemize}

\begin{comment}

\section{Supervision}

%Current:
\begin{itemize}
\item Ali Assi (Ph.D., 2012--\ldots);
\item Ferenc Molnar (M.Sc., 2012--\ldots);
\item Samy Chambi (Ph.D., 2012-\ldots);
\item Kenza Sakout Andaloussi (Ph.D., 2012--\ldots);
\item Tarek Khei (Ph.D., 2012-\ldots);
\item Ludovic Bocken (Ph.D., 2011--\ldots);
\item Badis Merdaoui (Ph.D., 2008--\ldots).
\item Rohit Patidar (M.Sc., 2012--\ldots);
\item Kushal Mehra (M.Sc., 2012--\ldots);
\item Danièle Massicotte (M.Sc., 2012--\ldots);
\item Charbel Matni (M.Sc., 2011--\ldots);
\item Djamel Garar (M.Sc., 2011--\ldots);
\item Julien Marais (M.Sc., 2011--\ldots);
\item Ikhtear Sharif (M.Sc., 2011--\ldots).
\end{itemize}

Recent graduates:
\begin{itemize}
\item Eduardo Gutarra (M.Sc., 2010--2012) is a software engineer with TIBCO in Texas. His thesis was on Multi-column Bitmap Indexing.
\item Hamid Douzi (D.E.S.S., 2010--2011) is  senior SQL \& B.I developer at IMBD (Montreal, Canada);
\item Hazel Webb (Ph.D., 2005--2010) is now Assistant Professor at University of New Brunswick (Saint John, Canada);
\item Minfang Tao (Master, 2011) is business analysist with BundleRank.
\item  Idris Laabassi (Master, 2010) is now a sales and Information Technology manager at OTIS (Lybia);
\item Kamel Aouiche (Post-doct., 2006--2008) is now a consultant at Objet direct (Lyon, France);
\item Steve Keith (M.Sc., 2004--2006) is now Data Architect at Innovatia (Saint John, Canada).
\end{itemize}


\end{comment}

%
% During the summer of 2003, Sean McGrath, a student from the University
% of New Brunswick, joined us to work on the Lemur Project. The Lemur
% Project is software written in C++ and Python for fast computations
% over data cubes. I then co-supervised Sean with
% Harold Boley on the project ``RACOFI Composer'', and then I
% provided NSERC funding to Sean so he could work with us a third term.
% One result of this work was the site inDiscover (http://www.indiscover.net).
%
% During the Winter 2003 term, I supervised Nancy Howse from Acadia
% University as a CO-OP student in the context of Scale and Translation
% Invariant Collaborative Filtering Systems. The project was called
% COFI Music and it was a subproject of the \textit{Sifter} project with Mosaic
% Corporation. Following the success of the first term, Nancy returned
% for a second term during the summer of 2003. The result of her work
% is a servlet-based web site (racofi.elg.ca) showing how the theory can
% be applied.
%
%  %As of February 2005, Nancy Howse works for Blue Cross in Halifax.
%
% Moreover, in 2002-2003, with Harold Boley, I co-supervised Marcel Ball (2 COOP terms), Stephen Greene (1 COOP term) and Michelle Anderson (1 COOP term).
%%I supervise UQAM students in the Computer Science  and Cognitive Computer Science Ph.D. programs.
%I coauthored several of my talks, articles and reports with my students and fellows~\cite{elearningsummit, COLA2003, elearningsummitpaper, TRD01, BELL2004-TR,TRD05001,apics2005,kamel1,dolap2007}.
%%%%%Abdallah, Moatez Ben

% \begin{small}\begin{center}\begin{tabular}{|p{0.2\columnwidth}|c|c|}
% \hline
% Name&
% Years&
% Program\tabularnewline
% \hline
% Nacima Hassaine &
% 2010--\ldots &
% M.Sc.  \tabularnewline
% Loukmane Arzim &
% 2010--\ldots &
% Ph.D. \tabularnewline
% Eduardo Gutarra &
% 2010--\ldots &
% M.Sc. \tabularnewline
% %Wilfrid Blais &
% %2009--\ldots &
% %Ph.D. \tabularnewline
% %Komi Sodoké &
% %2009--\ldots &
% %Ph.D. \tabularnewline
% Badis Merdaoui &
% 2008--\ldots &
% Ph.D. \tabularnewline
% Kamel Aouiche&
% 2006--2008&
% Post-doct. \tabularnewline
% Hazel Webb&
% 2005--2010&
% Ph.D.\tabularnewline
% Steven Keith&
% 2004--2006&
% M.Sc.  \tabularnewline\hline
% \end{tabular}\end{center}\end{small}


\section{Research program}
%My research program is focused on indexing multidimensional data sets in the context of
%(nearly) infinite storage capabilities. The data
%may come from time series, text documents or business transactions.
%\\
%
%\begin{itemize}
%\item OLAP and Very Large Databases: fast queries and updates
%by wavelets and efficient storage through hybrid OLAP~\cite{CASCON2002,IS2006,DOLAP2003} and
%text mining~\cite{apics2005}. This work is done
%in close collaboration with Owen Kaser (UNB/SJ).
%\item eBusiness Data Mining and Collaborative Filtering~\cite{elearningsummitpaper,SIGIR2003,slopeone,COLA2003}: some of the results of my work were licensed in October 2004 by Bell Canada and are now core
%to the Bell Sympatico/MSN Web site inDiscover.
%  Collaborators include
%Stephen Downes~(NRC~Moncton), Anna Maclachlan~(Idilia Inc.), Bruce Spencer~(NRC~Fredericton), and Harold Boley~(NRC~Fredericton). %RACOFI
%%Composer is part of the RuleML initiative from DFKI.
%\item Time Series Mining~\cite{ICDM-05,reactivecontrol,IJCAI05,ylbMBR2004,ylbMonet2004}. This work is
%done with Yuhong Yan~(NRC~Fredericton), Martin Brooks~(NRC~Ottawa), Will Fitzgerald (NASA Ames),  and Dan Kucerovsky~(UNB).
%\end{itemize}

\subsection{Individual Research Funding (external agencies, competitive grants)}
\begin{itemize}
\item NSERC Discovery Grant 2017--2022 (\$210,000);
\item NSERC Discovery Accelerator Supplement 2017--2020 (\$120,000);
\item NSERC Discovery Grant 2012--2017 (\$140,000);
\item NSERC Discovery Grant 2007--2012 (\$75,000);
\item NSERC Discovery Grant 2003--2007 (\$48,000);
\item FQRNT New Researcher Grant 2006--2008 (\$54,249).
\end{itemize}

\subsection{Team-Based Research Funding (external agencies, competitive grants)}
\begin{itemize}
\item CFI Grant 2016--2017 (\$797,481) with N.~Bélanger (lead) and E.~Filotas;
\item CFI Grant 2008--2009 (\$999,618) with G.~Paquette (lead) and P.~Valtchev;
\item New Brunswick Innovation Fund 2007--2008
 (\$10,000)  with O. Kaser (lead).
\item CAA Foundation (2013--2014,  \$189~994) with E.~Vallières (lead) et al.
\end{itemize}



%\section{Services (internal)}
%
%\subsection{Pedagogical}
%\begin{itemize}
%\item Member of UQAM's online-learning committee (2008-\ldots). %(Comité sur l’environnement numérique d’apprentissage)
%%\item With Olga Marino and Richard Hotte, member of the ``FAQ Project'' lead by Jacques Rivard in 2005-2006 (internal funding).
%%\item Member of the committee on science teaching quality policy (2007-2008).
%%\item With Richard Hotte, member of the ``Des Blogues pour IAO'' project lead by Jean-François Savard (\$17,940 in budget from internal funding) from  2004 to 2006.
%\end{itemize}
%
%\subsection{Career}
%\begin{itemize}
%\item Member of the new faculty support and integration committee  (2007-2008);
%\item Vice-President of the Faculty Council  (2005-2007);
%\item Hiring committees (2005, 2007, 2008, and 2009).
%%\item In October 2003, I was a member of the IIT Awards committee responsible
%%for giving prizes to outstanding employees.
%\end{itemize}
%
%
%\subsection{Academic Programs}
%\begin{itemize}
%\item Chair of the graduate certificate in Computer Science and Knowledge Management~(2009-\ldots);
%\item Chair of the certificate in Applied Computer Science (2007-\ldots);
%\item Chair of the distance learning Science B.Sc. (2005-2007);
%\item Chair of the science and technology certificate (2005-2007).
%%\item Chair of the ``comité d'orientation et de développement des programmes en sciences'' (Science Programs Strategic Committee) in 2004-2005.
%\end{itemize}
%
%\subsection{Research}
%\begin{itemize}
%\item Researcher at the  LICEF Research Center since 2006.
%%\item Collaborator at the \textit{Laboratoire de combinatoire et d'informatique mathématique} (LACIM) since 2006.
%\end{itemize}
%



\section{Services (external)}

\subsection{Blog}
\begin{itemize}
%\item I host the Data Warehouse and OLAP bibliography
%which was previously
%hosted by Alberto Mendelzon (UofT). This page alone receives more than
%350~hits per day: \url{http://lemire.me/OLAP}.
\item I run a blog at \url{http://lemire.me/}. It was featured on Reddit, Hacker News and Slashdot. It has over 25\,000~unique visitors a month.
\end{itemize}


%\subsection{Organizing Committees}
%\begin{itemize}
%\item I was steering chair of the  Second Canadian Semantic Web Working Symposium  (CSWWS 2009).
%\item I organized the Second KDD Workshop on Large Scale Recommenders Systems and the
%Netflix Prize at KDD 2008 with
%Yehuda Koren,	Alex Tuzhilin, Jim Bennett and Charles Elkan.
%\item I was co-chair of the Emerging Technologies for Web-based Communities (ET-WBC) Workshop (Mondragon University, Spain, February 25th 2006) with Mamadou Tadiou.
%\item I was co-chair of the ``Canadian Semantic Web Symposium 2006 (CSWWS'06)''  with Mamadou Tadiou. %The proceedings are published by Springer. %The event received financial support from the Rule Markup Initiative, Ontotext,  UQAM, and Laval University.
%\item In 2004, I was on the SWIG'04 organizing committee with Griff Richards, Bruce Spencer, Harold Boley, Fred Popowich, and Olga Marino.
%\end{itemize}


%
%\subsection{Program committees (workshops)}
%\begin{itemize}
%\item Journées francophones sur les Entrepôts de Données et l’Analyse en ligne
%(EDA)~2009, 2010;
%\item Workshop on Music Recommendation and Discovery (WOMRAD) 2010;
%\item ACM Conference on Management of Emergent Digital EcoSystems - Student Workshop (MEDES SW) 2009;
%\item International Workshop on Business Intelligence for Emerging e-Business Applications	(BieeBa) 2009;
%\item Fouille de données complexes dans un processus d'extraction des connaissances (FDC-EGC) 2007;
%\item International Advanced Database Conference (IADC) 2007 --- Advances in Querying Non-Conventional Data Sources track;
%\item  Atelier Systèmes Décisionnels (ASD) 2006, 2007, 2008, 2009, 2010;
%\item BaseWeb'05: workshop held in conjunction with Canadian AI'05.
%\end{itemize}
%
%
%\subsection{Program committees (international symposia)}
%\begin{itemize}
%\item Learners and their new expectations from ICTT --- TICE 2008 and 2010.
%\end{itemize}


\subsection{Program committees (incomplete list)} % {international conferences)}
\begin{itemize}
\item ACM Conference on Information Retrieval (SIGIR) 2015;
\item ACM Conference on Recommender Systems (RecSys) 2009--2014, 2017;
\item ACM Conference on Information and Knowledge Management (CIKM) 2012--2017;
\item ACM Conference on Web Search and Data Mining (WSDM) 2013--2015;
%\item Business Information Systems (BIS) 2011--2012;
%\item Web Intelligence, Mining and Semantics (WIMS) 2011--2012;
%\item Enterprise Information Systems (ICEIS) 2009--2012;
%\item IADIS Data Mining 2012;
%\item Database and Expert Systems Applications~(DEXA)~2009--2012;
%\item Web Information Systems and Technologies (WEBIST) 2007--2012;
\item  World Wide Web Conference (WWW) 2017;
\item ACM/IEEE Joint Conference on Digital Libraries (JCDL) 2011--2017.
%\item Logistics, Informatics and Service Science (LISS) 2011--2012;
%\item Association for the Advancement of Artificial Intelligence (AAAI) 2008.
\end{itemize}


%\subsection{Editorial Board}
%\begin{itemize}
%\item Open Journal of Information Systems;
%\item Open Journal of Databases;
%\item Atlantic Electronic Journal of Mathematics;
%\item Journal of Computers;
%\item International Journal on Advances in Intelligent Systems;
%\item Journal of Emerging Technologies in Web Intelligence.
%\end{itemize}


%\subsection{Referee}
%I have been a referee for a few international journals and conferences
%including:
%\begin{itemize}
%\item IEEE Transactions on Pattern Analysis and Machine Intelligence (2008);
%\item Data \& Knowledge Engineering --- DKE (2008, 2009);
%\item ACM CHI (2008);
%\item International Journal of Electronic Business --- IJEB (2007,2008);
%\item Data \& Knowledge Engineering --- DKE (2007);
%\item Electronic Commerce Research and Applications --- ECRA (2007);
%\item  Knowledge and Information Systems --- KAIS (2007);
%\item  I2LOR 2006;
%\item Information Retrieval (2006, 2007);
%\item M2USIC 2006;
%\item ACM SIGITE 2006;
%\item Information and Software Technology (2005);
%\item Data Mining and Knowledge Discovery (2005);
%\item Communications of the ACM (2005);
%\item ACM Transactions on Graphics (2005);
%\item WWW'05;
%\item ACM SAC'04;
%\item PST'04;
%\item IBM CASCON'04;
%\item AI'04;
%\item Computational Intelligence (2004);
%\item ISWC (2004);
%\item Web Intelligence (2004);
%\item IEEE Transactions in Medical Imaging (2003);
%\item IEEE Transactions in Signal Processing (2003);
%\item Curves and Surfaces (2003).
%\end{itemize}

\subsection{Funding Agencies}
\begin{itemize}
\item At the FQRNT, I have been a member of the review committee~03F (Theoretical Computer Science)  since 2007.
\item Still at the FQRNT, I was member of the review committee~309 (Computer Science team projects)  in 2006--2007, in
 2013--2014, 2014--2015 and 2016--2017.
\item At NSERC, I was a member of the Computer Science Research Tools and Instruments (RTI) Review Committee in 2012--2013, in 2013--2014
and in 2014--2015.
\end{itemize}

\subsection{External examiner/referee}
\begin{itemize}
\item External examiner for Ph.D. thesis:
\begin{itemize}
\item  Mohammed Shaaban from Université Pierre et Marie Curie, France (2017) --- supervised by Patrick Garda;
\item Mehdi Boukhechba from UQAC, Canada (2016) --- supervised by Abdenour Bouzouane and Charles Gouin-Vallerand;
\item Hicham Assoudi from UQAM, Canada (2016) --- supervised by Hakim Lounis;
\item Khaled Dehdouh from Lyon~2, France (2015) --- supervised by Omar Boussaid;
\item Martin Leginus from Aalborg University, Denmark (2015) --- supervised by Peter Dolog;
\item Ahmad Taleb from Concordia University, Canada (2011) --- supervised by Todd Eavis.
\end{itemize}
\item External referee for habilitation: \\ Sabine Loudcher Rabaseda from Lyon~2 University, France (2011).
\item External referee for promotion:
\begin{itemize}
\item Jason Sawin from the University of St. Thomas (2015);
\item Amer Nizar AbuAli from Philadelphia University (2013);
\item Ken Pu from the Ontario Institute of Technology (2011);
\item Jinan Fiaidhi from Lakehead University (2005).
\end{itemize}
\end{itemize}

%\subsection{Thesis Review}
%\begin{itemize}
%\item Sreeraman Rajan, Ph.D., UNB, 2004.
%\item Fang Wang,  M.Sc., UNB, 2004.
%\end{itemize}

%\subsection{Standards}
%\begin{itemize}
%\item ISO Committee Member (Data Format and Interchange, 2003).
%\end{itemize}




%\section{Media}

%\begin{itemize}
%\item In 2008, I gave an interview
%on Web technologies for professors to University Affairs.
%\item In  the winter of 2001, I was invited to the TV show {}``C'est mathématiques!'' ({}``It's
%mathematics!'') on {}``Canal Z'' to discuss Canadian
%innovations in geophysical data processing. {}``Canal Z'' is the
%major French TV channel in Canada featuring technology and science
%programs. {}``C'est mathématiques!'' has been on the air for several
%years.
%\item I gave an interview for the {}``Bulletin de l'AMQ'' (Bulletin of
%the {}``Quebec Mathematical Society'') on the theme {}``consulting
%in the industry as a mathematician'' (Autumn 2001). The {}``AMQ''
%is the only major Mathematical Society in Quebec and has thousands
%of members who all receive the {}``Bulletin de l'AMQ.''
%\end{itemize}

%\section{Entrepreneurship and Consulting}
%\begin{itemize}
%%\item Since October 2007, I am on the advisory board of
%%FavQuest~Inc.
%\item Since May 2006, I'm a scientific consultant for Intalgent (Virginia, USA).
%\item My ``Slope One'' Collaborative Filtering algorithm is  implemented in the Vogoo (\url{http://www.vogoo-api.com}),
%Apache Mahout (\url{http://lucene.apache.org/mahout/}) and Taste (\url{http://taste.sourceforge.net/}) software libraries.
%\item Together with NRC, we licensed \url{http://inDiscover.net} to Bell Canada in 2004. %In October 2004, in a three-way agreement between Daniel Lemire, NRC, and Bell Canada, Bell
%Canada licensed the technology behind the \url{http://inDiscover.net} Web site.
%The Web site became a Bell Sympatico/MSN web site on November 2004.%5$^{\textrm{th}}$ 2004. The term ``indiscover'' is found in over 20,000~Web pages  according to Google.
%NRC promotes the technology on its site\footnote{\url{http://iit-iti.nrc-cnrc.gc.ca/projects-projets/racofi-composer_e.html}} stating that:
%\begin{quote}The performance of Slope-One is remarkable; recommendations can be culled from collection of a million rated songs in a matter of seconds.
%\end{quote}
%\item I created the Waaves format for the first wavelet-based image server
%at the {}``Serveur national de radiologie'' in France ({}``National
%Radiology Server'') in 2000. % \cite{key-42,key-44}. %Serving medical images
%(DICOM) on the Web is a major undertaking since the average radiologist
%will generate over 4~gigabytes of raw data per day. The result of this work is known as the {}``Waaves''
%format and it is now in use in France for storing and retrieving
%digital radiology images and thousands of images are sent every day
%using this system.
% \textbf{This product was nominated for an IST prize.}
% \item In   2001, I was invited to the TV show {}``C'est mathématiques!''  on Canal Z to discuss my work in geophysical data processing. The interview is online at \url{http://video.google.com/videoplay?docid=-4750490412418990197}.
%\item Over several years, I worked with THEM Geophysics Inc. with funding
%from FalconBridge and CAMIRO (Canadian Mining Industry Research Organization)
%on improving geophysical data processing for the THEM system. The signal-to-noise ratio was improved by at least
%100\% using better mathematical modeling. % namely by multiscale methods.
%The
%THEM system is the only Canadian airborne EM system for exploring
%rough terrains.
%The software has been used successfully in Northern Canada and Sudan
%in exploration contracts worth several millions of dollars. %I gave
%a 5 minutes interview on television concerning this project.
%\item I was a scientific advisor for MathSoft Inc. in the design of
%the following Mathcad extension packs: signal processing, image processing,
%and wavelet. Mathcad is one of the most important mathematical software
%on the market.
%\end{itemize}





%%%%%%%%%%%%%%%%%%%%%%%%%%%%%%%%%%%%%%OLD%%%%%%%%%%%%%%%%%%%%%%%%%%%%%%%%%%%%%%%%%%%%%%%%%%%


%\section{Under Review}
%\begin{thebibliography}{10}
%\bibitem{shortdesign} Antonia Badia and Daniel Lemire, Functional dependencies with null markers.
%\end{thebibliography}

\section{Peer-Reviewed Journal Articles}

Impact factors of some journals: \textit{Information Sciences} (2.8),
\textit{Pattern Recognition}~(2.6), \textit{Journal of the Association for Information Science and Technology} (2.0), \textit{Data \& Knowledge Engineering} (1.7), \textit{ACM Transactions on Database Systems} (1.4), \textit{Computer Speech and Language} (1.4),  \textit{Information Retrieval} (1.3),
\textit{ACM Transactions on Information Systems} (1.3) and \textit{Information Systems} (1.2).


\begin{thebibliography}{10}
\bibitem{popcnt} Wojciech Muła, Nathan Kurz, Daniel Lemire, Faster Population Counts Using AVX2 Instructions, Computer Journal~(to appear)  \url{https://doi.org/10.1093/comjnl/bxx046}
  \bibitem{vague} Antonio Badia and Daniel Lemire,  On Desirable Semantics of Functional Dependencies over Databases with Incomplete Information, Fundamenta Informaticae~(to appear) \url{https://arxiv.org/abs/1703.08198}
  \bibitem{pseudomersenne} Dmytro Ivanchykhin, Sergey Ignatchenko, Daniel Lemire, Regular and almost universal hashing: an efficient implementation, Software: Practice and Experience~(to appear) \url{http://dx.doi.org/10.1002/spe.2461}
  \bibitem{upscaledb} Daniel Lemire, Christoph Rupp,  Upscaledb: Efficient Integer-Key Compression in a Key-Value Store using SIMD Instructions, Information Systems~\textbf{66}, 2017.
   \url{http://dx.doi.org/10.1016/j.is.2017.01.002}
  \bibitem{full} Jing Li, Yuhong Yan, Daniel Lemire, Full Solution Indexing for top-K Web Service Composition, IEEE Transactions on Services Computing~\textbf{99}, 2016. \url{http://dx.doi.org/10.1109/TSC.2016.2578924}
  \bibitem{vers} Samy Chambi, Daniel Lemire, Robert Godin, Vers de meilleures performances avec des Roaring bitmaps, Technique et Science Informatiques~\textbf{35}~3, 2016.
  \bibitem{broar}  Daniel Lemire, Gregory Ssi-Yan-Kai, Owen Kaser, Consistently faster and smaller compressed bitmaps with Roaring, Software: Practice \& Experience~\textbf{46}~11, 2016. \url{http://dx.doi.org/10.1002/spe.2402}
\bibitem{clhash}  Daniel Lemire, Owen Kaser, Faster 64-bit universal hashing using carry-less multiplications, Journal of Cryptographic Engineering~\textbf{6}~3, 2016. \url{http://dx.doi.org/10.1007/s13389-015-0110-5}
\bibitem{simdinter} Daniel Lemire, Nathan Kurz, Leonid Boytsov, SIMD Compression and the Intersection of Sorted Integers, Software: Practice \& Experience~\textbf{46}~6, 2016. \url{http://dx.doi.org/10.1002/spe.2326}
\bibitem{roaring} Samy Chambi, Daniel Lemire, Owen Kaser, Robert Godin, Better bitmap performance with Roaring bitmaps, Software: Practice \& Experience~\textbf{46}~5, 2016. \url{http://dx.doi.org/10.1002/spe.2325}
\bibitem{symbitmap} Owen Kaser and  Daniel Lemire, Compressed bitmap indexes: beyond unions and intersections, Software: Practice \& Experience~\textbf{46}~2, 2016. \url{http://dx.doi.org/10.1002/spe.2289}
\bibitem{bloofi} Adina Crainiceanu and  Daniel Lemire, Multidimensional Bloom Filters, Information Systems~\textbf{54}, 2015. \url{http://dx.doi.org/10.1016/j.is.2015.01.002}
\bibitem{shortdesign} Antonio Badia and Daniel Lemire,
Functional dependencies with \texttt{null} markers, Computer Journal~\textbf{58}~5, 2015.
\url{http://dx.doi.org/10.1093/comjnl/bxu039}
\bibitem{tois2015} Wayne Xin Zhao, Xudong Zhang, Daniel Lemire, Dongdong Shan, Jian-Yun
  Nie, Hongfei Yan, Ji-Rong Wen, A General SIMD-based Approach to Accelerating Compression Algorithms, ACM Transactions on Information Systems~\textbf{45}~1, 2015.
\url{http://dx.doi.org/10.1145/2735629}
\bibitem{jasist} Xiaodan Zhu, Peter Turney, Daniel Lemire, Andre Vellino,
Measuring academic influence: Not all citations are equal, Journal of the Association for Information Science and Technology~\textbf{66}~2, 2015.
\url{http://dx.doi.org/10.1002/asi.23179}
\bibitem{billions} Daniel Lemire and Leonid Boytsov, Decoding billions of integers per second through vectorization, Software: Practice \& Experience~\textbf{45}~1, 2015. \url{http://dx.doi.org/10.1002/spe.2203}
\bibitem{truncatedhashing}  Owen Kaser and Daniel Lemire, Strongly universal string hashing is fast, Computer Journal~\textbf{57} 11, 2014. \url{http://dx.doi.org/10.1093/comjnl/bxt070}
\bibitem{hazel1} Hazel Webb, Daniel Lemire, Owen Kaser, Diamond Dicing, Data \& Knowledge Engineering~\textbf{86}, 2013.  \url{http://arxiv.org/abs/1006.3726}
\bibitem{mahalanobis}	Zoltán Prekopcsák and Daniel Lemire, Time Series Classification by Class-Specific Mahalanobis Distances, \textit{Advances in Data Analysis and Classification}~\textbf{6}~3, 2012.  \url{http://arxiv.org/abs/1010.1526}
\bibitem{tods2012} Daniel Lemire, Owen Kaser, Eduardo Gutarra, Reordering Rows for Better Compression: Beyond the Lexicographic Order, \textit{ACM Transactions on Database Systems}~\textbf{37}~3, 2012.  \url{http://arxiv.org/abs/1207.2189}
\bibitem{ithash} Daniel Lemire, The universality of iterated hashing over variable-length strings, \textit{Discrete Applied Mathematics}~\textbf{160}~(4-5), 2012. \url{http://arxiv.org/abs/1008.1715}
 \bibitem{badiarecord} Antonio Badia and Daniel Lemire, A Call to Arms: Revisiting Database Design, \textit{SIGMOD Record}~\textbf{40}~3, 2011. \url{http://arxiv.org/abs/1105.6001}
\bibitem{rlemodel} Daniel Lemire and Owen Kaser, Reordering columns for smaller indexes, \textit{Information Sciences}~\textbf{181}~12, 2011.  \url{http://arxiv.org/abs/0909.1346}
\bibitem{csl2010} Daniel Lemire and Owen Kaser, Recursive n-gram hashing is pairwise independent, at best, \textit{Computer Speech and Language}~\textbf{24}~4, 2010.  \url{http://arxiv.org/abs/0705.4676}
\bibitem{longdolap2008}Daniel Lemire, Owen Kaser, Kamel Aouiche, Sorting improves word-aligned bitmap indexes, \textit{Data \& Knowledge Engineering}~\textbf{69}~1, 2010.   \url{http://arxiv.org/abs/0901.3751}

Cited at least \textbf{39~times}.

\bibitem{lbimproved}Daniel Lemire, Faster retrieval with a two-pass  Dynamic-Time-Warping  lower bound, \textit{Pattern Recognition}~\textbf{42}~9, 2009. %(Impact Factor: 3.279)
\url{http://arxiv.org/abs/0811.3301}

Cited at least \textbf{31~times}.

\bibitem{optiquasmon} Daniel Lemire, Martin Brooks, Yuhong Yan,  An optimal linear time algorithm for quasi-monotonic segmentation,  \textit{International Journal of Computer Mathematics}~\textbf{86}~7, 2009. %(Impact Factor: 0.254)
\url{http://arxiv.org/abs/0709.1166}
 \bibitem{longwebist2008} Kamel Aouiche, Daniel Lemire and Robert Godin, Web 2.0 OLAP: From data cubes to tag clouds, \textit{Lecture Notes in Business Information Processing}~\textbf{18}, 2009.
 \url{http://arxiv.org/abs/0905.2657}
\bibitem{ola} Daniel Lemire and Owen Kaser, Hierarchical Bin Buffering: Online local moments for dynamic external memory arrays, \textit{ACM Transactions on Algorithms}~\textbf{4}~1, 2008.
\bibitem{maxmin} Daniel Lemire, Streaming maximum-minimum filter using no more than three comparisons per element, \textit{Nordic Journal of Computing}~\textbf{13}~4,  2006. %(Impact Factor: 0,80)


Cited at least \textbf{24~times}.
\bibitem{IS2006}Owen Kaser and Daniel Lemire, Attribute value reordering for efficient hybrid OLAP, \textit{Information Sciences}~\textbf{176}~16, 2006. %(Impact Factor: 0.23)


Cited at least \textbf{33~times}.
\bibitem{elearningsummitpaper} Daniel Lemire, Harold Boley, Sean McGrath, Marcel Ball, Collaborative filtering and inference
rules for context-aware learning object recommendation, \textit{International Journal of Interactive Technology \& Smart Education}~\textbf{2}~3, 2005.

Cited at least \textbf{48~times}.

\bibitem{SIGIR2003}Daniel Lemire, Scale and Translation Invariant Collaborative Filtering
Systems, \textit{Information Retrieval}~\textbf{8}~1, 2005.  %(Impact Factor: 0.61)

Cited at least \textbf{64~times}.
\bibitem{firstref}Serge Dubuc, Daniel Lemire, Jean-Louis Merrien, Fourier analysis of
2-point Hermite interpolatory subdivision schemes, \textit{Journal of Fourier
Analysis and Applications}~\textbf{7}~5, 2001.
\bibitem{key-40}Daniel Lemire, Chantal Pharand, Jean-Claude Rajaonah, Bruno Dubé, A.-Robert LeBlanc, Wavelet time entropy, T wave morphology and myocardial
ischemia, \textit{IEEE Transactions in Biomedical Engineering}~\textbf{47}~7, 2000. %(Impact Factor: 1.665)

Cited at least \textbf{56~times}.
\bibitem{key-41}Gilles Deslauriers, Serge Dubuc, and Daniel Lemire, Une famille d'ondelettes
biorthogonales sur l'intervalle obtenue par un schéma d'interpolation
itérative, \textit{Annales des Sciences Mathématiques du Québec}~\textbf{23} 1, 1999.
\end{thebibliography}

\section{In Peer-Reviewed Proceedings}

%According to Google Scholar, ``Slope one predictors for online rating-based collaborative filtering'' has \textbf{282~citations} whereas ``Tag-cloud drawing: Algorithms for cloud visualization'' received \textbf{167~citations}.



\begin{thebibliography}{10}
\bibitem{WIMS2017} Dara Aghamirkarimi, Daniel Lemire,  Discovering the Smart Forests with Virtual Reality,
WIMS 2017, 2017.
\bibitem{IDEAS2016} Samy Chambi,  Daniel Lemire, Robert Godin, Kamel Boukhalfa, Charles Allen, Fangjin Yang,
Optimizing Druid with Roaring bitmaps, IDEAS 2016, 2016.
\bibitem{ICWS2016} Jing Li, Yuhong Yan, Daniel Lemire, Scaling up Web Service Composition with the Skyline Operator, IEEE ICWS 2016, 2016.
\bibitem{EDA2016} Samy Chambi, Daniel Lemire, Robert Godin, Nouveaux modèles d'index bitmap compressés à 64 bits, EDA 2016, 2016.
\bibitem{CONTEXT2015} Perrine Ruer, Charles Gouin-Vallerand, Le Zhang, Daniel Lemire, and Evelyne F.  Vallières,
An analysis tool for the contextual information from field experiments on driving fatigue,
CONTEXT~2015, 2015.
\bibitem{SCC2015} Jing Li, Yuhong Yan, Daniel Lemire, A web service composition method based on compact K2-trees, IEEE SCC~2015, 2015.
\bibitem{iswag2015} Jeff Plaisance, Nathan Kurz, Daniel Lemire, Vectorized VByte Decoding, International Symposium on Web Algorithms~2015, 2015.
\bibitem{EDA2014}Samy Chambi, Daniel Lemire,  Robert Godin, Owen Kaser, Roaring bitmap~: nouveau modèle de compression bitmap, EDA 2014, 2014.
\bibitem{SCC2014}Jing Li, Yuhong Yan, Daniel Lemire, Full Solution Indexing Using Database for QoS-aware Web Service Composition, IEEE SCC~2014, 2014.
\textbf{(Best Student Paper Award)}
\bibitem{VLDL11}Andre Vellino and Daniel Lemire, Extracting, Transforming and Archiving Scientific Data, VLDL~2011, 2011.
\bibitem{dolap08}Owen Kaser, Daniel Lemire, Kamel Aouiche,
Histogram-aware sorting for Enhanced Word-Aligned Compression in bitmap indexes, DOLAP~2008, 2008.  %(Acceptance Rate: 41\%)
\bibitem{bda08}Kamel Aouiche, Daniel Lemire, Owen Kaser, Tri de la table de faits et compression des index bitmaps avec alignement sur les mots, BDA~2008, 2008. %(Acceptance Rate: 33\%)
\bibitem{diamond08}Hazel Webb, Owen Kaser, Daniel Lemire, Pruning attributes from data cubes with Diamond Dicing, IDEAS~2008, 2008. %(Acceptance Rate: 34\%)
\bibitem{olaptagcloud}Kamel Aouiche, Daniel Lemire, Robert Godin, Collaborative OLAP with tag clouds: Web 2.0 OLAP formalism and experimental evaluation, WEBIST 2008, 2008. %(Acceptance Rate: 15\%)
\bibitem{dolap2007} Kamel Aouiche and Daniel Lemire, A comparison of five probabilistic view-size estimation techniques in OLAP, DOLAP 2007, 2007. %(Acceptance Rate: 43\%)
\bibitem{gutheader} Owen Kaser and Daniel Lemire, Removing manually-generated boilerplate from
electronic texts: Experiments with project Gutenberg e-books, CASCON~2007, 2007.
\bibitem{tagcloud} Owen Kaser and Daniel Lemire, Tag-Cloud Drawing: Algorithms for cloud visualization, Tagging and Metadata for Social Information Organization (WWW 2007), 2007.

Cited at least \textbf{167~times}.
\bibitem{kamel1} Kamel Aouiche and Daniel Lemire, Unassuming view-size estimation techniques in OLAP: An experimental comparison, ICEIS~2007, 2007. %(Impact Factor: 0.10)
\bibitem{overfit} Daniel Lemire,  A better alternative to piecewise linear time series segmentation, SIAM Data Mining 2007, 2007. %(Acceptance Rate: 25\%, Impact Factor: 0,10)


Cited at least \textbf{38~times}.
\bibitem{mon} Dan Kucerovsky and Daniel Lemire, Monotonicity analysis over chains and curves, Curves and Surfaces 2006, 2007.
\bibitem{CaSTA2006} Owen Kaser, Daniel Lemire, Steven Keith,
The LitOLAP Project: data warehousing with literature, CaSTA 2006,  2006.
\bibitem{ICDM-05} Daniel Lemire, Martin Brooks,  Yuhong Yan,
An optimal linear time algorithm for quasi-monotonic segmentation,
ICDM~2005, 2005. %(Acceptance Rate: 22\%) (Impact Factor: 0.35)
\bibitem{reactivecontrol} Will Fitzgerald, Daniel Lemire,  Martin Brooks, Quasi-monotonic segmentation of state variable behavior for reactive control,  AAAI~2005,  2005. %(Acceptance Rate: 27\%) (Impact Factor: 1.49)
\bibitem{IJCAI05}Yuhong Yan, Martin Brooks, Daniel Lemire, Scale-based monotonicity analysis in qualitative modelling with flat segments, IJCAI~2005,  2005. %(Acceptance Rate: 18\%) (Impact Factor: 1.82)
\bibitem{slopeone}Daniel Lemire and Anna Maclachlan, Slope One predictors for online rating-based collaborative filtering. SIAM Data Mining 2005, 2005.

Cited at least \textbf{282~times}.
 %(Acceptance Rate: 36\%, Impact Factor: 0,10)
%\bibitem{ylbMBR2004}Yuhong Yan, Daniel Lemire, Martin Brooks, Monotonicity analysis for constructing qualitative models,  MBR~2004,  2004
\bibitem{ylbMonet2004}Yuhong Yan, Daniel Lemire, Martin Brooks, Monotone pieces analysis for qualitative modeling, MONET 2004,  2004.
\bibitem{COLA2003}Michelle Anderson, Marcel Ball, Harold Boley, Stephen Greene, Nancy
Howse, Daniel Lemire, Sean McGrath, RACOFI: A rule-applying collaborative
filtering system, IEEE/WIC COLA~2003, 2003. %(NRC 46507)

Cited at least \textbf{102~times}.

\bibitem{DOLAP2003} Owen Kaser and Daniel Lemire, Attribute value reordering for efficient hybrid OLAP.
 DOLAP~2003, 2003. %(NRC  46510) (Acceptance Rate: 43\%) (Impact Factor: 0.55)
\bibitem{ProceedingCurvesandSurfaces}Daniel Lemire, A family of 4-point dyadic high resolution subdivision
schemes, Curves and Surfaces 2002,
2003. %(NRC 45811)
\bibitem{CASCON2002}Daniel Lemire, Wavelet-based relative prefix sum methods for range
sum queries in data cubes, CASCON~2002,
2002. %(NRC 44967)
\textbf{(Best Paper Award)}
\end{thebibliography}


\section{Books}

\begin{thebibliography}{10}
\bibitem{CSWWS2006} Mamadou Tadiou Koné and Daniel Lemire (Eds.), Canadian Semantic Web,  Springer, 2006.
\end{thebibliography}

\section{Special Issues and editorials}


\begin{thebibliography}{10}
\bibitem{neylon2012} Cameron Neylon, Jan Aerts, C. Titus Brown, Daniel Lemire, Jarrod Millman, Peter Murray-Rust, Fernando Perez, Neil Saunders, Arfon Smith, Gaël Varoquaux and Egon Willighagen, Changing computational research: The challenges ahead,
\textit{Source Code for Biology and Medicine}~\textbf{7} (2), 2012.
\bibitem{JETWI2010} Daniel Lemire and Richard Hotte (Eds.), Special issue on learning and the social web, Journal of Emerging Technologies in Web Intelligence \textbf{2}~1, 2010.
\bibitem{CI2007} Mamadou Tadiou Koné and Daniel Lemire (Eds.), Special issue on Canadian Semantic Web, Computational Intelligence \textbf{23}~3, 2007.
\end{thebibliography}


\section{Book chapters}
\begin{thebibliography}{10}
\bibitem{Collab2010} Sylvie Noël and Daniel Lemire, On the challenges of collaborative data processing, in
Collaborative Information Behaviour: User Engagement and Communication Sharing (edited by Jonathan Foster), IGI Global, 2010.
\end{thebibliography}

%\section{Abstracts (Peer-Reviewed)}
%
%\begin{thebibliography}{10}
%\bibitem[ECAP2007]{ecap2007}Jean Robillard and Daniel Lemire, The Semantic Web and its Flaws, ECAP'07,
%Twente, Netherland, June 2007.
%\bibitem[CS2006]{cs2006}Dan Kucerovsky and Daniel Lemire, Monotonicity Analysis and Sphere-Preserving Filters, \textit{Curves and Surfaces},
%Avignon, France, June-July 2006.
%\bibitem[eLS2004]{elearningsummit}Marcel Ball, Harold Boley, Daniel Lemire, Sean McGrath,
%Collaborative Filtering and Inference Rules for Context-Aware Learning Object Recommendation, e-Learning
%Summit, Fredericton, April 2004.
%\bibitem[CS2002]{cs2002}Daniel Lemire, High Resolution Subdivision Schemes, \textit{Curves and Surfaces},
%Saint-Malo, France, June-July 2002.
%\bibitem[CMBEC25]{key-45}Daniel Lemire, Chantal Pharand, Jean-Claude Rajaonah, Bruno Dubé, A.-Robert LeBlanc, Wavelet Time Entropy and T Wave Morphology, \textit{CMBEC
%25}, London, June 1999.
%\end{thebibliography}

%
\section{Magazines}
\begin{thebibliography}{10}
\bibitem{XRDS2010}
Daniel Lemire, Marketing your ideas: Don't sell yourself short, ACM~XRDS: Crossroads \textbf{16} (4), 2010.
\end{thebibliography}


\section{Technical Reports}


\begin{thebibliography}{10}
\bibitem{TRD14001} Owen Kaser and Daniel Lemire, Threshold and Symmetric Functions over Bitmaps, UNBSJ CSAS Technical Report TR-14-001, 2014.
%\bibitem{0807.1734} Daniel Lemire, Faster Sequential Search with a Two-Pass Dynamic-Time-Warping Lower Bound, 2008.
%(arXiv:0807.1734)
\bibitem{TRD08011}Hazel Webb, Owen Kaser, Daniel Lemire, Pruning attributes from data cubes with Diamond Dicing, UNBSJ CSAS Technical Report TR-08-011, 2008.
\bibitem{TRD07001}Owen Kaser and Daniel Lemire, Removing manually-generated boilerplate from electronic texts: Experiments with project Gutenberg e-books. UNBSJ CSAS Technical Report TR-07-001, 2007.
\bibitem{TRD06001}Daniel Lemire and Owen Kaser, One-pass, one-hash n-gram statistics estimation, UNBSJ CSAS Technical Report  TR-06-001,  2006.
\bibitem{TRD05001}Steven Keith, Owen Kaser, Daniel Lemire, Analyzing large collections of electronic text using OLAP, UNBSJ CSAS Technical Report TR-05-001, 2005.
%\bibitem[TRD01]{TRD01}Daniel Lemire, Sean McGrath, Implementing a Rating-Based Item-to-Item Recommender System in PHP/SQL, Rapport technique D-01, 2005.
%\bibitem[BELL2004-TR]{BELL2004-TR} Daniel Lemire, Sean McGrath et Bruce Spencer, The Design and Architecture of a Collaborative Filtering Web site --- inDiscover.net, Rapport technique du CNRC présenté à  Bell Canada, octobre 2004, 14~p.
%\bibitem[DOLAP2003-TR]{DOLAP2003-TR}Owen Kaser et Daniel Lemire, Attribute Value Reordering for Efficient Hybrid OLAP, janvier 2004, 29~p., CNRC  46509.
\bibitem{Franco}Jean-Michel Nonglaton, Franco Lenardon, and Daniel Lemire, Wavelet
shrinkage of LINAC III and protons synchrotron booster transformers
by the Haar Transform, CERN Technical Report Number AB-Note-2003-033,
avril 2003, VNRC 45816.
%\bibitem[CHIPP2003]{TeleoncologyJan2003}Irina Kondratova, Scott Flinn, and Daniel Lemire, Teleoncology Technology
%Assessment Report for the Teleoncology Project. Soumis au projet de
%téléoncologie, CHIPP, Health Canada, Beauséjour Hospital Corporation.
%Moncton, New Brunswick, février 2003, 25 p., CNRC 45799.
\bibitem{key-74}Daniel Lemire, Local interpolation by high resolution subdivision
schemes. Technical Report 200205-01, Acadia University, 2002.
%\bibitem[THEM2001-02]{key-81}Daniel Lemire, Baseline asymmetry, Tau projection, B-field estimation
%and automatic half-cycle rejection, THEM Geophysics Inc., THEM2001-02,
%février 2001.
%\bibitem[THEM2001-01]{key-80}Daniel Lemire, Off-time denoising: is stacking the solution?, THEM
%Geophysics Inc., THEM2001-01, janvier 2001.
%\bibitem[THEM1999-02]{THEM1999-02}Daniel Lemire, Rapport sur un nouvel algorithme de correction de la
%ligne de base pour THEM Geophysics, Splines interpolation des moyennes
%et moindres carrés (SIM), THEM Geophysics Inc., THEM1999-02, 1999.
%\bibitem[CIRA1999]{key-42}Alain Béliveau et Daniel Lemire, White paper sur la compression d'images
%par la Fast Wavelet Transform, société Compression
%d'Images, Réseaux et Automatismes, mars 1999.
%\bibitem[EPM/RT-97/28]{key-43}Gilles Deslauriers, Serge Dubuc et Daniel Lemire, Dérivées de l'interpolation
%itérative de Lagrange et les ondelettes b-adiques de Cohen-Daubechies-Feauveau,
%Rapport technique EPM/RT-97/28, École Polytechnique de Montréal, Montréal,
%avril 1997.
\end{thebibliography}


\section{Lecture Notes}
\begin{thebibliography}{10}
\bibitem[INF9004]{INF9004} Daniel Lemire, Informatique des entrepôts de données (Data Warehousing and Computer Science),  2012. Online course: \url{http://benhur.teluq.ca/SPIP/inf9004/}.
\bibitem[INF6104]{INF6104} Daniel Lemire, Recherche d'informations et web (Information Retrieval and the Web), 2008. Online course: \url{http://benhur.teluq.ca/SPIP/inf6104/}.
\bibitem[INF6460]{INF6460} Daniel Lemire, Recherche et filtrage d'informations (Information Retrieval and Filtering), 2007. Online course: \url{http://benhur.teluq.ca/SPIP/inf6460/}.
\bibitem[INF6450]{INF6450} Daniel Lemire, Gestion de l'information avec XML (Information Management with XML), ISBN 2762418747, 2005. (Revised in 2006, 2007 and 2008.) Online course: \url{http://www.teluq.ca/inf6450/}.
\end{thebibliography}

%\section{Thesis}
%\begin{thebibliography}{10}
%\bibitem[PhD]{phdd} Daniel Lemire,  Schémas d'interpolation et ondelettes, Ecole Polytechnique de Montréal, ISBN 0612373770, 1998.
%\end{thebibliography}


%\section{Supervised Thesis}
%\begin{thebibliography}{10}
%\bibitem[KeithThesis]{KeithThesis}  Steven W. Keith, Efficient Storage Methods for a Literary Data Warehouse, Master of Computer Science, University of New Brunswick, May 2006.
%\end{thebibliography}


%\section{Invited Talks}
%\begin{thebibliography}{10}
%\bibitem[Sher2010]{sher2010}Daniel Lemire, Write good papers, Université de Sherbrooke, May 2010.
%\bibitem[UQO2009]{uqo2009}Daniel Lemire, Compressing column-oriented indexes, Séminaire du département d'informatique, UQO, November 2009.
%\bibitem[UQAM2009]{uqam2009}Daniel Lemire, All About Bitmap Indexes\ldots And Sorting Them, Séminaire du département d'informatique, UQAM, February 2009.
%\bibitem[Lyon2007]{lyon2007}Daniel Lemire, Time Series Indexing and Aggregation, Laboratoire ERIC, Universit\'e Lyon~2, Lyon, France, January 2007.
%\bibitem[NRC2006]{nrc2006}Daniel Lemire, Time Series Indexing and Aggregation, NRC, Fredericton, November 2006.
%\bibitem[CIC2006]{cic2006}Daniel Lemire, Asemantic searching, CIC, UQAM, May 2006.
%\bibitem[LACIM2006]{lacim2006}Daniel Lemire, Analyse multi-échelle de la monotonicité, LACIM, UQAM, January 2006.
%\bibitem[Ottawa2006]{ottawa2006}Daniel Lemire, Rating-Based Collaborative Filtering, TAMALE Seminar, Computer Science Department, Ottawa University, January 2006.
%\bibitem[UQAM2005]{uqam2005}Daniel Lemire, Analyse multi-échelle de la monotonicité, Séminaire du département d'informatique, UQAM, October 2005.
%\bibitem[CRG2005]{crg2005}Daniel Lemire, Accélérations des bases de données multidimensionnelles par ondelettes et tri des valeurs d'attribut, Séminaire du Centre de recherche en géomatique, Université Laval, February 2005.
%\bibitem[CIRTA2005]{cirta2005}Daniel Lemire (with Anna Maclachlan), Filtrage collaboratif par évaluations (Rating-Based Collaborative Filtering), Ateliers du CIRTA, January 2005.
%\bibitem[RALI2004]{rali2004}Daniel Lemire (with Anna Maclachlan), Filtrage collaboratif par évaluations (Rating-Based Collaborative Filtering), Séminaires RALI, Université de Montréal, October 2004.
%\bibitem[AILIA2004]{ailia2004}Daniel Lemire (with Harold Boley and Bruce Spencer), Semantic Web in Brief, Annual Language Industry Association (AILIA) Meeting, September 2004.
%\bibitem[IEEE2003]{ieeesj2003}Daniel Lemire, 20 years of Wavelets, Annual IEEE New Brunswick Chapter Talk, December 2003.
%\bibitem[Arts2003]{artsnetlantic2003}Daniel Lemire, RACOFI: Rule-Applying Collaborative Filtering,
%Guest Speaker, Arts-Netlantic Network, October 2003.
%\bibitem[Acadia2003]{acadia2003}Daniel Lemire, Quasi-Optimal Collaborative Filtering Systems,
%CS Department, Acadia University, October 2003.
%\bibitem[CMS2003]{key-84}Daniel Lemire, Quadratic and Cubic 2-Step Subdivision Schemes, Approximation
%Theory and Harmonic Analysis Symposium, Summer CMS 2003, Edmonton,
%June 2003.
%\bibitem[NRC-Ottawa2002]{NRC-Ottawa2002}Daniel Lemire, Life Behind a Web Portal: Large Data Sets, Approximation
%Theory and Wavelets, NRC-IIT Colloquium, Ottawa, Canada, December
%2002. (Repeated at UNB/SJ in Saint John, New Brunswick, Canada a week later)
%\bibitem[UNB2002]{UNB2002}Daniel Lemire, High Resolution Subdivision Schemes, UNB Math. \& Stat.
%Colloquium, Fredericton, Canada, October 2002.
%\bibitem[CAMIRO2001]{key-47}Daniel Lemire, Recent mathematical advances in EM modeling for mineral
%exploration, Canadian Mining Industry Research Organization (CAMIRO),
%Toronto, Ontario, 2001.
%\bibitem[GeoQc1999]{key-48}Alain Béliveau and Daniel Lemire, Géophysique, traitement du
%signal et analyse numérique, congrès Géologie Québec, gouvernement
%du Québec, Québec, 1999.
%\end{thebibliography}

%\section{Panel}
%\begin{thebibliography}{10}
%\bibitem[SWIG04Panel]{SWIG04Panel} Daniel Lemire (with Bruce Spencer, Fred Popowich, and Éric Duval),  Towards the Semantic Web: When, Where and How will it impact?, Semantic Web Symposium (SWIG'04) Opening, 2004.
%\end{thebibliography}


%\section{Other Communications}
%
%\begin{thebibliography}{10}
%\bibitem[CREPUQ2006]{crepuq2006} Richard Hotte, Nicolas Lecomte, Daniel Lemire, Jean-François Savard, L'utilisation des blogues dans l'enseignement supérieur en ligne, Colloque CREPUQ - SCTIC 2006, April 2006.
%\bibitem[APICS2005]{apics2005}Steven Keith, Owen Kaser, Daniel Lemire, Analyzing Large Collections of Electronic Text Using OLAP, APICS 2005, Wolfville, Canada, October 2005.
%\bibitem[CMS2004]{cms2004}Dan Kucerovsky and Daniel Lemire, Quasi-monotone sequences, Winter Meeting of the Canadian Mathematical Society, December 2004.
%\bibitem[Acfas2003]{key-88}Daniel Lemire, Schémas de subdivision à résolution élevée, 71$^\textrm{e}$~Congrès de l'ACFAS, Rimouski, May 2003.
%\bibitem[Moncton2003]{key-87}Daniel Lemire, Schémas de subdivisions à étapes de degré quadratique
%et cubiquee, 69$^{\textrm{e}}$~Colloque des sciences mathématiques du Québec, Moncton, April 2003.
%\bibitem[WebTek2000]{key-44}Daniel Lemire, Intégration Web de formats propriétaires par le Java,
%Congrès WebTek, Québec, March 2000.
%\bibitem[Acfas1998]{key-46}Gilles Deslauriers, Serge Dubuc, and Daniel Lemire, Interpolation
%itérative et ondelettes, 66$^\textrm{e}$~Congrès de l'ACFAS, Université Laval,
%May 1998.
%\bibitem[CSMQ1998]{key-79}Daniel Lemire, Traitement du signal par ondelettes pour l'analyse
%de la morphologie de l'onde T et du segment ST chez des porcs, Colloque
%des sciences mathématiques du Québec,Université de Montréal, 1998.
%\bibitem[CSMQ1997]{key-50}Gilles Deslauriers, Serge Dubuc,~ Daniel Lemire, Traitement d'images
%par ondelettes non séparables, Colloque des sciences mathématiques
%du Québec, UQTR, Trois-Rivières, 1997.
%\bibitem[Moncton1996]{key-51}Gilles Deslauriers, Serge Dubuc, Daniel Lemire, Interpolation des
%moments, Colloque des sciences mathématiques du Québec et de l'Atlantique,
%Université de Moncton, Moncton, 1996.\end{thebibliography}



\section{Open Source Software (sample)}

 I am a beneficiary of the Google Open Source Peer Bonus Program.

\begin{itemize}
\item Daniel Lemire et al., Roaring Bitmaps, 2014. (\url{https://github.com/RoaringBitmap/RoaringBitmap})

This library is used by Apache Spark, Apache Kylin (eBay's OLAP engine) and Druid. A reimplementation of this library
is part of Apache Lucene, and used by systems such as Solr and Elastic.

\item Daniel Lemire et al., JavaFastPFOR: A simple integer compression library in Java, 2012. (\url{https://github.com/lemire/JavaFastPFOR})

A derivative of this library is \href{http://lucene.apache.org/core/4_6_1/core/org/apache/lucene/util/PForDeltaDocIdSet.html}{part of Apache Lucene}.


\item Daniel Lemire et al., The FastPFOR C++ library: Fast integer compression, 2012. (\url{https://github.com/lemire/FastPFor})
\item Daniel Lemire et al., JavaEWAH, A compressed alternative to the Java BitSet class, 2009. (\url{http://code.google.com/p/javaewah/})

This project was recognized by Google’s Open Source Programs in 2012.

JavaEWAH is used by Apache Hive for bitmap indexing, \textbf{part of major Linux distributions} like Redhat and Ubuntu. It is used
by JGit.

\item Daniel Lemire, EWAHBoolArray: A compressed bitmap class in C++, 2008. (\url{http://code.google.com/p/lemurbitmapindex/})

Used by the \href{http://chango.github.io/hustle/}{OLAP engine Hustle}.

Ported to C by GitHub and integrated in the Git version control system.

\end{itemize}

\section{Hobbies}

\begin{itemize}
\item My favorite science-fiction authors are Charles Stross, Peter F. Hamilton and Alastair Reynolds.
\item My family loves geocaching.
\item I build radio-controlled sailboat models.
\item Since I was twelve years old, I have been programming computers for fun.
\end{itemize}


\end{document}
