%% LyX 1.1 created this file.  For more info, see http://www.lyx.org/.
%% Do not edit unless you really know what you are doing.
\documentclass[english]{article}
\usepackage[T1]{fontenc}
\usepackage[latin1]{inputenc}
\usepackage{babel}

\makeatletter

%%%%%%%%%%%%%%%%%%%%%%%%%%%%%% LyX specific LaTeX commands.
\providecommand{\LyX}{L\kern-.1667em\lower.25em\hbox{Y}\kern-.125emX\@}

\makeatother
\begin{document}

\title{Tom's draft }


\author{Daniel Lemire}

\maketitle
\begin{enumerate}
\item Derivatives

\begin{enumerate}
\item Find the derivatives of the following functions. DO NOT SIMPLIFY anymore
than necessary.

\begin{enumerate}
\item \( f(x)=\frac{x^{2}+\sqrt{x}}{x^{4}-e^{x}} \)\\
Answer: seems like a lot of messy algebra???
\item \( h(x)=x^{\sin x^{2}} \)\\
Answer: Log. diff. \( x^{\sin (x^{2})}\left( \frac{\sin x^{2}}{x}+x\ln \left( x^{2}\right) \cos \left( x^{2}\right) \right)  \).
A bit on the hard side, but doable. There is the problem of \( x<0 \)
with log???
\item \( r(x)=\sec ^{2}\left( 3x+5\right)  \)\\
Answer: \( 6\sec ^{2}(3x+5)\tan (3x+5) \). Fair if you know what
the der. of \( \sec  \) is. I spent very little time on this in class. 
\item \( g(t)=t^{\ln (t)} \)\\
Answer: \( 2^{t^{\ln t-1}}\ln t \) log. diff. (easy?)
\end{enumerate}
\item Given \( \sqrt{x}+\sqrt{y}=xy+1 \) find

\begin{enumerate}
\item \( \frac{dy}{dx} \)\\
Answer: (in terms of \( x \) and \( y \) ?)
\item The equation of the tangent to the curve at the point \( (1,1) \).\\
Answer: quite fair since you give both \( x \) and \( y \).
\end{enumerate}
\end{enumerate}
\item Limits.

\begin{enumerate}
\item Calculate the following limits or explain why they don't exist.

\begin{enumerate}
\item \( \lim _{h\rightarrow 0}\frac{(h+1)^{2}-1}{h} \)\\
Answer: 2 (easy, basic).
\item \( \lim _{x\rightarrow \infty }x^{3}e^{-x^{2}} \)\\
Answer: clearly 0... Hospital?
\item \( \lim _{x\rightarrow 2}\frac{x-2}{x^{2}-4x+4} \)\\
Answer: doesn't exist... hard?
\item \( \lim _{x\rightarrow \infty }\left( 1+\frac{2}{x}\right) ^{x} \)\\
Answer: \( e^{2} \). Standard log-limit.
\end{enumerate}
\item If \( 1\leq f(x)\leq x^{2}+2x+2 \) for all \( x \), find \( \lim _{x\rightarrow -1}f(x) \).\\
Answer: 1. Isn't it a bit easy????
\item IVT

\begin{enumerate}
\item State IVT\\
Answer: Fair.
\item Given the function \( f(x)=\left\{ \begin{array}{c}
2t+1\, 0\leq t\leq 3\\
t^{2}\, 3<t\leq 5
\end{array}\right.  \)defined on the interval \( [0,5] \). Note that \( f(0)=1 \), \( f(5)=25 \)
but \( f(x) \) is never equal to \( 8 \).\\
Answer: is there a question?
\end{enumerate}
\end{enumerate}
\item Extrema/Optimization

\begin{enumerate}
\item State the Extreme Value Theorem\\
Answer: ok
\item Find the absolute maximum and absolute minimum of \( f(x)=e^{x^{3}-x} \)
on the interval \( [-1,0] \).\\
Answer: \( f'(x)=(3x^{2}-1)e^{x^{3}-x} \) and \( f'(x)=0 \) implies
\( x=\pm 1/\sqrt{3} \), so by CIM, we have to choose among the values
\( f(0)=1 \), \( f(-1)=1, \) \( e^{-1/3\sqrt{3}+1/\sqrt{3}}>1 \)
. Ok, fair question (should give plot though!).
\item A piece of paper \( 16\, cm \) by \( 24\, cm \) is folded (see sketch)
how (...)\\
Answer: I'm not sure I understand... 
\end{enumerate}
\item I can't reproduce the graph very easily but it seems like a fair question.
\item Applications of the derivative

\begin{enumerate}
\item Use Newton's method with an initial guess of \( x_{0}=1 \) to find
where \( f(x)=x^{3}-3 \) has a zero. Calculate the value correct
to two decimal places, assuming that each step brings you closer to
the correct answer.\\
Answer: I'm not sure the {}``assuming that each step(...)'' helps
them. \( x_{1}=1-\frac{-2}{3}=5/3 \), \( x_{2}\cong 1.47 \).
\item Use a local linear approximation to \( f(x)=x^{1/3} \) at \( x=3.375 \)
to approximate \( ^{3}\sqrt{3} \). Note that \( (3.375)^{1/3}=1.5 \).
\\
Answer: \( f(3)=f(3.375)-0.375f'(3.375) \). \( f'(x)=x^{-2/3}/3 \).
\( f'(3.375)=\frac{1}{3(1.5)^{2}}=0.\overline{148} \) . \( f(3)=1.5-0.375(0.\overline{148}) \)
(and so on...)
\item <<survivor in Nova Scotia>>... I'm missing a few things...\end{enumerate}
\end{enumerate}

\end{document}
