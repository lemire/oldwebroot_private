\documentclass[12pt]{article}

\usepackage{amsmath,amssymb,amsfonts,fullpage,fancy,array}

\textheight13.25in
\textwidth6.5in
\topmargin-.75in
\oddsidemargin-.25in
\pagestyle{fancy}

\newfont{\first}{msbm10 at 12pt}
\newcommand{\C}{\mbox{\first{C}}}
\newcommand{\N}{\mbox{\first{N}}}
\newcommand{\Q}{\mbox{\first{Q}}}
\newcommand{\R}{\mbox{\first{R}}}
\newcommand{\T}{\mbox{\first{T}}}
\newcommand{\Z}{\mbox{\first{Z}}}

\newcommand{\rmp}{\reversemarginpar\marginpar}
\renewcommand{\marginparsep}{-.4in}


\lhead{\small{Math 2433 -- Final Exam -- F01}}
\rhead {Name: \rule{2in}{.1mm}}
\chead{Page \, \thepage \, of \, \pageref{`last'} }
\cfoot{}

\setlength{\headheight}{18pt}
\setlength{\headrulewidth}{.5 pt}
%\nopagenumbers
\title{}\date{}
\begin{document}

\thispagestyle{empty}

\noindent
Page 1 of \pageref{`last'} \hfill NAME: \rule{2in}{.1mm} \\
\mbox{} \hfill Please Print \hspace{1in}\\

\mbox{} \hfill Id. No.: \rule{2in}{.1mm}  \\

 \begin{center}

{\large \bf Department of Mathematics \& Statistics \\ [2mm]
Acadia University\\ [2mm]
MATH 2433 \hfill  FINAL EXAMINATION  \hfill Fall 2001\\  [3mm]
Instructor:  J. Hooper \\ [3mm]
NO CALCULATORS}

 \end{center}
\noindent
Time: 3 hours \hfill  December 5, 2001  \hfill 9:00 a.m.-12:00 noon.
 \\

\noindent
{\bf Instructions:}

 \begin{enumerate}

 \item
Put your name and ID number at the top of this page.

 \item
Answer the questions in the spaces provided, using the backs of pages for
overflow or rough work.

 \end{enumerate}

 \begin{center}

$~$ \\

 \begin{tabular}{|c|r|} \hline
& \\
  Question & \hspace{5mm} Mark  \hspace{5mm} \\ \hline \hline &  \\
  1 & /8 \\ \hline
&  \\
  2 & /12 \\ \hline
&  \\
  3 & /10 \\ \hline
&  \\
  4 & /10 \\ \hline
&  \\
  5 & /12 \\ \hline
&  \\
6 & /16\\ \hline
&  \\
7 & /20 \\ \hline
&  \\
8 & /12\\ \hline
&  \\
Bonus & /10\\ \hline
&  \\
  Total & /100 \\ \hline
 \end{tabular}

 \end{center}

\newpage





\begin{enumerate}
\item If \rmp{[8]}  $a^2 + b^2 = c^2$, show that one of $a$, $b$ and $c$ must be
         a multiple of 5.

\newpage
\item
    Solve \rmp{[12]}the congruence
       $$
         x^{33} \equiv 27 \ \ (\textrm{mod\ } 112)\, .
       $$

\newpage
\item Let \rmp{[10]} $\T  = \{\, a+b\sqrt{5} i \,|\, a,b\in
         \Z\,\}$ and let $N(a+b\sqrt{5} i) = a^2 + 5b^2$ be the \textbf{norm} of
         $a+b\sqrt{5} i$. An element of  $\T$ is \textbf{prime} if it cannot be
         factored in $\T$ without using elements of norm 1.
    \begin{enumerate}
      \item  Is $1+\sqrt{5} i$ prime in $\T$? Give reasons.
\vskip 4 in
      \item  Show that the prime factorization of 6 in $\T$ is not unique.
    \end{enumerate}

\newpage
\item
    \begin{enumerate}
      \item  Find \rmp{[10]}the least positive integer $r$ such that
             $ 20^r \equiv 1 \ \ (\textrm{mod } 61)$.
\vskip 5 in
      \item  Find all the positive integers $n$ such that
             $ 3^n \equiv 13 \ \ (\textrm{mod } 23)$.
    \end{enumerate}

\newpage
\item
    \begin{enumerate}
      \item  Explain \rmp{[12]}what is meant by the statement that $M_n$ is a
Mersenne prime.
\vskip 1.75 in
      \item  Find the first four Mersenne primes.
\vskip 1.75 in
      \item  What is the connection between Mersenne primes and perfect numbers?
\vskip 2 in
      \item  If $M_n$ is any Mersenne prime, show that 3 is a quadratic non-residue (mod
$M_n$).
    \end{enumerate}

\newpage
\item
    \begin{enumerate}
      \item  Explain \rmp{[16]}what it means for $g$ to be a {\bf primitive
             root} modulo the prime $p$.
\vskip 1.5 in
      \item  Explain what is meant by the {\bf index} of $a$ modulo the prime $p$ for the
             primitive root $g$.
\vskip 1.5 in
      \item  Find a primitive root modulo 13.
%      \item  Using the primitive root determined in part (c), determine all primitive
%             roots modulo 17.
\vskip 2 in
      \item  Use the primitive root from part (c) to construct a table of indices
             modulo $13$.
\vskip 2.5 in
      \item  Use your table to find all solutions to
             $$
                  11 x^{10} \equiv 7\ \ (\textrm{mod } 13)\, .
             $$
%             answer.
    \end{enumerate}

\newpage
\item
    \begin{enumerate}
    \item Explain \rmp{[20]}what it means for $a$ to be a quadratic residue
          modulo the prime $p$.
\vskip 1.25 in
                \item Define the Legendre symbol
$\displaystyle{\left(\frac{a}{p}\right)}$ and state Euler's criterion.
\vskip 1.25 in
    \item Compute $11^{504}$
          (mod 1009).
\vskip 1.75 in
    \item Prove that the product of two non-residues modulo $p$ is a quadratic residue
          modulo $p$.
\vskip 3 in
    \item Does the following congruence have a solution? Give reasons for your answer.
             $$
                  7 x^{2} \equiv 16 \ \ (\textrm{mod } 139)\, .
             $$
    \end{enumerate}



\newpage
\item
       Let \rmp{[12]} $x=x_1$ and $y=y_1$ be the smallest
       positive solution to the Diophantine equation
       $$
             x^2 - D y^2 = 1\, .
       $$
    \begin{enumerate}
    \item Find all positive integer solutions to this equation.
\vskip 2 in
    \item Prove that the solutions given in (a) do in fact satisfy the given equation.
\vskip 4 in
    \item Find the positive integer solutions to \ $x^2 - 11 y^2 = 1$ having the three
          smallest $x$ values.
    \end{enumerate}

\renewcommand{\marginparsep}{-.2in}


\newpage
\item [Bonus:]
    \begin{enumerate}
    \item   Use \rmp{[10]} Euler's criterion to show that $x^4 \equiv -1$ (mod
$p$) has a solution if and only if $p\equiv 1$ (mod 8).
\vskip 5 in
    \item Using (a), show that there are infinitely many primes of the form $8k+1$.
    \end{enumerate}


\end{enumerate}


 \label{`last'}

\end{document}
