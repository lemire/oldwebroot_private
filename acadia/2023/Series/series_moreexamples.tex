%% LyX 1.1 created this file.  For more info, see http://www.lyx.org/.
%% Do not edit unless you really know what you are doing.
\documentclass[12pt,english]{amsart}
\usepackage[T1]{fontenc}
\usepackage[latin1]{inputenc}
\usepackage{fancyhdr}
\pagestyle{fancy}
\usepackage{babel}

\makeatletter

%%%%%%%%%%%%%%%%%%%%%%%%%%%%%% LyX specific LaTeX commands.
\providecommand{\LyX}{L\kern-.1667em\lower.25em\hbox{Y}\kern-.125emX\@}

%%%%%%%%%%%%%%%%%%%%%%%%%%%%%% Textclass specific LaTeX commands.
 \theoremstyle{plain}    
 \newtheorem{thm}{Theorem}[section]
 \numberwithin{equation}{section} %% Comment out for sequentially-numbered
 \numberwithin{figure}{section} %% Comment out for sequentially-numbered
 \theoremstyle{definition}
  \newtheorem*{example*}{Example}
 \theoremstyle{definition}
  \newtheorem{xca}[section]{Exercise}%%Delete [section] for sequential numbering
 \newcommand{\lyxaddress}[1]{
   \par {\raggedright #1 
   \vspace{1.4em}
   \noindent\par}
 }

%%%%%%%%%%%%%%%%%%%%%%%%%%%%%% User specified LaTeX commands.
\usepackage {graphicx}


\usepackage{epsfig}
\usepackage{amsmath,amssymb}
\usepackage{pslatex}
\usepackage{color}

 
 \definecolor{veryblackblue}{rgb}{0.0,0.0,0.1}
 \usepackage[pdftex,urlcolor=webblackblue,colorlinks=true,citecolor=blue,linkcolor=webblackblue]{hyperref}
 \pdfinfo{
            /Title      (Mathematics 2023 - Infinite Series I)
            /Author     (Daniel Lemire)
            /Subject    (A document on series for the course topics from advanced calculs offered at Acadia University.)
            /Keywords   ( series, convergence, Taylor, ratio test, cauchy sequences)
          }


\topmargin  = 0pt
\headheight = 0pt
\headsep    = 0pt

\voffset    = 0in
\hoffset    = 0in
\textheight = 230mm
\textwidth  = 164mm

\evensidemargin = 0pt
\oddsidemargin  = 0pt

\pagestyle{empty}
\usepackage{float}

%% Background of blue palette
 \definecolor{webblackblue}{rgb}{0.0,0.0,0.2}
 \definecolor{webblue}{rgb}{0.0, 0.0, 0.6}

%% Background of red palette
 \definecolor{webblackred}{rgb}{0.2,0.0,0.0}
 \definecolor{webred}{rgb}{0.6,0.0,0.0}

%% Background of green palette
 \definecolor{webblackgreen}{rgb}{0.0,0.2,0.0}                                   
 \definecolor{webgreen}{rgb}{0.0,0.6,0.0} 

%% Background of magenta palette
 \definecolor{webblackmagenta}{rgb}{0.14,0.0,0.14}                                   
 \definecolor{webmagenta}{rgb}{0.42,0.0,0.42} 

%% Background of cyan palette
 \definecolor{webblackcyan}{rgb}{0.0,0.14,0.14}                                   
 \definecolor{webcyan}{rgb}{0.0,0.42,0.42} 

%% Background of yellow pallete
 \definecolor{webblackyellow}{rgb}{0.14,0.14,0.0}                                   
 \definecolor{webyellow}{rgb}{0.85,0.85,0.0} 

 \definecolor{webdarkgray}{rgb}{0.2,0.2,0.2}

 \definecolor{webgray}{rgb}{0.75,0.75,0.75}
 \definecolor{weborange}{rgb}{1.0, 0.6, 0.0}

\renewcommand\labelenumi{\textcolor{webdarkgray}{\arabic{enumi}.}}
\newcommand{\setenumi}[1]{#1.\setcounter{enumi}{#1}}
\usepackage{dsfont}





\makeatother
\begin{document}

\title{Mathematics 2023 - Infinite Series I (Examples)\\
Acadia University}


\lyxaddress{Copyright 2002 D. Lemire}

\maketitle

\section{Questions}

\begin{xca}
Compute the Taylor expansion of \( 1/x \) about \( 1 \). Find the
radius of convergence. Using Taylor Theorem, find a bound on \( |1/x-S_{2}(x)| \)
where \( S_{2} \) is the sum of the first two terms over the interval
\( [1,2) \). Can you bound the error over \( [1,2) \) using another
theorem? What result do you get?
\end{xca}
~\vspace{3cm}

\begin{xca}
Will \( \sum _{k=0}^{\infty }\frac{e^{k}}{k!} \) converge???
\end{xca}
~\vspace{3cm}

\begin{xca}
Will \( \sum _{k=0}^{\infty }\frac{k!}{(2k)!} \) converge???
\end{xca}
~\vspace{3cm}

\begin{xca}
Will \( \sum _{k=2}^{\infty }\frac{(-1)^{k}}{\ln k} \) converge?\newpage
\end{xca}

\section{Solutions}

\begin{example*}
\( 1/x=1-(x-1)+(x-1)^{2}-(x-1)^{3}+... \). The ratio test gives us
\( R=|x-1| \) and thus the ratio of convergence is 1. Taylor Theorem
bounds the error after two terms by \( \frac{\frac{2}{\xi }(x-1)^{2}}{2}=\frac{(x-1)^{2}}{\xi }<\frac{(1)^{2}}{1}=1 \).
Using AST, we check that this is an alternating series (assuming \( 1>(x-1)>0 \)
makes this easy), hence the error after two terms is given by \( (x-1)^{2}<1 \).
\end{example*}
~

\begin{example*}
We apply the ratio test, \( \frac{e^{k+1}}{(k+1)!}\frac{k!}{e^{k}}=\frac{e}{k+1} \)
which goes to zero when \( k \) becomes big.
\end{example*}
~

\begin{example*}
Again, we apply the ratio test, \( \frac{(k+1)!}{(2k+2)!}\frac{(2k)!}{k!}=\frac{k+1}{(2k+2)(2k+1)} \)
which clearly goes to zero as \( k \) becomes big.
\end{example*}
~

\begin{example*}
Applying the ratio test, we get \( \frac{\ln k}{\ln (k+1)} \) which
tends to \( 1 \) as \( k \) becomes big (just use l'Hospital rule!)
However, since \( \frac{1}{\ln (k+1)}<\frac{1}{\ln k} \) (\( \ln  \)
is monotone increasing!) the series is alternating and thus, it converges!\end{example*}

\end{document}
